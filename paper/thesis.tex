%
% you should only have one "documentclass" line.  the following lines
% are samples that give various options.  the nofrontmatter option is
% nice because it suppresses the title and signature pages when you want
% to focus only on the main body of the thesis
%
% Friday April 10 2010 Ray Hylock <ray-hylock@uiowa.edu>
% documentclass options:
%   abstractpage            if you want to add an internal abstract (optional)
%   ackpage                 if you would like to add an acknowledgements page (optional)
%   algorithms              if you want a list of algorithms (optional)
%   appendix                if you have an appendix (optional)
%   copyrightpage           if you wish to copyright your thesis (optional)
%   dedicationpage          if you wish to make a dedication (optional)
%   epigraphpage            if you would like to add an epigraph to the beginning of your thesis (optional)
%   examples                if you want a list of examples (this uses the ntheorem package)
%   exampleslemmas          if you want a combined list of examples and lemmas (this uses the ntheorem package) (optional)
%   examplestheorems        if you want a combined list of examples and theorems (this uses the ntheorem package) (optional)
%   exampleslemmastheorems  if you want a combined list of examples, lemmas, and theorems (this uses the ntheorem package) (optional)
%   figures                 if you have any figures (this is required if you have even one figure)
%   lemmas                  if you want a list of lemmas (this uses the ntheorem package) (optional)
%   lemmastheorems          if you want a combined list of lemmas and theorems (this uses the ntheorem package) (optional)
%   nofrontmatter           suppresses the title and signiture pages for working on the body
%   tables                  if you have any tables (this is required if you have even one table)
%   theorems                if you want a list of theorems (this uses the ntheorem package) (optional)
%   phd                     if phd student; this will add the doctoral abstract (mandatory for PhD and DMA thesis candidates only)
%

% full options
%\documentclass[phd,abstractpage,copyrightpage,dedicationpage,epigraphpage,ackpage,figures,tables,lemmas,appendix]{uithesis}

% common options
%\documentclass[phd,dedicationpage,ackpage,figures,tables,appendix]{uithesis}

% example
\documentclass[phd,appendix,dedicationpage]{uithesis}

%=============================================================================
% User packages
%=============================================================================
\usepackage{bookmark}		% [recommended] for PDF bookmark generation
\usepackage{blindtext} 	% example text generation
\usepackage{mathpartir}
\usepackage{amsmath, amssymb,amsthm}
\usepackage{todonotes}
\usepackage{tikz}
\usepackage{diagrams}
\usepackage{supertabular}
\usepackage{chronology}
\usepackage{import}

% Theorems
\newcounter{thm}
%% Resets the thm counter at the start of each section.
\makeatletter
\@addtoreset{thm}{section}
\makeatother
%% This makes sure \ref has the section, subsection, and thm number.
\def\thethm{\thesubsection.\arabic{thm}}
%% This makes the figure numbering match the other environments.
% \renewcommand{\thefigure}{\thesubsection.\arabic{figure}}

\newenvironment{thm}{
 \refstepcounter{thm} \par \addvspace{\baselineskip} \noindent \textbf{\underline{Theorem \thethm.}} \begin{em}
   \begin{changemargin}{1px}{2px}\noindent
}{
  \end{changemargin}
  \end{em} \par \addvspace{\baselineskip} 
}
\newenvironment{lemma}{
  \refstepcounter{thm} \par \addvspace{\baselineskip} \noindent \textbf{\underline{Lemma \thethm.}} \begin{em}
    \begin{changemargin}{1px}{2px}\noindent
}{
  \end{changemargin}
  \end{em} \par \addvspace{\baselineskip}
}
\newenvironment{lemmaItem}{
  \refstepcounter{thm} \par \addvspace{\baselineskip} \noindent \textbf{\underline{Lemma \thethm.}} \begin{em}
}{
  \end{em} \par \addvspace{\baselineskip}
}
\newenvironment{proposition}{
  \refstepcounter{thm} \par \addvspace{\baselineskip} \noindent \textbf{\underline{Proposition \thethm.}} \begin{em}
    \begin{changemargin}{1px}{2px}\noindent
}{
    \end{changemargin}
  \end{em} \par \addvspace{\baselineskip}
}
\newenvironment{corollary}{
  \refstepcounter{thm} \par \addvspace{\baselineskip} \noindent \textbf{\underline{Corollary \thethm.}} \begin{em}
  \begin{changemargin}{1px}{2px}\noindent
}{
  \end{changemargin}
  \end{em} \par \addvspace{\baselineskip}
}
\newenvironment{definition}{
  \refstepcounter{thm} \par \addvspace{\baselineskip} \noindent \textbf{\underline{Definition \thethm.}} 
  \begin{em}\begin{changemargin}{1px}{2px}\noindent
}{
\end{changemargin}
  \end{em} \par \addvspace{\baselineskip} 
}
\newenvironment{example}{
  \refstepcounter{thm} \par \addvspace{\baselineskip} \noindent
  \textbf{\underline{Example \thethm.}} \begin{em}
    \begin{changemargin}{1px}{2px}\noindent
}{
  \end{changemargin}
  \end{em} \par \addvspace{\baselineskip}
}

% Commands that are useful for writing about type theory and programming language design.
\newcommand{\case}[4]{\text{case}\ #1\ \text{of}\ #2\text{.}#3\text{,}#2\text{.}#4}
\newcommand{\interp}[1]{[\negthinspace[#1]\negthinspace]}
\newcommand{\normto}[0]{\rightsquigarrow^{!}}
\newcommand{\join}[0]{\downarrow}
\newcommand{\redto}[0]{\rightsquigarrow}
\newcommand{\nat}[0]{\mathbb{N}}
\newcommand{\terms}[0]{\mathcal{T}}
\newcommand{\fun}[2]{\lambda #1.#2}
\newcommand{\CRI}[0]{\text{CR-Norm}}
\newcommand{\CRII}[0]{\text{CR-Pres}}
\newcommand{\CRIII}[0]{\text{CR-Prog}}
\newcommand{\subexp}[0]{\sqsubseteq}
\newcommand{\napprox}[2]{\lfloor #1 \rfloor_{#2}}
\newcommand{\interpset}{\mathcal{I}}
\newcommand{\powerset}[1]{\mathcal{P}(#1)}
\newcommand{\vinterp}[1]{\mathcal{V}[\negthinspace[#1]\negthinspace]}
\newcommand{\vbinterp}[2]{\bar{\mathcal{V}}_{#1}[\negthinspace[#2]\negthinspace]}
\newcommand{\ginterp}[1]{\mathcal{G}[\negthinspace[#1]\negthinspace]}
\newcommand{\dinterp}[1]{\mathcal{D}[\negthinspace[#1]\negthinspace]}
\newcommand{\tinterp}[1]{\mathcal{T}[\negthinspace[#1]\negthinspace]}

%=============================================================================
% prelude
%=============================================================================

\title{The Semantic Analysis of Advanced Programming Languages}
\author{Harley Daniel Eades III}
\dept{Computer Science}

% multipleSupervisors=true for two advisors
\setboolean{multipleSupervisors}{false}
\advisor{Associate Professor Aaron Stump}
% for multiple advisors; change <value> to line up the names
%\setboolean{multipleSupervisors}{true}
%\advisor{Advisor 1\\\hspace{<value>mm}Advisor 2...}
%
% edit the names below to have your committee members names appear
% on the signature page.  memberOne should be your advisor.
%
\memberOne{Aaron Stump}
\memberTwo{Cesare Tinelli}
\memberThree{Stephanie Weirich}
\memberFour{Gregory Landini}
\memberFive{Kasturi Varadarajan}
\submitdate{June 30}
\copyrightyear{2014}

\Abstract{
\blindtext
}

\dedication{To my lovely wife, Jenny Eades.}

%\epigraph{Epigraph here (optional)}

%% \acknowledgements{Acknowledgements here (optional)}

\newcommand{\LBMMT}[0]{\bar{\lambda}\mu\tilde\mu}

\begin{document}

\frontmatter

\section{Introduction}
\label{sec:introduction}

There are two major problems growing in two areas.  The first is in
Computer Science, in particular software engineering. Software is
becoming more and more complex, and hence more susceptible to software
defects.  Software bugs have two critical repercussions: they cost
companies lots of money and time to fix, and they have the potential
to cause harm. 

The National Institute of Standards and Technology estimated that
software errors cost the United State's economy approximately sixty
billion dollars annually, while the Federal Bureau of Investigations
estimated in a 2005 report that software bugs cost U.S. companies
approximately sixty-seven billion a year \cite{nist02,fbi05}.

Software bugs have the potential to cause harm.  In 2010 there were a
approximately a hundred reports made to the National Highway Traffic
Safety Administration of potential problems with the braking system of
the 2010 Toyota Prius \cite{Consumer:2010}.  The problem was that the
anti-lock braking system would experience a ``short delay'' when
the brakes where pressed by the driver of the vehicle
\cite{thedetroitbureau.com:2009}.  This actually caused some crashes.
Toyota found that this short delay was the result of a software bug,
and was able to repair the the vehicles using a software update
\cite{Reuters:2009}.  Another incident where substantial harm was
caused was in 2002 where two planes collided over \"{U}berlingen in
Germany. A cargo plane operated by DHL collided with a passenger
flight holding fifty-one passengers.  Air-traffic control did not
notice the intersecting traffic until less than a minute before the
collision occurred.  Furthermore, the on-board collision detection
system did not alert the pilots until seconds before the collision.
It was officially ruled by the German Federal Bureau of Aircraft
Accidents Investigation that the on-board collision detection was
indeed faulty \cite{Collision:2004}.

The second major problem affects all of science.  Scientific
publications are riddled with errors.  A portion of these errors are
mathematical.  In 2012 Casey Klein et al. used specialized computer
software to verify the correctness of nine papers published in the
proceedings of the International Conference on Functional Programming
(ICFP).  Two of the papers where used as a control which where known
to have been formally verified before.  In their paper
\cite{Klein:2012} they show that all nine papers contained
mathematical errors.  This is disconcerting especially since most
researchers trust published work and base their own work off of these
papers.  Kline's work shows that trusting published work might result
in wasted time for the researchers basing their work off of these
error prone publications.  Faulty research hinders scientific
progress.

Both problems outlined above have been the focus of a large body of
research over the course of the last forty years.  These challenges
have yet to be completed successfully.  The work we present here makes
up the foundations of one side of the programs leading the initiative
to build theory and tools which can be used to verify the correctness
of software and mathematics.  This program is called program
verification using dependent type theories.  The second program is
automated theorem proving.  In this program researchers build tools
called model checkers and satisfiability modulo-theories solvers.
These tools can be used to model and prove properties of large complex
systems carrying out proofs of the satisfiability of certain
constraints on the system nearly automatically, and in some cases
fully automatically.  As an example Andr\'{e} Platzer and Edmund
Clarke in 2009 used automated theorem proving to verify the
correctness of the in flight collision detection systems used in
airplanes.  They actually found that there were cases where two plans
could collide, and gave a way to fix the problem resulting in a fully
verified algorithm for collision detection.  That is he mathematically
proved that there is no possible way for two plans to collide if the
systems are operational \cite{DBLP:conf/fm/PlatzerC09}.  Automated
theorem provers, however, are tools used to verify the correctness of
software externally to the programming language and compiler one uses
to write the software.  In contrast with verification using dependent
types we wish to include the ability to verify software within the
programming language being used to write the software. Both programs
have their merits and are very fruitful and interesting.

Every formal language within this article has been formally defined in
a tool called Ott \cite{Sewell:2010}.  The full Ott specification of
every type theory defined with in this article can be found in the
appendix.  Ott is a tool for writing definitions of programming
languages, type theories, and $\lambda$-calculi.  Ott generates a
parser and a type checker which is used to check the accuracy of all
objects definable with in the language given to Ott as input.  Ott's
strongest application is to check for syntax errors within research
articles.  Ott is a great example of a tool using the very theory we
are presenting in this article.  It clearly stands as a successful
step towards the solution of the second major problem outlined above.
% section introduction (end)


\part{Background}
\label{part:background}

\import*{background/}{background}

% part background (end)

\part{Design}
\label{part:design}

% part design (end)

\part{Analysis}
\label{part:analysis}

% part analysis (end)

%=============================================================================
\appendix
%=============================================================================

\chapter{Type Theories}
\label{chap:type_theories}
\section{The $\lambda$-Calculus}
\label{sec:the_lambda-calculus}
\Lamall{}
% section the_lambda-calculus (end)

\newpage
\section{Church-Style Simply Typed $\lambda$-Calculus}
\label{sec:church_style_simply_typed_lambda-calculus}
\CHSTLCall{}
% section church_style_simply_typed_lambda-calculus (end)

\newpage
\section{Curry-Style Simply Typed $\lambda$-Calculus}
\label{sec:curry_style_simply_typed_lambda-calculus}
\CSTLCall{}
% section curry_style_simply_typed_lambda-calculus (end)

\newpage
\section{Combinatory Logic}
\label{sec:combinatory_logic}
\Comball{}
% section combinatory_logic (end)

\newpage
\section{G\"odel's System T}
\label{sec:godels_system_t}
\Tall{}
% section godels_system_t (end)

\newpage
\section{Girard/Reynold's System F}
\label{sec:girard-reynolds_system_f}
\Fall{}
% section girard-reynolds_system_f (end)

\newpage
\section{Stratified System F}
\label{sec:stratified_system_f}
\SSFall{}
% section girard-reynolds_system_f (end)

\newpage
\section{System $\text{F}^\omega$}
\label{sec:system_fw}
\Fwall{}
% section system_fw (end)

\newpage
\section{The $\lambda\mu$-Calculus}
\label{sec:lamu_all}
\Lamuall{}
% section lamu_all (end)

\newpage
\section{The $\lambda\Delta$-Calculus}
\label{sec:lamd_all}
\Lamdall{}
% section lamd_all (end)

\newpage
\section{The $\LBMMT$-Calculus}
\label{sec:lbmmt_all}
\LBMMTall{}
% section lbmmt_all (end)

\newpage
\section{The Dual-Calculus}
\label{sec:dc_all}
\DCall{}
% section dual_calculus_all (end)

\newpage
\section{Martin-L\"of's Type Theory}
\label{sec:tt_all}
\TTall{}
% section tt_all (end)

\newpage
\section{The Calculus of Constructions}
\label{sec:coc_all}
\CoCall{}
% section coc_all (end)

\newpage
\section{The Separated Calculus of Constructions}
\label{sec:coc_sep_all}
\CoCSall{}
% section coc_sep_all (end)
% section type_theories (end)

%=============================================================================
% bibliography
%=============================================================================
\interlinepenalty=10000	% prevents bib items from splitting across pages
\bibliographystyle{uithesis}
\bibliography{thesis}

\end{document}

%%% Local Variables: 
%%% mode: latex
%%% TeX-master: t
%%% End: 
