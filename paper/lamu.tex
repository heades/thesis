The reason intuitionistic logic was the focus is that it lends itself
very nicely to being interpreted as a system of computation.  That's
the entire point behind the BHK-interpretation and the work of
Brouwer.  This, it seemed, was not the case for classical logic, until
Michel Parigot constructed the $\lambda\mu$-calculus in 1992
\cite{Parigot:1992}.  Parigot was able to define a classical sequent
calculus called free deduction which had a cut-elimination procedure
validating the cut-theorem for classical logic \cite{Parigot:1992b}.
This allowed for Parigot to define a computational perspective of free
deduction which he called the $\lambda\mu$-calculus.  We now briefly
introduce the $\lambda\mu$-calculus.  The syntax and reduction rules
are in Fig.~\ref{fig:lamu_syntax}.
\begin{figure}
  \begin{center}
    \begin{tabular}{lll}
      Syntax: & \\
      & 
      \begin{math}
        \begin{array}{lll}
          [[T]],[[A]],[[B]],[[C]] & ::= & [[X]]\,|\,[[_|_]]\,|\,[[A -> B]]\\
          [[t]] & ::= & [[x]]\,|\,[[\x.t]]\,|\,[[\m a.s]]\,|\,[[t1 t2]]\\
          [[s]] & ::= & [[ [a]t]] 
        \end{array}
      \end{math}
      & \\
      Full $\beta$-reduction: & \\
      & 
      \begin{math}
        \begin{array}{lll}
          \LamudruleRXXBeta{} & \LamudruleRXXStruct{}\\
          &\\
          \LamudruleRXXRenaming{} & \LamudruleRXXLam{}\\
          &\\
          \LamudruleRXXMu{} & \LamudruleRXXNaming{}\\
          &\\
          \LamudruleRXXAppOne{} & \LamudruleRXXAppTwo{}
        \end{array}
      \end{math}
    \end{tabular}
  \end{center}

  \caption{Syntax and reduction rules for the $\lambda\mu$-calculus}
  \label{fig:lamu_syntax}
\end{figure}
We can think of the language of the $\lambda\mu$-calculus as an
extension of the $\lambda$-calculus.  We extend it with two new
operators.  The first is the $\mu$-abstraction $[[\m a.s]]$ where
$[[a]]$ is called a co-variable, an output port, or an output
variable.  We call the $\mu$-abstraction a control operator.  This
name conveys the fact that the $\mu$-abstraction has to ability to
control whether a value is returned or placed into its bound output
port.  The body of the $\mu$-abstraction must be a term called a
statement denoted by the metavariable $[[s]]$ which have the form $[[
    [a]t]]$.  We can think of this as assigning (or naming) an output
port to a term.  Now we extend the reduction rules with two new
reduction rules and two new congruence rules for the $\mu$-abstraction
and naming operator.  The \Lamudrulename{R\_Struct} is called the
structural reduction rule.  This allows one to target reduction to a
named subterm of the body of the $\mu$-abstraction.  This rule uses
a special substitution operation $[[ [t /* a]s]]$ which says to 
replace every subterm of $[[s]]$ matching the pattern $[[ [a]t']]$
with $[[ [a](t' t)]]$.  We may also write $[[ [t /* a]t']]$ for the
similar operation on terms. This is called structural substitution.

As we said above the language of the $\lambda\mu$-calculus is an
extension of the $\lambda$-calculus, but its type assignment is very
different than STLC.  The type assignment rules are defined in
Fig.\ref{fig:lamu_typing}.
\begin{figure}
  \begin{center}
    \begin{math}
      \begin{array}{lll}
        \LamudruleVar{} & \LamudruleLam{}\\
          & \\
          \LamudruleMu{} & \LamudruleApp{}\\
          &\\
          \LamudruleNameApp{}
      \end{array}
    \end{math}
  \end{center}
  \caption{Type-checking algorithm for the $\lambda\mu$-calculus}
  \label{fig:lamu_typing}
\end{figure}
Right away we can see a difference in the form of judgment.  We now
have $[[e : G |- D]]$ rather than $\Gamma \vdash t :
\tau$.  The former is in sequent form.  This is because the
$\lambda\mu$-calculus corresponds to a classical sequent calculus
unlike STLC which corresponds to intuitionistic natural deduction.  So
think of $[[e : G |- D]]$ as $[[e]]$ being a
witness\footnote{Actually, ``the witness'', because typing in unique.}
of the sequent $[[G]] [[|-]] [[D]]$.  Just like the other type
theories we have seen $[[G]]$ is the typing context or the set of
assumptions (input ports).  The environment $[[D]]$ is either empty
$[[.]]$, a formula $[[A]]$, one or more co-assumptions or output
ports, or a formula $[[A]]$ followed by one or more output ports.
Note that in $[[D]]$ we always have $[[_|_ a]]$ (false) and in $[[G]]$ we
always have $\top^[[x]]$ (true) where $\top \equiv [[_|_ -> _|_]]$ for any
$[[D]]$ and $[[G]]$ trivially.  We often leave these left implicit to
make the presentation clean unless absolutely necessary.  These two
facts hold because a sequent $[[A1]]^{[[x1]]},\cdots,[[Ai]]^{[[xi]]}
[[|-]] [[B]],[[B1]]^{[[a1]]},\cdots,[[Bi]]^{[[ai]]}$ 
can be interpreted as 
$([[A1]]^{[[x1]]} \land \cdots \land [[Ai]]^{[[xi]]}) \implies ([[B]] \lor [[B1]]^{[[a1]]} \lor \cdots \lor [[Bi]]^{[[ai]]})$
where $\implies$ is implication.  Using this interpretation we can see that
adding true to the left and/or false to the right does not impact the logical
truth of the statement.  This implies the following definition.
\begin{definition}
  \label{def:lamu_neg}
  The following rules are admissible w.r.t. the $\lambda\mu$-calculus:
  \begin{center}
    \begin{math}
      \begin{array}{lll}
        $$\mprset{flushleft}
        \inferrule* [right=BtmInt] {
          [[a fresh in D]]
          \\
            [[s : G |- D]]
        }{[[\m a.s : G |- _|_, D]]}
        &
        $$\mprset{flushleft}
        \inferrule* [right=BtmElim] {
          [[a fresh in D]]
          \\
            [[t : G |- _|_,D]]          
        }{[[ [a]t : G |- D]]}
      \end{array}
    \end{math}
  \end{center}
\end{definition}

The $\lambda\mu$-calculus is a classical type theory so it should be
the case that the law of excluded middle (LEM), $[[A]] \lor [[{-A}]]$,
holds, or equivalently the law of double negation (LDN) $[[{- {- A}} -> A]]$.
Since we do not have disjunction as a primitive we show LDN.
Before showing the derivation of the LDN we first define some
derived rules for handling negation and sequent manipulation rules.
The following definition defines all derivable rules.  We will take
these as primitive to make things cleaner.  We do not show the
derivations here, because they are rather straightforward.
\begin{definition}
  \label{def:lamu_derived_rules}
  The following rules are derivable using the typing rules and the rules of
  Def.~\ref{def:lamu_neg}:
  \begin{center}
    \begin{math}
      \begin{array}{lll}
        $$\mprset{flushleft}
        \inferrule* [right=NegInt1] {
          [[t : G,A x |- _|_,D]]
        }{[[\x.t:G |- {-A},D]]}
        &
        $$\mprset{flushleft}
        \inferrule* [right=NegElim1] {
          [[t1 : G |- {-A},D]]
          \\
          [[t2 : G |- A, D]]
        }{[[t1 t2 : G |- _|_,D]]}\\       
        &\\
        $$\mprset{flushleft}
        \inferrule* [right=NegInt2] {
          [[a fresh in D]]
          \\
          [[s : G, A x |- D]]
        }{[[\x.\m a.s : G |- {-A},D]]}
        &
        $$\mprset{flushleft}
        \inferrule* [right=NegElim2] {
          [[t1 : G |- {-A},D]]
          \\
          [[t2 : G |- A, D]]
        }{[[t1 t2 : G |- D]]}
      \end{array}
    \end{math}
  \end{center}
\end{definition}
We are now in the state where we can prove $[[{-{-A}} -> A]]$ in the $\lambda\mu$-calculus.
\begin{example}
  \label{ex:lamu_ldn}
  The proof of $[[{-{-A}} -> A]]$ is as follows:
  \begin{center}
    \begin{math}
      $$\mprset{flushleft}
      \inferrule* [right=Lam] {
        $$\mprset{flushleft}
        \inferrule* [right=Mu] {
          $$\mprset{flushleft}
          \inferrule* [right=NegElim2] {
            $$\mprset{flushleft}
            \inferrule* [right=Var] {
              \ 
            }{[[y : {-{-A}} y |- {-{-A}}]]} 
            \\
            $$\mprset{flushleft}
            \inferrule* [right=NegInt2] {
              $$\mprset{flushleft}
              \inferrule* [right=NameApp] {
                $$\mprset{flushleft}
                \inferrule* [right=Var] {
                  \ 
                }{[[x : {-{-A}} y, A x |- A a]]}
              }{[[ [a]x : {-{-A}} y, A x |- A a]]}
            }{[[\x.\m b.[a]x : {-{-A}} y |- {-A}, A a]]}
          }{[[ [b'](\x.\m b.[a]x) : {-{-A}} y |- A a]]}
        }{[[\m a.[b'](\x.\m b.[a]x) : {-{-A}} y |- A]]}
      }{[[\y.\m a.[b'](\x.\m b.[a]x):. |- {-{-A}} -> A]]}
    \end{math}
  \end{center}
\end{example}
In the above example we leave out freshness constraints to make the
presentation cleaner.  This example shows that the
$\lambda\mu$-calculus really is classical.  So from the logical
perspective of computation we gain classical reasoning, but do we
gain anything programmatically?  It turns out that we do.  We can
think of the $\mu$-abstraction and naming application as continuations
which allow us to define exceptions.  In fact a great way of thinking
about the $\mu$-abstraction $[[\m a.[b]t]]$ is due to Geuvers et al.:
\begin{center}
  \begin{quote}
    \emph{From a computational point of view one should think of $[[\m a.[b]t]]$
    as a combined operation that catches exceptions labeled $[[a]]$
    in $[[t]]$ and throws the results of $[[t]]$ to $[[b]]$.} \cite{Geuvers:2012}
  \end{quote}
\end{center}
Using this point of view we can define $[[catch a t]]$ and $[[throw a t]]$.
\begin{definition}
  \label{def:lamu_catch_throw}
  The following defines exceptions within the $\lambda\mu$-calculus:
  \begin{center}
    \begin{math}
      \begin{array}{lll}
        [[catch a t]] & := & [[\m a.[a]t]]\\
        & \\
        [[throw a t]] & := & [[\m b.[a]t]] \text{, where } [[b]] \text{ is fresh }
      \end{array}
    \end{math}
  \end{center}
\end{definition}
Using our reduction rules with the addition of $[[\m a.[a]t ~> t]]$
provided that $[[a]]$ is fresh in $[[t]]$\footnote{This is sometimes called
$\eta$-reduction for control operators.} we can easily define some
nice reduction rules for these definitions. 
\begin{definition}
  \label{def:lamu_catch_throw_red}
  Reduction rules for exceptions:
  \begin{center}
    \begin{math}
      \begin{array}{lll}
        [[catch a (throw a t) ~> catch a t]]\\
        & \\
        [[throw a (catch b t) ~> throw a ([a /* b]t)]]
      \end{array}
    \end{math}
  \end{center}  
\end{definition}
\noindent
There are other reductions one might want.  For the others and an
extension of the $\lambda\mu$-calculus see \cite{Geuvers:2012}.

\begin{openproblem}
  This is more of an investigation.  Most type theories are given in
  natural deduction form with introduction and elimination rules.  Why
  is this?  This we believe is a matter of perspective.  Some would
  claim that natural deduction provides a simple ``natural''
  formulation of computation and that sequent style formulations would
  burden the programmer, because they are used to programming with
  natural deduction.  However, sequent style type theories yield
  more symmetries with left and right rules.  We will talk more about
  symmetries and dualities in Sec.~\ref{subsec:beautiful_dualities}.
  We believe that the argument against sequent style programming
  languages is not convincing.  It seems sequent style formulations
  reveal more structure which yield better programming constructs.
  Thus, it would be worth while investigating using sequent style type
  theories as programming languages.

  \ \\ \textbf{Task:} Define a classical and/or intuitionistic sequent
  style type theory with programming in mind.  Investigate the
  possibility of constructing a surface language for programming in
  this core type theory.  Questions to consider are, does the sequent
  style formulation provide more of a burden for the programmer?  Can
  the surface language relieve some of this?  A lot of research has
  gone into sequent style logics.  Proof search for example is usually
  conducted on sequent style logics.  Can we pull research from proof
  search into a sequent style type theory to yield more automation?
\end{openproblem}
%%% Local Variables: 
%%% mode: latex
%%% TeX-master: "paper"
%%% End:
