\newcommand{\ccon}[4]{\mbox{case}\ #1\ \text{of}\ #2\text{.}#3\text{,}#2\text{.}#4}
\newcommand{\rcase}[5]{\mbox{rcase}_{#1}\ #2\ #3\ #4\ #5}
\newcommand{\F}[0]{\mbox{F}^{=}}
\newcommand{\Fp}[0]{\mbox{F}^{+}}
\newcommand{\STLCeq}[0]{\textnormal{STLC}^=}
\newcommand{\A}[0]{\mbox{\textbf{A}}}
\newcommand{\bredto}[0]{\rightsquigarrow_\beta}
\newcommand{\hlmn}[2]{
  \marginpar{
    \small
    {\color{blue}{
        \tiny #2
    }}}
    {\color{red}{
        #1
    }}
}
\input{SSF+-ott}
\section{Stratified System $\Fp$}
\label{sec:stratified_system_f_with_sum_types}
Stratified System $\Fp$ (SS$\Fp$) is an extension of SSF with sum types denoted
$T_1 + T_2$, whose elimination form $\textit{case}\ t\
\textit{of}\ x_1.t_1,x_2.t_2$ is used to case split on a whether or
not term $t$ with a sum type is truly $x_1$ of type $T_1$, or else
$x_2$ of type $T_2$.  We consider sum types\index{Sum Type} with so-called
commuting conversions\index{Commuting Conversion}, which allow independent cases to be permuted
past each other (see Fig~\ref{fig:syntax_ssfp} below).  Commuting
conversions are well-known to pose technical difficulties for
normalization proofs based on reducibility (see~\cite{tatsuta+05} and
Chapter 10 of~\cite{Girard:1989}).  We will see that they can be
handled straightforwardly with hereditary substitution.

\begin{figure}
  \begin{center}
    \begin{tabular}{l}
      Syntax:\\
      \begin{tabular}{lll}
        $K$ & $:=$ & $*_0$ $|$ $*_1$             $|$ $\ldots$\\
        $T$ & $:=$ & $X$   $|$ $T \rightarrow T$ $|$ $\forall X:K.T$  $|$ $T + T$\\
        $t$ & $:=$ & $x$   $|$ $\lambda x:T.t$   $|$ $t\ t$ $|$ 
        $\Lambda X:K.t$ $|$ $t[T]$ $|$ $inl(t)$ $|$ $inr(t)$ $|$ $\ccon{t}{x}{t}{t}$\\
      \end{tabular}
      \\ \\
      Reduction Rules:\\
      \begin{tabular}{cc}
        \begin{tabular}{lll}
          $(\Lambda X:*_p.t)[T]$ & $\rightsquigarrow$ & $[T/X]t$\\
          $(\lambda x:T.t)t'$    & $\rightsquigarrow$ & $[t'/x]t$\\
        \end{tabular}
        &
        \begin{tabular}{lll}
          $\ccon{inl(t)}{x}{t_1}{t_2}$ & $\rightsquigarrow$ & $[t/x]t_1$\\
          $\ccon{inr(t)}{x}{t_1}{t_2}$ & $\rightsquigarrow$ & $[t/x]t_2$
        \end{tabular}
      \end{tabular}
      \\ \\
      Commuting Conversions:\\
      \begin{tabular}{lll}
        $(\ccon{t}{x}{t_1}{t_2})\ t'$ & $\rightsquigarrow$ & 
        $\ccon{t}{x}{(t_1\ t')}{(t_2\ t')}$\\
        $\ccon{(\ccon{t}{x}{t_1}{t_2})}{y}{s_1}{s_2}$ & $\rightsquigarrow$ & 
        $\mbox{case}\ t\ \mbox{of}\ $\\
        &                    & \ \ \ $x.(\ccon{t_1}{y}{s_1}{s_2}),$\\
        &                    & \ \ \ $x.(\ccon{t_2}{y}{s_1}{s_2})$
      \end{tabular}
    \end{tabular}
  \end{center}
  
  \caption{Syntax, Reduction Rules, and Commuting Conversions for SS$\Fp$}
  \label{fig:syntax_ssfp}
\end{figure}

\begin{figure}[h]
  \begin{center}    
      \begin{mathpar}
        \inferrule* [right=] {
          \ 
        }{\cdot\ Ok}
        \and
        \inferrule* [right=] {
          \Gamma\ Ok
        }{\Gamma,X:*_p\ Ok}
        \and
        \inferrule* [right=] {
          \Gamma \vdash T:*_p
          \\
          \Gamma\ Ok
        }{\Gamma,x :T\ Ok}
      \end{mathpar} 
    
    \caption{Well-formedness of Contexts for SS$\Fp$}
    \label{fig:well-formed_ssfp}
  \end{center}
\end{figure}

\begin{figure}
  \begin{center}
    \setlength{\tabcolsep}{1pt}
    \begin{mathpar}    
        \inferrule* [right=] {
          \Gamma \vdash T_1 : *_p
          \\
          \Gamma \vdash T_2 : *_q
        }{\Gamma \vdash T_1 \rightarrow T_2 : *_{max(p,q)}}
        \and
        \inferrule* [right=] {
          \Gamma,X : *_q \vdash T : *_p
        }{\Gamma \vdash \forall X:*_q.T : *_{max(p,q)+1}}
        \and
        \inferrule* [right=] {
          \Gamma \vdash T_1 : *_p
          \\
          \Gamma \vdash T_2 : *_q
        }{\Gamma \vdash T_1 + T_2 : *_{max(p,q)}}
        \and
        \inferrule* [right=] {
          \Gamma(X) = *_p
          \\\\
          \Gamma\ Ok
          \\
          p \leq q
        }{\Gamma \vdash X : *_q}
      \end{mathpar}
    
    \caption{SS$\Fp$ Kinding Rules}
    \label{fig:kinding_rules_ssfp}
  \end{center}
\end{figure}

\begin{figure}
  \setlength{\tabcolsep}{1pt}
  \begin{mathpar}
        \inferrule* [right=] {
          \Gamma(x) = T
          \\\\
          \Gamma\ Ok
        }{\Gamma \vdash x : T}
        \and
        \inferrule* [right=] {
          \Gamma,x : T_1 \vdash t : T_2
        }{\Gamma \vdash \lambda x : T_1.t : T_1 \rightarrow T_2}
        \and
        \inferrule* [right=] {
          \Gamma \vdash t_1 : T_1 \rightarrow T_2 
          \\
          \Gamma \vdash t_2 : T_1
        }{\Gamma \vdash t_1\,t_2 : T_2}
        \and
        \inferrule* [right=] {
          \Gamma, X : *_l \vdash t : T
        }{\Gamma \vdash \Lambda X:*_l.t:\forall X : *_l.T}
        \and
        \inferrule* [right=] {
          \Gamma \vdash t:\forall X:*_l.T_1
          \\\\
          \Gamma \vdash T_2:*_l
        }{\Gamma \vdash t[T_2]: [T_2/X]T_1}
        \and
        \inferrule* [right=] {
          \Gamma \vdash t:T_1
          \\\\
	  \Gamma \vdash T_2:*_p
        }{\Gamma \vdash inl(t): T_1+T_2}
        \and
        \inferrule* [right=] {
          \Gamma \vdash t:T_2
          \\\\
	  \Gamma \vdash T_1:*_p
        }{\Gamma \vdash inr(t): T_1+T_2}
        \and
        \inferrule* [right=] {
          \Gamma \vdash t:T_1 + T_2
          \\\\
	  \Gamma,x:T_1 \vdash t_1:T
          \\\\
	  \Gamma,x:T_2 \vdash t_2:T
        }{\Gamma \vdash \ccon{t}{x}{t_1}{t_2}: T}
      \end{mathpar}
    
    \caption{SS$\Fp$ Type-Assignment Rules}
    \label{fig:typing_rules_ssfp}
  
\end{figure}
The syntax, reduction rules, and commuting conversions for SS$\Fp$ can
be found in Figure~\ref{fig:syntax_ssfp}.  The kind-assignment rules are
defined in Figure~\ref{fig:kinding_rules_ssfp} and the type-assignment
rules in defined in Figure~\ref{fig:typing_rules_ssfp}. The
kinding/typing relations depend on well-formed contexts which are
defined in Figure~\ref{fig:well-formed_ssfp}.  To ensure substitutions
over contexts behave in an expected manner, we rename variables as
necessary to ensure contexts have at most one declaration per
variable.  Lastly, the basic meta-theoretic results are used
throughout this chapter (we omit their proofs, because they are
similar to the proofs for SSF):

\begin{lemma}
  If $\Gamma \vdash T:*_p$ then $\Gamma\ Ok$.
  \label{lemma:kinding_ok_ssfp}
\end{lemma}
\begin{proof}
  This holds by straightforward induction on the form of the assumed
  kinding derivation.
\end{proof}

\begin{lemma}[Level Weakening for Kinding]
  If $\Gamma \vdash T:*_r$ and $r < s$ then $\Gamma \vdash T:*_s$.
  \label{lemma:level_weakening_for_kinding_ssfp}
\end{lemma}
\begin{proof}
  This holds by straightforward induction on the form of the assumed
  kinding derivation.
\end{proof}

\begin{lemma}[Substitution for Kinding, Context-Ok]
  Suppose $\Gamma \vdash T':*_p$.  \\ If $\Gamma,X:*_p,\Gamma' \vdash T:*_q$ 
  with a derivation of depth $d$, then $\Gamma,[T'/X]\Gamma' \vdash [T'/X]T:*_q$
  with a derivation of depth $d$.
  Also, if $\Gamma,X:*_p,\Gamma'\ Ok$ with a derivation of depth $d$, then 
  $\Gamma,[T'/X]\Gamma'\ Ok$ with a derivation of depth $d$.
  \label{lemma:substitution_for_kinding_ssfp}
\end{lemma}
\begin{proof}
  This holds by straightforward induction on the $d$.
\end{proof}

\begin{lemma}[Regularity]
  If $\Gamma \vdash t:T$ then $\Gamma \vdash T:*_p$ for some $p$.
  \label{lemma:regularity_ssfp}
\end{lemma}
\begin{proof}
  This holds by straightforward induction on the form of the assumed
  typing derivation.
\end{proof}

\subsection{Ordering on Types}
\label{subsec:ordering_on_types_ssfp}
In this section we define the ordering on types. It is a
straightforward extension of the ordering for SSF.

\begin{definition}\index{Ordering on Types}
  Suppose $\text{op} \in \{+,\to\}$.  Then the ordering $>_\Gamma$ is
  defined as the least relation satisfying the universal closures of
  the following formulas:
  \begin{center}
    \begin{tabular}{lll}
      \begin{tabular}{lll}
        $T_1 \mathrel{\text{op}} T_2$           & $>_\Gamma$ & $T_1$\\
        $T_1 \mathrel{\text{op}} T_2$           & $>_\Gamma$ & $T_2$\\
        $\forall X:*_l.T$        & $>_\Gamma$ & $[T'/X]T$ where 
        $\Gamma \vdash T':*_l$.\\
      \end{tabular}
    \end{tabular}
  \end{center}
  \label{def:ordering_ssfp}
\end{definition}

\begin{thm}[Well-Founded Ordering]
  The ordering $>_\Gamma$ is well-founded on types $T$ such that 
  $\Gamma \vdash T:*_l$ for some $l$.
  \label{thm:well-founded_ordering_ssfp}
\end{thm}
\begin{proof}
  The depth function, defined in the following definition, is used in the following proof.

  \begin{definition}
    The depth of a type $T$ is defined as follows:
    \begin{center}
      \begin{tabular}{lll}
        $depth(X)$                  & $=$ & $1$\\
        $depth(T \to T')$     & $=$ & $depth(T) + depth(T')$\\
        $depth(T + T')$       & $=$ & $depth(T) + depth(T')$\\
        $depth(\forall X:*_l.T)$ & $=$ & $depth(T) + 1$\\
      \end{tabular}
    \end{center}
  \end{definition}

  We define the metric $(l,d)$ in lexicographic combination, where $l$
  is the level of a type $T$ and $d$ is the depth of $T$.

  \begin{lemma}[Well-Founded Measure]
    \label{lemma:well-founded_measure_ssfp}
    If $T >_\Gamma T'$ then $(l,d) > (l',d')$, where $\Gamma \vdash T:*_l$, 
    $depth(T) = d$,  $\Gamma \vdash T:*_{l'}$, and $depth(T') = d'$.
  \end{lemma}
  \begin{proof}
    This holds by straightforward induction on the structure of
    $T$. 
  \end{proof}
  
  Finally, the proof of well-foundedness of $>_\Gamma$.  If there exists
  an infinite decreasing sequence using our ordering on types, then there
  is an infinite decreasing sequence using our measure by
  Lemma~\ref{lemma:well-founded_measure_ssfp}, but that is impossible.
\end{proof}
\noindent
We need transitivity in a number of places in the proof of the main
substitution lemma.  

\begin{lemma}[Transitivity of $>_\Gamma$]
  Let $T$, $T'$, and $T''$ be kindable types.  \\
  If $T >_\Gamma T'$ and $T' >_\Gamma T''$ then $T >_\Gamma T''$.
  \label{lemma:transitivity_ssfp}
\end{lemma}
\begin{proof}
  This proof is similar to the proof for the ordering used in the
  proof of normalization of SSF (Lemma~\ref{lemma:transitivity_ssf}).
\end{proof}
% subsection ordering_on_types_ssfp (end)

\subsection{Hereditary Substitution}
\label{sec:the_hereditary_substitution_function_ssfp}

The definition of the hereditary substitution function for SS$\Fp$ is
in Figure~\ref{fig:hereditary_substitution_function_part1} and
Figure~\ref{fig:hereditary_substitution_function_part2}.  First, one
should read this definition as a mutually recursive function in terms
of the hereditary substitution function $[t/x]^T t'$, the
application reduction function $app_T\ t_1\ t_2$, and case
construct reduction function $rcase_T\ t_0\ x\ t_1\ t_2$.  The
definitions of all these functions depend the same definition of
$ctype_T(x, t)$ as the proof of normalization of SSF
(Definition~\ref{def:ctype_function_ssf}).  So we do not repeat it
here.
\begin{figure}\index{Stratified System $\Fp$ (SS$\Fp$)!Hereditary
    Substitution Function}
  \begin{itemize}
  \item[] $app_T\ t_1\ t_2 = t_1\ t_2$
    \hscase{
      Where $t_1$ is not a $\lambda$-abstraction or a case construct.
    }
    
  \item[] $app_T\ (\lambda x:T'.t_1)\ t_2 = [t_2/x]^{T'} t_1$
  \item[] $app_T\ (\ccon{t_0}{x}{t_1}{t_2})\ t = 
    \ccon{t_0}{x}{(app_T\ t_1\ t)}{(app_T\ t_2\ t)}$
  \end{itemize}
  
  \begin{itemize}
  \item[] $rcase_{T}\ t_0\ y\ t_1\ t_2 = \ccon{t_0}{y}{t_1}{t_2}$
    \hscase{
      Where $t_0$ is not an inject-left or an inject-right term or a case construct.    
    }
  \item[] $rcase_{T}\ inl(t')\ y\ t_1\ t_2 = [t'/y]^{T_1}\ t_1$
  \item[] $rcase_{T}\ inr(t')\ y\ t_1\ t_2 = [t'/y]^{T_2}\ t_2$
  \item[] $rcase_{T}\,(\ccon{t'_0}{x}{t'_1}{t'_2})\,y\,t_1\,t_2 =$\\
    \vspace{-15px}
    \hscase{
      $\ccon{t'_0}{x}{(rcase_{T}\,t'_1\,y\,t_1\,t_2)}{(rcase_{T}\,t'_2\,y\,t_1\,t_2)}$
    }
  \end{itemize}
  \caption{Hereditary Substitution Function for Stratified System $\Fp$}
  \label{fig:hereditary_substitution_function_part1}
\end{figure}

\begin{figure}
  \begin{itemize}   
  \item[] $[t/x]^T x = t$
  \item[] $[t/x]^T y = y$
    \hscase{Where $y$ is a variable distinct from $x$.}
    
  \item[] $[t/x]^T (\lambda y:T'.t') = \lambda y:T'.([t/x]^T t')$
  \item[] $[t/x]^T (\Lambda X:*_l.t') = \Lambda X:*_l.([t/x]^T t')$
  \item[] $[t/x]^T inr(t') = inr([t/x]^T t')$
  \item[] $[t/x]^T inl(t') = inl([t/x]^T t')$
  \item[] $[t/x]^T (t_1\ t_2) = ([t/x]^T t_1)\ ([t/x]^T t_2)$
    \hscase{
      Where $([t/x]^T t_1)$ is not a $\lambda$-abstraction or a case construct, or both 
      $([t/x]^T t_1)$ and $t_1$ are $\lambda$-abstractions or case constructs, or $ctype_T(x,t_1)$ is 
      undefined.        
    }
  \item[] $[t/x]^{T} (t_1\ t_2) = [([t/x]^{T} t_2)/y]^{T''} s'_1$
    \hscase{
      Where $([t/x]^{T} t_1) = \lambda y:T''.s'_1$ for some $y$,
      $s'_1$, and $T''$ and $ctype_T(x,t_1) = T'' \to T'$.
    }

  \item[] $[t/x]^{T} (t_1\ t_2) = 
    \ccon{w}{y}{(app_T\ r\ ([t/x]^{T} t_2))}{(app_T\ s\ ([t/x]^{T} t_2))}$
    \hscase{
      Where $[t/x]^{T} t_1 = \ccon{w}{y}{r}{s}$ for some terms $w$, $r$, $s$ 
      and variable $y$, and 
      $ctype_T(x,t_1) = T'' \to T'$.
    }

  \item[] $[t/x]^T (t'[T']) = ([t/x]^T t')[T']$
    \hscase{
      Where $[t/x]^T t'$ is not a type abstraction or
      $t'$ and $[t/x]^T t'$ are type abstractions.    
    }
    
    \item[] $[t/x]^{T} (t'[T']) = [T'/X]s'_1$
      \hscase{
        Where $[t/x]^{T} t' = \Lambda X:*_l.s'_1$,
        for some $X$, $s'_1$ and $\Gamma \vdash T':*_q$, such that, $q \leq l$ and\\
        $t'$ is not a type abstraction.
      }

    \item[] $[t/x]^{T} (\ccon{t_0}{y}{t_1}{t_2}) = 
      \ccon{([t/x]^{T} t_0)}{y}{([t/x]^{T} t_1)}{([t/x]^{T} t_2)}$
      \hscase{
        Where $([t/x]^{T} t_0)$ is not an inject-left or an inject-right term or a case 
        construct, or $([t/x]^{T} t_0)$ and $t_0$ are both inject-left or inject-right terms or case 
        constructs, or $ctype_T(x,t_0)$ is undefined.
      }

    \item[] $[t/x]^{T} (\ccon{t_0}{y}{t_1}{t_2}) = rcase_{T}\ ([t/x]^{T} t_0)\ y\  ([t/x]^{T} t_1)\ ([t/x]^{T} t_2)$
      \hscase{
        Where $([t/x]^{T} t_0)$ is an inject-left or an inject-right term or 
        a case construct and $ctype_T(x,t_0) = T_1+T_2$.
      }
  \end{itemize}
  \caption{Hereditary Substitution Function for Stratified System $\Fp$ Continued}
  \label{fig:hereditary_substitution_function_part2}
\end{figure}
The hereditary substitution function is an extension of the hereditary
substitution function for SSF.  The most interesting case of the
definition is when a commuting conversion is created as a result of
substitution.  In this case we know by the $ctype_T$ function that
the head of the application is in the form $x\,t_1 \cdots t_2$.
Furthermore, we know that applying the hereditary substitution
function to the head of the application results in a case construct.
So we recursively reduce the created redex in the same way the
reduction rules do, but when we push the argument into the branches of
the resulting case construct more redexes may be created.  So to
handle recursively reducing all of the newly created redexes in the
branches we call the application reduction function $app_T$.  This
function reduces redexes by recursively calling itself and the
hereditary substitution function.  The remaining cases where new
redexes are potentially created are similar to these cases.  The
function $rcase_T$ handles reducing case constructs.
% subsection the_hereditary_substitution_function_ssfp (end)

\subsection{Main Properties}
\label{sec:properties_of_the_hereditary_substitution_function_ssfp}
We are now to the point where we can prove the properties of the
hereditary substitution function.  Just as we did in the proof for SSF
we first must show the properties of $ctype_T$ hold.

\begin{lemma}[Properties of $ctype_T$]
  \label{lemma:ctype_props_ssfp}
  \vspace{-30px}
  \begin{changemargin}{10px}{5px}\noindent
  \begin{itemize}
  \item[i.] If $ctype_T(x,t) = T'$ then $head(t) = x$ and $T'$ 
    is a subexpression of $T$.
    
  \item[ii.] If $\Gamma,x:T,\Gamma' \vdash t:T'$ and $ctype_T(x,t) = T''$ then
    $T' \equiv T''$.

  \item[iii.] If $\Gamma,x:T,\Gamma' \vdash t_1\ t_2:T'$, $\Gamma \vdash t:T$,
    $[t/x]^T t_1 = \lambda y:T_1.q$, and $t_1$ is not then there exists a type
    $T$ such that $ctype_T(x,t_1) = T$.

  \item[iv.] If $\Gamma,x:T,\Gamma' \vdash t_1\ t_2:T'$, $\Gamma \vdash t:T$,
    $[t/x]^T t_1 = \ccon{t'_0}{y}{t'_1}{t'_2}$, and $t_1$ is not then there exists a type
    $T$ such that $ctype_T(x,t_1) = T$.

  \item[v.] If $\Gamma,x:T,\Gamma' \vdash \ccon{t_0}{y}{t_1}{t_2}:T'$, 
    $\Gamma \vdash t:T$, $[t/x]^T t_0 = \ccon{t'_0}{z}{t'_1}{t'_2}$, and 
    $t_0$ is not then there exists a type $T$ such that $ctype_T(x,t_0) = T$.

  \item[vi.] If $\Gamma,x:T,\Gamma' \vdash \ccon{t_0}{y}{t_1}{t_2}:T'$, 
    $\Gamma \vdash t:T$, $[t/x]^T t_0 = inl(t')$, and $t_0$ is not then there 
    exists a type $T$ such that $ctype_T(x,t_0) = T$.

  \item[vii.] If $\Gamma,x:T,\Gamma' \vdash \ccon{t_0}{y}{t_1}{t_2}:T'$, 
    $\Gamma \vdash t:T$, $[t/x]^T t_0 = inr(t')$, and $t_0$ is not then 
    there exists a type $T$ such that $ctype_T(x,t_0) = T$.
  \end{itemize}
  \end{changemargin}
\end{lemma}
\begin{proof}
  The first two cases are equivalent to the proof of the properties
  for SSF (Lemma~\ref{lemma:ctype_props_ssf}).  Cases three through five are similar, so we only give the
  proof of part three.  This is a proof by induction on the structure
  of $t_1\ t_2$.

  The only possiblities for the form of $t_1$ is $x$ or $\hat{t}_1\ \hat{t}_2$.  All other 
  forms would not result in $[t/x]^T t_1$ being a $\lambda$-abstraction and $t_1$ not.
  If $t_1 \equiv x$ then there exist a type $T''$ such that $T \equiv T'' \to T'$ and
  $ctype_T(x,x\ t_2) = T'$ when $ctype_T(x,x) = T \equiv T'' \to T'$ in this case.  We know
  $T''$ to exist by inversion on $\Gamma,x:T,\Gamma' \vdash t_1\ t_2:T'$.

  Now suppose $t_1 \equiv (\hat{t}_1\ \hat{t}_2)$.  Now knowing $t'_1$ to not be a $\lambda$-abstraction
  implies that $\hat{t}_1$ is also not a $\lambda$-abstraction or $[t/x]^T t_1$ would be an application
  instead of a $\lambda$-abstraction.  So it must be the case that $[t/x]^T \hat{t}_1$ is a $\lambda$-abstraction
  and $\hat{t}_1$ is not.  Since $t_1\ t_2 > t_1$ we can apply the induction hypothesis to obtain there exists
  a type $T$ such that $ctype_T(x,\hat{t}_1) = T$.  
  Now by inversion on $\Gamma,x:T,\Gamma' \vdash t_1\ t_2:T'$ we know there exists a type $T''$ such that
  $\Gamma,x:T,\Gamma' \vdash t_1:T'' \to T'$.  We know $t_1 \equiv (\hat{t}_1\ \hat{t}_2)$ so by inversion on
  $\Gamma,x:T,\Gamma' \vdash t_1:T'' \to T'$ we know there exists a type $T''$ such that
  $\Gamma,x:T,\Gamma' \vdash \hat{t}_1:T'' \to (T'' \to T')$.
  By part two of Lemma~\ref{lemma:ctype_props_ssfp} we know $T \equiv T'' \to (T'' \to T')$ and
  $ctype_T(x,t_1) = ctype_T(x,\hat{t}_1\ \hat{t}_2) = T'' \to T'$ 
  when $ctype_T(x,\hat{t}_1) = T'' \to (T'' \to T')$, because we know $ctype_T(x,\hat{t}_1) = T$.
\end{proof}

We now move on to proving the main properties of the hereditary
substitution function.  First, we show that for typeable terms it is a
total function and the output maintains the same type as the principle
term of substitution.
\begin{lemma}[Total and Type Preserving]
  \label{lemma:total_ssfp}
  Suppose $\Gamma \vdash t : T$ and $\Gamma, x:T, \Gamma' \vdash t':T'$. Then
  there exists a term $t''$ such that $[t/x]^T t' = t''$ and 
  $\Gamma,\Gamma' \vdash t'':T'$.
\end{lemma}
\begin{proof}
  This is a proof by induction on the lexicorgraphic combination
  $(T, t')$ of $>_{\Gamma,\Gamma'}$ and the strict subexpression
  ordering.  We case split on $t'$.
\vspace{-20px}
\begin{changemargin}{10px}{5px}\noindent
\begin{itemize}
\item[Case.] Suppose $t'$ is either $x$ or a variable $y$ distinct from $x$.  
  Trivial in both cases.
  
\item[Case.] Suppose $t' \equiv \lambda y:T_1.t'_1$.  By inversion on the
  typing judgement we know $\Gamma,x:T,\Gamma',y:T_1 \vdash t'_1:T_2$.
  We also know $t' > t'_1$, hence we can apply the induction hypothesis to obtain
  $[t/x]^T t'_1 = \hat{t}'_1$ and $\Gamma,\Gamma',y:T_1 \vdash \hat{t}:T_2$
  for some term $\hat{t}'_1$.  By the definition of the hereditary substitution function 
  $[t/x]^T t' = \lambda y:T_1.[t/x]^T t'_1 = \lambda y:T_1.\hat{t}'_1$.  It suffices
  to show that $\Gamma,\Gamma' \vdash \lambda y:T_1.\hat{t}'_1:T_1 \to T_2$.  
  By simply applying the $\lambda$-abstraction typing rule using
  $\Gamma,\Gamma',y:T_1 \vdash \hat{t}:T_2$ we obtain 
  $\Gamma,\Gamma' \vdash \lambda y:T_1.\hat{t}'_1:T_1 \to T_2$.
  
\item[Case.] Suppose $t' \equiv \Lambda X:*_l.t'_1$.  Similar to the previous case.
  
\item[Case.] Suppose $t' \equiv t'_1\ t'_2$.  By inversion we know
  $\Gamma, x:T, \Gamma' \vdash t'_1 : T'' \to T'$ and
  $\Gamma, x:T, \Gamma' \vdash t'_2 : T''$ for some types $T'$ and $T''$.
  Clearly, $t' > t'_i$ for $i \in \{1,2\}$.  Thus, by the induction hypothesis
  there exists terms $m_1$ and $m_2$ such that $[t/x]^T t'_i = m_i$,
  $\Gamma, \Gamma' \vdash m_1 : T'' \to T'$ and
  $\Gamma, \Gamma' \vdash m_2 : T''$ for
  $i \in \{1,2\}$.  We case split on whether or not $m_1$ is a $\lambda$-abstraction,
  a case construct and $t'_1$ is not, or $ctype_T(x,t'_1)$ is undefined.  
  We only consider the non-trivial cases when 
  $m_1 \equiv \lambda y:T''.m'_1$ and $t'_1$ is not a $\lambda$-abstraction or $m_1 \equiv \ccon{m'_0}{y}{m'_1}{m'_2}$,
  $ctype_T(x,t'_1) = T'' \to T'$, and $t'_1$ is not a case construct.  Suppose the former.  
  Now by Lemma~\ref{lemma:ctype_props_ssfp} it is the case that 
  there exists a $T$ such that $ctype_T(x,t'_1) = T$, 
  $T \equiv T'' \to T'$, and $T$ is a subexpression of $T$, hence
  $T >_{\Gamma,\Gamma'} T''$.
  Then $[t/x]^T (t'_1\ t'_2) = [m_2/y]^{T''} m'_1$.  
  Therefore, by the induction hypothesis there exists a 
  term $m$ such that $[m_2/y]^{T''} m'_1 = m$ and $\Gamma,\Gamma' \vdash m:T''$.
   
  Suppose $m_1 \equiv \ccon{m'_0}{y}{m'_1}{m'_2}$ and $t'_1$ is not a case construct.
  Now $[t/x]^T t' = \ccon{m'_0}{y}{app_T\ m'_1\ m_2}{app_T\ m'_2\ m_2}$.  By inversion on
  $\Gamma,\Gamma' \vdash m_1 : T'' \to T'$ we know there exists terms $T_1$ and $T_2$ such that
  $\Gamma,\Gamma' \vdash m'_0:T_1+T_2$ and
  $\Gamma,\Gamma',y:T_i \vdash m'_i:T'' \to T'$
  for $i \in \{1,2\}$.  It suffcies to show that
  there exists terms $q$ and $q'$ such that $app_T\ m'_1\ m_2 = q$ and $app_T\ m'_2\ m_2 = q'$.  To obtain
  this result we prove the following proposition.  Note that $ctype_T(x,t'_1) = T'' \to T'$ which
  by Lemma~\ref{lemma:ctype_props_ssfp} is equivalent to $T'' \to T'$ and is 
  a subexpression of $T$, hence $T >_{\Gamma,\Gamma'} T''$ and $T >_{\Gamma,\Gamma'} T'$.
 
  {\bf Proposition.}  For all 
  $\Gamma \vdash m_2 : T''$ and $\Gamma \vdash m'_1:T'' \to T'$
  there exists a term $q$ such that $app_T\ m'_1\ m_2 = q$ and $\Gamma \vdash q:T'$.
  
  We prove this by nested induction on the ordering $(T, t', m'_1)$ and case splitting on 
  the structure of $m'_1$.
  \begin{itemize}
  \item[Case.] Suppose $m'_1$ is neither a $\lambda$-abstraction or a case construct.  Then\\
    $app_T\  m'_1\ m_2 = m'_1\ m_2$.  Take $m'_1\ m_2$ for $q$ and by applying the application typing rule
    we know $\Gamma \vdash m'_1\ m_2:T'$.
    
  \item[Case.] Suppose $m'_1 \equiv \lambda z:T''.m''_1$.  Then $app_T\ m'_1\ m_2 = [m_2/z]^{T''} m''_1$.
    By inversion on the assumption $\Gamma \vdash m'_1:T'' \to T'$ we know 
    $\Gamma,z:T'' \vdash m''_1:T'$.  Since $T >_{\Gamma} T''$ we can apply the outter induction
    hypothesis to obtain there there exists a $q$ such that $[m_2/z]^{T''} m''_1 = q$ and 
    $\Gamma \vdash q:T'$.  Therefore, $app_T\ m'_1\ m_2 = q$.
    
  \item[Case.] Suppose $m'_1 \equiv \ccon{m''_0}{z}{m''_1}{m''_2}$.  Then\\
    $app_T\ m'_1\ m_2 = \ccon{m''_0}{z}{app_T\ m''_1\ m_2}{app_T\ m''_2\ m_2}$.  By inversion on the assumption
    $\Gamma \vdash m'_1:T'' \to T'$ we know there exists types $T_1$ and $T_2$ such that
    $\Gamma \vdash m''_0:T_1+T_2$, $\Gamma,z:T_1 \vdash m''_1:T'' \to T'$
    and $\Gamma,z:T_2 \vdash m''_2:T'' \to T'$.  Since $m'_1 > m''_1$ and $m'_1 > m''_2$ we can 
    apply the inner induction hypothesis to obtain there exists terms $q'$ and $q''$ such that 
    $app_T\ m''_1\ m_2 = q'$, $\Gamma,z:T_1 \vdash q':T'$, $app_T\ m''_1\ m_2 = q''$ and $\Gamma,z:T_2 \vdash q'':T'$.  
    Hence, 
    $app_T\ m'_1\ m_2 = \ccon{m''_0}{z}{app_T\ m''_1\ m_2}{app_T\ m''_2\ m_2} = \ccon{m''_0}{z}{q'}{q''}$.  It suffices to 
    to show that $\Gamma \vdash \ccon{m''_0}{z}{q'}{q''}:T'$.  This is a simple consequence of applying the
    case-construct typing rule using $\Gamma \vdash m''_0:T_1+T_2$, $\Gamma,z:T_1 \vdash q':T'$, and
    $\Gamma,z:T_2 \vdash q'':T'$.        
  \end{itemize}
   
  By the previous proposition there exists terms $q$ and $q'$ such that \\
  $[t/x]^T t' = \ccon{m'_0}{y}{app_T\ m'_1\ m_2}{app_T\ m'_2\ m_2}
  = \ccon{m'_0}{y}{q}{q'}$, where $app_T\ m'_1\ m_2 = q$, $\Gamma,\Gamma',y:T_1 \vdash q:T'$, $app_T\ m'_1\ m_2 = q'$
  and $\Gamma,\Gamma',y:T_2 \vdash q':T'$.  It suffices to show that
  $\Gamma, \Gamma' \vdash \ccon{m'_0}{y}{q}{q'}:T'$.  From above we know that $\Gamma,\Gamma' \vdash m'_0:T_1+T_2$, 
  $\Gamma,\Gamma',y:T_1 \vdash q:T'$ and $\Gamma,\Gamma',y:T_2 \vdash q':T'$.  Thus,
  by applying the case-construct typing rule we obtain $\Gamma, \Gamma' \vdash \ccon{m'_0}{y}{q}{q'}:T'$.
  
\item[Case.] Suppose $t' \equiv t'_1[T'']$. Similar to the previous case.  
  
\item[Case.] Suppose $t' \equiv inl(t)$. Trivial.
  
\item[Case.] Suppose $t' \equiv inr(t)$. Trivial.
  
\item[Case.] Suppose $t' \equiv \ccon{m_0}{y}{m_1}{m_2}$. By inversion on the assumption
  $\Gamma,x:T,\Gamma' \vdash t':T'$ we know the following:
  \begin{center}
    \begin{math}
      \begin{array}{lll}
        \Gamma,x:T,\Gamma' \vdash m_0:T_1+T_2 \text{, for some types } T_1 \text{ and } T_2,\\
        \Gamma,x:T,\Gamma',y:T_1 \vdash m_1:T, and\\
        \Gamma,x:T,\Gamma',y:T_2 \vdash m_2:T.\\
      \end{array}
    \end{math}
  \end{center}
  It is easy to see that
  $t' > m_i$ for all $i \in \{0,1,2\}$.  Hence, by the induction hypothesis
  there exists terms $m'_0$, $m'_1$, and $m'_2$ such that $[t/x]^T m_i = m'_i$ for all $i \in \{0,1,2\}$,
  \begin{center}
    \begin{tabular}{lll}
      (i)   & $\Gamma,\Gamma \vdash m'_0:T_1+T_2$,  \\
      (ii)  & $\Gamma,\Gamma',y:T_1 \vdash m'_1:T'$, and\\
      (iii) & $\Gamma,\Gamma',y:T_2 \vdash m'_2:T'$.  
    \end{tabular}
  \end{center}
  We have two cases to consider.
  \begin{itemize}
  \item[Case.] Suppose $m_0$ and $m'_0$ are inject-left terms, inject-right terms, or case constructs,
    or $m_0$ is an inject-left term, inject-right term, or a case-construct and $m'_0$ is not, or
    $m_0$ and $m'_0$ are neither inject-left terms, inject-right terms, or case constructs, or
    $ctype_T(x,m_0)$ is undefined.
    Then \\
    $[t/x]^{T} (\ccon{m_0}{y}{m_1}{m_2}) = \ccon{m'_0}{y}{m'_1}{m'_2}$ and by applying the case-construct 
    typing rule to i - iii above we obtain \\ $\Gamma,\Gamma' \vdash \ccon{m'_0}{y}{m'_1}{m'_2}:T'$.
    
  \item[Case.] Suppose $m'_0$ is an inject-left term, inject-right term, or case construct and $ctype_T(x,m_0) = T_1+T_2$.  Then\\
    $[t/x]^{T} (\ccon{m_0}{y}{m_1}{m_2}) = rcase_{T}\ m'_0\ y\ m'_1\
    m'_2$ and by \\ Lemma~\ref{lemma:ctype_props_ssfp} we know
    $T_1 + T_2 \equiv T_1+T_2$ and is a subexpression of $T$.
    It suffices to show that there exists some term
    $q$ such that \\ $rcase_{T}\ m'_0\ y\ m'_1\ m'_2 = q$ and $\Gamma, \Gamma' \vdash q:T'$.  We obtain this
    result by the following proposition.
       
    {\bf Proposition.} For all $\Gamma \vdash q_0:T$, $\Gamma, y:T_1 \vdash q_1:T'$, 
    and $\Gamma, y:T_2 \vdash q_2:T'$ there exists a term $\hat{q}$ such that 
    $rcase_{T}\ q_0\ y\ q_1\ q_2 = \hat{q}$ and $\Gamma \vdash \hat{q}:T'$.  
    We prove this by induction on the the ordering $(T, t', q_0)$ and case split on the structure of $q_0$.
    \begin{itemize}
    \item[Case.] Suppose $q_0$ is not an inject-left term, inject-right term, or a case construct.  Then\\
      $rcase_{T}\ q_0\ y\ q_1\ q_2 = \ccon{q_0}{y}{q_1}{q_2}$ and by applying the case-construct typing rule
      using the assumptions $\Gamma \vdash q_0:T$, $\Gamma, y:T_1 \vdash q_1:T'$, 
      and $\Gamma, y:T_2 \vdash q_2:T'$ we obtain $\Gamma,\Gamma' \vdash \ccon{q_0}{y}{q_1}{q_2}:T'$.
      
    \item[Case.] Suppose $q_0 \equiv inl(q'_0)$.  Then $rcase_{T}\ q_0\ y\ q_1\ q_2 = [q'_0/y]^{T_1} q_1$ and
      by inversion on $\Gamma \vdash q_0:T$ we know $\Gamma \vdash q'_0:T_1$.  It suffices to show that there
      exists a term $\hat{q}$ such that $[q'_0/y]^{T_1} q_1 = \hat{q}$  and $\Gamma \vdash \hat{q}:T'$.  
      Since $T >_{\Gamma} T'$ we can apply the the outer induction hypothesis to obtain that there exists 
      such a term $\hat{q}$.
      
    \item[Case.] Suppose $q_0 \equiv inl(q'_0)$.  Similar to the previous case.
      
    \item[Case.] Suppose $q_0 \equiv \ccon{q'_0}{z}{q'_1}{q'_2}$.  Then \\
      $rcase_{T}\ q_0\ y\ q_1\ q_2 = \ccon{q'_0}{z}{(rcase_{T}\ q'_1\ y\ q_1\ q_2)}{(rcase_{T}\ q'_2\ y\ q_1\ q_2)}$.
      We know by assumption that $\Gamma \vdash q_0:T$, $\Gamma \vdash q_0:T$, and $\Gamma, y:T_1 \vdash q_1:T'$
      so by inversion we know the following:
      \begin{center}
        \begin{math}
          \begin{array}{lll}
            (i) & \Gamma \vdash q'_0:T'_1 + T'_2 \text{, for some types } T'_1 \text{ and } T'_2,\\
            (ii) & \Gamma, z:T'_1 \vdash q'_1:T, \text{ and }\\
            (iii) & \Gamma, z:T'_2 \vdash q'_2:T.\\
          \end{array}
        \end{math}
      \end{center}
      Now $q_0 > q'_1$ and $q_0 > q'_1$ so we can apply the induction hypothesis twice to obtain terms $\hat{q}_1$ and
      $\hat{q}_2$ such that $rcase_{T}\ q'_1\ y\ q_1\ q_2 = \hat{q}_1$, $rcase_{T}\ q'_1\ y\ q_1\ q_2 = \hat{q}_1$,
      $\Gamma, z:T'_1 \vdash \hat{q}_1:T'$ and $\Gamma, z:T'_2 \vdash \hat{q}_2:T'$. So
      $\ccon{q'_0}{z}{(rcase_{T}\ q'_1\ y\ q_1\ q_2)}{(rcase_{T}\ q'_2\ y\ q_1\ q_2)} = 
      \ccon{q'_0}{z}{\hat{q}_1}{\hat{q}_2}$.  It suffices to show that $\ccon{q'_0}{z}{\hat{q}_1}{\hat{q}_2} = \hat{q}$ 
      and $\Gamma \vdash \hat{q}:T$ for some term $\hat{q}$.  Now $q_0 > q'_0$ so we can apply the induction hypothesis
      to obtain our result, but before we can we must show that $\Gamma \vdash \ccon{q'_0}{z}{\hat{q}_1}{\hat{q}_2}:T'$.
      This is a direct consequence of applying the case-construct typing rule using i, $\Gamma, z:T'_1 \vdash \hat{q}_1:T'$
      and $\Gamma, z:T'_2 \vdash \hat{q}_2:T'$.  Therefore, by the induction hypothesis there exists a term $\hat{q}$ 
      such that $\ccon{q'_0}{z}{\hat{q}_1}{\hat{q}_2} = \hat{q}$ and $\Gamma \vdash \hat{q}:T$.
    \end{itemize}
  \end{itemize}
\end{itemize}
\end{changemargin}
\end{proof}
The next result we show is that the hereditary substitution function
cannot create new redexes, but this requires we first show that
redexes are either preserved or destroyed. Just as we have seen before
this requires the following definition.
\begin{definition}
  \label{def:rset_ssfp}
  The following function constructs the set of redexes within a
  term:\\
  \begin{changemargin}{10px}{5px}\noindent
    \begin{math}
      \begin{array}{lll}    
      rset(x) = \emptyset\\
      & \\
      rset(\lambda x:T.t) = rset(t)\\
      & \\
      rset(\Lambda X:*_l.t) = rset(t)\\
      & \\
      rset(t_1\ t_2) = rset(t_1, t_2)\\      
      \hscase{
        \text{ Where } t_1 \text{ is not a } \lambda \text{-abstraction.}
      }\\
      & \\
      rset(t_1\ t_2) = \{t_1\ t_2\} \cup rset(t'_1, t_2)\\
      \hscase{
        \text{ Where } t_1 \equiv \lambda x:T.t'_1.
      }\\
      & \\
      rset(t''[T'']) = rset(t'')\\
      \hscase{
        \text{ Where } t'' \text{ is not a type absraction.} 
      }\\
      & \\
      rset(t''[T'']) = \{t''[T'']\} \cup rset(t''') \\
      \hscase{
        \text{ Where } t'' \equiv \Lambda X:*_l.t'''.
      }\\
      & \\
      rset(inl(t)) = rset(t)\\
      & \\
      rset(inr(t)) = rset(t)\\
    \end{array}
  \end{math}
  \ \\
  \begin{math}
    \begin{array}{lll}     
      rset(\ccon{t_0}{x}{t_1}{t_2}) = rset(t_0) \cup rset(t_1,t_2)\\
      \hscase{
        \text{ Where } t_1 \text{ is not an inject-left term or an inject-right term.}
      }\\
      & \\
      rset(\ccon{t_0}{x}{t_1}{t_2}) = \{\ccon{t_0}{x}{t_1}{t_2}\} \cup rset(t_0) \cup rset(t_1,t_2)\\
      \hscase{
        \text{ Where } t_1 \text{ is an inject-left term or an inject-right term.}
      }\\
    \end{array}
    \end{math}
    \end{changemargin}
  \ \\
  The extention of $rset$ to multiple arguments is defined as follows:
  \begin{center}
    $rset(t_1, \ldots, t_n) =^{def} rset(t_1) \cup \cdots \cup rset(t_n)$.
  \end{center}
\end{definition}

\begin{lemma}[Redex Preserving]
  \label{lemma:redex_preserving_ssfp}
  \small
  If $\Gamma \vdash t : T$, $\Gamma, x:T, \Gamma' \vdash t':T'$, and
  $t'$ then $|rset(t', t)| \geq |rset([t/x]^T t')|$.
\end{lemma}
\begin{proof}
  This is a proof by induction on the lexicorgraphic combination
  $(T, t')$ of $>_{\Gamma,\Gamma'}$ and the strict subexpression
  ordering.  We case split on the structure of $t'$, and we only show
  the cases that differ from the proof of the same lemma for SSF\\
  (Lemma~\ref{lemma:redex_preserving_ssf}).
\vspace{-25px}
\begin{changemargin}{10px}{5px}\noindent
\begin{itemize}  
  
\item[Case.] Let $t' \equiv inl(t'')$. We know $rset(t', t) = rset(t'', t)$.  Since $t' > t''$ we can apply
  the induction hypothesis to obtain $|rset(t'', t)| \geq |rset([t/x]^T t'')|$.  This implies
  $|rset(t', t)| \geq |rset([t/x]^T t')|$.
  
\item[Case.] Let $t' \equiv inr(t'')$. Similar to the previous case.  
  
\item[Case.] Let $t' \equiv t'_1\ t'_2$.  First consider when $t_1'$ is not a $\lambda$-abstraction or a case construct. Then
  \begin{center}
    $rset(t'_1\ t'_2, t) = rset(t'_1, t'_2, t)$
  \end{center}  
  Clearly,  $t' > t'_i$ for $i \in \{1,2\}$, hence, by the induction hypothesis $|rset(t'_i,t)| \geq |rset([t/x]^T t'_i)|$.  
  We have three cases to consider.  That is whether or not $[t/x]^T t'_1$ is a $\lambda$-abstraction, 
  a case construct, or neither, or $ctype_T(x,t'_1)$ is undefined. However, we only show the new cases. 
  Suppose $[t/x]^T t'_1 = \ccon{t''_0}{y}{t''_1}{t''_2}$. Then 
  \begin{center}
    \small
    \begin{math}
      \begin{array}{lll}
        |rset([t/x]^T (t'_1\ t'_2))| & = & |rset(\ccon{t''_0}{y}{(app_T\ t''_1\ [t/x]^T t'_2)}{(app_T\ t''_2\ [t/x]^T t'_2)})|\\
        & = & |rset(t''_0, (app_T\ t''_1\ [t/x]^T t'_2), (app_T\ t''_2\ [t/x]^T t'_2))|.
      \end{array}
    \end{math}
  \end{center}
  We know $t' > t'_1$ and $t' > t'_2$ so by the induction hypothesis
  \begin{center}
    \begin{math}
      \begin{array}{lll}
        |rset([t/x]^T t'_1)| & =    & |rset(t''_0, t''_1,t''_2)|\\
        & \leq & |rset(t'_1,t)|
      \end{array}
    \end{math}
  \end{center}
  and
  \begin{center}
    \begin{math}
      |rset([t/x]^T t'_2)| \leq |rset(t'_2,t)|.
    \end{math}
  \end{center}
  By inversion on $\Gamma,x:T,\Gamma' \vdash t_1\ t_2:T'$ there exists a type $T''$ such that
  $\Gamma,x:T,\Gamma' \vdash t'_1:T'' \to T'$.  So by Lemma~\ref{lemma:ctype_props_ssfp},
  $ctype_T(x,t'_1) = T$, $T \equiv T'' \to T'$, and $T$ is a subexpression of $T$.  Hence,
  $T >_{\Gamma,\Gamma'} T''$ and $T >_{\Gamma,\Gamma'} T'$.  At this point we must show the following
  proposition.
  
  {\bf Proposition.}  For all $\Gamma \vdash t_1:T'' \to T'$ and $\Gamma \vdash t_2:T''$ we have \\
  $|rset(app_T\ t_1\ t_2)| \leq |rset(t_1,t_2)|$.
  
  
  We prove this by nested induction on the ordering $(T, t', t_1)$ and case split on the structure of $t_1$. 
  
  \begin{itemize}
  \item[Case.] Suppose $t_1$ is not a $\lambda$-abstraction or a case construct.  Then
    $app_T\ t_1\ t_2 = t_1\ t_2$ and $|rset(t_1\ t_2)| = |rset(t_1,t_2)|$.  Thus,\\
    $|rset(app_T\ t_1\ t_2)| \leq |rset(t_1,t_2)|$.  
    
  \item[Case.] Suppose $t_1 \equiv \lambda y:T''.t''_1$.  Then $app_T\ t_1\ t_2 = [t_2/y]^{T''} t''_1$.
    By inversion on the assumption $\Gamma \vdash t_1:T'' \to T'$ we know $\Gamma,y:T'' \vdash t''_1:T'$.  
    We know $T >_{\Gamma,\Gamma'} T''$ so by the outter induction hypothesis 
    \begin{center}
      \begin{math}
        \begin{array}{lll}
          |rset([t_2/y]^{T''} t''_1)| & \leq & |rset(t''_1,t_2)\\
          & =    & |rset(t_1,t_2).
        \end{array}
      \end{math}
    \end{center}
    Thus, $|rset(app_T\ t_1\ t_2) \leq |rset(t_1,t_2)|$.
    
  \item[Case.] Suppose $t_1 \equiv \ccon{t''_0}{y}{t''_1}{t''_2}$.  Then 
    \begin{center}
      \begin{math}
        app_T\ t_1\ t_2 = \ccon{t''_0}{y}{app_T\ t''_1\ t_2}{app_T\ t''_2\ t_2}.
      \end{math}
    \end{center}
    By inversion on the assumption $\Gamma \vdash t_1:T'' \to T'$ we know 
    $\Gamma,y:T''_1 \vdash t''_1:T'' \to T'$ and $\Gamma,y:T''_2 \vdash t''_2:T'' \to T'$.
    Since $t_1 > t''_1$ and $t_1 > t''_2$ we can apply the induction hypothesis to obtain
    \begin{center}
      \begin{math}
        |rset(app_T\ t''_1\ t_2)| \leq |rset(t''_1,t_2)|
      \end{math}
    \end{center}
    and
    \begin{center}
      \begin{math}
        |rset(app_T\ t''_2\ t_2)| \leq |rset(t''_2,t_2)|.
      \end{math}
    \end{center}
    Suppose $t''_0$ is not an inject-left or an inject-right term.  Then
    \begin{center}
      \begin{math}
        \begin{array}{lll}
          |rset(t_1,t_2)| & = & |rset(\ccon{t''_0}{y}{t''_1}{t''_2}, t_2)|\\
          & = & |rset(t''_0,t''_1,t''_2,t_2)|
        \end{array}
      \end{math}
    \end{center}
    and
    \begin{center}
      \begin{math}
        \begin{array}{lll}
          |rset(app_T\ t_1\ t_2)| & = & |rset(\ccon{t''_0}{y}{app_T\ t''_1\ t_2}{app_T\ t''_2\ t_2})|\\
          & = & |rset(t''_0) \cup rset(app_T\ t''_1\ t_2) \cup rset(app_T\ t''_2\ t_2)|\\
          & \leq & |rset(t''_0) \cup rset(t''_1, t_2) \cup rset(t''_2, t_2)|\\
          & = & |rset(t''_0) \cup rset(t''_1, t''_2, t_2)|\\
          & = & |rset(t''_0, t''_1, t''_2, t_2)|.
        \end{array}
      \end{math}
    \end{center}
    Therefore, $|rset(app_T\ t_1\ t_2)| \leq |rset(t_1,t_2)|$.
    
    Now suppose $t''_0 \equiv inl(\hat{t}_0)$.  We only show the case when $t''_0$ is an
    inject-left term, because the case when it is an inject-right term is similar.  By
    definition we know 
    \begin{center}
      \begin{math}
        \begin{array}{lll}
          |rset(t_1,t_2)| & = & |rset(\ccon{t''_0}{y}{t''_1}{t''_2}, t_2)|\\
          & = & |\{\ccon{t''_0}{y}{t''_1}{t''_2}\} \cup rset(t''_0,t''_1,t''_2,t_2)|
        \end{array}
      \end{math}
    \end{center}
    and 
    \begin{center}
      \small
      \begin{math}
        \begin{array}{lll}
          |rset(app_T\ t_1\ t_2)| \\
          \,\,\,\,\,= |rset(\ccon{t''_0}{y}{app_T\ t''_1\ t_2}{app_T\ t''_2\ t_2})|\\
%%          \,\,\,\,\,= |\{\ccon{t''_0}{y}{app_T\ t''_1\ t_2}{app_T\ t''_2\ t_2}\} \cup rset(t''_0) \cup rset(app_T\ t''_1\ t_2) \cup rset(app_T\ t''_2\ t_2)|\\
%%          \,\,\,\,\,\leq |\{\ccon{t''_0}{y}{app_T\ t''_1\ t_2}{app_T\ t''_2\ t_2}\} \cup rset(t''_0) \cup rset(t''_1, t_2) \cup rset(t''_2, t_2)|\\
          \,\,\,\,\,\leq |\{\ccon{t''_0}{y}{app_T\ t''_1\ t_2}{app_T\ t''_2\ t_2}\} \cup rset(t''_0) \cup rset(t''_1, t''_2, t_2)|\\
          \,\,\,\,\,= |\{\ccon{t''_0}{y}{app_T\ t''_1\ t_2}{app_T\ t''_2\ t_2}\} \cup rset(t''_0, t''_1, t''_2, t_2)|.
        \end{array}
      \end{math}
    \end{center}
    Therefore, $|rset(app_T\ t_1\ t_2)| \leq |rset(t_1,t_2)|$.
  \end{itemize}
  % end proposition.

  Finally, suppose $t'_1 \equiv \ccon{t''_0}{y}{t''_1}{t''_2}$.  Then
  \begin{center}
    \begin{math}
      \begin{array}{lll}
        |rset([t/x]^T (t'_1\ t'_2))| \\
        \,\,\,\,= |rset((\ccon{[t/x]^T t''_0}{y}{[t/x]^T t''_1}{[t/x]^T t''_2})\ [t/x]^T t'_2)|\\
        \,\,\,\,= |\{[t/x]^T (t'_1\ t'_2)\} \cup rset([t/x]^T t''_0,[t/x]^T t''_1,[t/x]^T t''_2,[t/x]^T t'_2)|.
      \end{array}
    \end{math}
  \end{center}
  Now $t' > t''_0$, $t' > t''_1$, $t' > t''_2$, and $t' > t'_2$ so by the induction hypothesis
  \begin{center}
    \begin{math}
      \begin{array}{lll}
        |rset([t/x]^T t''_0| \leq |rset(t''_0, t)|,\\
        |rset([t/x]^T t''_1| \leq |rset(t''_1, t)|,\\
        |rset([t/x]^T t''_2| \leq |rset(t''_2, t)|, \text{ and }\\
        |rset([t/x]^T t'_2| \leq |rset(t'_2, t)|.\\
      \end{array}
    \end{math}
  \end{center}
  Hence, 
  \begin{center}
    \begin{math}
      |rset([t/x]^T t''_0,[t/x]^T t''_1,[t/x]^T t''_2,[t/x]^T t'_2)| \leq |rset(t''_0, t''_1, t''_2, t'_2, t)|.
    \end{math}
  \end{center}
  Now
  \begin{center}
    \begin{math}
      \begin{array}{lll}
        |rset(t'_1\ t'_2,t)| & = & |rset((\ccon{t''_0}{y}{t''_1}{t''_2})\ t'_2,t)|\\
        & = & |\{t'_1\ t'_2\} \cup rset(t''_0, t''_1, t''_2, t'_2, t)|.
      \end{array}
    \end{math}
  \end{center}
  Therefore, $|rset([t/x]^T (t'_1\ t'_2))| \leq |rset(t'_1\ t'_2,t)|$.
  
\item[Case.] 
  Let $t' \equiv \ccon{t'_0}{y}{t'_1}{t'_2}$.  Suppose $t'_0$ is not an inject-left term, and inject-right term,
  or a case-construct.  First we know 
  \begin{center}
    \begin{math}
      |rset(t', t)| = |rset(t'_0, t'_1, t'_2, t)|.
    \end{math}
  \end{center}
  Now we have several cases to consider, when $[t/x]^T t'_0$ is an inject-left term,
  an inject-right term, a case construct, something else entirely, or $ctype_T(x,t'_0)$ is undefined. Suppose it is something else entirely or
  $ctype_T(x,t'_0)$ is undefined.
  Then 
  \begin{center}
    \begin{math}
      \begin{array}{lll}
        |rset([t/x]^T t')| & = & |rset(\ccon{[t/x]^T t'_0}{y}{([t/x]^T t'_1)}{([t/x]^T t'_2)})|\\
        & = & |rset([t/x]^T t'_0,([t/x]^T t'_1),([t/x]^T t'_2))|.
      \end{array}
    \end{math}
  \end{center}
  We can see that $t' > t'_0$, $t' > t'_1$, $t' > t'_2$ so by the induction hyothesis
  \begin{center}
    \begin{math}
      \begin{array}{lll}
        |rset([t/x]^T t'_0| \leq |rset(t'_0, t)|,\\
        |rset([t/x]^T t'_1| \leq |rset(t'_1, t)|, \text{ and }\\
        |rset([t/x]^T t'_2| \leq |rset(t'_2, t)|.
      \end{array}
    \end{math}
  \end{center}
  This implies that $|rset([t/x]^T t'_0,([t/x]^T t'_1),([t/x]^T t'_2))| \leq |rset(t'_0,t'_1,t'_2,t)|$.\\
  Therefore, $|rset(t', t)| \geq |rset([t/x]^T t')$.
  
  Now suppose $[t/x]^T t'_0 \equiv inl(t''_0)$.  We only show the case for when $[t/x]^T t'_0$ is an inject-left
  term, because the case for when it is an inject-right term is similar.  We can see that 
  \begin{center}
    \begin{math}
      \begin{array}{lll}
        |rset(t',t)| & = & |rset(\ccon{t'_0}{y}{t'_1}{t'_2},t)|\\
        & = & |rset(t'_0,t'_1,t'_2,t)|
      \end{array}
    \end{math}
  \end{center}
  and $|rset([t/x]^T t')| = |rset([t''_0/y]^{T_1} ([t/x]^T t'_1))|$.    By inversion on $\Gamma,x:T,\Gamma' \vdash t':T'$ we know there exists
  types $T_1$ and $T_2$ such that $\Gamma,x:T,\Gamma' \vdash t_0:T_1+T_2$.
  So by Lemma~\ref{lemma:ctype_props_ssfp} there exists a type $T$ such that $ctype_T (x,t'_0) = T$, $T \equiv T_1+T_2$, and $T$ is a 
  subexpression of $T$.  Thus, $T >_{\Gamma,\Gamma'} T_1$, $T >_{\Gamma,\Gamma'} T_2$, $t > t'_0$, and $t > t'_1$ so we can apply the induction hypothesis
  to obtain 
  \begin{center}
    \begin{math}
      \begin{array}{lll}
        |rset([t''_0/y]^{T_1} ([t/x]^T t'_1))| & \leq & |rset(t''_0,[t/x]^T t'_1)|\\
        & =    & |rset([t/x]^T t'_0,[t/x]^T t'_1)|\\
        & \leq & |rset(t'_0, t'_1, t)|\\
        & \leq & |rset(t'_0, t'_1, t'_2, t)|.\\
      \end{array}
    \end{math}
  \end{center}
     
  Next suppose $[t/x]^T t'_0 \equiv \ccon{t''_0}{y}{t''_1}{t''_2}$.  Then 
  \begin{center}
    \begin{math}
      |rset([t/x]^T t')| = |rset(rcase_{T}\ [t/x]^T t'_0\ y\ [t/x]^T t'_1\ [t/x]^T t'_2)|
    \end{math}
  \end{center}
  and
  \begin{center}
    \begin{math}
      \begin{array}{lll}
        |rset(t', t)| & = & |rset(\ccon{t'_0}{y}{t'_1}{t'_2}, t)|\\
        & = & |rset(t'_0,t_1,t'_2,t)|.
      \end{array}
    \end{math}
  \end{center}
  Note that by inverision on $\Gamma,x:T,\Gamma' \vdash t':T'$ we know there exists
  types $T_1$ and $T_2$ such that $\Gamma,x:T,\Gamma' \vdash t'_0:T_1+T_2$.
  So by Lemma~\ref{lemma:ctype_props_ssfp} there exists a type $T$ such that $ctype_T (x,t'_0) = T$, $T \equiv T_1+T_2$, and $T$ is a 
  subexpression of $T$. Thus, $T >_{\Gamma,\Gamma'} T_1$ and $T >_{\Gamma,\Gamma'} T_2$.
  It suffices to show that $|rset(rcase_T\ [t/x]^T t'_0\ [t/x]^T t'_1\ [t/x]^T t'_2)| \leq |rset([t/x]^T t'_0, [t/x]^T t'_1, [t/x]^T t'_2)|$ which
  is a consequence of the following proposition.
      
  {\bf Proposition.}  For all $\Gamma \vdash t:T_1+T_2$, $\Gamma, y:T_1 \vdash t'_1:T'$, and 
  $\Gamma, y:T_2 \vdash t'_2:T'$ we have $|rset(rcase_{T}\ t\ y\ t'_1\ t'_2)| \leq |rset(t, t'_1, t'_2)|$.
      
  We prove this proposition by nested induction on $(T, t', t)$ and we case split on $t$.
  \begin{itemize}
  \item[Case.] Suppose $t \equiv inl(t')$.  Then
    \begin{center}
      \begin{math}
        \begin{array}{lll}
          rset(rcase_{T}\ t\ y\ t'_1\ t'_2) = rset([t'/y]^{T_1} t'_1).
        \end{array}
      \end{math}
    \end{center}
    By inversion on $\Gamma \vdash t:T_1+T_2$ we know $\Gamma \vdash t':T_1$.  So by the outer induction
    hypothesis 
    \begin{center}
      \begin{math}
        \begin{array}{lll}
          |rset([t'/y]^{T_1}t'_1)| & \leq & |rset(t'_1,t')|\\
          & =    & |rset(t,t'_1)|\\
          & \leq & |rset(t,t'_1,t'_2)|.
        \end{array}
      \end{math}
    \end{center}
    Therefore, $|rset(rcase_{T}\ t\ y\ t'_1\ t'_2)| \leq |rset(t,t'_1,t'_2)|$.

  \item[Case.] Suppose $t \equiv inr(t')$.  This case is similar to the previous case.

  \item[Case.] Suppose $t \equiv \ccon{t_0}{z}{t_1}{t_2}$.  Then
    \begin{center}
      \begin{math}
        \rcase{T}{t}{y}{t'_1}{t'_2} = 
        \ccon{t_0}{z}{(\rcase{T}{t_1}{y}{t'_1}{t'_2})}{(\rcase{T}{t_2}{y}{t'_1}{t'_2})}.
      \end{math}
    \end{center}
    Now $t > t'_i$ for $i \in \{0,1,2\}$. Before we can apply the inductive hypothesis we must show that $t_1$ and $t_2$ 
    are typeable.  By inversion on the assumption $\Gamma \vdash t:T_1+T_2$ we know 
    $\Gamma,z:T'_1 \vdash t_1:T_1+T_2$ and $\Gamma,z:T'_2 \vdash t_2:T_1+T_2$.  So by the inner induction 
    hypothesis $|rset(\rcase{T}{t_i}{y}{t'_1}{t'_2})| \leq |rset(t_i,t'_1,t'_2)|$.
            
    We have two cases to consider either $t_0$ is not an inject-left term or an inject-right term, or it is.
    If not then
    \begin{center}
      \begin{math}
        rset(\rcase{T}{t}{y}{t'_1}{t'_2}) =
        rset(t_0,\rcase{T}{t_1}{y}{t'_2}{t'_2},\rcase{T}{t_2}{y}{t'_2}{t'_2}) 
      \end{math}
    \end{center}
    otherwise 
    \begin{center}
      \begin{math}
        \begin{array}{lll}
          rset(\rcase{T}{t}{y}{t'_1}{t'_2}) \\
          \text{\ \ \ \ } = \{\rcase{T}{t}{y}{t'_1}{t'_2}\} \cup
          rset(t_0,\rcase{T}{t_1}{y}{t'_1}{t'_2},\rcase{T}{t_2}{y}{t'_1}{t'_2}).
        \end{array}
      \end{math}
    \end{center}
    Suppose $t_0$ is not an inject-left or an inject-right term.  Then
    \begin{center}
      \begin{math}
        |rset(t, t'_1, t'_2)| = |rset(t_0, t_1, t_2, t'_1, t'_2)|.
      \end{math}
    \end{center}
    Now we know 
    \begin{center}
      \begin{math}
        \begin{array}{ll}
          |rset(t_0,\rcase{T}{t_1}{y}{t'_2}{t'_2},\rcase{T}{t_2}{y}{t'_2}{t'_2})|\\
          \text{\ \ \ } = 
          |rset(t_0) \cup rset(\rcase{T}{t_1}{y}{t'_2}{t'_2}) \cup rset(\rcase{T}{t_2}{y}{t'_2}{t'_2})|\\
          \text{\ \ \ } \leq  |rset(t_0) \cup rset(t_1,t'_2,t'_2) \cup rset(t_1,t'_2,t'_2)| \\
          \text{\ \ \ } =   |rset(t_0, t_1,t_2,t'_2,t'_2)|.
        \end{array}
      \end{math}
    \end{center}
    Therefore, $|rset(\rcase{T}{t}{y}{t'_1}{t'_2})| \leq |rset(t, t'_1, t'_2)|$.
        
    Now suppose $t_0 \equiv inl(t'_0)$.  Then
    \begin{center}
      \begin{math}
        |rset(t, t'_1, t'_2)| = |\{t\} \cup rset(t_0, t_1, t_2, t'_1, t'_2)|.
      \end{math}
    \end{center}
    It suffices to show that 
    \begin{center}
      \begin{math}
        \begin{array}{lll}
          |\{\rcase{T}{t}{y}{t'_1}{t'_2}\} \cup
          rset(t_0,\rcase{T}{t_1}{y}{t'_1}{t'_2},\rcase{T}{t_2}{y}{t'_1}{t'_2})|\\
          \text{\ \ \ \ } \leq |\{t\} \cup rset(t_0, t_1, t_2, t'_1, t'_2)|.
        \end{array}
      \end{math}
    \end{center}
    Let $A = \rcase{T}{t_1}{y}{t'_1}{t'_2}$ and $B = \rcase{T}{t_2}{y}{t'_1}{t'_2}$. Since \\
    $|rset(\rcase{T}{t_i}{y}{t'_1}{t'_2})| \leq |rset(t_i,t'_1,t'_2)|$ we obtain the following:
    \begin{center}
      \begin{math}
        \begin{array}{lll}
          |\{\rcase{T}{t}{y}{t'_1}{t'_2}\} \cup rset(t_0,A,B)|\\
          \text{\ \ \ \ } = |\{\rcase{T}{t}{y}{t'_1}{t'_2}\}| + |rset(t_0)| + |rset(A)| + |rset(B)|\\
          \text{\ \ \ \ } \leq |\{\rcase{T}{t}{y}{t'_1}{t'_2}\}| +
          |rset(t_0)| + |rset(t_1,t'_1,t'_2)| + |rset(t_2,t'_1,t'_2)|\\
          \text{\ \ \ \ } = |\{\rcase{T}{t}{y}{t'_1}{t'_2}\}| + |rset(t_0, t_1,t_2,t'_1,t'_2)|\\
          \text{\ \ \ \ } = |\{t\} \cup rset(t_0, t_1, t_2, t'_1, t'_2)|.\\
        \end{array}
      \end{math}
    \end{center}
    The case when $t_0$ is an inject-right term is similar to the case when it is an inject-left term.
  \end{itemize}
  Now by the previous proposition we know 
  \begin{center}
    $|rset(rcase_T\ [t/x]^T t'_0\ [t/x]^T t'_1\ [t/x]^T t'_2)| \leq |rset([t/x]^T t'_0, [t/x]^T t'_1, [t/x]^T t'_2)|$,
  \end{center}
  becuase by Lemma~\ref{lemma:total_ssfp} $t'_0$, $t'_1$, and $t'_2$ have the same types as $[t/x]^T t'_0$, $[t/x]^T t'_1$, and $[t/x]^T t'_2$.
  Now $t' > t'_0$, $t' > t'_1$, and $t' > t'_2$, so 
  \begin{center}
    \begin{math}
      \begin{array}{lll}
        |rset([t/x]^T t'_0)| \leq |rset(t'_0, t),\\
        |rset([t/x]^T t'_1)| \leq |rset(t'_1, t), \text{ and }\\
        |rset([t/x]^T t'_2)| \leq |rset(t'_2, t).
      \end{array}
    \end{math}
  \end{center}
  Thus, 
  \begin{center}
    \begin{math}
      \begin{array}{lll}
        |rset([t/x]^T t'_0, [t/x]^T t'_1, [t/x]^T t'_2)| & \leq & |rset(t'_0,t'_1,t'_2,t)|\\
        & =    & |rset(t',t)|.
      \end{array}
    \end{math}
  \end{center}
    
  Suppose $t'_0 \equiv inl(t''_0)$.  Again, we only show the case for when $t'_0$ is an inject-left term.
  We know 
  \begin{center}
    \begin{math}
      \begin{array}{lll}
        |rset([t/x]^T t')| & = & |rset(\ccon{[t/x]^T t'_0}{y}{[t/x]^T t'_1}{[t/x]^T t'_2})|\\
        & = & |rset(\ccon{inl([t/x]^T t''_0)}{y}{[t/x]^T t'_1}{[t/x]^T t'_2})|\\
        & = & |\{[t/x]^T t'\} \cup rset([t/x]^T t''_0, [t/x]^T t'_1, [t/x]^T t'_2)|
      \end{array}
    \end{math}
  \end{center}
  and
  \begin{center}
    \begin{math}
      \begin{array}{lll}
        |rset(t',t)| & = & |rset(\ccon{t'_0}{y}{t'_1}{t'_2},t)|\\
        & = & |rset(\ccon{inl(t''_0)}{y}{t'_1}{t'_2},t)|\\
        & = & |\{t'\} \cup rset(t''_0, t'_1,t'_2)|.
      \end{array}
    \end{math}
  \end{center}
  Now $t' > t''_0$, $t' > t'_1$, and $t' > t'_2$ so by the induction hypothesis
  \begin{center}
    \begin{math}
      \begin{array}{lll}
        |rset([t/x]^T t''_0)| & \leq & |rset(t''_0,t)|\\
        |rset([t/x]^T t'_1)|  & \leq & |rset(t'_1,t)|\\
        |rset([t/x]^T t'_2)|  & \leq & |rset(t'_2,t)|.
      \end{array}
    \end{math}
  \end{center}
  Therefore, $|rset([t/x]^T t''_0, [t/x]^T t'_1, [t/x]^T t'_2)| \leq |rset(t''_0, t'_1, t'_2, t)|$
  which implies that $|rset([t/x]^T t')| \leq |rset(t',t)|$.

  Finally, suppose $t'_0 \equiv \ccon{t''_0}{z}{t''_1}{t''_2}$.  Then 
  \vspace{-20px}
  \begin{center}
    \begin{math}
      \begin{array}{lll}
        |rset([t/x]^T t')| \\
        = |rset(\ccon{[t/x]^T t'_0}{y}{[t/x]^T t'_1}{[t/x]^T t'_2})|\\
        = |rset(\ccon{\ccon{[t/x]^T t''_0}{z}{[t/x]^T t''_0 t''_1}{[t/x]^T t''_0 t''_2}}{y}{[t/x]^T t'_1}{[t/x]^T t'_2})|\\
        = |\{[t/x]^T t'\} \cup rset([t/x]^T t'_0, [t/x]^T t'_1, [t/x]^T t'_2)|
      \end{array}
    \end{math}
  \end{center}
  and
  \begin{center}
    \begin{math}
      \begin{array}{lll}
        |rset(t',t)| & = & |rset(\ccon{t'_0}{y}{t'_1}{t'_2},t)|\\
        & = & |rset(\ccon{\ccon{t''_0}{z}{t''_1}{t''_2}}{y}{t'_1}{t'_2},t)|\\
        & = & |\{t'\} \cup rset(t'_0, t'_1,t'_2)|.
      \end{array}
    \end{math}
  \end{center}
  Now $t' > t''_0$, $t' > t'_1$, and $t' > t'_2$ so by the induction hypothesis
  \begin{center}
    \begin{math}
      \begin{array}{lll}
        |rset([t/x]^T t'_0)| & \leq & |rset(t'_0,t)|,\\
        |rset([t/x]^T t'_1)|  & \leq & |rset(t'_1,t)|, \text{ and }\\
        |rset([t/x]^T t'_2)|  & \leq & |rset(t'_2,t)|.
      \end{array}
    \end{math}
  \end{center}
  Therefore, $|rset([t/x]^T t''_0, [t/x]^T t'_1, [t/x]^T t'_2)| \leq |rset(t''_0, t'_1, t'_2, t)|$
  which implies that $|rset([t/x]^T t')| \leq |rset(t',t)|$.  
\end{itemize}
\end{changemargin}
\end{proof}
\noindent
We now use the previous result to show the hereditary substitution
function to be normality preserving.
\begin{lemma}[Normality Preserving]
  \label{corollary:normalization_preserving_ssfp}
  If $\Gamma \vdash n:T$ and $\Gamma, x:T' \vdash n':T'$ then there exists 
  a normal term $n''$ such that $[n/x]^T n' = n''$.
\end{lemma}
\begin{proof}
  By Lemma~\ref{lemma:total_ssfp} we know there exists a term $n''$ such that $[n/x]^T n' = t$ and by 
  Lemma~\ref{lemma:redex_preserving_ssfp} 
  $|rset(n', n)| \geq |rset([n/x]^T n')|$.  Hence, $|rset(n', n)| \geq |rset(t)|$, but
  $|rset(n', n)| = 0$.  Therefore, $|rset(t)| = 0$ which implies $n''$ has no redexes.  It suffices to show
  that $n''$ has no structural redexes.  We prove this by induction on the lexicographic ordering $(T,n')$.
  We case split on the structure of $n'$.
\vspace{-25px}
\begin{changemargin}{10px}{5px}\noindent
\begin{itemize}
\item[Case.] Suppose $n'$ is a variable $x$ or $y$ distinct from $x$.  Trivial in both cases.
  
\item[Case.] Suppose $n' \equiv \lambda y:T''.\hat{n'}$.  Then
  $[n/x]^T n' = \lambda y:T''.[t/x]^T \hat{n'}$. By inversion on the assumption  
  $\Gamma, x:T' \vdash n':T'$ we know $\Gamma, x:T',\Gamma',y:T'' \vdash \hat{n'}:T'$.  Since
  $n' > \hat{n}$ we can apply the induction hypothesis to obtain there exists a term $t'$ such that
  $[t/x]^T \hat{n} = t'$ and $t'$ has no structural redexes.  Therefore, neither does 
  $\lambda y:T''.[t/x]^T \hat{n'}$.
  
\item[Case.] Suppose $n' \equiv \Lambda X:*_l.\hat{n}$.  Similar to the previous case.
  
\item[Case.] Suppose $n' \equiv inl(n'_0)$.  Similar to the $\lambda$-abstraction case.
  
\item[Case.] Suppose $n' \equiv n'_1\ n'_2$.  By inversion we know
  $\Gamma, x:T, \Gamma' \vdash n'_1 : T'' \to T'$ and
  $\Gamma, x:T, \Gamma' \vdash n'_2 : T''$ for some types $T'$ and $T''$.
  Clearly, $n' > n'_i$ for $i \in \{1,2\}$.  Thus, by the induction hypothesis
  there exists normal terms $m_1$ and $m_2$ such that $[n/x]^T n'_i = m_i$ such that $m_i$ have no
  structural redexes.  We case split on whether or not $m_1$ is a $\lambda$-abstraction
  or a case construct and $n'_1$ is not, or $ctype_T(x,n'_1)$ is undefined.  
  We only consider the non-trivial cases when 
  $m_1 \equiv \lambda y:T''.m'_1$ or $m_1 \equiv \ccon{m'_0}{y}{m'_1}{m'_2}$ and $n'_1$
  is not a $\lambda$-abstraction or a case construct.
  Suppose the former.  
  Now by Lemma~\ref{lemma:ctype_props_ssfp} there exists a type $T$ such that 
  $ctype_T(x, n'_1) = T$, $T \equiv T'' \to T'$, and $T$ is a subexpression
  of $T$, hence $T >_{\Gamma,\Gamma'} T''$. So $[n/x]^T (n'_1\ n'_2) = [m_2/y]^{T''} m'_1$ and
  by the induction hypothesis there exists a term $m$ such that 
  $[m_2/y]^{T''} m'_1 = m$ and $m$ has no structural redexes..  
  
  Suppose $m_1 \equiv \ccon{m'_0}{y}{m'_1}{m'_2}$.
  By inversion on
  $\Gamma,\Gamma' \vdash m_1 : T'' \to T'$ we know there exists terms $T_1$ and $T_2$ such that
  $\Gamma,\Gamma' \vdash m'_0:T_1+T_2$ and
  $\Gamma,\Gamma',y:T_i \vdash m'_i:T'' \to T'$
  for $i \in \{1,2\}$.  Note that by Lemma~\ref{lemma:ctype_props_ssfp} there exists a type $T$ such that 
  $ctype_T(x, n'_1) = T$, $T \equiv T'' \to T'$, and $T$ is a subexpression
  of $T$, hence $T >_{\Gamma,\Gamma'} T'$ and $T >_{\Gamma,\Gamma'} T''$.  
  Now $[t/x]^T t' = \ccon{m'_0}{y}{app_T\ m'_1\ m_2}{app_T\ m'_2\ m_2}$.  It suffcies to show that
  there exists terms $q$ and $q'$ such that $app_T\ m'_1\ m_2 = q$, $app_T\ m'_2\ m_2 = q'$ and $q$ and $q'$ have
  no structural redexes.  To obtain this result we prove the following proposition.
  
  {\bf Proposition.}  For all normal terms $m_2$ and $m'_1$ such that  
  $\Gamma \vdash m_2 : T''$ and $\Gamma \vdash m'_1:T'' \to T'$
  there exists a term $q$ such that $app_T\ m'_1\ m_2 = q$ and $q$ has no structural redexes.
  
  We prove this by nested induction on the ordering $(T, n', m'_1)$ and case splitting on 
  the structure of $m'_1$.
  \begin{itemize}
  \item[Case.] Suppose $m'_1$ is neither a $\lambda$-abstraction or a
    case construct.  Then \\
    $app_T\  m'_1\ m_2 = m'_1\ m_2$.  Take $m'_1\ m_2$ for $q$ and we know $q$ has no structural
    redexes, because $m'_1$ and $m_2$ are normal.
    
  \item[Case.] Suppose $m'_1 \equiv \lambda z:T''.m''_1$.  Then $app_T\ m'_1\ m_2 = [m_2/z]^{T''} m''_1$.
    By inversion on the assumption $\Gamma \vdash m'_1:T'' \to T'$ we know 
    $\Gamma,z:T'' \vdash m''_1:T'$.  Since $T >_{\Gamma} T''$ we can apply the outter induction
    hypothesis to obtain there there exists a $q$ such that $[m_2/z]^{T''} m''_1 = q$ and 
    $q$ has no structural redexes.
    
  \item[Case.] Suppose $m'_1 \equiv \ccon{m''_0}{z}{m''_1}{m''_2}$.  Then\\
    $app_T\ m'_1\ m_2 = \ccon{m''_0}{z}{app_T\ m''_1\ m_2}{app_T\ m''_2\ m_2}$.  By inversion on the assumption
    $\Gamma \vdash m'_1:T''\to T'$ we know there exists types $T_1$ and $T_2$ such that
    $\Gamma \vdash m''_0:T_1+T_2$, $\Gamma,z:T_1 \vdash m''_1:T''\to T'$
    and $\Gamma,z:T_2 \vdash m''_2:T''\to T'$.  Since $m'_1 > m''_1$ and $m'_1 > m''_2$ we can 
    apply the inner induction hypothesis to obtain there exists terms $q'$ and $q''$ such that 
    $app_T\ m''_1\ m_2 = q'$, $q'$ has no structural redexes, $app_T\ m''_1\ m_2 = q''$ and $q''$ has no structural redexes.  
    Hence, 
    $app_T\ m'_1\ m_2 = \ccon{m''_0}{z}{app_T\ m''_1\ m_2}{app_T\ m''_2\ m_2} = \ccon{m''_0}{z}{q'}{q''}$ and
    $\ccon{m''_0}{z}{q'}{q''}$ has no structural redexes.  Note that $m''_0$ is normal, because $m'_1$ is
    normal.  
  \end{itemize}

  By the previous proposition there exists terms $q$ and $q'$ such that \\
  $[n/x]^T n' = \ccon{m'_0}{y}{app_T\ m'_1\ m_2}{app_T\ m'_2\ m_2}
  = \ccon{m'_0}{y}{q}{q'}$, where $app_T\ m'_1\ m_2 = q$, $app_T\ m'_1\ m_2 = q'$, and
  $q$ and $q'$ have no structural redexes.  Thus, $\ccon{m'_0}{y}{q}{q'}$ has no
  structural redexes.  
  
\item[Case.] Suppose $n' \equiv \ccon{m_0}{y}{m_1}{m_2}$. By inversion on the assumption
  $\Gamma,x:T,\Gamma' \vdash n':T'$ we know the following:
  \begin{center}
    \begin{math}
      \begin{array}{lll}
        \Gamma,x:T,\Gamma' \vdash m_0:T_1+T_2 \text{, for some types } T_1 \text{ and } T_2,\\
        \Gamma,x:T,\Gamma',y:T_1 \vdash m_1:T, and\\
        \Gamma,x:T,\Gamma',y:T_2 \vdash m_2:T.\\
      \end{array}
    \end{math}
  \end{center}
  It is easy to see that
  $n' > m_i$ for all $i \in \{0,1,2\}$.  Hence, by the induction hypothesis
  there exists terms $m'_0$, $m'_1$, and $m'_2$ such that $[t/x]^T m_i = m'_i$ and $m'_i$ have no structural redexes
  for all $i \in \{0,1,2\}$.  We have two cases to consider.
  \begin{itemize}
  \item[Case.] Suppose $m'_0$ is not an inject-left term, inject-right term, or case construct, or
    $m_0$ is an inject-left term, an inject-right term, or a case construct, or $ctype_T(x,m_0)$ is undefined.
    Then \\
    $[n/x]^{T} (\ccon{m_0}{y}{m_1}{m_2}) = \ccon{m'_0}{y}{m'_1}{m'_2}$ which has no structural redexes.
    
  \item[Case.] Suppose $m'_0$ is an inject-left term, inject-right term, or case construct and $m_0$ is not
    an inject-left term, an inject-right term, or a case construct.  Then
    $[n/x]^{T} (\ccon{m_0}{y}{m_1}{m_2}) = rcase_{T}\ m'_0\ y\ m'_1\ m'_2$, where by \\
    Lemma~\ref{lemma:ctype_props_ssfp}
    there exists a type $T$ such that $ctype_T(x,m_0) = T$, $T \equiv T_1+T_2$, and $T$ is a subexpression
    of $T$, hence $T >_{\Gamma,\Gamma'} T_1$ and $T >_{\Gamma,\Gamma'} T_2$.
    Consider the case when $m'_0 \equiv inl(m''_0)$.  Then we know by the definition of $rcase$ that
    $rcase_{T}\ m'_0\ y\ m'_1\ m'_2 = [m''_0/y]^{T_1} m'_1$.  Clearly, $T >_{\Gamma,\Gamma'} T_1$ hence
    by the the induction hypothesis there exists a term $r$ such that $[m''_0/y]^{T_1} m'_1 = r$ and
    $r$ has no structural redexes. Similarly for when $m'_0 \equiv inr(m''_0)$.  
    So suppose $m'_0 \equiv \ccon{m''_0}{z}{m''_1}{m''_2}$ then it suffices to show that there exists some term
    $q$ such that $rcase_{T}\ m'_0\ y\ m'_1\ m'_2 = q$ and $q$ has no structural redexes.  We obtain this
    result by the following proposition.
         
    {\bf Proposition.} For all normal terms $q$ and $q_1$ such that 
    $\Gamma \vdash q_0:T$, $\Gamma, y:T_1 \vdash q_1:T'$, 
    and $\Gamma, y:T_2 \vdash q_2:T'$ there exists a term $\hat{q}$ such that 
    $rcase_{T}\ q_0\ y\ q_1\ q_2 = \hat{q}$ and $\hat{q}$ has no structural redexes.
    We prove this by induction on the the ordering $(T, n', q_0)$ and case split on the structure of $q_0$.
    \begin{itemize}
    \item[Case.] Suppose $q_0$ is not an inject-left term, inject-right term, or a case construct.  Then\\
      $rcase_{T}\ q_0\ y\ q_1\ q_2 = \ccon{q_0}{y}{q_1}{q_2}$ which has no structural redexes.
      
    \item[Case.] Suppose $q_0 \equiv inl(q'_0)$.  Then $rcase_{T}\ q_0\ y\ q_1\ q_2 = [q'_0/y]^{T_1} q_1$ and
      by inversion on $\Gamma \vdash q_0:T$ we know $\Gamma \vdash q'_0:T_1$.  It suffices to show that there
      exists a term $\hat{q}$ such that $[q'_0/y]^{T_1} q_1 = \hat{q}$  and $\hat{q}$ has no structural redexes.
      Clearly, $T >_{\Gamma} T'$ so by the outer induction hypothesis there exists such a term $\hat{q}$.
      
    \item[Case.] Suppose $q_0 \equiv inl(q'_0)$.  Similar to the previous case.
      
    \item[Case.] Suppose $q_0 \equiv \ccon{q'_0}{z}{q'_1}{q'_2}$.  Then \\
      $rcase_{T}\ q_0\ y\ q_1\ q_2 = \ccon{q'_0}{z}{(rcase_{T}\ q'_1\ y\ q_1\ q_2)}{(rcase_{T}\ q'_2\ y\ q_1\ q_2)}$.
      We know by assumption that $\Gamma \vdash q_0:T$, $\Gamma \vdash q_0:T$, and $\Gamma, y:T_1 \vdash q_1:T'$
      so by inversion we know the following:
      \begin{center}
        \begin{math}
          \begin{array}{lll}
            (i) & \Gamma \vdash q'_0:T'_1 + T'_2 \text{, for some types } T'_1 \text{ and } T'_2,\\
            (ii) & \Gamma, z:T'_1 \vdash q'_1:T, \text{ and }\\
            (iii) & \Gamma, z:T'_2 \vdash q'_2:T.\\
          \end{array}
        \end{math}
      \end{center}
      Now $q_0 > q'_1$ and $q_0 > q'_1$ so we can apply the inner induction hypothesis twice to obtain terms $\hat{q}_1$ and
      $\hat{q}_2$ such that $rcase_{T}\ q'_1\ y\ q_1\ q_2 = \hat{q}_1$, $rcase_{T}\ q'_1\ y\ q_1\ q_2 = \hat{q}_1$ where 
      $\hat{q}_1$ and $\hat{q}_2$ have no structural redexes. So
      $\ccon{q'_0}{z}{(rcase_{T}\ q'_1\ y\ q_1\ q_2)}{(rcase_{T}\ q'_2\ y\ q_1\ q_2)} = 
      \ccon{q'_0}{z}{\hat{q}_1}{\hat{q}_2}$.  It suffices to show that $\ccon{q'_0}{z}{\hat{q}_1}{\hat{q}_2} = \hat{q}$ 
      for some normal term $\hat{q}$.  Now $q_0 > q'_0$ so we can apply the induction hypothesis
      to obtain our result, but before we can we must show that $\Gamma \vdash \ccon{q'_0}{z}{\hat{q}_1}{\hat{q}_2}:T'$.
      This is a direct consequence of applying the case-construct typing rule using i, $\Gamma, z:T'_1 \vdash \hat{q}_1:T'$
      and $\Gamma, z:T'_2 \vdash \hat{q}_2:T'$.  Therefore, by the inner induction hypothesis there exists a term $\hat{q}$ 
      such that $\ccon{q'_0}{z}{\hat{q}_1}{\hat{q}_2} = \hat{q}$ and $\hat{q}$ is has no structural redexes.
    \end{itemize}
  \end{itemize}
\end{itemize}
\end{changemargin}
\end{proof}
\noindent
Finally, we show that the hereditary substitution function is
consistent with respect to reduction.
\begin{lemma}[Soundness with Respect to Reduction]
  \label{lemma:soundness_reduction_ssfp}
  If $\Gamma \vdash t : T$ and $\Gamma, x:T, \Gamma' \vdash t':T'$ then
  $[t/x]t' \redto^* [t/x]^T t'$.
\end{lemma}
\begin{proof}
  This is a proof by induction on the lexicorgraphic combination
  $(T, t')$ of $>_\Gamma$ and the strict subexpression ordering.
  We case split on the structure of $t'$.  When applying the induction
  hypothesis we must show that the input terms to the substitution and
  the hereditary substitution functions are typeable.  We do not
  explicitly state typing results that are simple conseqences of
  inversion.  Furthermore, we only give the cases that differ from the
  proof of the same result for SSF
  (Lemma~\ref{lemma:soundness_reduction_ssf}).
  \vspace{-25px}
  \begin{changemargin}{10px}{5px}\noindent
  \begin{itemize}
  \item[Case.] Suppose $t' \equiv inl(t'_0)$.  Then $[t/x]^T t' = inl([t/x]^T t'_0)$.  We can
    see that $t' > t'_0$ so by the induction hypothesis $[t/x]t'_0 \redto^* [t/x]^T t'_0$.  Hence,
    $inl([t/x]t'_0) \redto^* inl([t/x]^T t'_0)$ which implies that 
    $[t/x](inl(t'_0)) \redto^* [t/x]^T (inl(t'_0))$.
    
  \item[Case.] Suppose $t' \equiv inr(t'_0)$.  Similar to the previous case.
    
  \item[Case.] Suppose $t' \equiv \ccon{t'_0}{y}{t'_1}{t'_2}$.  Clearly, $t' > t'_0$,
    $t' > t'_1$, and $t' > t'_2$, so we can apply the induction hypothesis to conclude 
    $[t/x]t'_0 \redto^* [t/x]^T t'_0$, $[t/x]t'_1 \join [t/x]^T t'_1$, and 
    $[t/x]t'_2 \redto^* [t/x]^T t'_2$.  We have several cases to consider, either when $[t/x]^T t'_0$ is an
    inject-left term or an inject-right term and $t'_0$ is not, when $[t/x]^T t'_0$ is a case construct
    and $t'_0$ is not, or $[t/x]^T t'_0$ is not an inject-left term, an inject-right term, or a case construct, or
    $ctype_T(x,t'_0)$ is undefined.  The cases when $[t/x]^T t'_0$ is not an inject-left term, an inject-right term, 
    or a case construct, or $ctype_T(x,t'_0)$ is undefined are trivial.
    
    Let's consider the case when $[t/x]^T t'_0$ is an inject-left term or an inject-right term and 
    $t'_0$ is not.  Since the case when $[t/x]^T t'_0$ is an inject-left term is similar to the case when
    it is an inject-right term we only consider the former.  Suppose $[t/x]^T t'_0 = inl(t''_0)$ and 
    $t'_0$ is not an inject-left term.  By Lemma~\ref{lemma:ctype_props_ssfp} there exists a type
    $T$ such that $ctype_T(x,t'_0) = T$, $T \equiv T_1+T_2$, and $T$ is a subexpression
    of $T$, where by inversion on $\Gamma,x:T,\Gamma' \vdash t':T'$ there exists types $T_1$ and
    $T_2$ such that $\Gamma,x:T,\Gamma' \vdash t'_0:T_1+T_2$.  Thus, $T >_{\Gamma,\Gamma'} T_1$
    and $T >_{\Gamma,\Gamma'} T_2$.  So $[t/x]^T t' = [t''_0/y]^{T_1} ([t/x]^{T} t'_1)$ and we know from above
    that $[t/x]t'_1 \redto^* [t/x]^{T} t'_1$.  Now $T >_{\Gamma,\Gamma'} T_1$, so by the induction
    hypothesis, $[t''_0/y]([t/x]^{T} t'_1) \redto^* [t''_0/y]^{T_1}([t/x]^T t'_1)$.  Thus,
    $[t''_0/y]([t/x] t'_1) \redto^* [t''_0/y]^{T_1}([t/x]^T t'_1)$.  It suffices to show 
    $ [t/x]t' \redto^* [t''_0/y]([t/x] t'_1)$.  We can see that 
    \begin{center}
      \begin{math}
        \begin{array}{lll}
          [t/x]t' & = & [t/x](\ccon{x}{y}{t'_1}{t'_2})\\
          & \equiv & \ccon{[t/x]x}{y}{[t/x]t'_1}{[t/x]t'_2}\\
          & \equiv & \ccon{inl(t''_0)}{y}{[t/x]t'_1}{[t/x]t'_2}\\
          & \redto & [t''_0/y]([t/x]t'_1).
        \end{array}
      \end{math}
    \end{center}

    Suppose $[t/x]t'_0 = \ccon{t''_0}{z}{t''_1}{t''_2}$ and $t'_0$ is not.  It suffices to show that 
    $[t/x]t \redto^* [t/x]^T t'$, which is equivalent to showing 
    $[t/x](\ccon{t'_0}{y}{t'_1}{t'_2}) \redto^* [t/x]^T (\ccon{t'_0}{y}{t'_1}{t'_2})$.  Now
    \begin{center}
      \begin{math}
        \begin{array}{lll}
          [t/x]^T (\ccon{t'_0}{y}{t'_1}{t'_2})\\
          \,\,\,\,= \ccon{t''_0}{z}{(\rcase{T}{t''_1}{y}{t'_1}{t'_2})}{(\rcase{T}{t''_1}{y}{t'_1}{t'_2})}
        \end{array}
      \end{math}
    \end{center}
    and
    \begin{center}
      \begin{math}
        \begin{array}{lll}
          [t/x] (\ccon{t'_0}{y}{t'_1}{t'_2})\\
          \,\,\,\,= \ccon{[t/x]t'_0}{y}{[t/x]t'_1}{[t/x]t'_2} \\
          \,\,\,\,\redto^* \ccon{(\ccon{t''_0}{z}{t''_1}{t''_2})}{y}{[t/x]t'_1}{[t/x]t'_2} \\
          \,\,\,\,\redto \ccon{t''_0}{z}{(\ccon{t''_1}{y}{t'_1}{t'_2})}{(\ccon{t''_2}{y}{t'_1}{t'_2})},
        \end{array}
      \end{math}
    \end{center}
    because we know from above that $[t/x]t'_0 \redto^* [t/x]^T t'_0$.  So it suffices to show that
    $(\ccon{t''_1}{y}{t'_1}{t'_2}) \redto^* (\rcase{T}{t''_1}{y}{t'_1}{t'_2})$ and
    $(\ccon{t''_2}{y}{t'_1}{t'_2}) \redto^* (\rcase{T}{t''_2}{y}{t'_1}{t'_2})$, because we know from above that 
    $[t/x]t_i \redto^* [t/x]^T t'_i$.  This is a consequence of the following proposition.  First note that 
    again by Lemma~\ref{lemma:ctype_props_ssfp} there exists a type $T$ such that $ctype_T(x,t'_0) = T$,
    $T \equiv T_1+T_2$, and $T$ is a subexpression of $T$, where by inversion on the assumption
    $\Gamma,x:T,\Gamma' \vdash t':T'$ there exists types $T_1$ and $T_2$ such that 
    $\Gamma,x:T,\Gamma' \vdash t'_0:T_1+T_2$.  Hence, $T >_{\Gamma,\Gamma'} T_1$ and
    $T >_{\Gamma,\Gamma'} T_2$.  
    
    {\bf Proposition.} For all $\Gamma \vdash t_0:T_1+T_2$, $\Gamma,y:T_1 \vdash t_1:T''$ and
    $\Gamma,y:T_2 \vdash t_2:T''$ we have 
    $(\ccon{t_0}{y}{t_1}{t_2}) \redto^* (\rcase{T}{t_0}{y}{t_1}{t_2})$.

    We prove this by nested induction on the ordering $(T,t',t_0)$ and case splitting on
    the structure of $t_0$.  
    \begin{itemize}
    \item[Case.] Suppose $t_0$ is not an inject-left term, an inject-right term, or a case construct.  Then
      \begin{center}
        $\rcase{T}{t_0}{y}{t_1}{t_2} = \ccon{t_0}{y}{t_1}{t_2}$.
      \end{center}
      
    \item[Case.] Suppose $t_0 \equiv inl(t'_0)$.  Then 
      \begin{center}
        $\rcase{T}{t_0}{y}{t_1}{t_2} = [t'_0/y]^{T_1} t_1$
      \end{center}
      and
      \begin{center}
        \begin{math}
          \begin{array}{lll}
            \ccon{t_0}{y}{t_1}{t_2} & \equiv & \ccon{inl(t'_0)}{y}{t_1}{t_2}\\
            & \redto & [t'_0/y]t_1.
          \end{array}
        \end{math}
      \end{center}
      Now $T >_{\Gamma} T_1$ so by the outer-induction hypothesis 
      $[t'_0/y]t_1 \redto^* [t'_0/y]^{T_1} t_1$.  Therefore, 
      $(\ccon{t_0}{y}{t_1}{t_2}) \redto^* (\rcase{T}{t_0}{y}{t_1}{t_2})$.
      
    \item[Case.] Suppose $t_0 \equiv inl(t'_0)$.  Similar to the previous case.
      
    \item[Case.] Suppose $t_0 \equiv \ccon{t'_0}{z}{t'_1}{t'_2}$.  Then 
      \begin{center}
        \begin{math}
          \rcase{T}{t_0}{y}{t_1}{t_2} = 
          \ccon{t'_0}{z}{(\rcase{T}{t'_1}{y}{t_1}{t_2})}
          {(\rcase{T}{t'_2}{y}{t_1}{t_2})}
        \end{math}
      \end{center}
      and
      \begin{center}
        \begin{math}
          \begin{array}{lll}
            \ccon{t_0}{y}{t_1}{t_2} \\
            \,\,\,\,\equiv  \ccon{(\ccon{t'_0}{z}{t'_1}{t'_2})}{y}{t_1}{t_2}\\
            \,\,\,\,\redto \ccon{t'_0}{z}{(\ccon{t'_1}{y}{t_1}{t_2})}{(\ccon{t'_2}{y}{t_1}{t_2})}.
          \end{array}
        \end{math}
      \end{center}
      Trivially, $t_0 > t'_1$ and $t_0 > t'_2$ so by the inner-induction hypothesis \\
      \[ (\ccon{t'_1}{y}{t_1}{t_2}) \redto^* (\rcase{T}{t'_1}{y}{t_1}{t_2})\] and 
      \[(\ccon{t'_2}{y}{t_1}{t_2}) \redto^* (\rcase{T}{t'_2}{y}{t_1}{t_2}).\]  Therefore,
      $(\ccon{t_0}{y}{t_1}{t_2}) \redto^* (\rcase{T}{t_0}{y}{t_1}{t_2})$.
    \end{itemize}
    
  \item[Case.] Suppose $t' \equiv t'_1\ t'_2$.  By Lemma~\ref{lemma:total_ssfp}
    there exists terms $\hat{t}'_1$ and $\hat{t}'_2$
    such that $[t/x]^T t'_1 = \hat{t}'_1$ and $[t/x]^T t'_2 = \hat{t}'_2$.  Since
    $t' > t'_1$ and $t' > t'_2$ we can apply the induction hypothesis to obtain
    $[t/x]t'_1 \redto^* \hat{t}'_1$ and $[t/x]t'_2 \redto^* \hat{t}'_2$.  Now we case
    split on whether or not $\hat{t}'_1$ is a $\lambda$-abstraction and $t'_1$ is not, $\hat{t}'_1$ is a case construct and
    $t'_1$ is not, $ctype_T(x,t'_1)$ is undefined, or $\hat{t}'_1$ is neither a $\lambda$-abstraction or a case construct.  
    We only show the case where $\hat{t}'_1$ is a case construct.

    Finally, suppose $\hat{t}'_1 \equiv \ccon{t_0}{y}{t_1}{t_2}$ and $t'_0$ is not a case construct.  By Lemma~\ref{lemma:ctype_props_ssfp}
    there exists a type $T$ such that $ctype_T(x,t'_1) = T$, $T \equiv T'' \to T'$ and $T$ is a subexpression
    of $T$, where by inversion on the assumption $\Gamma,x:T,\Gamma' \vdash t':T'$ there exists a type $T''$ such that
    $\Gamma,x:T,\Gamma' \vdash t'_1:T'' \to T'$.  Now 
    \begin{center}
      \begin{math}
        [t/x]^T (t'_1\ t'_2) = \ccon{t_0}{y}{(app_T\ t_1\ ([t/x]^T t'_2))}{(app_T\ t_1\ ([t/x]^T t'_2))}
      \end{math}
    \end{center}
    and
    \begin{center}
      \begin{math}
        [t/x](t'_1\ t'_2) = ([t/x]t'_1)([t/x]t'_2).
      \end{math}
    \end{center}
    Clearly, $t' > t'_1$ and $t' > t'_2$, so by the induction hypothesis $[t/x]t'_1 \redto^* [t/x]^T t'_1$ and $[t/x]t'_2 \redto^* [t/x]^T t'_2$.  Thus,
    \begin{center}
      \begin{math}
        \begin{array}{lll}
          ([t/x]t'_1)\ ([t/x]t'_2) & \redto^* & (\ccon{t_0}{y}{t_1}{t_2})\ ([t/x]t'_2)\\
          & \redto & \ccon{t_0}{y}{(app_T\ t_1 ([t/x]t'_2))}{(app_T\ t_2 ([t/x]t'_2))}
        \end{array}
      \end{math}
    \end{center}
    and
    \begin{center}
      \begin{math}
        ((\ccon{t_0}{y}{(app_T\ t_1\ ([t/x] t'_2))}{(app_T\ t_1\ ([t/x] t'_2))})) \redto^* (\ccon{t_0}{y}{(app_T\ t_1\ ([t/x]^T t'_2))}{(app_T\ t_1\ ([t/x]^T t'_2))}).
      \end{math}
    \end{center}
    It suffices to show that $(t_1\ ([t/x] t'_2)) \redto^* (app_T\ t_1\ ([t/x] t'_2))$ and 
    $(t_2\ ([t/x] t'_2)) \redto^* (app_T\ t_2\ ([t/x] t'_2))$.  This is a consequence of the following proposition:

    {\bf Proposition.} For all $\Gamma \vdash t_1:T_1 \to T_2$ and $\Gamma \vdash t_2:T_1$ we have
    $(t_1\ t_2) \redto^* (app_T\ t_1\ t_2)$.

    We prove this by nested induction on the ordering $(T, t', t_1)$ and case split on the
    structure of $t_1$.  
    \begin{itemize}
    \item[Case.] Suppose $t_1$ is not a $\lambda$-abstraction or a case construct.  Then 
      $app_T\ t_1\ t_2 = t_1\ t_2$.
      
    \item[Case.] Suppose $t_1 \equiv \lambda y:T_1.t''_1$.  Then $app_T\ t_1\  t_2 = [t_2/y]^{T_1} t''_1$.
      Clearly, $T >_\Gamma T_1$ so by the outer-induction hypothesis 
      $[t_2/y]t''_1 \redto^* [t_2/y]^{T_1} t''_1$.  Therefore, $(t_1\ t_2) \redto^* (app_T\ t_1\ t_2)$.
      
    \item[Case.] Suppose $t_1 \equiv \ccon{t'_0}{y}{t'_1}{t'_2}$.  Then 
      \begin{center}
        \begin{math}
          app_T\ t_1\ t_2 = \ccon{t'_0}{y}{(app_T\ t'_1\ t_2)}{(app_T\ t'_2\ t_2)}
        \end{math}
      \end{center}
      and
      \begin{center}
        \begin{math}
          \begin{array}{lll}
            (t_1\ t_2) & =      & (\ccon{t'_0}{y}{t'_1}{t'_2})\ t_2\\
            & \redto & \ccon{t'_0}{y}{(t'_1\ t_2)}{(t'_2\ t_2)}.
          \end{array}
        \end{math}
      \end{center}
      We can see that $t_1 > t'_1$ and $t_1 > t'_2$ so by the inner-induction hypothesis, $(t'_1\ t_2) \redto^* (app_T\ t'_1\ t_2)$ and
      $(t'_2\ t_2) \redto^* (app_T\ t'_2\ t_2)$.  Therefore, 
      \begin{center}
        \begin{math}
          (\ccon{t'_0}{y}{(t'_1\ t_2)}{(t'_2\ t_2)}) \redto^* (\ccon{t'_0}{y}{(app_T\ t'_1\ t_2)}{(app_T\ t'_2\ t_2)}),
        \end{math}
      \end{center}
      which implies $(t_1\ t_2) \redto^* (app_T\ t_1\ t_2)$.
    \end{itemize} 
  \end{itemize}
  \end{changemargin}
\end{proof}
% subsection properties_of_the_hereditary_substitution_function_ssfp (end)

\subsection{Concluding Normalization}
\label{subsec:concluding_normalization_ssfp}
Similarly to SSF, the definition of the interpretation of types is
identical to the definition for STLC
(Definition~\ref{def:interpretation_of_types_stlc}), so we do not
repeat it here.  We define $t \normto t'$ to be $t \redto^{*} t'$ and
$t'$ is normal.  Before moving on to proving soundness of typing and
concluding normalization we need a basic result about the
interpretation of types: type substitution.  It is used in the proof
of the type soundness theorem (Theorem~\ref{thm:soundness_ssfp}).

\begin{lemma}[Type Substitution for the Interpretation of Types]
  \ \\
  If $n \in \interp{T'}_{\Gamma,X:*_l,\Gamma'}$ and 
  $\Gamma \vdash T:*_l$ then 
  $[T/X]n \in \interp{[T/X]T'}_{\Gamma,[T/X]\Gamma'}$.
  \label{lemma:type_sub_ssfp}
\end{lemma}
\begin{proof}
  This proof is similar to the proof of the same lemma for SSF (Lemma~\ref{lemma:type_sub_ssf}).
\end{proof}
\noindent
Next we now show substitution for the interpretation of types.
\begin{lemma}[Hereditary Substitution for the Interpretation of Types]
  If $n' \in \interp{T'}_{\Gamma,x:T,\Gamma'}$, $n \in \interp{T}_\Gamma$, then 
  $[n/x]^T n' \in \interp{T'}_{\Gamma,\Gamma'}$.
  \label{lemma:interpretation_of_types_closed_substitution_ssfp}
\end{lemma}
\begin{proof}
  By Lemma~\ref{lemma:total_ssfp} we know there exists a term $\hat{n}$ 
  such that $[n/x]^T n' = \hat{n}$ and $\Gamma,\Gamma' \vdash \hat{n}:T'$ and by 
  Lemma~\ref{corollary:normalization_preserving_ssfp} $\hat{n}$ is normal.  Therefore,
  $[n/x]^T n' = \hat{n} \in \interp{T'}_{\Gamma,\Gamma'}$.
\end{proof}
\noindent
Finally, we are ready to present our main result, which implies
normalization of SS$\Fp$.

\begin{thm}[Type Soundness]
  If $\Gamma \vdash t:T$ then $t \in \interp{T}_\Gamma$.
  \label{thm:soundness_ssfp}
\end{thm}
\begin{proof}
  This is a proof by induction on the assumed typing derivation.  We
  only show the cases that differ from the proof of type soundness for
  SSF (Theorem~\ref{thm:soundness_ssf}).
\vspace{-25px}
\begin{changemargin}{10px}{5px}\noindent
\begin{itemize}  
\item[Case.]\ \\
  \begin{center}
    \begin{math}
      $$\mprset{flushleft}
      \inferrule* [right=] {
        \Gamma \vdash t:T_1
	\\
	\Gamma \vdash T_2:*_p
      }{\Gamma \vdash inl(t): T_1+T_2}
    \end{math}
  \end{center}
  By the induction hypothesis, $t \in \interp{T_1}_\Gamma$ and by the definition of the 
  interpretation of types,
  $t \normto n \in \interp{T_1}_\Gamma$, and $inl(n) \in \interp{T_1 + T_2}_{\Gamma}$.  
  Again, by the definition of the interpretation of types
  $inl(t) \normto inl(n) \in \interp{T_1 + T_2}_\Gamma$.
  
\item[Case.]\ \\
  \begin{center}
    \begin{math}
      $$\mprset{flushleft}
      \inferrule* [right=] {
        \Gamma \vdash t:T_2
	\\
	\Gamma \vdash T_1:*_p
      }{\Gamma \vdash inr(t): T_1+T_2}
    \end{math}
  \end{center}
  Similar the inject-left case above.
  
\item[Case.]\ \\
  \begin{center}
    \begin{math}
      $$\mprset{flushleft}
      \inferrule* [right=] {
        \Gamma \vdash t_0:T_1 + T_2
        \\
	\Gamma,x:T_1 \vdash t_1:T
        \\
	\Gamma,x:T_2 \vdash t_2:T
      }{\Gamma \vdash \ccon{t_0}{x}{t_1}{t_2}: T}
    \end{math}
  \end{center}
  By the induction hypothesis and the definition of the interpretation of types
  $t_0 \normto n_0 \in \interp{T_1 + T_2}_\Gamma$ and $\Gamma \vdash n_0:T_1+T_2$, 
  $t_1 \normto n_1 \in \interp{T}_{\Gamma,x:T_1}$ and $\Gamma,x:T_1 \vdash n_1:T$, and
  $t_2 \normto n_2 \in \interp{T}_{\Gamma,x:T_2}$ and  $\Gamma,x:T_2 \vdash n_2:T$.  
  Clearly, 
  \begin{center}
    \begin{math}
      \begin{array}{lll}
        \ccon{t_0}{x}{t_1}{t_2} & \redto^* & \ccon{n_0}{x}{n_1}{n_2}\\
        & =        & [n_0/z](\ccon{z}{x}{n_1}{n_2}),
      \end{array}
    \end{math}
  \end{center}
  for some variable $z \not \in FV(n_0,n_1,n_2) \cup \{x\}$.  Lemma~\ref{lemma:total_ssfp}, 
  Lemma~\ref{lemma:soundness_reduction_ssfp}, and Lemma~\ref{corollary:normalization_preserving_ssfp} 
  allow us to conclude that $[n_0/z](\ccon{z}{x}{n_1}{n_2}) \redto^* [n_0/z]^{T_1+T_2}(\ccon{z}{x}{n_1}{n_2})$,
  $\Gamma \vdash [n_0/z]^{T_1+T_2}(\ccon{z}{x}{n_1}{n_2}) :T$, and $[n_0/z]^{T_1+T_2}(\ccon{z}{x}{n_1}{n_2})$
  is normal.  Thus, \\ $[n_0/z]^{T_1+T_2}(\ccon{z}{x}{n_1}{n_2}) \in \interp{T}_\Gamma$ and we obtain 
  $\ccon{t_0}{x}{t_1}{t_2} \in \interp{T}_\Gamma$. 
\end{itemize}
\end{changemargin}
\end{proof}

\begin{corollary}[Normalization]
  If $\Gamma \vdash t:T$, then there exists a normal form $n$, such
  that $t \normto n$.
\end{corollary}
% subsubsection concluding_normalization_ssfp (end)
% section stratified_system_f_with_sum_types (end)

%%% Local Variables: 
%%% mode: latex
%%% reftex-default-bibliography: ("/Users/hde/thesis/paper/thesis.bib")
%%% TeX-master: "/Users/hde/thesis/paper/thesis.tex"
%%% End: 