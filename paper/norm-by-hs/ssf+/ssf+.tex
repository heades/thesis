\newcommand{\ccon}[4]{\mbox{case}\ #1\ \text{of}\ #2\text{.}#3\text{,}#2\text{.}#4}
\newcommand{\rcase}[5]{\mbox{rcase}_{#1}\ #2\ #3\ #4\ #5}
\newcommand{\F}[0]{\mbox{F}^{=}}
\newcommand{\Fp}[0]{\mbox{F}^{+}}
\newcommand{\STLCeq}[0]{\textnormal{STLC}^=}
\newcommand{\A}[0]{\mbox{\textbf{A}}}
\newcommand{\bredto}[0]{\rightsquigarrow_\beta}
\newcommand{\hlmn}[2]{
  \marginpar{
    \small
    {\color{blue}{
        \tiny #2
    }}}
    {\color{red}{
        #1
    }}
}

\section{Stratified System $\Fp$ (SS$\Fp$)}
\label{sec:stratified_system_f_with_sum_types}
Stratified System $\Fp$ is a predicative polymorphic type theory.
Stratified polymorphism is used in predicative type theories for
universe hierarchies.  SS$\Fp$ also has sum types $\phi_1 + \phi_2$,
whose elimination form
$\textit{case}\ t\ \textit{of}\ x_1.t_1,x_2.t_2$ is used to case split
on a whether or not term $t$ with a sum type is truly $x_1$ of type
$\phi_1$, or else $x_2$ of type $\phi_2$.  We consider sum types with
so-called commuting conversions, which allow independent cases to be
permuted past each other (see Fig~\ref{fig:syntax_ssfp}
below).  Commuting conversions are well-known to pose technical
difficulties for normalization proofs based on reducibility
(see~\cite{tatsuta+05} and Chapter 10 of~\cite{Girard:1989}).  We will
see that they can be handled straightforwardly with hereditary
substitution.

\begin{figure}
  \begin{center}
    \begin{tabular}{l}
      Syntax:\\
      \begin{tabular}{lll}
        $K$ & $:=$ & $*_0$ $|$ $*_1$             $|$ $\ldots$\\
        $T$ & $:=$ & $X$   $|$ $T \rightarrow T$ $|$ $\forall X:K.T$  $|$ $T + T$\\
        $t$ & $:=$ & $x$   $|$ $\lambda x:T.t$   $|$ $t\ t$ $|$ 
        $\Lambda X:K.t$ $|$ $t[T]$ $|$ $inl(t)$ $|$ $inr(t)$ $|$ $\ccon{t}{x}{t}{t}$\\
      \end{tabular}
      \\ \\
      Reduction Rules:\\
      \begin{tabular}{cc}
        \begin{tabular}{lll}
          $(\Lambda X:*_p.t)[\phi]$ & $\rightsquigarrow$ & $[\phi/X]t$\\
          $(\lambda x:\phi.t)t'$    & $\rightsquigarrow$ & $[t'/x]t$\\
        \end{tabular}
        &
        \begin{tabular}{lll}
          $\ccon{inl(t)}{x}{t_1}{t_2}$ & $\rightsquigarrow$ & $[t/x]t_1$\\
          $\ccon{inr(t)}{x}{t_1}{t_2}$ & $\rightsquigarrow$ & $[t/x]t_2$
        \end{tabular}
      \end{tabular}
      \\ \\
      Commuting Conversions:\\
      \begin{tabular}{lll}
        $(\ccon{t}{x}{t_1}{t_2})\ t'$ & $\rightsquigarrow$ & 
        $\ccon{t}{x}{(t_1\ t')}{(t_2\ t')}$\\
        $\ccon{(\ccon{t}{x}{t_1}{t_2})}{y}{s_1}{s_2}$ & $\rightsquigarrow$ & 
        $\mbox{case}\ t\ \mbox{of}\ $\\
        &                    & \ \ \ $x.(\ccon{t_1}{y}{s_1}{s_2}),$\\
        &                    & \ \ \ $x.(\ccon{t_1}{y}{s_1}{s_2})$
      \end{tabular}
    \end{tabular}
  \end{center}
  
  \caption[]{Syntax, Reduction Rules, Commuting Conversions for SS$\Fp$}
  \label{fig:syntax_ssfp}
\end{figure}

\begin{figure}[h]
  \begin{center}
    \begin{tabular}{ccc}
      \begin{math}
        $$\mprset{flushleft}
        \inferrule* [right=] {
          \ 
        }{\cdot\ Ok}
      \end{math}
      & 
      \begin{math}
        $$\mprset{flushleft}
        \inferrule* [right=] {
          \Gamma\ Ok
        }{\Gamma,X:*_p\ Ok}
      \end{math}
      &
      \begin{math}
        $$\mprset{flushleft}
        \inferrule* [right=] {
          \Gamma \vdash \phi:*_p
          \\
          \Gamma\ Ok
        }{\Gamma,x :\phi\ Ok}
      \end{math} 
    \end{tabular}	
    
    \caption{Well-formedness of Contexts for SS$\Fp$}
    \label{fig:well-formed_ssfp}
  \end{center}
\end{figure}

\begin{figure}
  \begin{center}
    \setlength{\tabcolsep}{1pt}
    \begin{tabular}{cccc}
      \begin{math}
        $$\mprset{flushleft}
        \inferrule* [right=] {
          \Gamma \vdash \phi_1 : *_p
          \\
          \Gamma \vdash \phi_2 : *_q
        }{\Gamma \vdash \phi_1 \rightarrow \phi_2 : *_{max(p,q)}}
      \end{math}      
      & 
      \begin{math}
        $$\mprset{flushleft}
        \inferrule* [right=] {
          \Gamma,X : *_q \vdash \phi : *_p
        }{\Gamma \vdash \forall X:*_q.\phi : *_{max(p,q)+1}}
      \end{math}
      &
      \begin{math}
        $$\mprset{flushleft}
        \inferrule* [right=] {
          \Gamma \vdash \phi_1 : *_p
          \\
          \Gamma \vdash \phi_2 : *_q
        }{\Gamma \vdash \phi_1 + \phi_2 : *_{max(p,q)}}
      \end{math}
      & 
      \begin{math}
        $$\mprset{flushleft}
        \inferrule* [right=] {
          \Gamma(X) = *_p
          \\\\
          \Gamma\ Ok
          \\
          p \leq q
        }{\Gamma \vdash X : *_q}
      \end{math}
    \end{tabular}	
    
    \caption[]{SS$\Fp$ Kinding Rules}
    \label{fig:kinding_rules_ssfp}
  \end{center}
\end{figure}

\begin{figure}
  \setlength{\tabcolsep}{1pt}
    \begin{tabular}{cccc}
      \begin{math}
        $$\mprset{flushleft}
        \inferrule* [right=] {
          \Gamma(x) = \phi
          \\\\
          \Gamma\ Ok
        }{\Gamma \vdash x : \phi}
      \end{math}  
      &
      \begin{math}
        $$\mprset{flushleft}
        \inferrule* [right=] {
          \Gamma,x : \phi_1 \vdash t : \phi_2
        }{\Gamma \vdash \lambda x : \phi_1.t : \phi_1 \rightarrow \phi_2}
      \end{math} 
      &
      \begin{math}
        $$\mprset{flushleft}
        \inferrule* [right=] {
          \Gamma \vdash t_1 : \phi_1 \rightarrow \phi_2 
          \\\\
          \Gamma \vdash t_2 : \phi_1
        }{\Gamma \vdash t_1t_2 : \phi_2}
      \end{math}  
      &
      \begin{math}
        $$\mprset{flushleft}
        \inferrule* [right=] {
          \Gamma, X : *_l \vdash t : \phi
        }{\Gamma \vdash \Lambda X:*_l.t:\forall X : *_l.\phi}
      \end{math} 
      \\ \\
      \begin{math}
        $$\mprset{flushleft}
        \inferrule* [right=] {
          \Gamma \vdash t:\forall X:*_l.\phi_1
          \\\\
          \Gamma \vdash \phi_2:*_l
        }{\Gamma \vdash t[\phi_2]: [\phi_2/X]\phi_1}
      \end{math} 
      &
      \begin{math}
        $$\mprset{flushleft}
        \inferrule* [right=] {
          \Gamma \vdash t:\phi_1
          \\\\
	  \Gamma \vdash \phi_2:*_p
        }{\Gamma \vdash inl(t): \phi_1+\phi_2}
      \end{math}
      &
      \begin{math}
        $$\mprset{flushleft}
        \inferrule* [right=] {
          \Gamma \vdash t:\phi_2
          \\\\
	  \Gamma \vdash \phi_1:*_p
        }{\Gamma \vdash inr(t): \phi_1+\phi_2}
      \end{math}
      &
      \begin{math}
        $$\mprset{flushleft}
        \inferrule* [right=] {
          \Gamma \vdash t:\phi_1 + \phi_2
          \\\\
	  \Gamma,x:\phi_1 \vdash t_1:\psi
          \\\\
	  \Gamma,x:\phi_2 \vdash t_2:\psi
        }{\Gamma \vdash \ccon{t}{x}{t_1}{t_2}: \psi}
      \end{math}
    \end{tabular}
    
    \caption[]{SS$\Fp$ Type-Assignment Rules}
    \label{fig:typing_rules_ssfp}
  
\end{figure}
The syntax, reduction rules, and commuting conversions for SS$\Fp$ can
be found in Fig.~\ref{fig:syntax_ssfp}.  This extension of SSF is
based on the version of SSF used in \cite{Eades:2010}.  The
kind-assignment rules are defined in Fig.~\ref{fig:kinding_rules_ssfp}
and the type-assignment rules in defined in
Fig.~\ref{fig:typing_rules_ssfp}. The kinding/typing relations
depend on well-formed contexts which are defined in
Fig.~\ref{fig:well-formed_ssfp}.  To ensure substitutions over
contexts behave in an expected manner, we rename variables as
necessary to ensure contexts have at most one declaration per
variable.  Lastly, throughout this chapter we the following basic
meta-theoretic results:  

\begin{lemma}
  If $\Gamma \vdash \phi:*_p$ then $\Gamma\ Ok$.
  \label{lemma:kinding_ok_ssfp}
\end{lemma}
\begin{proof}
  This holds by straightforward induction on the form of the assumed
  kinding derivation.
\end{proof}

\begin{lemma}[Level Weakening for Kinding]
  If $\Gamma \vdash \phi:*_r$ and $r < s$ then $\Gamma \vdash \phi:*_s$.
  \label{lemma:level_weakening_for_kinding_ssfp}
\end{lemma}
\begin{proof}
  This holds by straightforward induction on the form of the assumed
  kinding derivation.
\end{proof}

\begin{lemma}[Substitution for Kinding, Context-Ok]
  Suppose $\Gamma \vdash \phi':*_p$.  If $\Gamma,X:*_p,\Gamma' \vdash \phi:*_q$ 
  with a derivation of depth $d$, then $\Gamma,[\phi'/X]\Gamma' \vdash [\phi'/X]\phi:*_q$
  with a derivation of depth $d$.
  Also, if $\Gamma,X:*_p,\Gamma'\ Ok$ with a derivation of depth $d$, then 
  $\Gamma,[\phi'/X]\Gamma'\ Ok$ with a derivation of depth $d$.
  \label{lemma:substitution_for_kinding_ssfp}
\end{lemma}
\begin{proof}
  This holds by straightforward induction on the $d$.
\end{proof}

\begin{lemma}[Regularity]
  If $\Gamma \vdash t:\phi$ then $\Gamma \vdash \phi:*_p$ for some $p$.
  \label{lemma:regularity_ssfp}
\end{lemma}
\begin{proof}
  This holds by straightforward induction on the form of the assumed
  typing derivation.
\end{proof}

\subsection{Ordering on Types}
\label{subsec:ordering_on_types_ssfp}
In this section we define an ordering on types.  This ordering is
curcial for the hereditary substitution method.  We will see in
Section~\ref{sec:properties_of_the_hereditary_substitution_function_ssfp}
that we prove several properties of the hereditary substitution
function defined in
Section~\ref{sec:the_hereditary_substitution_function_ssfp}.  The
ordering used in these proofs is the lexicographic ordering consisting
of the the ordering we are about to define and the strict
subexpression ordering on terms.  In fact if no ordering on types exists then
one cannot prove the necessary properties of the hereditary
substitution function needed to conclude normalization.

\begin{definition}
  The ordering $>_\Gamma$ is defined as the least relation satisfying the universal closures of 
  the following formulas:\\
  \begin{center}
    \begin{tabular}{lll}
      \begin{tabular}{lll}
        $\phi_1 \rightarrow \phi_2$ & $>_\Gamma$ & $\phi_1$\\
        $\phi_1 \rightarrow \phi_2$ & $>_\Gamma$ & $\phi_2$\\
        & & \\
      \end{tabular}
      &
      \ \ \ \ \ \ \ \ \ \ \ \
      \begin{tabular}{lll}
        $\phi_1 + \phi_2$           & $>_\Gamma$ & $\phi_1$\\
        $\phi_1 + \phi_2$           & $>_\Gamma$ & $\phi_2$\\
        $\forall X:*_l.\phi$        & $>_\Gamma$ & $[\phi'/X]\phi$ where 
        $\Gamma \vdash \phi':*_l$.\\
      \end{tabular}
    \end{tabular}
  \end{center}
  \label{def:ordering_ssfp}
\end{definition}

\begin{thm}[Well-Founded Ordering]
  The ordering $>_\Gamma$ is well-founded on types $\phi$ such that 
  $\Gamma \vdash \phi:*_l$ for some $l$.
  \label{thm:well-founded_ordering_ssfp}
\end{thm}
\begin{proof}
  The depth function, defined in the following definition, is used in the following proof.

  \begin{definition}
    The depth of a type $\phi$ is defined as follows:
    \begin{center}
      \begin{tabular}{lll}
        $depth(X)$                  & $=$ & $1$\\
        $depth(\phi \to \phi')$     & $=$ & $depth(\phi) + depth(\phi')$\\
        $depth(\phi + \phi')$       & $=$ & $depth(\phi) + depth(\phi')$\\
        $depth(\forall X:*_l.\phi)$ & $=$ & $depth(\phi) + 1$\\
      \end{tabular}
    \end{center}
  \end{definition}

  We define the metric $(l,d)$ in lexicographic combination, where $l$
  is the level of a type $\phi$ and $d$ is the depth of $\phi$.

  \begin{lemma}[Well-Founded Measure]
    \label{lemma:well-founded_measure_ssfp}
    If $\phi >_\Gamma \phi'$ then $(l,d) > (l',d')$, where $\Gamma \vdash \phi:*_l$, 
    $depth(\phi) = d$,  $\Gamma \vdash \phi:*_{l'}$, and $depth(\phi') = d'$.
  \end{lemma}
  \begin{proof}
    This holds by straightforward induction on the structure of
    $\phi$. 
  \end{proof}
  
  Finally, the proof of well-foundedness of $>_\Gamma$.  If there exists
  an infinite decreasing sequence using our ordering on types, then there
  is an infinite decreasing sequence using our measure by
  Lemma~\ref{lemma:well-founded_measure_ssfp}, but that is impossible.
\end{proof}
\noindent
We need transitivity in a number of places in the proof of the main
substitution lemma.  

\begin{lemma}[Transitivity of $>_\Gamma$]
  Let $\phi$, $\phi'$, and $\phi''$ be kindable types.  If $\phi >_\Gamma \phi'$ and 
  $\phi' >_\Gamma \phi''$ then $\phi >_\Gamma \phi''$.
  \label{lemma:transitivity_ssfp}
\end{lemma}
\begin{proof}
  Suppose $\phi >_\Gamma \phi'$ and $\phi' >_\Gamma \phi''$.  If 
  $\phi \equiv \phi_1 \rightarrow \phi_2$ or $\phi \equiv \phi_1 + \phi_2$ then,
  $\phi'$ must be a subexpression of $\phi$.  Now if $\phi' \equiv \phi'_1 \rightarrow \phi'_2$ 
  or $\phi' \equiv \phi'_1 + \phi'_2$ then,
  $\phi''$ must be a subexpression of $\phi'$, which implies that $\phi''$ is a subexpression of 
  $\phi$.  Thus, $\phi >_\Gamma \phi''$.
  
  If $\phi' \equiv \forall X:*_l.\phi'_1$ then, there exists a type $\phi'_2$ where, 
  $\Gamma \vdash \phi'_2:*_l$, such that, 
  $\phi'' \equiv [\phi'_2/X]\phi'_1$.  The level of $\phi'$ is $max(l,l')+1$, where $l'$ is the 
  level of $\phi'_1$, the level of 
  $\phi''$ is $max(l,l')$, and the level of $\phi$ is $max(max(l,l')+1,p)$, where $p$ is the 
  level of the type, which is, the second subexpression of $\phi$.
  Clearly, $max(max(l,l')+1,p) \geq max(l,l')$, thus, $\phi >_\Gamma \phi''$.

  If $\phi \equiv \forall X:*_l.\phi_1$, then $\phi' \equiv [\phi_2/X]\phi_1$ for some type 
  $\phi_2$, where $\Gamma \vdash \phi_2:*_l$.  If
  $[\phi_2/X]\phi_1 \equiv \phi'_1 \rightarrow \phi'_2$ or 
  $[\phi_2/X]\phi_1 \equiv \phi'_1 + \phi'_2$, then the level of $\phi'$ is $max(p,q)$, where 
  $p$ is the level
  of $\phi'_1$ and $q$ is the level of $\phi'_2$.  Now $\phi''$ must be a subexpression of 
  $\phi'$, hence the level of $\phi''$ is either $p$ or $q$.  Now, since
  the level of $\phi$ is greater than the level of $\phi'$ and we know, $max(p,q)$ is greater 
  than both $p$ and $q$ then $\phi >_\Gamma \phi''$.  If 
  $[\phi_2/X]\phi_1 \equiv \forall Y:*_{l'}.\phi'_1$, then $\phi'' \equiv [\phi'_2/X]\phi'_1$.  
  Now if $p$ is the level of $\phi_1$, then the level of $\phi$ is
  $max(l,p)+1$ and the level of $\phi'$ must be $max(l,p)$ since we know the level of $\phi'$ is 
  greater than the level of $\phi''$ then clearly, the level of 
  $\phi$ is greater than the level of $\phi''$.  Thus, $\phi >_\Gamma \phi''$.
\end{proof}
% subsection ordering_on_types_ssfp (end)

\subsection{The Hereditary Substitution Function}
\label{sec:the_hereditary_substitution_function_ssfp}
We will work through the hereditary substitution method in more detail
for this first example system than for the other two.  The difference
between ordinary capture avoiding substitution and hereditary substitution
is if as a result of substitution a new redex is created, then
that redex is recursively reduced.  The hereditary substitution
function is denoted $[t/x]^\phi t'$ where $\phi$ is called the cut
type, due to the relation of the hereditary substitution function and
cut elimination, and $t'$ is called the principle term of
substitution.  The purpose of the cut type is that it is used to
ensure that the definition of the function is well founded.  We will
see this connection below.  We now turn to defining this function for
SS$\Fp$.

The definition of the hereditary substitution function for SS$\Fp$ is
in Fig.~\ref{fig:hereditary_substitution_function_part1} and
Fig.~\ref{fig:hereditary_substitution_function_part2}.  First, one
should read this definition as a mutually recursive function in terms
of the hereditary substitution function $[t/x]^\phi t'$,
the application reduction function $app_\phi\ t_1\ t_2$, and 
case construct reduction function $rcase_\phi\ t_0\ x\ t_1\ t_2$.  The
definitions of all these functions depend on a partial function called
$ctype_\phi(x, t)$ which returns the type of a term $t$ if that term
is in weak-head normal form.  This function is equivalent to the
$treduce$ function used in \cite{Watkins:2004}.
%
A brief explanation of the definition of the hereditary substitution
function is the following.  One should read the
``where''-conditions below some of the cases of the definitions of
each function as preconditions.  The definition of $ctype_\phi$ is
straightforward.  Now the definition of the hereditary substitution
function has the form one would expect for a substitution function.
In fact the only difference between the definition of the hereditary
substitution function and capture avoiding substitution is really
present in the cases for applications, type instantiation, and case
constructs.  These are the locations where new redexes can be created
by substitution.  

Consider application.  Based on the definition of the reduction rules
we have two forms of redexes: either a $\beta$-redex or a commuting
coversion where a case construct is applied to an argument.  In order
to create a new $\beta$-redex as a result of substitution we must be
substituting into a term of the form $x\ t_1 \cdots t_n$.  The
hereditary substitution function detects this by using the
$ctype_\phi$ function.  We will show that if this function is defined
then the head of the input term must be a variable $x$ thus implying
that the input term is of the form we expect.  Next we apply the
hereditary substitution to the head of the application and if that
results in a $\lambda$-abstraction then we know that a new redex will
be created by substitution.  Thus, we recursively reduce this redex
using the hereditary substitution function.  We will see that the
result of the $ctype_\phi$ function is in fact a subexpression of the
cut type $\phi$.  This tells us that the cut type in the case for
creation of a new $\beta$-redex has reduced in the recursive call to
the hereditary substitution function.

Now in the definition of the hereditary substitution function where a
new redex in the form of the commuting conversion is created -- in
this case a case construct is applied to an argument -- we again know
by the $ctype_\phi$ function that the head of the application is in
the form $x\,t_1 \cdots t_2$.  Furthermore, we know applying the
hereditary substitution function to the head of the application
results in a case construct.  So we recursively reduce the created
redex in the same way the reduction rules do, but when we push the
argument into the branches of the resulting case construct more
redexes may be created.  So to handle recursively reducing all of the
newly created redexes in the branches we call the application
reduction function $app_\phi$.  This function reduces redexes by
recursively calling itself and the hereditary substitution function.
The remaining cases where new redexes are potentially created are
similar to these cases.  The function $rcase_\phi$ handles reducing 
case constructs.
\begin{figure}[t]
  \small
  \begin{itemize}
  \item[] $ctype_\phi(x,x) = \phi$
  \item[] $ctype_\phi(x,t_1\ t_2) = \phi''$\\
    \begin{tabular}{lll}
      & Where $type_\phi(x,t_1) = \phi' \to \phi''$.
    \end{tabular}    
  \end{itemize}
  
  \begin{itemize}
  \item[] $app_\phi\ t_1\ t_2 = t_1\ t_2$\\
    \begin{tabular}{lll}
      & Where $t_1$ is not a $\lambda$-abstraction or a case construct.
    \end{tabular}
  \item[] $app_\phi\ (\lambda x:\phi'.t_1)\ t_2 = [t_2/x]^{\phi'} t_1$
  \item[] $app_\phi\ (\ccon{t_0}{x}{t_1}{t_2})\ t = 
    \ccon{t_0}{x}{(app_\phi\ t_1\ t)}{(app_\phi\ t_2\ t)}$
  \end{itemize}
  
  \begin{itemize}
  \item[] $rcase_{\phi}\ t_0\ y\ t_1\ t_2 = \ccon{t_0}{y}{t_1}{t_2}$\\
    \begin{tabular}{lll}
      & Where $t_0$ is not an inject-left or an inject-right term or a case construct.
    \end{tabular}
  \item[] $rcase_{\phi}\ inl(t')\ y\ t_1\ t_2 = [t'/y]^{\phi_1}\ t_1$
  \item[] $rcase_{\phi}\ inr(t')\ y\ t_1\ t_2 = [t'/y]^{\phi_2}\ t_2$
  \item[] $rcase_{\phi}\ (\ccon{t'_0}{x}{t'_1}{t'_2})\ y\ t_1\ t_2 = 
    \ccon{t'_0}{x}{(rcase_{\phi}\ t'_1\ y\ t_1\ t_2)}{(rcase_{\phi}\ t'_2\ y\ t_1\ t_2)}$
  \end{itemize}
  \caption{Hereditary Substitution Function for Stratified System $\Fp$}
  \label{fig:hereditary_substitution_function_part1}
\end{figure}

\begin{figure}[t]
  \begin{itemize}
  \item[] $[t/x]^\phi x = t$
  \item[] $[t/x]^\phi y = y$\\
    \begin{tabular}{lll}
      & Where $y$ is a variable distinct from $x$.\\
    \end{tabular}
  \item[] $[t/x]^\phi (\lambda y:\phi'.t') = \lambda y:\phi'.([t/x]^\phi t')$
  \item[] $[t/x]^\phi (\Lambda X:*_l.t') = \Lambda X:*_l.([t/x]^\phi t')$
  \item[] $[t/x]^\phi inr(t') = inr([t/x]^\phi t')$
  \item[] $[t/x]^\phi inl(t') = inl([t/x]^\phi t')$
  \item[] $[t/x]^\phi (t_1\ t_2) = ([t/x]^\phi t_1)\ ([t/x]^\phi t_2)$\\
    \begin{tabular}{lll}
      & Where $([t/x]^\phi t_1)$ is not a $\lambda$-abstraction or a case construct, or both 
        $([t/x]^\phi t_1)$\\
      & and $t_1$ are $\lambda$-abstractions or case constructs, or $ctype_\phi(x,t_1)$ is 
        undefined.
    \end{tabular}
  \item[] $[t/x]^{\phi} (t_1\ t_2) = [([t/x]^{\phi} t_2)/y]^{\phi''} s'_1$\\
    \begin{tabular}{lll}
      & Where $([t/x]^{\phi} t_1) = \lambda y:\phi''.s'_1$ 
      for some $y$, $s'_1$, and $\phi''$ and $ctype_\phi(x,t_1) = \phi'' \to \phi'$.\\
    \end{tabular}
  \item[] $[t/x]^{\phi} (t_1\ t_2) = 
    \ccon{w}{y}{(app_\phi\ r\ ([t/x]^{\phi} t_2))}{(app_\phi\ s\ ([t/x]^{\phi} t_2))}$\\
    \begin{tabular}{ll}
      & Where $[t/x]^{\phi} t_1 = \ccon{w}{y}{r}{s}$ for some terms $w$, $r$, $s$ 
      and variable $y$, and \\
      & $ctype_\phi(x,t_1) = \phi'' \to \phi'$.
    \end{tabular}
  \item[] $[t/x]^\phi (t'[\phi']) = ([t/x]^\phi t')[\phi']$\\
    \begin{tabular}{lll}
      & Where $[t/x]^\phi t'$ is not a type abstraction or
      $t'$ and $[t/x]^\phi t'$ are type abstractions.
    \end{tabular}
    \item[] $[t/x]^{\phi} (t'[\phi']) = [\phi'/X]s'_1$\\
      \begin{tabular}{lll}
        & Where $[t/x]^{\phi} t' = \Lambda X:*_l.s'_1$,
        for some $X$, $s'_1$ and $\Gamma \vdash \phi':*_q$, such that, $q \leq l$ and\\
	& $t'$ is not a type abstraction.
      \end{tabular}
    \item[] $[t/x]^{\phi} (\ccon{t_0}{y}{t_1}{t_2}) = 
      \ccon{([t/x]^{\phi} t_0)}{y}{([t/x]^{\phi} t_1)}{([t/x]^{\phi} t_2)}$\\
      \begin{tabular}{lll}
        & Where $([t/x]^{\phi} t_0)$ is not an inject-left or an inject-right term or a case 
        construct, or\\
        & $([t/x]^{\phi} t_0)$ and $t_0$ are both inject-left or inject-right terms or case 
        constructs, or\\
        & $ctype_\phi(x,t_0)$ is undefined.
      \end{tabular}
    \item[] $[t/x]^{\phi} (\ccon{t_0}{y}{t_1}{t_2}) = 
      rcase_{\phi}\ ([t/x]^{\phi} t_0)\ y\ 
      ([t/x]^{\phi} t_1)\ ([t/x]^{\phi} t_2)$\\
      \begin{tabular}{lll}
        & Where $([t/x]^{\phi} t_0)$ is an inject-left or an inject-right term or 
        a case construct and\\
        & $ctype_\phi(x,t_0) = \phi_1+\phi_2$.
      \end{tabular}
  \end{itemize}
  \caption{Hereditary Substitution Function for Stratified System $\Fp$ Continued}
  \label{fig:hereditary_substitution_function_part2}
\end{figure}
% subsection the_hereditary_substitution_function_ssfp (end)

\subsection{Properties of The Hereditary Substitution Function}
\label{sec:properties_of_the_hereditary_substitution_function_ssfp}
We now turn to proving several properites of the hereditary
substitution function.  Since the various functions involved in the
definition of the hereditary substitution function including the
hereditary substitution function itself depends on the $ctype_\phi$
function we first establish its main properties.  The major property of
this function is that its output is a subexpression of
the cut type $\phi$.  This is stated in part one of the next lemma.
Part two is a sanity check which shows that the type
returned by $ctype_\phi$ is the right type.  In other words it is the
type of the second argument of the function.  The remaining parts of
the lemma are used in the proofs of the other properties of the
hereditary substitution function.  Recall in certain parts of the
definition of the hereditary substitution function it must be the case
that $ctype_\phi$ is defined so the remaining parts of the properties
lemma ensure this is the case.

\begin{lemma}[Properties of $ctype_\phi$]
  \label{lemma:ctype_props_ssfp}

  \begin{itemize}
  \small
  \item[]
  \item[i.] If $ctype_\phi(x,t) = \phi'$ then $head(t) = x$ and $\phi'$ 
    is a subexpression of $\phi$.
    
  \item[ii.] If $\Gamma,x:\phi,\Gamma' \vdash t:\phi'$ and $ctype_\phi(x,t) = \phi''$ then
    $\phi' \equiv \phi''$.

  \item[iii.] If $\Gamma,x:\phi,\Gamma' \vdash t_1\ t_2:\phi'$, $\Gamma \vdash t:\phi$,
    $[t/x]^\phi t_1 = \lambda y:\phi_1.q$, and $t_1$ is not then there exists a type
    $\psi$ such that $ctype_\phi(x,t_1) = \psi$.

  \item[iv.] If $\Gamma,x:\phi,\Gamma' \vdash t_1\ t_2:\phi'$, $\Gamma \vdash t:\phi$,
    $[t/x]^\phi t_1 = \ccon{t'_0}{y}{t'_1}{t'_2}$, and $t_1$ is not then there exists a type
    $\psi$ such that $ctype_\phi(x,t_1) = \psi$.

  \item[v.] If $\Gamma,x:\phi,\Gamma' \vdash \ccon{t_0}{y}{t_1}{t_2}:\phi'$, 
    $\Gamma \vdash t:\phi$, $[t/x]^\phi t_0 = \ccon{t'_0}{z}{t'_1}{t'_2}$, and 
    $t_0$ is not then there exists a type $\psi$ such that $ctype_\phi(x,t_0) = \psi$.

  \item[vi.] If $\Gamma,x:\phi,\Gamma' \vdash \ccon{t_0}{y}{t_1}{t_2}:\phi'$, 
    $\Gamma \vdash t:\phi$, $[t/x]^\phi t_0 = inl(t')$, and $t_0$ is not then there 
    exists a type $\psi$ such that $ctype_\phi(x,t_0) = \psi$.

  \item[vii.] If $\Gamma,x:\phi,\Gamma' \vdash \ccon{t_0}{y}{t_1}{t_2}:\phi'$, 
    $\Gamma \vdash t:\phi$, $[t/x]^\phi t_0 = inr(t')$, and $t_0$ is not then 
    there exists a type $\psi$ such that $ctype_\phi(x,t_0) = \psi$.
  \end{itemize}
\end{lemma}
\begin{proof}
 We prove part one first. This is a proof by induction on the structure of $t$.

\begin{itemize}
\item[Case.] Suppose $t \equiv x$.  Then $ctype_\phi(x,x) = \phi$.  Clearly,
  $head(x) = x$ and $\phi$ is a subexpression of itself.
  
\item[Case.] Suppose $t \equiv t_1\ t_2$.  Then $ctype_\phi(x,t_1\ t_2) = \phi''$
  when $ctype_\phi(x,t_1) = \phi' \to \phi''$.  Now $t > t_1$ so by the induciton
  hypothesis $head(t_1) = x$ and $\phi' \to \phi''$ is a subexpression of $\phi$.
  Therefore, $head(t_1\ t_2) = x$ and certainly $\phi''$ is a subexpression of $\phi$.
\end{itemize}

\ \\
We now prove part two.  This is also a proof by induction on the structure of $t$.

\begin{itemize}
\item[Case.] Suppose $t \equiv x$.  Then $ctype_\phi(x,x) = \phi$.  Clearly,
  $\phi \equiv \phi$.
  
\item[Case.] Suppose $t \equiv t_1\ t_2$.  Then $ctype_\phi(x,t_1\ t_2) = \phi_2$
  when $ctype_\phi(x,t_1) = \phi_1 \to \phi_2$.  By inversion on the assumed typing
  derivation we know there exists type $\phi''$ such that $\Gamma,x:\phi,\Gamma' \vdash t_1:\phi'' \to \phi'$.
  Now $t > t_1$ so by the induciton hypothesis $\phi_1 \to \phi_2 \equiv \phi'' \to \phi'$.
  Therefore, $\phi_1 \equiv \phi''$ and $\phi_2 \equiv \phi'$.
\end{itemize}

\noindent 
Next we prove part three.  This is a proof by induction on the structure of $t_1\ t_2$.

\ \\
The only possiblities for the form of $t_1$ is $x$ or $\hat{t}_1\ \hat{t}_2$.  All other 
forms would not result in $[t/x]^\phi t_1$ being a $\lambda$-abstraction and $t_1$ not.
If $t_1 \equiv x$ then there exist a type $\phi''$ such that $\phi \equiv \phi'' \to \phi'$ and
$ctype_\phi(x,x\ t_2) = \phi'$ when $ctype_\phi(x,x) = \phi \equiv \phi'' \to \phi'$ in this case.  We know
$\phi''$ to exist by inversion on $\Gamma,x:\phi,\Gamma' \vdash t_1\ t_2:\phi'$.

\ \\
Now suppose $t_1 \equiv (\hat{t}_1\ \hat{t}_2)$.  Now knowing $t'_1$ to not be a $\lambda$-abstraction
implies that $\hat{t}_1$ is also not a $\lambda$-abstraction or $[t/x]^\phi t_1$ would be an application
instead of a $\lambda$-abstraction.  So it must be the case that $[t/x]^\phi \hat{t}_1$ is a $\lambda$-abstraction
and $\hat{t}_1$ is not.  Since $t_1\ t_2 > t_1$ we can apply the induction hypothesis to obtain there exists
a type $\psi$ such that $ctype_\phi(x,\hat{t}_1) = \psi$.  
Now by inversion on $\Gamma,x:\phi,\Gamma' \vdash t_1\ t_2:\phi'$ we know there exists a type $\phi''$ such that
$\Gamma,x:\phi,\Gamma' \vdash t_1:\phi'' \to \phi'$.  We know $t_1 \equiv (\hat{t}_1\ \hat{t}_2)$ so by inversion on
$\Gamma,x:\phi,\Gamma' \vdash t_1:\phi'' \to \phi'$ we know there exists a type $\psi''$ such that
$\Gamma,x:\phi,\Gamma' \vdash \hat{t}_1:\psi'' \to (\phi'' \to \phi')$.
By part two of Lemma~\ref{lemma:ctype_props_ssfp} we know $\psi \equiv \psi'' \to (\phi'' \to \phi')$ and
$ctype_\phi(x,t_1) = ctype_\phi(x,\hat{t}_1\ \hat{t}_2) = \phi'' \to \phi'$ 
when $ctype_\phi(x,\hat{t}_1) = \psi'' \to (\phi'' \to \phi')$, because we know $ctype_\phi(x,\hat{t}_1) = \psi$.

\ \\
The proofs of the remaining parts are similar to the proof of part three.
\end{proof}

We now move on to proving the main properties of the hereditary
substitution function.  First, we show that for typeable terms it is a
total function and the output maintains the same type as the principle
term of substitution.
\begin{lemma}[Total and Type Preserving]
  \label{lemma:total_ssfp}
  Suppose $\Gamma \vdash t : \phi$ and $\Gamma, x:\phi, \Gamma' \vdash t':\phi'$. Then
  there exists a term $t''$ such that $[t/x]^\phi t' = t''$ and 
  $\Gamma,\Gamma' \vdash t'':\phi'$.
\end{lemma}
\begin{proof}
  This is a proof by induction on the lexicorgraphic combination $(\phi, t')$ of $>_{\Gamma,\Gamma'}$ and
the strict subexpression ordering.  We case split on $t'$.

\begin{itemize}
\item[Case.] Suppose $t'$ is either $x$ or a variable $y$ distinct from $x$.  
  Trivial in both cases.
  
\item[Case.] Suppose $t' \equiv \lambda y:\phi_1.t'_1$.  By inversion on the
  typing judgement we know $\Gamma,x:\phi,\Gamma',y:\phi_1 \vdash t'_1:\phi_2$.
  We also know $t' > t'_1$, hence we can apply the induction hypothesis to obtain
  $[t/x]^\phi t'_1 = \hat{t}'_1$ and $\Gamma,\Gamma',y:\phi_1 \vdash \hat{t}:\phi_2$
  for some term $\hat{t}'_1$.  By the definition of the hereditary substitution function 
  $[t/x]^\phi t' = \lambda y:\phi_1.[t/x]^\phi t'_1 = \lambda y:\phi_1.\hat{t}'_1$.  It suffices
  to show that $\Gamma,\Gamma' \vdash \lambda y:\phi_1.\hat{t}'_1:\phi_1 \to \phi_2$.  
  By simply applying the $\lambda$-abstraction typing rule using
  $\Gamma,\Gamma',y:\phi_1 \vdash \hat{t}:\phi_2$ we obtain 
  $\Gamma,\Gamma' \vdash \lambda y:\phi_1.\hat{t}'_1:\phi_1 \to \phi_2$.
  
\item[Case.] Suppose $t' \equiv \Lambda X:*_l.t'_1$.  Similar to the previous case.
  
\item[Case.] Suppose $t' \equiv t'_1\ t'_2$.  By inversion we know
  $\Gamma, x:\phi, \Gamma' \vdash t'_1 : \phi'' \to \phi'$ and
  $\Gamma, x:\phi, \Gamma' \vdash t'_2 : \phi''$ for some types $\phi'$ and $\phi''$.
  Clearly, $t' > t'_i$ for $i \in \{1,2\}$.  Thus, by the induction hypothesis
  there exists terms $m_1$ and $m_2$ such that $[t/x]^\phi t'_i = m_i$,
  $\Gamma, \Gamma' \vdash m_1 : \phi'' \to \phi'$ and
  $\Gamma, \Gamma' \vdash m_2 : \phi''$ for
  $i \in \{1,2\}$.  We case split on whether or not $m_1$ is a $\lambda$-abstraction,
  a case construct and $t'_1$ is not, or $ctype_\phi(x,t'_1)$ is undefined.  
  We only consider the non-trivial cases when 
  $m_1 \equiv \lambda y:\phi''.m'_1$ and $t'_1$ is not a $\lambda$-abstraction or $m_1 \equiv \ccon{m'_0}{y}{m'_1}{m'_2}$,
  $ctype_\phi(x,t'_1) = \psi'' \to \psi'$, and $t'_1$ is not a case construct.  Suppose the former.  
  Now by Lemma~\ref{lemma:ctype_props_ssfp} it is the case that 
  there exists a $\psi$ such that $ctype_\phi(x,t'_1) = \psi$, 
  $\psi \equiv \phi'' \to \phi'$, and $\psi$ is a subexpression of $\phi$, hence
  $\phi >_{\Gamma,\Gamma'} \phi''$.
  Then $[t/x]^\phi (t'_1\ t'_2) = [m_2/y]^{\psi''} m'_1$.  
  Therefore, by the induction hypothesis there exists a 
  term $m$ such that $[m_2/y]^{\phi''} m'_1 = m$ and $\Gamma,\Gamma' \vdash m:\phi''$.
  
  \ \\
  Suppose $m_1 \equiv \ccon{m'_0}{y}{m'_1}{m'_2}$ and $t'_1$ is not a case construct.
  Now $[t/x]^\phi t' = \ccon{m'_0}{y}{app_\phi\ m'_1\ m_2}{app_\phi\ m'_2\ m_2}$.  By inversion on
  $\Gamma,\Gamma' \vdash m_1 : \phi'' \to \phi'$ we know there exists terms $\phi_1$ and $\phi_2$ such that
  $\Gamma,\Gamma' \vdash m'_0:\phi_1+\phi_2$ and
  $\Gamma,\Gamma',y:\phi_i \vdash m'_i:\phi'' \to \phi'$
  for $i \in \{1,2\}$.  It suffcies to show that
  there exists terms $q$ and $q'$ such that $app_\phi\ m'_1\ m_2 = q$ and $app_\phi\ m'_2\ m_2 = q'$.  To obtain
  this result we prove the following proposition.  Note that $ctype_\phi(x,t'_1) = \psi'' \to \psi'$ which
  by Lemma~\ref{lemma:ctype_props_ssfp} is equivalent to $\phi'' \to \phi'$ and is 
  a subexpression of $\phi$, hence $\phi >_{\Gamma,\Gamma'} \phi''$ and $\phi >_{\Gamma,\Gamma'} \phi'$.

  \ \\
  {\bf Proposition.}  For all 
  $\Gamma \vdash m_2 : \phi''$ and $\Gamma \vdash m'_1:\phi'' \to \phi'$
  there exists a term $q$ such that $app_\phi\ m'_1\ m_2 = q$ and $\Gamma \vdash q:\phi'$.
  
  \ \\
  We prove this by nested induction on the ordering $(\phi, t', m'_1)$ and case splitting on 
  the structure of $m'_1$.
  \begin{itemize}
  \item[Case.] Suppose $m'_1$ is neither a $\lambda$-abstraction or a case construct.  Then
    $app_\phi\  m'_1\ m_2 = m'_1\ m_2$.  Take $m'_1\ m_2$ for $q$ and by applying the application typing rule
    we know $\Gamma \vdash m'_1\ m_2:\phi'$.
    
  \item[Case.] Suppose $m'_1 \equiv \lambda z:\phi''.m''_1$.  Then $app_\phi\ m'_1\ m_2 = [m_2/z]^{\phi''} m''_1$.
    By inversion on the assumption $\Gamma \vdash m'_1:\phi'' \to \phi'$ we know 
    $\Gamma,z:\phi'' \vdash m''_1:\phi'$.  Since $\phi >_{\Gamma} \phi''$ we can apply the outter induction
    hypothesis to obtain there there exists a $q$ such that $[m_2/z]^{\phi''} m''_1 = q$ and 
    $\Gamma \vdash q:\phi'$.  Therefore, $app_\phi\ m'_1\ m_2 = q$.
    
  \item[Case.] Suppose $m'_1 \equiv \ccon{m''_0}{z}{m''_1}{m''_2}$.  Then\\
    $app_\phi\ m'_1\ m_2 = \ccon{m''_0}{z}{app_\phi\ m''_1\ m_2}{app_\phi\ m''_2\ m_2}$.  By inversion on the assumption
    $\Gamma \vdash m'_1:\phi''\to\phi'$ we know there exists types $\phi_1$ and $\phi_2$ such that
    $\Gamma \vdash m''_0:\phi_1+\phi_2$, $\Gamma,z:\phi_1 \vdash m''_1:\phi''\to\phi'$
    and $\Gamma,z:\phi_2 \vdash m''_2:\phi''\to\phi'$.  Since $m'_1 > m''_1$ and $m'_1 > m''_2$ we can 
    apply the inner induction hypothesis to obtain there exists terms $q'$ and $q''$ such that 
    $app_\phi\ m''_1\ m_2 = q'$, $\Gamma,z:\phi_1 \vdash q':\phi'$, $app_\phi\ m''_1\ m_2 = q''$ and $\Gamma,z:\phi_2 \vdash q'':\phi'$.  
    Hence, 
    $app_\phi\ m'_1\ m_2 = \ccon{m''_0}{z}{app_\phi\ m''_1\ m_2}{app_\phi\ m''_2\ m_2} = \ccon{m''_0}{z}{q'}{q''}$.  It suffices to 
    to show that $\Gamma \vdash \ccon{m''_0}{z}{q'}{q''}:\phi'$.  This is a simple consequence of applying the
    case-construct typing rule using $\Gamma \vdash m''_0:\phi_1+\phi_2$, $\Gamma,z:\phi_1 \vdash q':\phi'$, and
    $\Gamma,z:\phi_2 \vdash q'':\phi'$.        
  \end{itemize}

   
  \ \\
  By the previous proposition there exists terms $q$ and $q'$ such that \\
  $[t/x]^\phi t' = \ccon{m'_0}{y}{app_\phi\ m'_1\ m_2}{app_\phi\ m'_2\ m_2}
  = \ccon{m'_0}{y}{q}{q'}$, where $app_\phi\ m'_1\ m_2 = q$, $\Gamma,\Gamma',y:\phi_1 \vdash q:\phi'$, $app_\phi\ m'_1\ m_2 = q'$
  and $\Gamma,\Gamma',y:\phi_2 \vdash q':\phi'$.  It suffices to show that
  $\Gamma, \Gamma' \vdash \ccon{m'_0}{y}{q}{q'}:\phi'$.  From above we know that $\Gamma,\Gamma' \vdash m'_0:\phi_1+\phi_2$, 
  $\Gamma,\Gamma',y:\phi_1 \vdash q:\phi'$ and $\Gamma,\Gamma',y:\phi_2 \vdash q':\phi'$.  Thus,
  by applying the case-construct typing rule we obtain $\Gamma, \Gamma' \vdash \ccon{m'_0}{y}{q}{q'}:\phi'$.
  
\item[Case.] Suppose $t' \equiv t'_1[\phi'']$. Similar to the previous case.  
  
\item[Case.] Suppose $t' \equiv inl(t)$. Trivial.
  
\item[Case.] Suppose $t' \equiv inr(t)$. Trivial.
  
\item[Case.] Suppose $t' \equiv \ccon{m_0}{y}{m_1}{m_2}$. By inversion on the assumption
  $\Gamma,x:\phi,\Gamma' \vdash t':\phi'$ we know the following:
  \begin{center}
    \begin{math}
      \begin{array}{lll}
        \Gamma,x:\phi,\Gamma' \vdash m_0:\phi_1+\phi_2 \text{, for some types } \phi_1 \text{ and } \phi_2,\\
        \Gamma,x:\phi,\Gamma',y:\phi_1 \vdash m_1:\phi, and\\
        \Gamma,x:\phi,\Gamma',y:\phi_2 \vdash m_2:\phi.\\
      \end{array}
    \end{math}
  \end{center}
  It is easy to see that
  $t' > m_i$ for all $i \in \{0,1,2\}$.  Hence, by the induction hypothesis
  there exists terms $m'_0$, $m'_1$, and $m'_2$ such that $[t/x]^\phi m_i = m'_i$ for all $i \in \{0,1,2\}$,
  \begin{center}
    \begin{tabular}{lll}
      (i)   & $\Gamma,\Gamma \vdash m'_0:\phi_1+\phi_2$,  \\
      (ii)  & $\Gamma,\Gamma',y:\phi_1 \vdash m'_1:\phi'$, and\\
      (iii) & $\Gamma,\Gamma',y:\phi_2 \vdash m'_2:\phi'$.  
    \end{tabular}
  \end{center}
  We have two cases to consider.
  \begin{itemize}
  \item[Case.] Suppose $m_0$ and $m'_0$ are inject-left terms, inject-right terms, or case constructs,
    or $m_0$ is an inject-left term, inject-right term, or a case-construct and $m'_0$ is not, or
    $m_0$ and $m'_0$ are neither inject-left terms, inject-right terms, or case constructs, or
    $ctype_\phi(x,m_0)$ is undefined.
    Then \\
    $[t/x]^{\phi} (\ccon{m_0}{y}{m_1}{m_2}) = \ccon{m'_0}{y}{m'_1}{m'_2}$ and by applying the case-construct 
    typing rule to i - iii above we obtain $\Gamma,\Gamma' \vdash \ccon{m'_0}{y}{m'_1}{m'_2}:\phi'$.
    
  \item[Case.] Suppose $m'_0$ is an inject-left term, inject-right term, or case construct and $ctype_\phi(x,m_0) = \psi_1+\psi_2$.  Then\\
    $[t/x]^{\phi} (\ccon{m_0}{y}{m_1}{m_2}) = rcase_{\phi}\ m'_0\ y\ m'_1\ m'_2$ and by Lemma~\ref{lemma:ctype_props_ssfp} we know
    $\psi_1 + \psi_2 \equiv \phi_1+\phi_2$ and is a subexpression of $\phi$.
    It suffices to show that there exists some term
    $q$ such that $rcase_{\phi}\ m'_0\ y\ m'_1\ m'_2 = q$ and $\Gamma, \Gamma' \vdash q:\phi'$.  We obtain this
    result by the following proposition.
    
    \ \\
    {\bf Proposition.} For all $\Gamma \vdash q_0:\phi$, $\Gamma, y:\phi_1 \vdash q_1:\phi'$, 
    and $\Gamma, y:\phi_2 \vdash q_2:\phi'$ there exists a term $\hat{q}$ such that 
    $rcase_{\phi}\ q_0\ y\ q_1\ q_2 = \hat{q}$ and $\Gamma \vdash \hat{q}:\phi'$.  
    We prove this by induction on the the ordering $(\phi, t', q_0)$ and case split on the structure of $q_0$.
    \begin{itemize}
    \item[Case.] Suppose $q_0$ is not an inject-left term, inject-right term, or a case construct.  Then\\
      $rcase_{\phi}\ q_0\ y\ q_1\ q_2 = \ccon{q_0}{y}{q_1}{q_2}$ and by applying the case-construct typing rule
      using the assumptions $\Gamma \vdash q_0:\phi$, $\Gamma, y:\phi_1 \vdash q_1:\phi'$, 
      and $\Gamma, y:\phi_2 \vdash q_2:\phi'$ we obtain $\Gamma,\Gamma' \vdash \ccon{q_0}{y}{q_1}{q_2}:\phi'$.
      
    \item[Case.] Suppose $q_0 \equiv inl(q'_0)$.  Then $rcase_{\phi}\ q_0\ y\ q_1\ q_2 = [q'_0/y]^{\phi_1} q_1$ and
      by inversion on $\Gamma \vdash q_0:\phi$ we know $\Gamma \vdash q'_0:\phi_1$.  It suffices to show that there
      exists a term $\hat{q}$ such that $[q'_0/y]^{\phi_1} q_1 = \hat{q}$  and $\Gamma \vdash \hat{q}:\phi'$.  
      Since $\phi >_{\Gamma} \phi'$ we can apply the the outer induction hypothesis to obtain that there exists 
      such a term $\hat{q}$.
      
    \item[Case.] Suppose $q_0 \equiv inl(q'_0)$.  Similar to the previous case.
      
    \item[Case.] Suppose $q_0 \equiv \ccon{q'_0}{z}{q'_1}{q'_2}$.  Then \\
      $rcase_{\phi}\ q_0\ y\ q_1\ q_2 = \ccon{q'_0}{z}{(rcase_{\phi}\ q'_1\ y\ q_1\ q_2)}{(rcase_{\phi}\ q'_2\ y\ q_1\ q_2)}$.
      We know by assumption that $\Gamma \vdash q_0:\phi$, $\Gamma \vdash q_0:\phi$, and $\Gamma, y:\phi_1 \vdash q_1:\phi'$
      so by inversion we know the following:
      \begin{center}
        \begin{math}
          \begin{array}{lll}
            (i) & \Gamma \vdash q'_0:\phi'_1 + \phi'_2 \text{, for some types } \phi'_1 \text{ and } \phi'_2,\\
            (ii) & \Gamma, z:\phi'_1 \vdash q'_1:\phi, \text{ and }\\
            (iii) & \Gamma, z:\phi'_2 \vdash q'_2:\phi.\\
          \end{array}
        \end{math}
      \end{center}
      Now $q_0 > q'_1$ and $q_0 > q'_1$ so we can apply the induction hypothesis twice to obtain terms $\hat{q}_1$ and
      $\hat{q}_2$ such that $rcase_{\phi}\ q'_1\ y\ q_1\ q_2 = \hat{q}_1$, $rcase_{\phi}\ q'_1\ y\ q_1\ q_2 = \hat{q}_1$,
      $\Gamma, z:\phi'_1 \vdash \hat{q}_1:\phi'$ and $\Gamma, z:\phi'_2 \vdash \hat{q}_2:\phi'$. So
      $\ccon{q'_0}{z}{(rcase_{\phi}\ q'_1\ y\ q_1\ q_2)}{(rcase_{\phi}\ q'_2\ y\ q_1\ q_2)} = 
      \ccon{q'_0}{z}{\hat{q}_1}{\hat{q}_2}$.  It suffices to show that $\ccon{q'_0}{z}{\hat{q}_1}{\hat{q}_2} = \hat{q}$ 
      and $\Gamma \vdash \hat{q}:\phi$ for some term $\hat{q}$.  Now $q_0 > q'_0$ so we can apply the induction hypothesis
      to obtain our result, but before we can we must show that $\Gamma \vdash \ccon{q'_0}{z}{\hat{q}_1}{\hat{q}_2}:\phi'$.
      This is a direct consequence of applying the case-construct typing rule using i, $\Gamma, z:\phi'_1 \vdash \hat{q}_1:\phi'$
      and $\Gamma, z:\phi'_2 \vdash \hat{q}_2:\phi'$.  Therefore, by the induction hypothesis there exists a term $\hat{q}$ 
      such that $\ccon{q'_0}{z}{\hat{q}_1}{\hat{q}_2} = \hat{q}$ and $\Gamma \vdash \hat{q}:\phi$.
    \end{itemize}
  \end{itemize}
\end{itemize}
\end{proof}
The next result we show is that the hereditary substitution function cannot 
create new redexes.  The following example shows when a particular redex is destroyed
by hereditary substitution.
\begin{example}
  \label{ex:commuting_conv_example_ssfp}
  Let $t \equiv inl(a)$ for some variable $a$ and $t' \equiv \ccon{(\ccon{x}{y}{y}{y})}{z}{z}{z}$
  where $\Gamma \vdash t : (\phi_1 + \phi_2) + \phi$, $\Gamma \vdash a : \phi_1 + \phi_2$, and 
  $\Gamma, x : (\phi_1 + \phi_2) + \phi \vdash t' : \phi'$.  Now lets trace the definition of 
  the hereditary substitution function on the input $t$, $x$, and $t'$ and compute what the 
  image of $[t/x]^{(\phi_1 + \phi_2) + \phi} t'$ will be.
  Well the first thing the hereditary substitution function does is apply itself to all the 
  parts of the case construct. So we must calculute the following results:
  \begin{itemize}
  \item[i.] First we have to compute $[t/x]^{(\phi_1 + \phi_2) + \phi} (\ccon{x}{y}{y}{y})$.  This 
    requires us to compute $[t/x]^{(\phi_1 + \phi_2) + \phi} x = t$.  Now the hereditary 
    substitution functions checks to see if $t$ is an inject-left term
    or an inject-right term, if it is then we have created a new redex.  It happens that 
    $t$ is so we know 
    $[t/x]^{(\phi_1 + \phi_2) + \phi} (\ccon{x}{y}{y}{y}) = [a/y]^{\phi_1 + \phi_2} y = a$.

  \item[ii.] Second we must compute $[t/x]^{(\phi_1 + \phi_2) + \phi} y = y$.
  \end{itemize}
  Now putting these pieces together we obtain 
  $[t/x]^{(\phi_1 + \phi_2) + \phi} t' = \ccon{a}{z}{z}{z}$.  Clearly,
  we no longer have the right-hand side of the commuting conversion for case constructs.
\end{example}
Showing redex preservation for the hereditary substitution function
depends on the following function which contructs the set of redexes
in a term.
\begin{definition}
  \label{def:rset_ssfp}
  The following function constructs the set of redexes within a term:

  \begin{center}
    \begin{itemize}
    \item[] $rset(x) = \emptyset$\\
    \item[] $rset(\lambda x:\phi.t) = rset(t)$\\
    \item[] $rset(\Lambda X:*_l.t) = rset(t)$\\
    \item[] $rset(t_1\ t_2)$\\
      \begin{math}
        \begin{array}{lll}
          = & rset(t_1, t_2) & \text{if } t_1 \text{ is not a } \lambda \text{-abstraction.}\\
          = & \{t_1\ t_2\} \cup rset(t'_1, t_2)\ & \text{if } t_1 \equiv \lambda x:\phi.t'_1.\\
        \end{array}
      \end{math}
    \item[] $rset(t''[\phi''])$\\
      \begin{math}
        \begin{array}{lll}
          = & rset(t'') & \text{if } t'' \text{ is not a type absraction.}\\
          = & \{t''[\phi'']\} \cup rset(t''') & \text{if } t'' \equiv \Lambda X:*_l.t'''.
        \end{array}
      \end{math}
    \item[] $rset(inl(t)) = rset(t)$\\
    \item[] $rset(inr(t)) = rset(t)$\\
    \item[] $rset(\ccon{t_0}{x}{t_1}{t_2})$\\
      \begin{math}
        \begin{array}{lll}
          = & rset(t_0) \cup rset(t_1,t_2) & \text{if } t_1 \text{ is not an inject-left term 
            or an inject-right term.}\\
          = & \{\ccon{t_0}{x}{t_1}{t_2}\} \cup rset(t_0) \cup rset(t_1,t_2) & \text{if } t_1 
          \text{ is an inject-left term or an inject-right term.}
        \end{array}
      \end{math}
    \end{itemize}
  \end{center}
  \ \\
  The extention of $rset$ to multiple arguments is defined as follows:
  \begin{center}
    $rset(t_1, \ldots, t_n) =^{def} rset(t_1) \cup \cdots \cup rset(t_n)$.
  \end{center}
\end{definition}

\begin{lemma}[Redex Preserving]
  \label{lemma:redex_preserving_ssfp}
  \small
  If $\Gamma \vdash t : \phi$, $\Gamma, x:\phi, \Gamma' \vdash t':\phi'$, and
  $t'$ then $|rset(t', t)| \geq |rset([t/x]^\phi t')|$.
\end{lemma}
\begin{proof}
  This is a proof by induction on the lexicorgraphic combination
$(\phi, t')$ of $>_{\Gamma,\Gamma'}$ and the strict subexpression ordering.
We case split on the structure of $t'$.  
\begin{itemize}
\item[Case.] Let $t' \equiv x$ or $t' \equiv y$ where $y$ is distinct from $x$.  Trivial. 
  
\item[Case.] Let $t' \equiv \lambda x:\phi_1.t''$.  Then $[t/x]^\phi t' \equiv \lambda x:\phi_1.[t/x]^\phi t''$.
  Now 
  \begin{center}
    \begin{math}
      \begin{array}{lll}
        rset(\lambda x:\phi_1.t'', t) & = & rset(\lambda x:\phi_1.t'') \cup rset(t)\\
        & = & rset(t'') \cup rset(t)\\
        & = & rset(t'', t).\\
      \end{array}
    \end{math}
  \end{center}
  We know that $t' > t''$ by the strict subexpression ordering, hence by the induction hypothesis
  $|rset(t'', t)| \geq |rset([t/x]^\phi t'')|$ which implies $|rset(t', t)| \geq |rset([t/x]^\phi t')|$.
  
\item[Case.] Let $t' \equiv \Lambda X:*_l.t''$.  Similar to the previous case.
  
\item[Case.] Let $t' \equiv inl(t'')$. We know $rset(t', t) = rset(t'', t)$.  Since $t' > t''$ we can apply
  the induction hypothesis to obtain $|rset(t'', t)| \geq |rset([t/x]^\phi t'')|$.  This implies
  $|rset(t', t)| \geq |rset([t/x]^\phi t')|$.
  
\item[Case.] Let $t' \equiv inr(t'')$. Similar to the previous case.  
  
\item[Case.] Let $t' \equiv t'_1\ t'_2$.  First consider when $t_1'$ is not a $\lambda$-abstraction or a case construct. Then
  \begin{center}
    $rset(t'_1\ t'_2, t) = rset(t'_1, t'_2, t)$
  \end{center}  
  Clearly,  $t' > t'_i$ for $i \in \{1,2\}$, hence, by the induction hypothesis $|rset(t'_i,t)| \geq |rset([t/x]^\phi t'_i)|$.  
  We have three cases to consider.  That is whether or not $[t/x]^\phi t'_1$ is a $\lambda$-abstraction, 
  a case construct, or neither, or $ctype_\phi(x,t'_1)$ is undefined.  
  Suppose it is a $\lambda$-abstraction.
  Then by Lemma~\ref{lemma:ctype_props_ssfp} $ctype_\phi(x.t'_1) = \psi$ and by inversion on $\Gamma,x:\phi,\Gamma' \vdash t'_1\ t'_2:\phi'$
  there exists a type $\phi''$ such that $\Gamma,x:\phi,\Gamma' \vdash t_1:\phi'' \to \phi'$.  Again, by Lemma~\ref{lemma:ctype_props_ssfp}
  $\psi \equiv \phi'' \to \phi'$. Thus, $ctype_\phi(x,t'_1) = \phi'' \to \phi'$ and $\phi'' \to \phi'$ is a subexpression of $\phi$.
  So by the definition of the hereditary substitution function $[t/x]^\phi t'_1\ t'_2 = [([t/x]^\phi t'_2)/y]^{\phi''} t''_1$, where
  $[t/x]^\phi t'_1 = \lambda y:\phi''.t''_1$.  Hence,
  \begin{center}
    \begin{math}
      |rset([t/x]^\phi t'_1\ t'_2)| = |rset([([t/x]^\phi t'_2)/y]^{\phi''} t''_1)|.
    \end{math}
  \end{center}
  Now $\phi >_{\Gamma,\Gamma'} \phi''$ so by the induction hypothesis 
  \begin{center}
    \begin{math}
      \begin{array}{lll}
        |rset([([t/x]^\phi t'_2)/y]^{\phi''} t''_1)| & \leq & |rset([t/x]^\phi t'_2, t''_1)|\\
        & \leq & |rset(t'_2, t''_1, t)|\\
        & = & |rset(t'_2, [t/x]^\phi t'_1, t)|\\
        & \leq & |rset(t'_2, t'_1, t)|\\
        & = & |rset(t'_1, t'_2, t)|.\\
      \end{array}
    \end{math}
  \end{center}
  
  \ \\
  \noindent
  Suppose $[t/x]^\phi t'_1 = \ccon{t''_0}{y}{t''_1}{t''_2}$. Then 
  \begin{center}
    \begin{math}
      \begin{array}{lll}
        |rset([t/x]^\phi (t'_1\ t'_2))| & = & |rset(\ccon{t''_0}{y}{(app_\phi\ t''_1\ [t/x]^\phi t'_2)}{(app_\phi\ t''_2\ [t/x]^\phi t'_2)})|\\
        & = & |rset(t''_0, (app_\phi\ t''_1\ [t/x]^\phi t'_2), (app_\phi\ t''_2\ [t/x]^\phi t'_2))|.
      \end{array}
    \end{math}
  \end{center}
  We know $t' > t'_1$ and $t' > t'_2$ so by the induction hypothesis
  \begin{center}
    \begin{math}
      \begin{array}{lll}
        |rset([t/x]^\phi t'_1)| & =    & |rset(t''_0, t''_1,t''_2)|\\
        & \leq & |rset(t'_1,t)|
      \end{array}
    \end{math}
  \end{center}
  and
  \begin{center}
    \begin{math}
      |rset([t/x]^\phi t'_2)| \leq |rset(t'_2,t)|.
    \end{math}
  \end{center}
  By inversion on $\Gamma,x:\phi,\Gamma' \vdash t_1\ t_2:\phi'$ there exists a type $\phi''$ such that
  $\Gamma,x:\phi,\Gamma' \vdash t'_1:\phi'' \to \phi'$.  So by Lemma~\ref{lemma:ctype_props_ssfp},
  $ctype_\phi(x,t'_1) = \psi$, $\psi \equiv \phi'' \to \phi'$, and $\psi$ is a subexpression of $\phi$.  Hence,
  $\phi >_{\Gamma,\Gamma'} \phi''$ and $\phi >_{\Gamma,\Gamma'} \phi'$.  At this point we must show the following
  proposition.
  \ \\
  {\bf Proposition.}  For all $\Gamma \vdash t_1:\phi'' \to \phi'$ and $\Gamma \vdash t_2:\phi''$ we have 
  $|rset(app_\phi\ t_1\ t_2)| \leq |rset(t_1,t_2)|$.
  
  \ \\
  We prove this by nested induction on the ordering $(\phi, t', t_1)$ and case split on the structure of $t_1$. 
  
  \begin{itemize}
  \item[Case.] Suppose $t_1$ is not a $\lambda$-abstraction or a case construct.  Then
    $app_\phi\ t_1\ t_2 = t_1\ t_2$ and $|rset(t_1\ t_2)| = |rset(t_1,t_2)|$.  Thus,
    $|rset(app_\phi\ t_1\ t_2)| \leq |rset(t_1,t_2)|$.  
    
  \item[Case.] Suppose $t_1 \equiv \lambda y:\phi''.t''_1$.  Then $app_\phi\ t_1\ t_2 = [t_2/y]^{\phi''} t''_1$.
    By inversion on the assumption $\Gamma \vdash t_1:\phi'' \to \phi'$ we know $\Gamma,y:\phi'' \vdash t''_1:\phi'$.  
    We know $\phi >_{\Gamma,\Gamma'} \phi''$ so by the outter induction hypothesis 
    \begin{center}
      \begin{math}
        \begin{array}{lll}
          |rset([t_2/y]^{\phi''} t''_1)| & \leq & |rset(t''_1,t_2)\\
          & =    & |rset(t_1,t_2).
        \end{array}
      \end{math}
    \end{center}
    Thus, $|rset(app_\phi\ t_1\ t_2) \leq |rset(t_1,t_2)|$.
    
  \item[Case.] Suppose $t_1 \equiv \ccon{t''_0}{y}{t''_1}{t''_2}$.  Then 
    \begin{center}
      \begin{math}
        app_\phi\ t_1\ t_2 = \ccon{t''_0}{y}{app_\phi\ t''_1\ t_2}{app_\phi\ t''_2\ t_2}.
      \end{math}
    \end{center}
    By inversion on the assumption $\Gamma \vdash t_1:\phi'' \to \phi'$ we know 
    $\Gamma,y:\phi''_1 \vdash t''_1:\phi'' \to \phi'$ and $\Gamma,y:\phi''_2 \vdash t''_2:\phi'' \to \phi'$.
    Since $t_1 > t''_1$ and $t_1 > t''_2$ we can apply the induction hypothesis to obtain
    \begin{center}
      \begin{math}
        |rset(app_\phi\ t''_1\ t_2)| \leq |rset(t''_1,t_2)|
      \end{math}
    \end{center}
    and
    \begin{center}
      \begin{math}
        |rset(app_\phi\ t''_2\ t_2)| \leq |rset(t''_2,t_2)|.
      \end{math}
    \end{center}
    Suppose $t''_0$ is not an inject-left or an inject-right term.  Then
    \begin{center}
      \begin{math}
        \begin{array}{lll}
          |rset(t_1,t_2)| & = & |rset(\ccon{t''_0}{y}{t''_1}{t''_2}, t_2)|\\
          & = & |rset(t''_0,t''_1,t''_2,t_2)|
        \end{array}
      \end{math}
    \end{center}
    and
    \begin{center}
      \begin{math}
        \begin{array}{lll}
          |rset(app_\phi\ t_1\ t_2)| & = & |rset(\ccon{t''_0}{y}{app_\phi\ t''_1\ t_2}{app_\phi\ t''_2\ t_2})|\\
          & = & |rset(t''_0) \cup rset(app_\phi\ t''_1\ t_2) \cup rset(app_\phi\ t''_2\ t_2)|\\
          & \leq & |rset(t''_0) \cup rset(t''_1, t_2) \cup rset(t''_2, t_2)|\\
          & = & |rset(t''_0) \cup rset(t''_1, t''_2, t_2)|\\
          & = & |rset(t''_0, t''_1, t''_2, t_2)|.
        \end{array}
      \end{math}
    \end{center}
    Therefore, $|rset(app_\phi\ t_1\ t_2)| \leq |rset(t_1,t_2)|$.

    \ \\
    Now suppose $t''_0 \equiv inl(\hat{t}_0)$.  We only show the case when $t''_0$ is an
    inject-left term, because the case when it is an inject-right term is similar.  By
    definition we know 
    \begin{center}
      \begin{math}
        \begin{array}{lll}
          |rset(t_1,t_2)| & = & |rset(\ccon{t''_0}{y}{t''_1}{t''_2}, t_2)|\\
          & = & |\{\ccon{t''_0}{y}{t''_1}{t''_2}\} \cup rset(t''_0,t''_1,t''_2,t_2)|
        \end{array}
      \end{math}
    \end{center}
    and 
    \begin{center}
      \begin{math}
        \begin{array}{lll}
          |rset(app_\phi\ t_1\ t_2)| & = & |rset(\ccon{t''_0}{y}{app_\phi\ t''_1\ t_2}{app_\phi\ t''_2\ t_2})|\\
          & = & |\{\ccon{t''_0}{y}{app_\phi\ t''_1\ t_2}{app_\phi\ t''_2\ t_2}\} \cup 
          rset(t''_0) \cup rset(app_\phi\ t''_1\ t_2) \cup rset(app_\phi\ t''_2\ t_2)|\\
          & \leq & |\{\ccon{t''_0}{y}{app_\phi\ t''_1\ t_2}{app_\phi\ t''_2\ t_2}\} \cup 
          rset(t''_0) \cup rset(t''_1, t_2) \cup rset(t''_2, t_2)|\\
          & = & |\{\ccon{t''_0}{y}{app_\phi\ t''_1\ t_2}{app_\phi\ t''_2\ t_2}\} \cup 
          rset(t''_0) \cup rset(t''_1, t''_2, t_2)|\\
          & = & |\{\ccon{t''_0}{y}{app_\phi\ t''_1\ t_2}{app_\phi\ t''_2\ t_2}\} \cup 
          rset(t''_0, t''_1, t''_2, t_2)|.
        \end{array}
      \end{math}
    \end{center}
    Therefore, $|rset(app_\phi\ t_1\ t_2)| \leq |rset(t_1,t_2)|$.
  \end{itemize}
  % end proposition.

  \ \\
  \noindent
  Suppose $[t/x]^\phi t'_1$ is not a $\lambda$-abstractions or a case construct, or $ctype_\phi(x,t'_1)$ is undefined.  Then
  \begin{center}
    \begin{math}
      \begin{array}{lll}
        rset([t/x]^\phi (t'_1\ t'_2)) & = & rset([t/x]^\phi t'_1\ [t/x]^\phi t'_2)\\
        & = & rset([t/x]^\phi t'_1, [t/x]^\phi t'_2).\\
        & \leq & rset(t'_1, t'_2, t).\\
      \end{array}
    \end{math}
  \end{center}
  
  \ \\
  Next suppose $t'_1 \equiv \lambda y:\phi_1.t''_1$.  Then 
  \begin{center}
    \begin{math}
      \begin{array}{lll}
        rset((\lambda y:\phi_1.t''_1)\ t'_2, t) & = & \{ (\lambda y:\phi_1.t''_1)\ t'_2\} \cup rset(t''_1, t'_2, t).
      \end{array}
    \end{math}
  \end{center}
  By the definition of the hereditary substitution function,
  \begin{center}
    \begin{math}
      \begin{array}{lll}
        rset([t/x]^\phi (\lambda y:\phi_1.t''_1)\ t'_2) & = & rset([t/x]^\phi (\lambda y:\phi_1.t''_1)\ [t/x]^\phi t'_2)\\
        & = & rset((\lambda y:\phi_1.[t/x]^\phi t''_1)\ [t/x]^\phi t'_2)\\
        & = & \{(\lambda y:\phi_1.[t/x]^\phi t''_1)\ [t/x]^\phi t'_2\} \cup 
        rset([t/x]^\phi t''_1) \cup rset([t/x]^\phi t'_2).\\
      \end{array}
    \end{math}
  \end{center}
  Since $t' > t''_1$ and $t' > t'_2$ we can apply the induction hypothesis to obtain,
  $|rset(t''_1, t)| \geq |rset([t/x]^\phi t''_1)|$ and $|rset(t'_2,t)| \geq |rset([t/x]^\phi t'_2)|$.  Therefore, \\
  $|\{ (\lambda y:\phi_1.t''_1)\ t'_2\}\ \cup\ rset(t''_1,t)\ \cup\ rset(t'_2,t)| \geq $ 
  $|\{(\lambda y:\phi_1.[t/x]^\phi t''_1)\ [t/x]^\phi t'_2\}\ \cup\ rset([t/x]^\phi t''_1)\ \cup\ rset([t/x]^\phi t'_2)|$.

  \ \\
  Finally, suppose $t'_1 \equiv \ccon{t''_0}{y}{t''_1}{t''_2}$.  Then
  \begin{center}
    \begin{math}
      \begin{array}{lll}
        |rset([t/x]^\phi (t'_1\ t'_2))| & = & |rset((\ccon{[t/x]^\phi t''_0}{y}{[t/x]^\phi t''_1}{[t/x]^\phi t''_2})\ [t/x]^\phi t'_2)|\\
        & = & |\{[t/x]^\phi (t'_1\ t'_2)\} \cup rset([t/x]^\phi t''_0,[t/x]^\phi t''_1,[t/x]^\phi t''_2,[t/x]^\phi t'_2)|.
      \end{array}
    \end{math}
  \end{center}
  Now $t' > t''_0$, $t' > t''_1$, $t' > t''_2$, and $t' > t'_2$ so by the induction hypothesis
  \begin{center}
    \begin{math}
      \begin{array}{lll}
        |rset([t/x]^\phi t''_0| \leq |rset(t''_0, t)|,\\
        |rset([t/x]^\phi t''_1| \leq |rset(t''_1, t)|,\\
        |rset([t/x]^\phi t''_2| \leq |rset(t''_2, t)|, \text{ and }\\
        |rset([t/x]^\phi t'_2| \leq |rset(t'_2, t)|.\\
      \end{array}
    \end{math}
  \end{center}
  Hence, 
  \begin{center}
    \begin{math}
      |rset([t/x]^\phi t''_0,[t/x]^\phi t''_1,[t/x]^\phi t''_2,[t/x]^\phi t'_2)| \leq |rset(t''_0, t''_1, t''_2, t'_2, t)|.
    \end{math}
  \end{center}
  Now
  \begin{center}
    \begin{math}
      \begin{array}{lll}
        |rset(t'_1\ t'_2,t)| & = & |rset((\ccon{t''_0}{y}{t''_1}{t''_2})\ t'_2,t)|\\
        & = & |\{t'_1\ t'_2\} \cup rset(t''_0, t''_1, t''_2, t'_2, t)|.
      \end{array}
    \end{math}
  \end{center}
  Therefore, $|rset([t/x]^\phi (t'_1\ t'_2))| \leq |rset(t'_1\ t'_2,t)|$.
  
\item[Case.] 
  Let $t' \equiv \ccon{t'_0}{y}{t'_1}{t'_2}$.  Suppose $t'_0$ is not an inject-left term, and inject-right term,
  or a case-construct.  First we know 
  \begin{center}
    \begin{math}
      |rset(t', t)| = |rset(t'_0, t'_1, t'_2, t)|.
    \end{math}
  \end{center}
  Now we have several cases to consider, when $[t/x]^\phi t'_0$ is an inject-left term,
  an inject-right term, a case construct, something else entirely, or $ctype_\phi(x,t'_0)$ is undefined. Suppose it is something else entirely or
  $ctype_\phi(x,t'_0)$ is undefined.
  Then 
  \begin{center}
    \begin{math}
      \begin{array}{lll}
        |rset([t/x]^\phi t')| & = & |rset(\ccon{[t/x]^\phi t'_0}{y}{([t/x]^\phi t'_1)}{([t/x]^\phi t'_2)})|\\
        & = & |rset([t/x]^\phi t'_0,([t/x]^\phi t'_1),([t/x]^\phi t'_2))|.
      \end{array}
    \end{math}
  \end{center}
  We can see that $t' > t'_0$, $t' > t'_1$, $t' > t'_2$ so by the induction hyothesis
  \begin{center}
    \begin{math}
      \begin{array}{lll}
        |rset([t/x]^\phi t'_0| \leq |rset(t'_0, t)|,\\
        |rset([t/x]^\phi t'_1| \leq |rset(t'_1, t)|, \text{ and }\\
        |rset([t/x]^\phi t'_2| \leq |rset(t'_2, t)|.
      \end{array}
    \end{math}
  \end{center}
  This implies that $|rset([t/x]^\phi t'_0,([t/x]^\phi t'_1),([t/x]^\phi t'_2))| \leq |rset(t'_0,t'_1,t'_2,t)|$.
  Therefore, $|rset(t', t)| \geq |rset([t/x]^\phi t')$.
  
  \ \\
  Now suppose $[t/x]^\phi t'_0 \equiv inl(t''_0)$.  We only show the case for when $[t/x]^\phi t'_0$ is an inject-left
  term, because the case for when it is an inject-right term is similar.  We can see that 
  \begin{center}
    \begin{math}
      \begin{array}{lll}
        |rset(t',t)| & = & |rset(\ccon{t'_0}{y}{t'_1}{t'_2},t)|\\
        & = & |rset(t'_0,t'_1,t'_2,t)|
      \end{array}
    \end{math}
  \end{center}
  and $|rset([t/x]^\phi t')| = |rset([t''_0/y]^{\phi_1} ([t/x]^\phi t'_1))|$.    By inversion on $\Gamma,x:\phi,\Gamma' \vdash t':\phi'$ we know there exists
  types $\phi_1$ and $\phi_2$ such that $\Gamma,x:\phi,\Gamma' \vdash t_0:\phi_1+\phi_2$.
  So by Lemma~\ref{lemma:ctype_props_ssfp} there exists a type $\psi$ such that $ctype_\phi (x,t'_0) = \psi$, $\psi \equiv \phi_1+\phi_2$, and $\psi$ is a 
  subexpression of $\phi$.  Thus, $\phi >_{\Gamma,\Gamma'} \phi_1$, $\phi >_{\Gamma,\Gamma'} \phi_2$, $t > t'_0$, and $t > t'_1$ so we can apply the induction hypothesis
  to obtain 
  \begin{center}
    \begin{math}
      \begin{array}{lll}
        |rset([t''_0/y]^{\phi_1} ([t/x]^\phi t'_1))| & \leq & |rset(t''_0,[t/x]^\phi t'_1)|\\
        & =    & |rset([t/x]^\phi t'_0,[t/x]^\phi t'_1)|\\
        & \leq & |rset(t'_0, t'_1, t)|\\
        & \leq & |rset(t'_0, t'_1, t'_2, t)|.\\
      \end{array}
    \end{math}
  \end{center}
    
  \ \\
  Next suppose $[t/x]^\phi t'_0 \equiv \ccon{t''_0}{y}{t''_1}{t''_2}$.  Then 
  \begin{center}
    \begin{math}
      |rset([t/x]^\phi t')| = |rset(rcase_{\phi}\ [t/x]^\phi t'_0\ y\ [t/x]^\phi t'_1\ [t/x]^\phi t'_2)|
    \end{math}
  \end{center}
  and
  \begin{center}
    \begin{math}
      \begin{array}{lll}
        |rset(t', t)| & = & |rset(\ccon{t'_0}{y}{t'_1}{t'_2}, t)|\\
        & = & |rset(t'_0,t_1,t'_2,t)|.
      \end{array}
    \end{math}
  \end{center}
  Note that by inverision on $\Gamma,x:\phi,\Gamma' \vdash t':\phi'$ we know there exists
  types $\phi_1$ and $\phi_2$ such that $\Gamma,x:\phi,\Gamma' \vdash t'_0:\phi_1+\phi_2$.
  So by Lemma~\ref{lemma:ctype_props_ssfp} there exists a type $\psi$ such that $ctype_\phi (x,t'_0) = \psi$, $\psi \equiv \phi_1+\phi_2$, and $\psi$ is a 
  subexpression of $\phi$. Thus, $\phi >_{\Gamma,\Gamma'} \phi_1$ and $\phi >_{\Gamma,\Gamma'} \phi_2$.
  It suffices to show that $|rset(rcase_\phi\ [t/x]^\phi t'_0\ [t/x]^\phi t'_1\ [t/x]^\phi t'_2)| \leq |rset([t/x]^\phi t'_0, [t/x]^\phi t'_1, [t/x]^\phi t'_2)|$ which
  is a consequence of the following proposition.
    
  \ \\
  {\bf Proposition.}  For all $\Gamma \vdash t:\phi_1+\phi_2$, $\Gamma, y:\phi_1 \vdash t'_1:\phi'$, and 
  $\Gamma, y:\phi_2 \vdash t'_2:\phi'$ we have $|rset(rcase_{\phi}\ t\ y\ t'_1\ t'_2)| \leq |rset(t, t'_1, t'_2)|$.
    
  \ \\
  We prove this proposition by nested induction on $(\phi, t', t)$ and we case split on $t$.
  \begin{itemize}
  \item[Case.] Suppose $t \equiv inl(t')$.  Then
    \begin{center}
      \begin{math}
        \begin{array}{lll}
          rset(rcase_{\phi}\ t\ y\ t'_1\ t'_2) = rset([t'/y]^{\phi_1} t'_1).
        \end{array}
      \end{math}
    \end{center}
    By inversion on $\Gamma \vdash t:\phi_1+\phi_2$ we know $\Gamma \vdash t':\phi_1$.  So by the outer induction
    hypothesis 
    \begin{center}
      \begin{math}
        \begin{array}{lll}
          |rset([t'/y]^{\phi_1}t'_1)| & \leq & |rset(t'_1,t')|\\
          & =    & |rset(t,t'_1)|\\
          & \leq & |rset(t,t'_1,t'_2)|.
        \end{array}
      \end{math}
    \end{center}
    Therefore, $|rset(rcase_{\phi}\ t\ y\ t'_1\ t'_2)| \leq |rset(t,t'_1,t'_2)|$.

  \item[Case.] Suppose $t \equiv inr(t')$.  This case is similar to the previous case.

  \item[Case.] Suppose $t \equiv \ccon{t_0}{z}{t_1}{t_2}$.  Then
    \begin{center}
      \begin{math}
        \rcase{\phi}{t}{y}{t'_1}{t'_2} = 
        \ccon{t_0}{z}{(\rcase{\phi}{t_1}{y}{t'_1}{t'_2})}{(\rcase{\phi}{t_2}{y}{t'_1}{t'_2})}.
      \end{math}
    \end{center}
    Now $t > t'_i$ for $i \in \{0,1,2\}$. Before we can apply the inductive hypothesis we must show that $t_1$ and $t_2$ 
    are typeable.  By inversion on the assumption $\Gamma \vdash t:\phi_1+\phi_2$ we know 
    $\Gamma,z:\phi'_1 \vdash t_1:\phi_1+\phi_2$ and $\Gamma,z:\phi'_2 \vdash t_2:\phi_1+\phi_2$.  So by the inner induction 
    hypothesis $|rset(\rcase{\phi}{t_i}{y}{t'_1}{t'_2})| \leq |rset(t_i,t'_1,t'_2)|$.
        
    \ \\
    We have two cases to consider either $t_0$ is not an inject-left term or an inject-right term, or it is.
    If not then
    \begin{center}
      \begin{math}
        rset(\rcase{\phi}{t}{y}{t'_1}{t'_2}) =
        rset(t_0,\rcase{\phi}{t_1}{y}{t'_2}{t'_2},\rcase{\phi}{t_2}{y}{t'_2}{t'_2}) 
      \end{math}
    \end{center}
    otherwise 
    \begin{center}
      \begin{math}
        \begin{array}{lll}
          rset(\rcase{\phi}{t}{y}{t'_1}{t'_2}) \\
          \text{\ \ \ \ } = \{\rcase{\phi}{t}{y}{t'_1}{t'_2}\} \cup
          rset(t_0,\rcase{\phi}{t_1}{y}{t'_1}{t'_2},\rcase{\phi}{t_2}{y}{t'_1}{t'_2}).
        \end{array}
      \end{math}
    \end{center}
    Suppose $t_0$ is not an inject-left or an inject-right term.  Then
    \begin{center}
      \begin{math}
        |rset(t, t'_1, t'_2)| = |rset(t_0, t_1, t_2, t'_1, t'_2)|.
      \end{math}
    \end{center}
    Now we know 
    \begin{center}
      \begin{math}
        \begin{array}{ll}
          |rset(t_0,\rcase{\phi}{t_1}{y}{t'_2}{t'_2},\rcase{\phi}{t_2}{y}{t'_2}{t'_2})|\\
          \text{\ \ \ } = 
          |rset(t_0) \cup rset(\rcase{\phi}{t_1}{y}{t'_2}{t'_2}) \cup rset(\rcase{\phi}{t_2}{y}{t'_2}{t'_2})|\\
          \text{\ \ \ } \leq  |rset(t_0) \cup rset(t_1,t'_2,t'_2) \cup rset(t_1,t'_2,t'_2)| \\
          \text{\ \ \ } =   |rset(t_0, t_1,t_2,t'_2,t'_2)|.
        \end{array}
      \end{math}
    \end{center}
    Therefore, $|rset(\rcase{\phi}{t}{y}{t'_1}{t'_2})| \leq |rset(t, t'_1, t'_2)|$.
    
    \ \\
    Now suppose $t_0 \equiv inl(t'_0)$.  Then
    \begin{center}
      \begin{math}
        |rset(t, t'_1, t'_2)| = |\{t\} \cup rset(t_0, t_1, t_2, t'_1, t'_2)|.
      \end{math}
    \end{center}
    It suffices to show that 
    \begin{center}
      \begin{math}
        \begin{array}{lll}
          |\{\rcase{\phi}{t}{y}{t'_1}{t'_2}\} \cup
          rset(t_0,\rcase{\phi}{t_1}{y}{t'_1}{t'_2},\rcase{\phi}{t_2}{y}{t'_1}{t'_2})|\\
          \text{\ \ \ \ } \leq |\{t\} \cup rset(t_0, t_1, t_2, t'_1, t'_2)|.
        \end{array}
      \end{math}
    \end{center}
    Let $A = \rcase{\phi}{t_1}{y}{t'_1}{t'_2}$ and $B = \rcase{\phi}{t_2}{y}{t'_1}{t'_2}$. Since 
    $|rset(\rcase{\phi}{t_i}{y}{t'_1}{t'_2})| \leq |rset(t_i,t'_1,t'_2)|$ we obtain the following:
    \begin{center}
      \begin{math}
        \begin{array}{lll}
          |\{\rcase{\phi}{t}{y}{t'_1}{t'_2}\} \cup rset(t_0,A,B)|\\
          \text{\ \ \ \ } = |\{\rcase{\phi}{t}{y}{t'_1}{t'_2}\}| + |rset(t_0)| + |rset(A)| + |rset(B)|\\
          \text{\ \ \ \ } \leq |\{\rcase{\phi}{t}{y}{t'_1}{t'_2}\}| +
          |rset(t_0)| + |rset(t_1,t'_1,t'_2)| + |rset(t_2,t'_1,t'_2)|\\
          \text{\ \ \ \ } = |\{\rcase{\phi}{t}{y}{t'_1}{t'_2}\}| + |rset(t_0, t_1,t_2,t'_1,t'_2)|\\
          \text{\ \ \ \ } = |\{t\} \cup rset(t_0, t_1, t_2, t'_1, t'_2)|.\\
        \end{array}
      \end{math}
    \end{center}
    The case when $t_0$ is an inject-right term is similar to the case when it is an inject-left term.
  \end{itemize}
  Now by the previous proposition we know 
  \begin{center}
    $|rset(rcase_\phi\ [t/x]^\phi t'_0\ [t/x]^\phi t'_1\ [t/x]^\phi t'_2)| \leq |rset([t/x]^\phi t'_0, [t/x]^\phi t'_1, [t/x]^\phi t'_2)|$,
  \end{center}
  becuase by Lemma~\ref{lemma:total_ssfp} $t'_0$, $t'_1$, and $t'_2$ have the same types as $[t/x]^\phi t'_0$, $[t/x]^\phi t'_1$, and $[t/x]^\phi t'_2$.
  Now $t' > t'_0$, $t' > t'_1$, and $t' > t'_2$, so 
  \begin{center}
    \begin{math}
      \begin{array}{lll}
        |rset([t/x]^\phi t'_0)| \leq |rset(t'_0, t),\\
        |rset([t/x]^\phi t'_1)| \leq |rset(t'_1, t), \text{ and }\\
        |rset([t/x]^\phi t'_2)| \leq |rset(t'_2, t).
      \end{array}
    \end{math}
  \end{center}
  Thus, 
  \begin{center}
    \begin{math}
      \begin{array}{lll}
        |rset([t/x]^\phi t'_0, [t/x]^\phi t'_1, [t/x]^\phi t'_2)| & \leq & |rset(t'_0,t'_1,t'_2,t)|\\
        & =    & |rset(t',t)|.
      \end{array}
    \end{math}
  \end{center}
  
  \ \\
  Suppose $t'_0 \equiv inl(t''_0)$.  Again, we only show the case for when $t'_0$ is an inject-left term.
  We know 
  \begin{center}
    \begin{math}
      \begin{array}{lll}
        |rset([t/x]^\phi t')| & = & |rset(\ccon{[t/x]^\phi t'_0}{y}{[t/x]^\phi t'_1}{[t/x]^\phi t'_2})|\\
        & = & |rset(\ccon{inl([t/x]^\phi t''_0)}{y}{[t/x]^\phi t'_1}{[t/x]^\phi t'_2})|\\
        & = & |\{[t/x]^\phi t'\} \cup rset([t/x]^\phi t''_0, [t/x]^\phi t'_1, [t/x]^\phi t'_2)|
      \end{array}
    \end{math}
  \end{center}
  and
  \begin{center}
    \begin{math}
      \begin{array}{lll}
        |rset(t',t)| & = & |rset(\ccon{t'_0}{y}{t'_1}{t'_2},t)|\\
        & = & |rset(\ccon{inl(t''_0)}{y}{t'_1}{t'_2},t)|\\
        & = & |\{t'\} \cup rset(t''_0, t'_1,t'_2)|.
      \end{array}
    \end{math}
  \end{center}
  Now $t' > t''_0$, $t' > t'_1$, and $t' > t'_2$ so by the induction hypothesis
  \begin{center}
    \begin{math}
      \begin{array}{lll}
        |rset([t/x]^\phi t''_0)| & \leq & |rset(t''_0,t)|\\
        |rset([t/x]^\phi t'_1)|  & \leq & |rset(t'_1,t)|\\
        |rset([t/x]^\phi t'_2)|  & \leq & |rset(t'_2,t)|.
      \end{array}
    \end{math}
  \end{center}
  Therefore, $|rset([t/x]^\phi t''_0, [t/x]^\phi t'_1, [t/x]^\phi t'_2)| \leq |rset(t''_0, t'_1, t'_2, t)|$
  which implies that $|rset([t/x]^\phi t')| \leq |rset(t',t)|$.
  
  \ \\
  Finally, suppose $t'_0 \equiv \ccon{t''_0}{z}{t''_1}{t''_2}$.  Then 
  \begin{center}
    \begin{math}
      \begin{array}{lll}
        |rset([t/x]^\phi t')| & = & |rset(\ccon{[t/x]^\phi t'_0}{y}{[t/x]^\phi t'_1}{[t/x]^\phi t'_2})|\\
        & = & |rset(\ccon{\ccon{[t/x]^\phi t''_0}{z}{[t/x]^\phi t''_0 t''_1}{[t/x]^\phi t''_0 t''_2}}{y}{[t/x]^\phi t'_1}{[t/x]^\phi t'_2})|\\
        & = & |\{[t/x]^\phi t'\} \cup rset([t/x]^\phi t'_0, [t/x]^\phi t'_1, [t/x]^\phi t'_2)|
      \end{array}
    \end{math}
  \end{center}
  and
  \begin{center}
    \begin{math}
      \begin{array}{lll}
        |rset(t',t)| & = & |rset(\ccon{t'_0}{y}{t'_1}{t'_2},t)|\\
        & = & |rset(\ccon{\ccon{t''_0}{z}{t''_1}{t''_2}}{y}{t'_1}{t'_2},t)|\\
        & = & |\{t'\} \cup rset(t'_0, t'_1,t'_2)|.
      \end{array}
    \end{math}
  \end{center}
  Now $t' > t''_0$, $t' > t'_1$, and $t' > t'_2$ so by the induction hypothesis
  \begin{center}
    \begin{math}
      \begin{array}{lll}
        |rset([t/x]^\phi t'_0)| & \leq & |rset(t'_0,t)|,\\
        |rset([t/x]^\phi t'_1)|  & \leq & |rset(t'_1,t)|, \text{ and }\\
        |rset([t/x]^\phi t'_2)|  & \leq & |rset(t'_2,t)|.
      \end{array}
    \end{math}
  \end{center}
  Therefore, $|rset([t/x]^\phi t''_0, [t/x]^\phi t'_1, [t/x]^\phi t'_2)| \leq |rset(t''_0, t'_1, t'_2, t)|$
  which implies that $|rset([t/x]^\phi t')| \leq |rset(t',t)|$.  
\end{itemize}
\end{proof}
It is easy to see that the previous lemma can be used to show that if
the input to the hereditary substitution function contains no redexes
then the result will contain no redexes.  That is, the function is
normality preserving.  This is the key result when using the
hereditary substitution function in normalization proofs.
\begin{lemma}[Normality Preserving]
  \label{corollary:normalization_preserving_ssfp}
  If $\Gamma \vdash n:\phi$ and $\Gamma, x:\phi' \vdash n':\phi'$ then there exists 
  a normal term $n''$ such that $[n/x]^\phi n' = n''$.
\end{lemma}
\begin{proof}
  By Lemma~\ref{lemma:total_ssfp} we know there exists a term $n''$ such that $[n/x]^\phi n' = t$ and by 
Lemma~\ref{lemma:redex_preserving_ssfp} 
$|rset(n', n)| \geq |rset([n/x]^\phi n')|$.  Hence, $|rset(n', n)| \geq |rset(t)|$, but
$|rset(n', n)| = 0$.  Therefore, $|rset(t)| = 0$ which implies $n''$ has no redexes.  It suffices to show
that $n''$ has no structural redexes.  We prove this by induction on the lexicographic ordering $(\phi,n')$.
We case split on the structure of $n'$.
\begin{itemize}
\item[Case.] Suppose $n'$ is a variable $x$ or $y$ distinct from $x$.  Trivial in both cases.
  
\item[Case.] Suppose $n' \equiv \lambda y:\phi''.\hat{n'}$.  Then
  $[n/x]^\phi n' = \lambda y:\phi''.[t/x]^\phi \hat{n'}$. By inversion on the assumption  
  $\Gamma, x:\phi' \vdash n':\phi'$ we know $\Gamma, x:\phi',\Gamma',y:\phi'' \vdash \hat{n'}:\phi'$.  Since
  $n' > \hat{n}$ we can apply the induction hypothesis to obtain there exists a term $t'$ such that
  $[t/x]^\phi \hat{n} = t'$ and $t'$ has no structural redexes.  Therefore, neither does 
  $\lambda y:\phi''.[t/x]^\phi \hat{n'}$.
  
\item[Case.] Suppose $n' \equiv \Lambda X:*_l.\hat{n}$.  Similar to the previous case.
  
\item[Case.] Suppose $n' \equiv inl(n'_0)$.  Similar to the $\lambda$-abstraction case.
  
\item[Case.] Suppose $n' \equiv n'_1\ n'_2$.  By inversion we know
  $\Gamma, x:\phi, \Gamma' \vdash n'_1 : \phi'' \to \phi'$ and
  $\Gamma, x:\phi, \Gamma' \vdash n'_2 : \phi''$ for some types $\phi'$ and $\phi''$.
  Clearly, $n' > n'_i$ for $i \in \{1,2\}$.  Thus, by the induction hypothesis
  there exists normal terms $m_1$ and $m_2$ such that $[n/x]^\phi n'_i = m_i$ such that $m_i$ have no
  structural redexes.  We case split on whether or not $m_1$ is a $\lambda$-abstraction
  or a case construct and $n'_1$ is not, or $ctype_\phi(x,n'_1)$ is undefined.  
  We only consider the non-trivial cases when 
  $m_1 \equiv \lambda y:\phi''.m'_1$ or $m_1 \equiv \ccon{m'_0}{y}{m'_1}{m'_2}$ and $n'_1$
  is not a $\lambda$-abstraction or a case construct.
  Suppose the former.  
  Now by Lemma~\ref{lemma:ctype_props_ssfp} there exists a type $\psi$ such that 
  $ctype_\phi(x, n'_1) = \psi$, $\psi \equiv \phi'' \to \phi'$, and $\psi$ is a subexpression
  of $\phi$, hence $\phi >_{\Gamma,\Gamma'} \phi''$. So $[n/x]^\phi (n'_1\ n'_2) = [m_2/y]^{\phi''} m'_1$ and
  by the induction hypothesis there exists a term $m$ such that 
  $[m_2/y]^{\phi''} m'_1 = m$ and $m$ has no structural redexes..  
  
  \ \\
  Suppose $m_1 \equiv \ccon{m'_0}{y}{m'_1}{m'_2}$.
  By inversion on
  $\Gamma,\Gamma' \vdash m_1 : \phi'' \to \phi'$ we know there exists terms $\phi_1$ and $\phi_2$ such that
  $\Gamma,\Gamma' \vdash m'_0:\phi_1+\phi_2$ and
  $\Gamma,\Gamma',y:\phi_i \vdash m'_i:\phi'' \to \phi'$
  for $i \in \{1,2\}$.  Note that by Lemma~\ref{lemma:ctype_props_ssfp} there exists a type $\psi$ such that 
  $ctype_\phi(x, n'_1) = \psi$, $\psi \equiv \phi'' \to \phi'$, and $\psi$ is a subexpression
  of $\phi$, hence $\phi >_{\Gamma,\Gamma'} \phi'$ and $\phi >_{\Gamma,\Gamma'} \phi''$.  
  Now $[t/x]^\phi t' = \ccon{m'_0}{y}{app_\phi\ m'_1\ m_2}{app_\phi\ m'_2\ m_2}$.  It suffcies to show that
  there exists terms $q$ and $q'$ such that $app_\phi\ m'_1\ m_2 = q$, $app_\phi\ m'_2\ m_2 = q'$ and $q$ and $q'$ have
  no structural redexes.  To obtain this result we prove the following proposition.
  
  \ \\
  {\bf Proposition.}  For all normal terms $m_2$ and $m'_1$ such that  
  $\Gamma \vdash m_2 : \phi''$ and $\Gamma \vdash m'_1:\phi'' \to \phi'$
  there exists a term $q$ such that $app_\phi\ m'_1\ m_2 = q$ and $q$ has no structural redexes.
  
  \ \\
  We prove this by nested induction on the ordering $(\phi, n', m'_1)$ and case splitting on 
  the structure of $m'_1$.
  \begin{itemize}
  \item[Case.] Suppose $m'_1$ is neither a $\lambda$-abstraction or a case construct.  Then
    $app_\phi\  m'_1\ m_2 = m'_1\ m_2$.  Take $m'_1\ m_2$ for $q$ and we know $q$ has no structural
    redexes, because $m'_1$ and $m_2$ are normal.
    
  \item[Case.] Suppose $m'_1 \equiv \lambda z:\phi''.m''_1$.  Then $app_\phi\ m'_1\ m_2 = [m_2/z]^{\phi''} m''_1$.
    By inversion on the assumption $\Gamma \vdash m'_1:\phi'' \to \phi'$ we know 
    $\Gamma,z:\phi'' \vdash m''_1:\phi'$.  Since $\phi >_{\Gamma} \phi''$ we can apply the outter induction
    hypothesis to obtain there there exists a $q$ such that $[m_2/z]^{\phi''} m''_1 = q$ and 
    $q$ has no structural redexes.
    
  \item[Case.] Suppose $m'_1 \equiv \ccon{m''_0}{z}{m''_1}{m''_2}$.  Then\\
    $app_\phi\ m'_1\ m_2 = \ccon{m''_0}{z}{app_\phi\ m''_1\ m_2}{app_\phi\ m''_2\ m_2}$.  By inversion on the assumption
    $\Gamma \vdash m'_1:\phi''\to\phi'$ we know there exists types $\phi_1$ and $\phi_2$ such that
    $\Gamma \vdash m''_0:\phi_1+\phi_2$, $\Gamma,z:\phi_1 \vdash m''_1:\phi''\to\phi'$
    and $\Gamma,z:\phi_2 \vdash m''_2:\phi''\to\phi'$.  Since $m'_1 > m''_1$ and $m'_1 > m''_2$ we can 
    apply the inner induction hypothesis to obtain there exists terms $q'$ and $q''$ such that 
    $app_\phi\ m''_1\ m_2 = q'$, $q'$ has no structural redexes, $app_\phi\ m''_1\ m_2 = q''$ and $q''$ has no structural redexes.  
    Hence, 
    $app_\phi\ m'_1\ m_2 = \ccon{m''_0}{z}{app_\phi\ m''_1\ m_2}{app_\phi\ m''_2\ m_2} = \ccon{m''_0}{z}{q'}{q''}$ and
    $\ccon{m''_0}{z}{q'}{q''}$ has no structural redexes.  Note that $m''_0$ is normal, because $m'_1$ is
    normal.  
  \end{itemize}

  
  \ \\
  By the previous proposition there exists terms $q$ and $q'$ such that \\
  $[n/x]^\phi n' = \ccon{m'_0}{y}{app_\phi\ m'_1\ m_2}{app_\phi\ m'_2\ m_2}
  = \ccon{m'_0}{y}{q}{q'}$, where $app_\phi\ m'_1\ m_2 = q$, $app_\phi\ m'_1\ m_2 = q'$, and
  $q$ and $q'$ have no structural redexes.  Thus, $\ccon{m'_0}{y}{q}{q'}$ has no
  structural redexes.  
  
\item[Case.] Suppose $n' \equiv \ccon{m_0}{y}{m_1}{m_2}$. By inversion on the assumption
  $\Gamma,x:\phi,\Gamma' \vdash n':\phi'$ we know the following:
  \begin{center}
    \begin{math}
      \begin{array}{lll}
        \Gamma,x:\phi,\Gamma' \vdash m_0:\phi_1+\phi_2 \text{, for some types } \phi_1 \text{ and } \phi_2,\\
        \Gamma,x:\phi,\Gamma',y:\phi_1 \vdash m_1:\phi, and\\
        \Gamma,x:\phi,\Gamma',y:\phi_2 \vdash m_2:\phi.\\
      \end{array}
    \end{math}
  \end{center}
  It is easy to see that
  $n' > m_i$ for all $i \in \{0,1,2\}$.  Hence, by the induction hypothesis
  there exists terms $m'_0$, $m'_1$, and $m'_2$ such that $[t/x]^\phi m_i = m'_i$ and $m'_i$ have no structural redexes
  for all $i \in \{0,1,2\}$.  We have two cases to consider.
  \begin{itemize}
  \item[Case.] Suppose $m'_0$ is not an inject-left term, inject-right term, or case construct, or
    $m_0$ is an inject-left term, an inject-right term, or a case construct, or $ctype_\phi(x,m_0)$ is undefined.
    Then \\
    $[n/x]^{\phi} (\ccon{m_0}{y}{m_1}{m_2}) = \ccon{m'_0}{y}{m'_1}{m'_2}$ which has no structural redexes.
    
  \item[Case.] Suppose $m'_0$ is an inject-left term, inject-right term, or case construct and $m_0$ is not
    an inject-left term, an inject-right term, or a case construct.  Then\\
    $[n/x]^{\phi} (\ccon{m_0}{y}{m_1}{m_2}) = rcase_{\phi}\ m'_0\ y\ m'_1\ m'_2$, where by Lemma~\ref{lemma:ctype_props_ssfp}
    there exists a type $\psi$ such that $ctype_\phi(x,m_0) = \psi$, $\psi \equiv \phi_1+\phi_2$, and $\psi$ is a subexpression
    of $\phi$, hence $\phi >_{\Gamma,\Gamma'} \phi_1$ and $\phi >_{\Gamma,\Gamma'} \phi_2$.
    Consider the case when $m'_0 \equiv inl(m''_0)$.  Then we know by the definition of $rcase$ that
    $rcase_{\phi}\ m'_0\ y\ m'_1\ m'_2 = [m''_0/y]^{\phi_1} m'_1$.  Clearly, $\phi >_{\Gamma,\Gamma'} \phi_1$ hence
    by the the induction hypothesis there exists a term $r$ such that $[m''_0/y]^{\phi_1} m'_1 = r$ and
    $r$ has no structural redexes. Similarly for when $m'_0 \equiv inr(m''_0)$.  
    So suppose $m'_0 \equiv \ccon{m''_0}{z}{m''_1}{m''_2}$ then it suffices to show that there exists some term
    $q$ such that $rcase_{\phi}\ m'_0\ y\ m'_1\ m'_2 = q$ and $q$ has no structural redexes.  We obtain this
    result by the following proposition.
      
    \ \\
    {\bf Proposition.} For all normal terms $q$ and $q_1$ such that 
    $\Gamma \vdash q_0:\phi$, $\Gamma, y:\phi_1 \vdash q_1:\phi'$, 
    and $\Gamma, y:\phi_2 \vdash q_2:\phi'$ there exists a term $\hat{q}$ such that 
    $rcase_{\phi}\ q_0\ y\ q_1\ q_2 = \hat{q}$ and $\hat{q}$ has no structural redexes.
    We prove this by induction on the the ordering $(\phi, n', q_0)$ and case split on the structure of $q_0$.
    \begin{itemize}
    \item[Case.] Suppose $q_0$ is not an inject-left term, inject-right term, or a case construct.  Then\\
      $rcase_{\phi}\ q_0\ y\ q_1\ q_2 = \ccon{q_0}{y}{q_1}{q_2}$ which has no structural redexes.
      
    \item[Case.] Suppose $q_0 \equiv inl(q'_0)$.  Then $rcase_{\phi}\ q_0\ y\ q_1\ q_2 = [q'_0/y]^{\phi_1} q_1$ and
      by inversion on $\Gamma \vdash q_0:\phi$ we know $\Gamma \vdash q'_0:\phi_1$.  It suffices to show that there
      exists a term $\hat{q}$ such that $[q'_0/y]^{\phi_1} q_1 = \hat{q}$  and $\hat{q}$ has no structural redexes.
      Clearly, $\phi >_{\Gamma} \phi'$ so by the outer induction hypothesis there exists such a term $\hat{q}$.
      
    \item[Case.] Suppose $q_0 \equiv inl(q'_0)$.  Similar to the previous case.
      
    \item[Case.] Suppose $q_0 \equiv \ccon{q'_0}{z}{q'_1}{q'_2}$.  Then \\
      $rcase_{\phi}\ q_0\ y\ q_1\ q_2 = \ccon{q'_0}{z}{(rcase_{\phi}\ q'_1\ y\ q_1\ q_2)}{(rcase_{\phi}\ q'_2\ y\ q_1\ q_2)}$.
      We know by assumption that $\Gamma \vdash q_0:\phi$, $\Gamma \vdash q_0:\phi$, and $\Gamma, y:\phi_1 \vdash q_1:\phi'$
      so by inversion we know the following:
      \begin{center}
        \begin{math}
          \begin{array}{lll}
            (i) & \Gamma \vdash q'_0:\phi'_1 + \phi'_2 \text{, for some types } \phi'_1 \text{ and } \phi'_2,\\
            (ii) & \Gamma, z:\phi'_1 \vdash q'_1:\phi, \text{ and }\\
            (iii) & \Gamma, z:\phi'_2 \vdash q'_2:\phi.\\
          \end{array}
        \end{math}
      \end{center}
      Now $q_0 > q'_1$ and $q_0 > q'_1$ so we can apply the inner induction hypothesis twice to obtain terms $\hat{q}_1$ and
      $\hat{q}_2$ such that $rcase_{\phi}\ q'_1\ y\ q_1\ q_2 = \hat{q}_1$, $rcase_{\phi}\ q'_1\ y\ q_1\ q_2 = \hat{q}_1$ where 
      $\hat{q}_1$ and $\hat{q}_2$ have no structural redexes. So
      $\ccon{q'_0}{z}{(rcase_{\phi}\ q'_1\ y\ q_1\ q_2)}{(rcase_{\phi}\ q'_2\ y\ q_1\ q_2)} = 
      \ccon{q'_0}{z}{\hat{q}_1}{\hat{q}_2}$.  It suffices to show that $\ccon{q'_0}{z}{\hat{q}_1}{\hat{q}_2} = \hat{q}$ 
      for some normal term $\hat{q}$.  Now $q_0 > q'_0$ so we can apply the induction hypothesis
      to obtain our result, but before we can we must show that $\Gamma \vdash \ccon{q'_0}{z}{\hat{q}_1}{\hat{q}_2}:\phi'$.
      This is a direct consequence of applying the case-construct typing rule using i, $\Gamma, z:\phi'_1 \vdash \hat{q}_1:\phi'$
      and $\Gamma, z:\phi'_2 \vdash \hat{q}_2:\phi'$.  Therefore, by the inner induction hypothesis there exists a term $\hat{q}$ 
      such that $\ccon{q'_0}{z}{\hat{q}_1}{\hat{q}_2} = \hat{q}$ and $\hat{q}$ is has no structural redexes.
    \end{itemize}
  \end{itemize}
  
\item[Case.] Suppose $n' \equiv n'_1[\phi'']$.
  Since $n' > n'_1$ we can apply the induction hypothesis to
  obtain $[n'/x]^\phi n'_1$ has no structural redexes.  We case split on whether or not $[n'/x]^\phi n'_1$ is
  a type abstraction and $n'_1$ is not.  The case where it is not is trivial so we only consider
  the case where $[n'/x]^\phi n'_1 \equiv \Lambda X:*_l.s'$ for some normal term $s'$.  Then 
  $[n'/x]^\phi n'  = [\phi'/X]s'$ has no structural redexes, because $s'$ is normal.
\end{itemize}
\end{proof}
The final property of the hereditary substitution that needs to be
proven is a safety property.  Which is that the hereditary substitution
function does not deviate from the reduction rules.  We call
this soundness with respect to reduction.  We will see this property
come into play in the proof of type soundness.
\begin{lemma}[Soundness with Respect to Reduction]
  \label{lemma:soundness_reduction_ssfp}
  If $\Gamma \vdash t : \phi$ and $\Gamma, x:\phi, \Gamma' \vdash t':\phi'$ then
  $[t/x]t' \redto^* [t/x]^\phi t'$.
\end{lemma}
\begin{proof}
  This is a proof by induction on the lexicorgraphic combination
$(\phi, t')$ of $>_\Gamma$ and the strict subexpression ordering.
We case split on the structure of $t'$.  When applying
the induction hypothesis we must show that the input terms to the
substitution and the hereditary substitution functions are typeable.
We do not explicitly state typing results that are simple
conseqences of inversion.

\begin{itemize}
\item[Case.] Suppose $t'$ is a variable $x$ or $y$ distinct from $x$.  
  Trivial in both cases.
  
\item[Case.] Suppose $t' \equiv \lambda y:\phi'.\hat{t}$.  Then
  $[t/x]^\phi (\lambda y:\phi'.\hat{t}) = \lambda y:\phi'.([t/x]^\phi \hat{t})$. 
  Now $t' > \hat{t}$ so we can apply the induction hypothesis to obtain 
  $[t/x]\hat{t} \redto^* [t/x]^\phi \hat{t}$.  At this point we can see that since 
  $\lambda y:\phi'.[t/x]\hat{t} \equiv [t/x](\lambda y:\phi'.\hat{t})$ and we may
  conclude that $\lambda y:\phi'.[t/x]\hat{t} \redto^* \lambda y:\phi'.[t/x]^\phi \hat{t}$.
  
\item[Case.] Suppose $t' \equiv \Lambda X:*_l.\hat{t}$.  Similar to the previous case.
  
\item[Case.] Suppose $t' \equiv inl(t'_0)$.  Then $[t/x]^\phi t' = inl([t/x]^\phi t'_0)$.  We can
  see that $t' > t'_0$ so by the induction hypothesis $[t/x]t'_0 \redto^* [t/x]^\phi t'_0$.  Hence,
  $inl([t/x]t'_0) \redto^* inl([t/x]^\phi t'_0)$ which implies that 
  $[t/x](inl(t'_0)) \redto^* [t/x]^\phi (inl(t'_0))$.
  
\item[Case.] Suppose $t' \equiv inr(t'_0)$.  Similar to the previous case.
  
\item[Case.] Suppose $t' \equiv \ccon{t'_0}{y}{t'_1}{t'_2}$.  Clearly, $t' > t'_0$,
  $t' > t'_1$, and $t' > t'_2$, so we can apply the induction hypothesis to conclude 
  $[t/x]t'_0 \redto^* [t/x]^\phi t'_0$, $[t/x]t'_1 \join [t/x]^\phi t'_1$, and 
  $[t/x]t'_2 \redto^* [t/x]^\phi t'_2$.  We have several cases to consider, either when $[t/x]^\phi t'_0$ is an
  inject-left term or an inject-right term and $t'_0$ is not, when $[t/x]^\phi t'_0$ is a case construct
  and $t'_0$ is not, or $[t/x]^\phi t'_0$ is not an inject-left term, an inject-right term, or a case construct, or
  $ctype_\phi(x,t'_0)$ is undefined.  The cases when $[t/x]^\phi t'_0$ is not an inject-left term, an inject-right term, 
  or a case construct, or $ctype_\phi(x,t'_0)$ is undefined are trivial.
  
  \ \\
  Let's consider the case when $[t/x]^\phi t'_0$ is an inject-left term or an inject-right term and 
  $t'_0$ is not.  Since the case when $[t/x]^\phi t'_0$ is an inject-left term is similar to the case when
  it is an inject-right term we only consider the former.  Suppose $[t/x]^\phi t'_0 = inl(t''_0)$ and 
  $t'_0$ is not an inject-left term.  By Lemma~\ref{lemma:ctype_props_ssfp} there exists a type
  $\psi$ such that $ctype_\phi(x,t'_0) = \psi$, $\psi \equiv \phi_1+\phi_2$, and $\psi$ is a subexpression
  of $\phi$, where by inversion on $\Gamma,x:\phi,\Gamma' \vdash t':\phi'$ there exists types $\phi_1$ and
  $\phi_2$ such that $\Gamma,x:\phi,\Gamma' \vdash t'_0:\phi_1+\phi_2$.  Thus, $\phi >_{\Gamma,\Gamma'} \phi_1$
  and $\phi >_{\Gamma,\Gamma'} \phi_2$.  So $[t/x]^\phi t' = [t''_0/y]^{\phi_1} ([t/x]^{\phi} t'_1)$ and we know from above
  that $[t/x]t'_1 \redto^* [t/x]^{\phi} t'_1$.  Now $\phi >_{\Gamma,\Gamma'} \phi_1$, so by the induction
  hypothesis, $[t''_0/y]([t/x]^{\phi} t'_1) \redto^* [t''_0/y]^{\phi_1}([t/x]^\phi t'_1)$.  Thus,
  $[t''_0/y]([t/x] t'_1) \redto^* [t''_0/y]^{\phi_1}([t/x]^\phi t'_1)$.  It suffices to show 
  $ [t/x]t' \redto^* [t''_0/y]([t/x] t'_1)$.  We can see that 
  \begin{center}
    \begin{math}
      \begin{array}{lll}
        [t/x]t' & = & [t/x](\ccon{x}{y}{t'_1}{t'_2})\\
        & \equiv & \ccon{[t/x]x}{y}{[t/x]t'_1}{[t/x]t'_2}\\
        & \equiv & \ccon{inl(t''_0)}{y}{[t/x]t'_1}{[t/x]t'_2}\\
        & \redto & [t''_0/y]([t/x]t'_1).
      \end{array}
    \end{math}
  \end{center}
  
  \ \\
  Suppose $[t/x]t'_0 = \ccon{t''_0}{z}{t''_1}{t''_2}$ and $t'_0$ is not.  It suffices to show that 
  $[t/x]t \redto^* [t/x]^\phi t'$, which is equivalent to showing 
  $[t/x](\ccon{t'_0}{y}{t'_1}{t'_2}) \redto^* [t/x]^\phi (\ccon{t'_0}{y}{t'_1}{t'_2})$.  Now
  \begin{center}
    \begin{math}
      \begin{array}{lll}
        [t/x]^\phi (\ccon{t'_0}{y}{t'_1}{t'_2}) & = &
        \ccon{t''_0}{z}{(\rcase{\phi}{t''_1}{y}{t'_1}{t'_2})}{(\rcase{\phi}{t''_1}{y}{t'_1}{t'_2})}
      \end{array}
    \end{math}
  \end{center}
  and
  \begin{center}
    \begin{math}
      \begin{array}{lll}
        [t/x] (\ccon{t'_0}{y}{t'_1}{t'_2}) & = & \ccon{[t/x]t'_0}{y}{[t/x]t'_1}{[t/x]t'_2} \\
        & \redto^* & \ccon{(\ccon{t''_0}{z}{t''_1}{t''_2})}{y}{[t/x]t'_1}{[t/x]t'_2} \\
        & \redto & \ccon{t''_0}{z}{(\ccon{t''_1}{y}{t'_1}{t'_2})}{(\ccon{t''_2}{y}{t'_1}{t'_2})},
      \end{array}
    \end{math}
  \end{center}
  because we know from above that $[t/x]t'_0 \redto^* [t/x]^\phi t'_0$.  So it suffices to show that
  $(\ccon{t''_1}{y}{t'_1}{t'_2}) \redto^* (\rcase{\phi}{t''_1}{y}{t'_1}{t'_2})$ and
  $(\ccon{t''_2}{y}{t'_1}{t'_2}) \redto^* (\rcase{\phi}{t''_2}{y}{t'_1}{t'_2})$, because we know from above that 
  $[t/x]t_i \redto^* [t/x]^\phi t'_i$.  This is a consequence of the following proposition.  First note that 
  again by Lemma~\ref{lemma:ctype_props_ssfp} there exists a type $\psi$ such that $ctype_\phi(x,t'_0) = \psi$,
  $\psi \equiv \phi_1+\phi_2$, and $\psi$ is a subexpression of $\phi$, where by inversion on the assumption
  $\Gamma,x:\phi,\Gamma' \vdash t':\phi'$ there exists types $\phi_1$ and $\phi_2$ such that 
  $\Gamma,x:\phi,\Gamma' \vdash t'_0:\phi_1+\phi_2$.  Hence, $\phi >_{\Gamma,\Gamma'} \phi_1$ and
  $\phi >_{\Gamma,\Gamma'} \phi_2$.  
  
  \ \\
  {\bf Proposition.} For all $\Gamma \vdash t_0:\phi_1+\phi_2$, $\Gamma,y:\phi_1 \vdash t_1:\phi''$ and
  $\Gamma,y:\phi_2 \vdash t_2:\phi''$ we have 
  $(\ccon{t_0}{y}{t_1}{t_2}) \redto^* (\rcase{\phi}{t_0}{y}{t_1}{t_2})$.
  
  \ \\
  We prove this by nested induction on the ordering $(\phi,t',t_0)$ and case splitting on
  the structure of $t_0$.  
  \begin{itemize}
  \item[Case.] Suppose $t_0$ is not an inject-left term, an inject-right term, or a case construct.  Then
    \begin{center}
      $\rcase{\phi}{t_0}{y}{t_1}{t_2} = \ccon{t_0}{y}{t_1}{t_2}$.
    \end{center}
    
  \item[Case.] Suppose $t_0 \equiv inl(t'_0)$.  Then 
    \begin{center}
      $\rcase{\phi}{t_0}{y}{t_1}{t_2} = [t'_0/y]^{\phi_1} t_1$
    \end{center}
    and
    \begin{center}
      \begin{math}
        \begin{array}{lll}
          \ccon{t_0}{y}{t_1}{t_2} & \equiv & \ccon{inl(t'_0)}{y}{t_1}{t_2}\\
          & \redto & [t'_0/y]t_1.
        \end{array}
      \end{math}
    \end{center}
    Now $\phi >_{\Gamma} \phi_1$ so by the outer-induction hypothesis 
    $[t'_0/y]t_1 \redto^* [t'_0/y]^{\phi_1} t_1$.  Therefore, 
    $(\ccon{t_0}{y}{t_1}{t_2}) \redto^* (\rcase{\phi}{t_0}{y}{t_1}{t_2})$.
    
  \item[Case.] Suppose $t_0 \equiv inl(t'_0)$.  Similar to the previous case.
    
  \item[Case.] Suppose $t_0 \equiv \ccon{t'_0}{z}{t'_1}{t'_2}$.  Then 
    \begin{center}
      \begin{math}
        \rcase{\phi}{t_0}{y}{t_1}{t_2} = 
        \ccon{t'_0}{z}{(\rcase{\phi}{t'_1}{y}{t_1}{t_2})}
             {(\rcase{\phi}{t'_2}{y}{t_1}{t_2})}
      \end{math}
    \end{center}
    and
    \begin{center}
      \begin{math}
        \begin{array}{lll}
          \ccon{t_0}{y}{t_1}{t_2} & \equiv & \ccon{(\ccon{t'_0}{z}{t'_1}{t'_2})}{y}{t_1}{t_2}\\
          & \redto & \ccon{t'_0}{z}{(\ccon{t'_1}{y}{t_1}{t_2})}{(\ccon{t'_2}{y}{t_1}{t_2})}.
        \end{array}
      \end{math}
    \end{center}
    Trivially, $t_0 > t'_1$ and $t_0 > t'_2$ so by the inner-induction hypothesis 
    $(\ccon{t'_1}{y}{t_1}{t_2}) \redto^* (\rcase{\phi}{t'_1}{y}{t_1}{t_2})$ and 
    $(\ccon{t'_2}{y}{t_1}{t_2}) \redto^* (\rcase{\phi}{t'_2}{y}{t_1}{t_2})$.  Therefore,\\
    $(\ccon{t_0}{y}{t_1}{t_2}) \redto^* (\rcase{\phi}{t_0}{y}{t_1}{t_2})$.
  \end{itemize}
  
\item[Case.] Suppose $t' \equiv t'_1\ t'_2$.  By Lemma~\ref{lemma:total_ssfp}
  there exists terms $\hat{t}'_1$ and $\hat{t}'_2$
  such that $[t/x]^\phi t'_1 = \hat{t}'_1$ and $[t/x]^\phi t'_2 = \hat{t}'_2$.  Since
  $t' > t'_1$ and $t' > t'_2$ we can apply the induction hypothesis to obtain
  $[t/x]t'_1 \redto^* \hat{t}'_1$ and $[t/x]t'_2 \redto^* \hat{t}'_2$.  Now we case
  split on whether or not $\hat{t}'_1$ is a $\lambda$-abstraction and $t'_1$ is not, $\hat{t}'_1$ is a case construct and
  $t'_1$ is not, $ctype_\phi(x,t'_1)$ is undefined, or $\hat{t}'_1$ is neither a $\lambda$-abstraction or a case construct.  If
  $ctype_\phi(x,t'_1)$ is undefined or $\hat{t}'_1$ is neither a $\lambda$-abstraction or a case construct then 
  $[t/x]^\phi t' = ([t/x]^\phi t'_1)\ ([t/x]^\phi t'_2) \equiv \hat{t}'_1\ \hat{t}'_2$. Thus,
  $[t/x]t' \redto^* [t/x]^\phi t'$, because $[t/x]t' = ([t/x] t'_1)\ ([t/x] t'_2)$.  So suppose 
  $\hat{t}'_1 \equiv \lambda y:\phi'.\hat{t}''_1$ and $t'_1$ is not a $\lambda$-abstraction.  
  By Lemma~\ref{lemma:ctype_props_ssfp} there exists a type $\psi$ such that
  $ctype_\phi(x,t'_1) = \psi$, $\psi \equiv \phi'' \to \phi'$, and $\psi$ is a subexpression
  of $\phi$, where by inversion on $\Gamma,x:\phi,\Gamma' \vdash t':\phi'$ there exists a type
  $\phi''$ such that $\Gamma,x:\phi,\Gamma' \vdash t'_1:\phi'' \to \phi'$.  
  Then by the definiton of the hereditary substitution function $[t/x]^\phi (t'_1\ t'_2) = 
  [\hat{t}'_2/y]^{\phi'} \hat{t}''_1$.
  Now we know $\phi >_{\Gamma,\Gamma'} \phi'$ so 
  we can apply the induction hypothesis to obtain 
  $[\hat{t}'_2/y] \hat{t}''_1 \redto^* [\hat{t}'_2/y]^{\phi'} \hat{t}''_1$.  Now by knowing that 
  $(\lambda y:\phi'.\hat{t}''_1)\ t'_2 \redto [\hat{t}'_2/y] \hat{t}''_1$ and
  by the previous fact we know $(\lambda y:\phi'.\hat{t}''_1)\ t'_2 \redto^* [\hat{t}'_2/y]^{\phi'} \hat{t}''_1$.
  We now make use of the well known result of full $\beta$-reduction.  The
  result is stated as
  \begin{center}
    \begin{math}
      $$\mprset{flushleft}
      \inferrule* [right=] {
        a \redto^* a'
        \\\\
        b \redto^* b'
        \\
        a'\ b' \redto^* c
      }{a\ b \redto^* c}
    \end{math}
  \end{center}
  where $a$, $a'$, $b$, $b'$, and $c$ are all terms.  We apply this
  result by instantiating $a$, $a'$, $b$, $b'$, and $c$ with
  $[t/x] t'_1$, $\hat{t}'_1$, $[t/x] t'_2$, $\hat{t}'_2$, and $[\hat{t}'_2/y]^{\phi'} \hat{t}''_1$ 
  respectively.  Therefore, $[t/x](t'_1\ t'_2) \redto^* [\hat{t}'_2/y]^{\phi'} \hat{t}''_1$.    
  
  \ \\
  Finally, suppose $\hat{t}'_1 \equiv \ccon{t_0}{y}{t_1}{t_2}$ and $t'_0$ is not a case construct.  By Lemma~\ref{lemma:ctype_props_ssfp}
  there exists a type $\psi$ such that $ctype_\phi(x,t'_1) = \psi$, $\psi \equiv \phi'' \to \phi'$ and $\psi$ is a subexpression
  of $\phi$, where by inversion on the assumption $\Gamma,x:\phi,\Gamma' \vdash t':\phi'$ there exists a type $\phi''$ such that
  $\Gamma,x:\phi,\Gamma' \vdash t'_1:\phi'' \to \phi'$.  Now 
  \begin{center}
    \begin{math}
      [t/x]^\phi (t'_1\ t'_2) = \ccon{t_0}{y}{(app_\phi\ t_1\ ([t/x]^\phi t'_2))}{(app_\phi\ t_1\ ([t/x]^\phi t'_2))}
    \end{math}
  \end{center}
  and
  \begin{center}
    \begin{math}
      [t/x](t'_1\ t'_2) = ([t/x]t'_1)([t/x]t'_2).
    \end{math}
  \end{center}
  Clearly, $t' > t'_1$ and $t' > t'_2$, so by the induction hypothesis $[t/x]t'_1 \redto^* [t/x]^\phi t'_1$ and $[t/x]t'_2 \redto^* [t/x]^\phi t'_2$.  Thus,
  \begin{center}
    \begin{math}
      \begin{array}{lll}
        ([t/x]t'_1)\ ([t/x]t'_2) & \redto^* & (\ccon{t_0}{y}{t_1}{t_2})\ ([t/x]t'_2)\\
        & \redto & \ccon{t_0}{y}{(app_\phi\ t_1 ([t/x]t'_2))}{(app_\phi\ t_2 ([t/x]t'_2))}
      \end{array}
    \end{math}
  \end{center}
  and
  \begin{center}
    \begin{math}
      ((\ccon{t_0}{y}{(app_\phi\ t_1\ ([t/x] t'_2))}{(app_\phi\ t_1\ ([t/x] t'_2))})) \redto^* (\ccon{t_0}{y}{(app_\phi\ t_1\ ([t/x]^\phi t'_2))}{(app_\phi\ t_1\ ([t/x]^\phi t'_2))}).
    \end{math}
  \end{center}
  It suffices to show that $(t_1\ ([t/x] t'_2)) \redto^* (app_\phi\ t_1\ ([t/x] t'_2))$ and 
  $(t_2\ ([t/x] t'_2)) \redto^* (app_\phi\ t_2\ ([t/x] t'_2))$.  This is a consequence of the following proposition:
    
  \ \\
  {\bf Proposition.} For all $\Gamma \vdash t_1:\phi_1 \to \phi_2$ and $\Gamma \vdash t_2:\phi_1$ we have
  $(t_1\ t_2) \redto^* (app_\phi\ t_1\ t_2)$.
  
  \ \\
  We prove this by nested induction on the ordering $(\phi, t', t_1)$ and case split on the
  structure of $t_1$.  
  \begin{itemize}
  \item[Case.] Suppose $t_1$ is not a $\lambda$-abstraction or a case construct.  Then 
    $app_\phi\ t_1\ t_2 = t_1\ t_2$.
    
  \item[Case.] Suppose $t_1 \equiv \lambda y:\phi_1.t''_1$.  Then $app_\phi\ t_1\  t_2 = [t_2/y]^{\phi_1} t''_1$.
    Clearly, $\phi >_\Gamma \phi_1$ so by the outer-induction hypothesis 
    $[t_2/y]t''_1 \redto^* [t_2/y]^{\phi_1} t''_1$.  Therefore, $(t_1\ t_2) \redto^* (app_\phi\ t_1\ t_2)$.
    
  \item[Case.] Suppose $t_1 \equiv \ccon{t'_0}{y}{t'_1}{t'_2}$.  Then 
    \begin{center}
      \begin{math}
        app_\phi\ t_1\ t_2 = \ccon{t'_0}{y}{(app_\phi\ t'_1\ t_2)}{(app_\phi\ t'_2\ t_2)}
      \end{math}
    \end{center}
    and
    \begin{center}
      \begin{math}
        \begin{array}{lll}
          (t_1\ t_2) & =      & (\ccon{t'_0}{y}{t'_1}{t'_2})\ t_2\\
          & \redto & \ccon{t'_0}{y}{(t'_1\ t_2)}{(t'_2\ t_2)}.
        \end{array}
      \end{math}
    \end{center}
    We can see that $t_1 > t'_1$ and $t_1 > t'_2$ so by the inner-induction hypothesis, $(t'_1\ t_2) \redto^* (app_\phi\ t'_1\ t_2)$ and
    $(t'_2\ t_2) \redto^* (app_\phi\ t'_2\ t_2)$.  Therefore, 
    \begin{center}
      \begin{math}
        (\ccon{t'_0}{y}{(t'_1\ t_2)}{(t'_2\ t_2)}) \redto^* (\ccon{t'_0}{y}{(app_\phi\ t'_1\ t_2)}{(app_\phi\ t'_2\ t_2)}),
      \end{math}
    \end{center}
    which implies $(t_1\ t_2) \redto^* (app_\phi\ t_1\ t_2)$.
  \end{itemize}
  
\item[Case.] Suppose $t' \equiv t'_1[\phi'']$.
  Since $t' > t'_1$ we can apply the induction hypothesis to
  obtain $[t/x] t'_1 \redto^* [t'/x]^\phi t'_1$.  We case split on whether or not $[t'/x]^\phi t'_1$ is
  a type abstraction and $t'_1$ is not.  The case where it is not is trivial so we only consider
  the case where $ [t'/x]^\phi t'_1 \equiv \Lambda X:*_l.s'$.  Then 
  $[t'/x]^\phi t'  = [\phi'/X]s'$.  Now we have $[t/x] t'_1 \redto^* [t'/x]^\phi t'_1$ and
  $[t/x](t'_1[\phi]) \equiv ([t/x]t'_1)[\phi] \redto^* ([t'/x]^\phi t'_1)[\phi] \redto [\phi/X]s'$.  Thus,
  $[t/x]t' \redto^* [t'/x]^\phi t'$.  
\end{itemize}
\end{proof}
At this point we are ready to move on to concluding normalization.
% subsection properties_of_the_hereditary_substitution_function_ssfp (end)

\subsection{Concluding Normalization}
\label{subsec:concluding_normalization_ssfp}
We now define the interpretation $\interp{\phi}_\Gamma$ of types $\phi$ in typing context 
$\Gamma$.  
\begin{definition}
  \label{def:interpretation_types_ssfp}
  The interpretation of types $\interp{\phi}$ is defined by:
  \begin{center}
    $n \in \interp{\phi}_\Gamma \iff \Gamma \vdash n:\phi$
  \end{center}
  We extend this definition to non-normal terms $t$ in the following way:
  \begin{center}
    $t \in \interp{\phi}_\Gamma \iff \exists n.t$ $\normto n \in \interp{\phi}_\Gamma$
  \end{center}
\end{definition}
We define $t \normto t'$ to be $t \redto^{*} t'$ and $t'$ is normal.

In the introduction we defined semantic inversion and asserted that it
must hold with respect to any interpretation of types used in a proof
by hereditary substitution.  It is easy to see that syntactic
inversion holds for every form of the SS$\Fp$ typing relation
trivially.  This fact yields semantic inversion by definition of the
interpretation of types.  In this paper we will freely use syntactic
and semantic inversion lemmas without explicit reference.

Before moving on to proving soundness of typing and concluding
normalization we need a basic result about the interpretation of
types: type substitution.  It is used in the proof of the type
soundness theorem (Theorem~\ref{thm:soundness_ssfp}).

\begin{lemma}[Type Substitution for the Interpretation of Types]
  If $n \in \interp{\phi'}_{\Gamma,X:*_l,\Gamma'}$ and 
  $\Gamma \vdash \phi:*_l$ then 
  $[\phi/X]n \in \interp{[\phi/X]\phi'}_{\Gamma,[\phi/X]\Gamma'}$.
  \label{lemma:type_sub_ssfp}
\end{lemma}
\begin{proof}
  This proof is by structural induction on $n$.
\begin{itemize}
\item[Case.] $n$ is a variable $y$.  Clearly, $[\phi/X]n \equiv $
  $[\phi/X]y = y \in \interp{\phi'}_{\Gamma,X:*_l,\Gamma'}$, and\\
  $(\Gamma,[\phi/X]\Gamma')(y) = [\phi/X]\phi'$. Also,
  we have $\Gamma,[\phi/X]\Gamma' \vdash [\phi/X]\phi':*_p$ for some $p$, by 
  Lemma~\ref{lemma:substitution_for_kinding_ssfp}. Hence,
  by the definition of the interpretation of types, 
  $y \in \interp{[\phi/X]\phi'}_{\Gamma,[\phi/X]\Gamma'}$.
  
\item[Case.] Let $n \equiv \lambda y:\psi.n'$.  By the definition of the
  interpretation of types $\phi' \equiv \psi \rightarrow \psi'$.  
  By the induction hypothesis 
  $[\phi/X]n' \in \interp{[\phi/X]\psi'}_{\Gamma,\Gamma',y:[\phi/X]\psi}$. 
  Again by the definition of the interpretation of types
  $\lambda y:[\phi/X]\psi.[\phi/X]n' \equiv $
  $[\phi/X](\lambda y:\psi.n') \in $
  $\interp{[\phi/X]\phi'}_{\Gamma,[\phi/X]\Gamma'}.$
  
\item[Case.]  Let $n \equiv n_1n_2$.  By the definition of the 
  interpretation of types $\phi' \equiv \psi$, 
  $n_1 \in \interp{\psi' \rightarrow \psi}_{\Gamma,X:*_q,\Gamma'}$, and
  $n_2 \in \interp{\psi'}_{\Gamma,X:*_q,\Gamma'}$.  By the induction hypothesis 
  $[\phi/X]n_1 \in $
  $\interp{[\phi/X](\psi' \rightarrow \psi)}_{\Gamma,[\phi/X]\Gamma'}$ and
  $[\phi/X]n_2 \in \interp{[\phi/X]\psi'}_{\Gamma,[\phi/X]\Gamma'}$.  Now by
  the definition of the interpretation of types 
  $([\phi/X]n_1)([\phi/X]n_2) \in \interp{[\phi/X]\psi}_{\Gamma,[\phi/X]\Gamma'}$,
  since $[\phi/X]n_1$, cannot be a $\lambda$-abstraction.
  
\item[Case.]  Let $n \equiv \Lambda Y:*_q.n'$.  By the definition of the
  interpretation of types $\phi' = \forall Y:*_q.\psi$ and 
  $n' \in \interp{\psi}_{\Gamma,X:*_l,\Gamma',Y:*_q}$.  By the induction 
  hypothesis \\
  $[\phi/X]n' \in \interp{[\phi/X]\psi}_{\Gamma,[\phi/X]\Gamma',Y:*_q}$ and by
  the definition of the interpretation of types 
  $\Lambda Y:*_q.[\phi/X]n' \in 
  \interp{\forall Y:*_q.[\phi/X]\psi}_{\Gamma,[\phi/X]\Gamma'}$ which is 
  equivalent to
  $[\phi/X](\Lambda Y:*_q.n') \in $
  $\interp{[\phi/X](\forall Y:*_q.\psi)}_{\Gamma,[\phi/X]\Gamma'}$.
  
\item[Case.]  Let $n \equiv n'[\psi]$.  By the definition of the
  interpretation of types $\phi' = [\psi/Y]\psi'$, for some $Y$, $\psi$, and 
  there exists a $q$ such that $\Gamma,X:*_l,\Gamma' \vdash \psi:*_q$, and 
  $n' \in \interp{\forall Y:*_q.\psi'}_{\Gamma,X:*_l,\Gamma'}$.  By the 
  induction hypothesis 
  $[\phi/X]n' \in \interp{[\phi/X](\forall Y:*_q.\psi')}_{\Gamma,[\phi/X]\Gamma'}$.
  Therefore, by the definition of the interpretation of types\\
  $([\phi/X]n')[\psi] \in$
  $ \interp{[\psi/Y]([\phi/X]\psi')}_{\Gamma,[\phi/X]\Gamma'}$, which is equivalent
  to \\
  $[\phi/X](n'[\psi]) \in \interp{[\phi/X]([\psi/Y]\psi')}_{\Gamma,[\phi/X]\Gamma'}$.
  
\item[Case.] Let $n \equiv inl(n')$.  By the definition of the interpretation of types,
  $\phi' \equiv \phi'_1 + \phi'_2$, for some types $\phi'_1$ and $\phi'_2$, and 
  $n' \in \interp{\phi'_1}_{\Gamma,X:*_l,\Gamma'}$.  By the induction hypothesis,
  $[\phi/X]n' \in \interp{[\phi/X]\phi'_1}_{\Gamma,[\phi/X]\Gamma'}$.  Thus, by the definition of
  the interpretation of types, 
  $inl([\phi/X]n') \equiv [\phi/X]inl(n') \in \interp{[\phi/X]\phi'}_{\Gamma,[\phi/X]\Gamma'}$.
  
\item[Case.] Let $n \equiv inr(n')$.  Similar to the inject-left case above.
  
\item[Case.] Let $n \equiv \ccon{n_0}{y}{n_1}{n_2}$.  By the definition of the 
  interpretation of types,
  $n_0 \in \interp{\phi'_1 + \phi'_2}_{\Gamma,X:\phi,\Gamma'}$, for some types $\phi'_1$ and 
  $\phi'_2$,
  $n_1 \in \interp{\phi'}_{\Gamma,X:\phi,\Gamma',y:\phi'_1}$, and 
  $n_2 \in \interp{\phi'}_{\Gamma,X:\phi,\Gamma',y:\phi'_2}$.
  By the induction hypothesis, 
  $[\phi/X]n_0 \in \interp{[\phi/X](\phi'_1 + \phi'_2)}_{\Gamma,[\phi/X]\Gamma'}$,
  $[\phi/X]n_1 \in \interp{[\phi/X]\phi'}_{\Gamma,[\phi/X]\Gamma',y:[\phi/X]\phi'_1}$, and\\ 
  $[\phi/X]n_2 \in \interp{[\phi/X]\phi'}_{\Gamma,[\phi/X]\Gamma',y:[\phi/X]\phi'_2}$.  Finally, by 
  the definition of the interpretation
  of types, 
  $case\ [\phi/X]n_0\ of\ y.[\phi/X]n_1,y.[\phi/X]n_2 \equiv$\\
  $[\phi/X](\ccon{n_0}{y}{n_1}{n_2}) 
  \in \interp{[\phi/X]\phi'}_{\Gamma,[\phi/X]\Gamma'}$.
\end{itemize}
\end{proof}
Substitution for the interpretation of types is as we have been
calling it the main substitution lemma.  It is a crucial result,
because it is needed in the proof of type soundness and it depends on
the hereditary substitution substitution function.

\begin{lemma}[Hereditary Substitution for the Interpretation of Types]
  If $n' \in \interp{\phi'}_{\Gamma,x:\phi,\Gamma'}$, $n \in \interp{\phi}_\Gamma$, then 
  $[n/x]^\phi n' \in \interp{\phi'}_{\Gamma,\Gamma'}$.
  \label{lemma:interpretation_of_types_closed_substitution_ssfp}
\end{lemma}
\begin{proof}
  By Lemma~\ref{lemma:total_ssfp} we know there exists a term $\hat{n}$ 
  such that $[n/x]^\phi n' = \hat{n}$ and $\Gamma,\Gamma' \vdash \hat{n}:\phi'$ and by 
  Lemma~\ref{corollary:normalization_preserving_ssfp} $\hat{n}$ is normal.  Therefore,
  $[n/x]^\phi n' = \hat{n} \in \interp{\phi'}_{\Gamma,\Gamma'}$.
\end{proof}
\noindent
We are now ready to present our main result.

\begin{thm}[Type Soundness]
  If $\Gamma \vdash t:\phi$ then $t \in \interp{\phi}_\Gamma$.
  \label{thm:soundness_ssfp}
\end{thm}
\begin{proof}
  This is a proof by induction on the structure of the typing derivation of $t$.

\begin{itemize}
\item[Case.]\ \\
  \begin{center}
    \begin{math}
      $$\mprset{flushleft}
      \inferrule* [right=] {
        \Gamma(x) = \phi
        \\
        \Gamma\ Ok
      }{\Gamma \vdash x:\phi}
    \end{math}
  \end{center}
  By regularity $\Gamma \vdash \phi:*_l$ for some $l$, hence $\interp{\phi}_\Gamma$ is nonempty.
  Clearly, $x \in \interp{\phi}_\Gamma$ by the definition of the interpretation of types.
  
\item[Case.]\ \\
  \begin{center}
    \begin{math}
      $$\mprset{flushleft}
      \inferrule* [right=] {
        \Gamma,x:\phi_1 \vdash t:\phi_2
      }{\Gamma \vdash \lambda x:\phi_1.t : \phi_1 \rightarrow \phi_2}
    \end{math}
  \end{center}
  By the induction hypothesis $t \in
  \interp{\phi_2}_{\Gamma,x:\phi_1}$ and by the definition of the
  interpretation of types $t \normto n \in $ 
  $\interp{\phi_2}_{\Gamma,x:\phi_1}$ and $\Gamma, x:\phi_1 \vdash
  n:\phi_2$.  Thus, by applying the $\lambda$-abstraction type-checking
  rule, $\Gamma \vdash \lambda x:\phi_1.n:\Pi x:\phi_1.\phi_2$ so 
  by the definition of the interpretation of types $\lambda x:\phi_1.n
  \in \interp{\phi_1 \rightarrow \phi_2}_\Gamma$.  Thus, according to the
  definition of the interpretation of types $\lambda x:\phi_1.t
  \normto \lambda x:\phi_1.n \in \interp{\phi_1 \rightarrow \phi_2}_\Gamma$.

\item[Case.]\ \\
  \begin{center}
    \begin{math}
      $$\mprset{flushleft}
      \inferrule* [right=] {
        \Gamma \vdash t_1 : \phi_2 \rightarrow \phi_1 
        \\
        \Gamma \vdash t_2 : \phi_2
      }{\Gamma \vdash t_1\ t_2 : \phi_1}
    \end{math}
  \end{center}
  By the induction hypothesis $t_1 \normto n_1 \in $
  $\interp{\phi_2 \rightarrow \phi_1}_\Gamma$,
  $t_2 \normto n_2 \in \interp{\phi_2}_\Gamma$, $\Gamma \vdash \phi_2 \rightarrow \phi_1:*_p$, 
  and $\Gamma \vdash \phi_2:*_q$.  Inversion on the arrow-type kind-checking rule yields, 
  $\Gamma \vdash \phi_1:*_r$, and by
  Lemma~\ref{lemma:context_weakening_for_kinding_ssfp}, 
  $\Gamma,x:\phi_2,\Gamma' \vdash \phi_1:*_r$.

  \ \\
  Now we know from above that $n_1 \in \interp{\phi_2 \rightarrow \phi_1}_\Gamma$ and
  $n_2 \in \interp{\phi_2}_\Gamma$, hence $\Gamma \vdash n_1:\phi_2 \to \phi_1$ and
  $\Gamma \vdash n_2:\phi_2$.  It suffices to show that $n_1\ n_2 \in \interp{\phi_2}_\Gamma$.
  Clearly, $n_1\ n_2 = [n_1/z](z\ n_2)$ for some variable $z \not \in FV(n_1,n_2)$.  
  Lemma~\ref{lemma:total_ssfp}, Lemma~\ref{lemma:soundness_reduction_ssfp}, 
  and Lemma~\ref{corollary:normalization_preserving_ssfp} allow us to conclude that 
  $[n_1/z](z\ n_2) \redto^* [n_1/z]^{\phi_2 \to \phi_1}(z\ n_2)$, $\Gamma \vdash [n_1/z]^{\phi_2 \to \phi_1}(z\ n_2):\phi_2$,
  and $[n_1/z]^{\phi_2 \to \phi_1}(z\ n_2)$ is normal.  Thus, 
  $t_1\ t_2 \redto^* n_1\ n_2 = [n_1/z](z\ n_2) \normto [n_1/z]^{\phi_2 \to \phi_1}(z\ n_2) \in \interp{\phi_2}_\Gamma$.
  %% We case split on whether or not $n_1$ is a $\lambda$-abstraction or a case-construct.  
  
  %% \ \\
  %% If $n_1$ is not a 
  %% $\lambda$-abstraction nor a case-construct then
  %% $n_1\ n_2 \in \interp{\phi_1}_\Gamma$.  Suppose $n_1 \equiv \lambda x:\phi_2.n'_1$.  Then
  %% $t_1\ t_2 \rightsquigarrow^{*} (\lambda x:\phi_2.n'_1)\ n_2 \rightsquigarrow [n_2/x]n'_1$.  
  %% Now by Lemma~\ref{lemma:soundness_reduction_ssfp} we know $[n_2/x]n'_1 \redto^* [n_1/x]^{\phi_2} n'_1$ and
  %% by Lemma~\ref{corollary:normalization_preserving_ssfp} $[n_1/x]^{\phi_2} n'_1$ is normal.
  %% By the definition of the interpretation of types $n'_1 \in \interp{\phi_1}_{\Gamma,x:\phi_2}$ 
  %% and by Lemma~\ref{lemma:context_weakening_interpretations_ssfp}, 
  %% $n'_1 \in \interp{\phi_1}_{\Gamma,x:\phi_2,\Gamma'}$.
  %% Therefore, by Lemma~\ref{lemma:interpretation_of_types_closed_substitution_ssfp} and
  %% the definition of the interpretation of types 
  %% $[n_2/x]n'_1 \normto [n_2/x]^{\phi_2} n'_1 \in \interp{\phi_1}_{\Gamma,\Gamma'}$.  
  
  %% \ \\
  %% Now suppose $n_1 \equiv \ccon{n'_0}{y}{n'_1}{n'_2}$.  Then
  %% \begin{center}
  %%   \begin{math}
  %%     \begin{array}{lll}
  %%       t_1\ t_2 & \redto^* & (\ccon{n'_0}{y}{n'_1}{n'_2})\ n_2\\
  %%                & =       & [\ccon{n'_0}{y}{n'_1}{n'_2}/z](z\ n_2),
  %%     \end{array}
  %%   \end{math}
  %% \end{center}
  %% where $z \not \in FV(n'_0, n'_1, n'_2) \cup \{x,y\}$.  Lemma~\ref{lemma:total_ssfp}, Lemma~\ref{lemma:soundness_reduction_ssfp}, 
  %% and Lemma~\ref{corollary:normalization_preserving_ssfp} allow us to conclude that 
  %% $[\ccon{n'_0}{y}{n'_1}{n'_2}/z](z\ n_2) \redto^* [\ccon{n'_0}{y}{n'_1}{n'_2}/z]^{\phi_2 \to \phi_1} (z\ n_2)$, 
  %% $\Gamma,\Gamma' \vdash [\ccon{n'_0}{y}{n'_1}{n'_2}/z]^{\phi_2 \to \phi_1} (z\ n_2) : \phi_1$, and $[\ccon{n'_0}{y}{n'_1}{n'_2}/z]^{\phi_2 \to \phi_1} (z\ n_2)$ is
  %% normal.  Thus, $[\ccon{n'_0}{y}{n'_1}{n'_2}/z] (z\ n_2) \normto [\ccon{n'_0}{y}{n'_1}{n'_2}/z]^{\phi_2 \to \phi_1} (z\ n_2) \in \interp{\phi_1}_{\Gamma,\Gamma'}$.  Therefore,
  %% $t_1\ t_2 \redto^* n_1\ n_2 \in \interp{\phi_1}_{\Gamma,\Gamma'}$.
  
\item[Case.]\ \\
  \begin{center}
    \begin{math}
      $$\mprset{flushleft}
      \inferrule* [right=] {
        \Gamma, X : *_p \vdash t : \phi
      }{\Gamma \vdash \Lambda X:*_p.t:\forall X:*_p.\phi}
    \end{math}
  \end{center}
  By the induction hypothesis and definition of the interpretation of types 
  $t \in \interp{\phi}_{\Gamma,X:*_p}$, $t \normto n \in \interp{\phi}_{\Gamma,X:*_p}$ and 
  $\Lambda X:*_p.n \in \interp{\phi}_{\Gamma}$.  Again, by definition of the interpretation 
  of types $\Lambda X:*_p.t \normto \Lambda X:*_p.n \in \interp{\phi}_{\Gamma}$.

\item[Case.]\ \\
  \begin{center}
    \begin{math}
      $$\mprset{flushleft}
      \inferrule* [right=] {
        \Gamma \vdash t:\forall X:*_l.\phi_1
        \\
        \Gamma \vdash \phi_2:*_l
      }{\Gamma \vdash t[\phi_2]: [\phi_2/X]\phi_1}
    \end{math}
  \end{center}
  By the induction hypothesis $t \in \interp{\forall X:*_l.\phi_1}_\Gamma$ and by the 
  definition of the interpretation of types we know 
  $t \normto n \in \interp{\forall X:*_l.\phi_1}_\Gamma$.  We case
  split on whether or not $n$ is a type abstraction. If not then again, by the 
  definition of the interpretation of types 
  $n[\phi_2] \in \interp{[\phi_2/X]\phi_1}_\Gamma$, therefore 
  $t \in \interp{[\phi_2/X]\phi_1}_\Gamma$.  Suppose $n \equiv \Lambda X:*_l.n'$.  Then 
  $t[\phi_2] \redto^* (\Lambda X:*_l.n')[\phi_2] \redto [\phi_2/X]n'$.  By the definition 
  of the interpretation of types $n' \in \interp{\phi_1}_{\Gamma,X:*_l}$. Therefore, by
  Lemma~\ref{lemma:type_sub_ssfp} $[\phi_2/X]n' \in \interp{[\phi_2/X]\phi_1}_{\Gamma}$.
  
\item[Case.]\ \\
  \begin{center}
    \begin{math}
      $$\mprset{flushleft}
      \inferrule* [right=] {
        \Gamma \vdash t:\phi_1
	\\
	\Gamma \vdash \phi_2:*_p
      }{\Gamma \vdash inl(t): \phi_1+\phi_2}
    \end{math}
  \end{center}
  By the induction hypothesis, $t \in \interp{\phi_1}_\Gamma$ and by the definition of the 
  interpretation of types,
  $t \normto n \in \interp{\phi_1}_\Gamma$, and $inl(n) \in \interp{\phi_1 + \phi_2}_{\Gamma}$.  
  Again, by the definition of the interpretation of types
  $inl(t) \normto inl(n) \in \interp{\phi_1 + \phi_2}_\Gamma$.
  
\item[Case.]\ \\
  \begin{center}
    \begin{math}
      $$\mprset{flushleft}
      \inferrule* [right=] {
        \Gamma \vdash t:\phi_2
	\\
	\Gamma \vdash \phi_1:*_p
      }{\Gamma \vdash inr(t): \phi_1+\phi_2}
    \end{math}
  \end{center}
  Similar the inject-left case above.
  
\item[Case.]\ \\
  \begin{center}
    \begin{math}
      $$\mprset{flushleft}
      \inferrule* [right=] {
        \Gamma \vdash t_0:\phi_1 + \phi_2
        \\
	\Gamma,x:\phi_1 \vdash t_1:\psi
        \\
	\Gamma,x:\phi_2 \vdash t_2:\psi
      }{\Gamma \vdash \ccon{t_0}{x}{t_1}{t_2}: \psi}
    \end{math}
  \end{center}
  By the induction hypothesis and the definition of the interpretation of types
  $t_0 \normto n_0 \in \interp{\phi_1 + \phi_2}_\Gamma$ and $\Gamma \vdash n_0:\phi_1+\phi_2$, 
  $t_1 \normto n_1 \in \interp{\psi}_{\Gamma,x:\phi_1}$ and $\Gamma,x:\phi_1 \vdash n_1:\psi$, and
  $t_2 \normto n_2 \in \interp{\psi}_{\Gamma,x:\phi_2}$ and  $\Gamma,x:\phi_2 \vdash n_2:\psi$.  
  Clearly, 
  \begin{center}
    \begin{math}
      \begin{array}{lll}
        \ccon{t_0}{x}{t_1}{t_2} & \redto^* & \ccon{n_0}{x}{n_1}{n_2}\\
                                & =        & [n_0/z](\ccon{z}{x}{n_1}{n_2}),
      \end{array}
    \end{math}
  \end{center}
  for some variable $z \not \in FV(n_0,n_1,n_2) \cup \{x\}$.  Lemma~\ref{lemma:total_ssfp}, 
  Lemma~\ref{lemma:soundness_reduction_ssfp}, and Lemma~\ref{corollary:normalization_preserving_ssfp} 
  allow us to conclude that $[n_0/z](\ccon{z}{x}{n_1}{n_2}) \redto^* [n_0/z]^{\phi_1+\phi_2}(\ccon{z}{x}{n_1}{n_2})$,
  $\Gamma \vdash [n_0/z]^{\phi_1+\phi_2}(\ccon{z}{x}{n_1}{n_2}) :\psi$, and $[n_0/z]^{\phi_1+\phi_2}(\ccon{z}{x}{n_1}{n_2})$
  is normal.  Thus, $[n_0/z]^{\phi_1+\phi_2}(\ccon{z}{x}{n_1}{n_2}) \in \interp{\psi}_\Gamma$ and we obtain 
  $\ccon{t_0}{x}{t_1}{t_2} \in \interp{\psi}_\Gamma$.
  

  %% We do a case split on whether or not 
  %% $n_0 \equiv inl(n'_0)$, 
  %% $n_0 \equiv inr(n'_0)$, or $n_0 \equiv \ccon{n'_0}{y}{n'_1}{n'_2}$.  If not, then, again 
  %% by the definition of the interpretation of types, 
  %% $\ccon{n_0}{x}{n_1}{n_2} \in \interp{\psi}_\Gamma$.  Therefore, 
  %% $\ccon{t_0}{x}{t_1}{t_2} \in \interp{\psi}_\Gamma$.  Suppose 
  %% $n_0 \equiv inl(n'_0)$.  Then $\ccon{n_0}{x}{n_1}{n_2} \equiv $
  %% $\ccon{inl(n'_0)}{x}{n_1}{n_2} \rightsquigarrow [n'_0/x]n_1$.  By
  %% Lemma~\ref{lemma:interpretation_of_types_closed_substitution_ssfp}, 
  %% $[n'_0/x]n_1 \normto \hat{n_1} \in \interp{\psi}_\Gamma$.  
  %% The case where, $n_0 \equiv inr(n'_0)$, is similar to the previous case.  Suppose 
  %% $n_0 \equiv \ccon{n'_0}{y}{n'_1}{n'_2}$.  Then\\
  %% $\ccon{n_0}{x}{n_1}{n_2} \equiv $\\
  %% $\ccon{(\ccon{n'_0}{y}{n'_1}{n'_2})}{x}{n_1}{n_2} \rightsquigarrow $\\
  %% $\ccon{n'_0}{y}{(\ccon{n'_1}{x}{n_1}{n_2})}{(\ccon{n'_2}{x}{n_1}{n_2})}$.  We know, \\
  %% $\ccon{n'_1}{x}{n_1}{n_2} \in \interp{\psi}_{\Gamma,y:\phi'_1}$,
  %% $\ccon{n'_2}{x}{n_1}{n_2} \in \interp{\psi}_{\Gamma,y:\phi'_2}$, and 
  %% $n'_2 \in \interp{\phi_1 + \phi_2}_{\Gamma}$, thus, by 
  %% Lemma~\ref{lemma:interpretations_closed_cases_ssfp},
  %% $\ccon{n'_1}{x}{n_1}{n_2} \normto \hat{n}_1 \in \interp{\psi}_{\Gamma,y:\phi'_1}$ and 
  %% $\ccon{n'_2}{x}{n_1}{n_2} \normto \hat{n}_2 \in \interp{\psi}_{\Gamma,y:\phi'_2}$.  By the 
  %% interpretation of types,\\
  %% $\ccon{n'_0}{y}{\hat{n}_1}{\hat{n}_2} \in \interp{\psi}_\Gamma$.  Therefore,\\
  %% $\ccon{n'_0}{y}{(\ccon{n'_1}{x}{n_1}{n_2})}{(\ccon{n'_2}{x}{n_1}{n_2})} \normto$\\
  %% $ \ccon{n'_0}{y}{\hat{n}_1}{\hat{n}_2} \in \interp{\psi}_\Gamma$.
\end{itemize}
\end{proof}
\begin{corollary}[Normalization]
  If $\Gamma \vdash t:\phi$ then $t \normto n$.
\end{corollary}
% subsubsection concluding_normalization_ssfp (end)
% section stratified_system_f_with_sum_types (end)