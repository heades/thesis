\input{lam-delta-ott}

\renewcommand{\LamDeltadrulename}[1]{{\normalfont\scriptsize \text{#1}}}

So far in this chapter we have introduced proofs of normalization for
several extensions of SSF, which is an intuitionistic logic.  In fact,
to the knowledge of the author the only type theories to which the
hereditary substitution proof technique has been applied to are
intuitionistic.  So a natural question is, can hereditary substitution
be used to prove normalization of a classical type
theory\index{Classical Logic}\index{Classical Type Theory}?  This would
then imply that hereditary substitution can be used to provide a
constructive proof of normalization for a non-constructive theory.  

In this section we answer this question positively.  We show
normalization using hereditary substitution for the classical type
theory the $\lambda\Delta$-calculus.  The $\lambda\Delta$-calculus was
introduced in Section~\ref{sec:the_lambda-delta-calculus}.  We do not
repeat its definition.  The reader can find the syntax and reduction
rules in Figure~\ref{fig:lamd_syntax} and the typing rules in
Figure~\ref{fig:lamd_typing}.  We define negation just as it is in
intuitionistic type theory, that is, $[[{-A}]] =^{def} [[A -> _|_]]$,
where $[[_|_]]$ is absurdity.  Arbitrary syntactically defined normal
forms will be denoted by the meta-variables $[[n]]$ and $[[m]]$, and
arbitrary typing contexts will be denoted by the meta-variable
$[[G]]$.  We assume at all times that all variables in the domain of
$[[G]]$ are unique.  In addition we rearrange the objects in $[[G]]$
freely without indication. 

\section{Basic Syntactic Lemmas}
\label{sec:basic_syntactic_lemmas}
The following meta-results are well-known so we omit their proofs.  We
do not always explicitly state the use of these results.  The first
two properties are weakening and substitution for the typing relation.

\begin{lemma}[Weakening for Typing]
  \label{lemma:weakening_for_typing}
  If $[[G |- t : T]]$ then $[[G,x:T' |- t : T]]$ for any fresh variable $[[x]]$ and
  type $[[T']]$.
\end{lemma}
\begin{proof}
  Straightforward induction on the assumed typing derivation.
\end{proof}

\begin{lemma}[Substitution for Typing]
  \label{lemma:substitution_for_typing}
  If $[[G |- t : T]]$ and $[[G,x:T,G' |- t':T']]$ then $[[G |- [t/x]t':T']]$.
\end{lemma}
\begin{proof}
  Straightforward induction on the second assumed typing derivation.
\end{proof}
\noindent
The final three properties are, confluence, type preservation and inversion of the typing
relation. The proof of the confluence and type preservation can be found in \cite{Rehof:1994}
and the proof of the latter is trivial.

\begin{thm}[Confluence]
  \label{thm:confluence}
  If $[[t1 ~*> t2]]$ and $[[t1 ~*> t3]]$, then there exists a term $[[t4]]$, such that,
  $[[t2 ~*> t4]]$ and $[[t3 ~*> t4]]$.
\end{thm}

\begin{thm}[Preservation]
  \label{thm:preservation}
  If $[[G |- t : T]]$ and $[[t]] \redto [[t']]$ then $[[G |- t' : T]]$.
\end{thm}

\begin{thm}[Inversion]
  \label{theorem:inversion}
  \begin{itemize}
  \item[i.] If $[[G |- x : T]]$ then $[[x]] \in [[G]]$.
  \item[ii.] If $[[G |- \x:T1.t:T1->T2]]$ then $[[G, x:T1 |- t:T2]]$.
  \item[iii.] If $[[G |- \m x:{-T}.t:T]]$ then $[[G, x:{-T} |- t: _|_]]$.
  \end{itemize}
\end{thm}
\begin{proof}
  This can be shown by straightforward induction on the assumed typing derivations.
\end{proof}
\noindent
At this point we have everything we need to state and prove correct
the hereditary substitution function.
% section basic_syntactic_lemmas (end)

\section{An Extension}
\label{sec:the_hereditary_substitution_function_for_the_ld-calculus}
Since the $\lambda\Delta$-calculus is an extension of
STLC\index{Simply Typed $\lambda$-Calculus}, we might expect that 
the hereditary substitution function for the $\lambda\Delta$-calculus is also an extension of the
hereditary substitution function for STLC.  In this section we show that this extension is
non-trivial by first considering the naive extension, and then discussing why it does not work.
Following this, we give the final extension and prove it correct.

\subsection{Problems with a Naive Extension}
\label{subsec:the_naive_extension}
Lets consider the definition of the hereditary substitution function
for STLC extended with two new cases. The first case for the $\Delta$-abstraction
whose definition parallels the definition for the
$\lambda$-abstraction.  The second is a new application case which
handles newly created structural redexes and is defined following the
same pattern as the case which handles $\beta$-redexes.  We use the same termination
metric we previously used. 
\begin{definition}
  \label{def:hereditary_substitution_function}
  The naive hereditary substitution function is defined as follows:
  \begin{itemize}
  \item[] $[[ [t/x] A x]] = [[t]]$
  \item[] $[[ [t/x] A y]] = [[y]]$

  \item[] $[[ [t/x] A (\ y:A'.t')]] = [[\y:A'.([t/x ] A t')]]$
  \item[] $[[ [t/x] A (\m y:A'.t')]] = [[\m y:A'.([t/x] A t')]]$

  \item[] $[[ [t/x] A (t1 t2)]] = [[([t/x] A t1) ([t/x] A t2)]]$\\
    \begin{tabular}{lll}
      & Where $[[([t/x] A t1)]]$ is not a $\lambda$-abstraction or $\Delta$-abstraction,  or both $[[([t/x] A t1)]]$ \\
      & and  $[[t1]]$ are $\lambda$-abstractions or $\Delta$-abstractions.\\
    \end{tabular}

  \item[] $[[ [t/x] A (t1 t2)]] = [[ [s'2/y] A'' s'1]]$\\
    \begin{tabular}{lll}
      & Where $[[([t/x] A t1)]] = [[\ y:A''.s'1]]$ for some $[[y]]$, $[[s'1]]$ and $[[A'']]$, \\
      & $[[ [t/x] A t2]] = [[s'2]]$, and $[[ctype A (x,t1)]] = [[A'' -> A']]$. \\
    \end{tabular}

  \item[] $[[ [t/x] A (t1 t2)]] = \Delta z:[[{-A'}]].[ [[\y:A'' -> A'.(z (y s2))]]/[[y]] ]^{[[{-(A''->A')}]]} [[s]]$\\
    \begin{tabular}{lll}
      & Where $[[([t/x] A t1)]] = [[\m y:{-(A''->A')}.s]]$ for some, $[[y]]$ $[[s]]$, 
      and $[[A'' -> A']]$, \\      
      & $[[([t/x] A t2)]] = [[s2]]$ for some $[[s2]]$, $[[ctype A (x,t1)]] = [[A'' -> A']]$, and \\
      & $[[z]]$ is completely fresh.
    \end{tabular}  
  \end{itemize}
\end{definition}

There is one glaring issue with this definition and it lies in the final case.  
We know from Lemma~\ref{lemma:ctype_props} and Lemma~\ref{lemma:ctype_props_cont} 
that $[[ctype A (x,t1)]] = [[A'' -> A']]$ 
implies that $[[A]] \geq [[A'' -> A']] < [[{-(A'' ->  A')}]]$. Thus, this 
definition is not well founded!  To fix this issue instead of naively following 
the structural reduction rule we immediately simultaneously hereditarily reduce 
all redexes created by replacing $[[y]]$ with the linear $\lambda$-abstraction
$[[\y:A'' ->A'.(z (y s2))]]$.  To accomplish this we will define mutually with
the hereditary substitution function a new function called the hereditary
structural substitution function.
% subsection the_naive_extension (end)

\subsection{A Correct Extension}
\label{subsec:the_final_extension}
In order to reduce structural redexes in the definition of the
hereditary substitution we will define by induction mutually with
the hereditary substitution function  a function called the
hereditary structural substitution function.  This function will
use the notion of a multi-substitution.  These are
given by the following grammar:
\begin{center}
  \begin{math}
    [[Th]] ::= [[.]]\,|\,[[Th,(y,z,t)]]
  \end{math}
\end{center}
We denote the hereditary structural substitution function by $[[<Th ,
A, A'>t']]$ and hereditary substitution by $[[ [t/x] A t']]$.  The
type of all the first projections of the elements of $[[Th]]$ is
$[[{-(A->A')}]]$ and the type of the second projections is $[[{-A'}]]$.
Both functions are defined by mutual induction using the metric
$([[A]],f,[[t']])$, where $f \in \{0,1\}$, in lexicographic
combination with the ordering on types, the natural number ordering,
and the strict subexpression on terms.  The meta-variable $f$ labels
each function and is equal to $0$ in the definition of the hereditary
substitution function and is equal to $1$ in the definition of the
hereditary structural substitution function.  Again, in the definitions of
the hereditary substitution and hereditary structural substitution
function it is assumed that all variables have been renamed as to
prevent variable capture.  The following is the final definition of the
hereditary substitution function for the $\lambda\Delta$-calculus.

\begin{definition}
  \label{def:hereditary_substitution_function}
  The hereditary substitution function is defined as follows:
  \begin{itemize}
  \item[] $[[<Th,A1,A2>x]] = [[\y:A1 -> A2.(z (y t))]]$\\
    \begin{tabular}{lll}
      & Where $[[(x,z,t)]] \in [[Th]]$, for some $[[z]]$ and $[[t]]$, and $[[y]]$ is fresh in $[[x]]$, $[[z]]$, and $[[t]]$.\\
      & \\
    \end{tabular}
  \item[] $[[<Th,A1,A2>x]] = x$\\
    \begin{tabular}{lll}
      & Where $[[(x,z,t)]] \not\in [[Th]]$ for any $[[z]]$ or $[[t]]$.\\
      & \\
    \end{tabular}
  \item[] $[[<Th,A1,A2>(\y:A.t)]] = [[\y:A.<Th, A1, A2>t]]$
  \item[] $[[<Th,A1,A2>(\m y:A.t)]] = [[\m y:A.<Th, A1, A2>t]]$

  \item[] $[[<Th,A1,A2>(x t')]] = [[z [t/y] A1 s]]$\\
    \begin{tabular}{lll}
      & Where $[[(x,z,t)]] \in [[Th]]$, $t' \equiv [[\y:A1.t'']]$, for some $[[y]]$ and
      $[[t'']]$, and $[[<Th,A1,A2>t'']] = [[s]]$.\\
      & \\
    \end{tabular}
  \item[] $[[<Th,A1,A2>(x t')]] = [[z (\m z2:{-A2}.s)]]$\\
    \begin{tabular}{lll}
      & Where $[[(x,z,t)]] \in [[Th]]$, $t' \equiv [[\m y:{-(A1 -> A2)}.t'']]$, for some $[[y]]$ and $[[t'']]$, and \\
      & $[[<Th,(y,z2,t),A1,A2>t'']] = [[s]]$, for some fresh $[[z2]]$.\\
      & \\
    \end{tabular}
  \item[] $[[<Th,A1,A2>(x t')]] = [[z s']]$\\
    \begin{tabular}{lll}
      & Where $[[(x,z,t)]] \in [[Th]]$, $[[t']]$ is not an abstraction, and $[[<Th,A1,A2>t']] = [[s']]$.\\
      & \\
    \end{tabular}
  \item[] $[[<Th,A1,A2>(t1 t2)]] = [[s1 s2]]$\\
    \begin{tabular}{lll}
      & Where $[[t1]]$ is either not a variable, or it is both a variable
      and $([[t1]],[[z']],[[t']]) \not \in [[Th]]$ \\
      & for any $[[t']]$ and $[[z']]$, $[[<Th,A1,A2>t1]] = [[s1]]$, and $[[<Th,A1,A2>t2]] = [[s2]]$.\\
      & \\
    \end{tabular}

  \item[] $[[ [t/x] A x]] = [[t]]$
  \item[] $[[ [t/x] A y]] = [[y]]$

  \item[] $[[ [t/x] A (\ y:A'.t')]] = [[\y:A'.([t/x ] A t')]]$
  \item[] $[[ [t/x] A (\m y:A'.t')]] = [[\m y:A'.([t/x] A t')]]$

  \item[] $[[ [t/x] A (t1 t2)]] = [[([t/x] A t1) ([t/x] A t2)]]$\\
    \begin{tabular}{lll}
      & Where $[[([t/x] A t1)]]$ is not a $\lambda$-abstraction or $\Delta$-abstraction,  or both $[[([t/x] A t1)]]$ \\
      & and  $[[t1]]$ are $\lambda$-abstractions or $\Delta$-abstractions.\\
      & \\
    \end{tabular}

  \item[] $[[ [t/x] A (t1 t2)]] = [[ [s'2/y] A'' s'1]]$\\
    \begin{tabular}{lll}
      & Where $[[([t/x] A t1)]] = [[\ y:A''.s'1]]$ for some $[[y]]$, $[[s'1]]$ and $[[A'']]$, \\
      & $[[ [t/x] A t2]] = [[s'2]]$, and $[[ctype A (x,t1)]] = [[A'' -> A']]$. \\
      & \\
    \end{tabular}

  \item[] $[[ [t/x] A (t1 t2)]] = \Delta z:[[{-A'}]].[[<(y,z,s2),A'',A'>s]]$\\
    \begin{tabular}{lll}
      & Where $[[([t/x] A t1)]] = [[\m y:{-(A''->A')}.s]]$ for some $[[y]]$ $[[s]]$, 
      and $[[A'' -> A']]$, \\      
      & $[[([t/x] A t2)]] = [[s2]]$ for some $[[s2]]$, $[[ctype A (x,t1)]] = [[A'' -> A']]$, and $[[z]]$ is fresh.\\
      & \\
    \end{tabular}  
  \end{itemize}
\end{definition}
We can see in the final case of the hereditary substitution function that the cut type has decreased.  Hence, this
case is now well founded. Lets consider an example which illustrates how our new definition operates. 
\begin{example}
  \label{ex:struct_reduction}
  Consider the terms $[[t]] \equiv [[\m f:{-(base->base)}.(f (\m f':{-(base -> base)}.(f' (\z:base.z))))]]$ and $[[t']] \equiv [[x u]]$,
  where $[[u]]$ is a free variable of type $[[base]]$.  Again, our goal is to compute $[[ [t/x] (base -> base) t']]$ using
  the definition of the hereditary substitution function in Definition~\ref{def:hereditary_substitution_function}.
  Now
  \begin{center}
    $[[ [t/x] (base->base) (x u) ]] = [[\m z1:{-base}.(z1 (\m z2:{-base}.(z2 u)))]]$,
  \end{center} 
  because 
  \begin{center}
    \begin{tabular}{llllll}
      $[[ctype (base -> base) (x,x)]] = [[(base -> base)]]$, &
      &
      $[[ [t/x] (base->base) x]] = [[t]]$,
      &
      $[[ [t/x] (base->base) u]] = [[u]]$,
    \end{tabular}
  \end{center}
  and for some fresh variable $[[z1]]$ of type $[[{-base}]]$
  \begin{center}
    \begin{math}
      \begin{array}{lll}
        [[\m z1:{-base}.<(f,z1,u),base,base>(f (\m f':{-(base->base)}.(f' (\z:base.z))))]]  & = & \\
        [[\m z1:{-base}.(z1 (\m z2:{-base}.(z2 u)))]]\\
      \end{array}
    \end{math}
  \end{center}
  where
  \begin{center}
    \begin{math}
      \begin{array}{lll}
        [[<(f,z1,u),base,base>(f (\m f':{-(base->base)}.(f' (\z:base.z))))]]  & = & \\
        [[z1 (\m z2:{-base}.<(f,z1,u),(f',z2,u),base,base>(f' (\z:base.z)))]]
      \end{array}
    \end{math}
    \end{center}
  because
  \begin{center}
    \begin{math}
      \begin{array}{lll}
        [[(f,z1,u)]] \in \langle[[(f,z1,u)]]\rangle, [[\m f':{-(base->base)}.(f' (\z:base.z))]] \equiv [[\m f':{-(base->base)}.(f' (\z:base.z))]],
      \end{array}
    \end{math}
  \end{center}
    and for some fresh variable $[[z2]]$ of type $[[{-base}]]$
    \begin{center}
    \begin{math}
      \begin{array}{lll}
        [[<(f,z1,u),(f',z2,u),base,base>(f' (\z:base.z))]]  & = & [[z2 u]]\\
      \end{array}
    \end{math}
  \end{center}
  because
  \begin{center}
    \begin{math}
      \begin{array}{llllll}
        [[(f',z2,u)]] \in \langle[[(f,z1,u),(f',z2,u)]]\rangle, &
        &
        [[\z:base.z]] \equiv [[\z:base.z]],
        &
        [[<(f,z1,u),base,base>z]] = z
      \end{array}
    \end{math}
  \end{center}
\end{example}
In the next section we prove the definition of the hereditary substitution function correct.
% subsection the_final_extension (end)

\subsection{Main Properties}
\label{subsec:properties_of_the_hereditary_substitution_function}
% There are two main ways the hereditary substitution function is used.
% It either replaces capture-avoiding substitution in ones' type theory
% or it is used in some other way.  For example, in Canonical LF
% hereditary substitution replaces capture-avoiding substitution
% \cite{Watkins:2004,Adams:2004}.  However, in \cite{Abel:2008} it is
% used only as a normalization function.  No matter how it is used there
% are three correctness results which must be proven.  These are
% totality, type preservation, and normality preservation.  There is an
% additional correctness property we feel one must prove when hereditary
% substitution is used as a normalization function.  This property is
% called soundness with respect to reduction.  It shows that hereditary
% substitution does nothing more than what capture-avoiding substitution
% and $\beta$-reduction can do.
Just as we did for the previous system we now prove the properties of
hereditary substitution.  We introduce some notation to make working
with \\ multi-substitutions a bit easier.  The sets of all first, second,
and third projections of the triples in $[[Th]]$ are denoted $[[Th
1]]$, $[[Th 2]]$, and $[[Th 3]]$ respectively.  We denote the
assumption of all elements of $[[Th]]^i$ having the type $[[T]]$ as
$[[Th]]^i : [[T]]$. This latter notation is used in typing contexts to
indicate the addition of all the variables in $[[Th]]^j$ for $j \in
\{1,2\}$ to the context with the specified type.  We denote this as
$[[G]],[[Th]]^j : [[T]],[[G']]$ for some contexts $[[G]]$ and
$[[G']]$.  The notation $[[G |- Th 3 : T]]$ is defined as for all
$[[t]] \in [[Th 3]]$ the typing judgment $[[G |- t : T]]$ holds.
Finally, we denote terms in $[[Th 3]]$ being normal as $[[norm(Th
3)]]$.

All of the following properties will depend on a few 
properties of the $[[ctype]]$ function.  They are listed in 
the following lemma.
\begin{lemma}[Properties of $[[ctype]]$]
  \label{lemma:ctype_props}
  \begin{itemize}
  \item[i.] If $[[ctype T (x,t)]] = [[T']]$ then $[[head(t)]] = [[x]]$ and $[[T']] \leq [[T]]$.
    
  \item[ii.] If $[[G,x:T,G' |- t:T']]$ and $[[ctype T (x,t)]] = [[T'']]$ then
    $[[T']] \equiv [[T'']]$.    
  \end{itemize}
\end{lemma}
\begin{proof}
  We prove part one first. This is a proof by induction on the structure of $t$.
  \vspace{-25px}
  \begin{changemargin}{10px}{5px}\noindent
  \begin{itemize}
  \item[Case.] Suppose $[[t]] \equiv [[x]]$.  Then $[[ctype T (x,x)]] = [[T]]$.  Clearly,
    $[[head(x)]] = [[x]]$ and $[[T]]$ is a subexpression of itself.
    
  \item[Case.] Suppose $[[t]] \equiv [[t1 t2]]$.  Then $[[ctype T (x,t1 t2)]] = [[T'']]$
    when $[[ctype T (x,t1)]] = [[T' -> T'']]$.  Now $[[t]] > [[t1]]$ so by the induction
    hypothesis $[[head(t1)]] = [[x]]$ and $[[T' -> T'']]$ is a subexpression of $[[T]]$.
    Therefore, $[[head(t1 t2)]] = [[x]]$ and certainly $[[T'']]$ is a subexpression of $[[T]]$.
  \end{itemize}

  We now prove part two.  This is also a proof by induction on the structure of $t$.

  \begin{itemize}
  \item[Case.] Suppose $[[t]] \equiv [[x]]$.  Then $[[ctype T (x,x)]] = [[T]]$.  Clearly,
    $[[T]] \equiv [[T]]$.
    
  \item[Case.] Suppose $[[t]] \equiv [[t1 t2]]$.  Then $[[ctype T (x,t1 t2)]] = [[T2]]$
    when $[[ctype T (x,t1)]] = [[T1 -> T2]]$.  By inversion on the assumed typing
    derivation we know there exists type $[[T'']]$ such that $[[G,x:T,G' |- t1:T'' -> T']]$.
    Now $[[t]] > [[t1]]$ so by the induction hypothesis $[[T1 -> T2]] \equiv [[T'' -> T']]$.
    Therefore, $[[T1]] \equiv [[T'']]$ and $[[T2]] \equiv [[T']]$.
  \end{itemize}
  \end{changemargin}
\end{proof}

\begin{lemma}[Properties of $[[ctype]]$ Continued]
  \label{lemma:ctype_props_cont}
  \begin{itemize}
  \item[i.] If $[[G,x:T,G' |- t1 t2:T']]$, $[[G |- t:T]]$, 
    $[[ [t/x] T t1]] = [[\y:T1.t']]$, and $[[t1]]$ is not a
    $\lambda$-abstraction, then $t_1$ is in head normal form and
    there exists a type $[[A]]$ such that $[[ctype T (x,t1)]] = [[A]]$.
    
  \item[ii.] If $[[G,x:T,G' |- t1 t2:T']]$, $[[G |- t:T]]$, 
    $[[ [t/x] T t1]] = [[\m y:{-(T''->T')}.t']]$, and $[[t1]]$ is not a
    $\Delta$-abstraction, then there exists a type $[[A]]$ such that
    $[[ctype T (x,t1)]] = [[A]]$.
  \end{itemize}
\end{lemma}
\begin{proof}
  We prove part one first. This is a proof by induction on the structure of $[[t1 t2]]$.

  The only possibilities for the form of $[[t1]]$ is $[[x]]$ or $[[s1 s2]]$.  All other 
  forms would not result in $[[ [t/x] T t1]]$ being a $\lambda$-abstraction and $[[t1]]$ not.
  If $[[t1]] \equiv [[x]]$ then there exist a type $[[T'']]$ such that $[[T]] \equiv [[T'' -> T']]$ and
  $[[ctype T (x,x t2)]] = [[T']]$ when $[[ctype T (x,x)]] = [[T]] \equiv [[T'' -> T']]$ in this case.  We know
  $[[T'']]$ to exist by inversion on $[[G,x:T,G' |- t1 t2:T']]$.

  Now suppose $[[t1]] \equiv [[s1 s2]]$.  Now knowing $[[t1]]$ to not a $\lambda$-abstraction
  implies that $[[s1]]$ is also not a $\lambda$-abstraction or $[[ [t/x] T t1]]$ would be an application
  instead of a $\lambda$-abstraction.  So it must be the case that $[[ [t/x] T s1]]$ is a $\lambda$-abstraction
  and $[[s1]]$ is not.  Since $[[s1]] < [[t1]]$ we can apply the induction hypothesis to obtain there exists
  a type $[[A]]$ such that $[[ctype T (x,s1)]] = [[A]]$.  
  Now by inversion on $[[G,x:T,G' |- t1 t2:T']]$ we know there exists a type $[[T'']]$ such that
  $[[G,x:T,G' |- t1:T'' -> T']]$.  We know $[[t1]] \equiv [[s1 s2]]$ so by inversion on
  $[[G,x:T,G' |- t1:T'' -> T']]$ we know there exists a type $[[A'']]$ such that
  $[[G,x:T,G' |- s1:A'' -> (T'' -> T')]]$.
  By part two of Lemma~\ref{lemma:ctype_props} we know $[[A]] \equiv [[A'' -> (T'' -> T')]]$ and
  $[[ctype T (x,t1)]] = [[ctype T (x,s1 s2)]] = [[T'' -> T']]$ 
  when $[[ctype T (x,s1)]] = [[A'' -> (T'' -> A')]]$, because we know $[[ctype T (x,s1)]] = [[A]]$.

  \noindent
  The proof of part two is similar to the proof of part one.
\end{proof}
The first two properties of the hereditary substitution function are
totality and type preservation.  
\begin{lemma}[Totality and Type Preservation]
  \label{lemma:totality_and_type_preservation}
  \begin{itemize}
  \item[i.] If $[[G |- Th 3 : A]]$ and $[[G, Th 1:{-(A->A')} |- t' : B]]$, then there
    exists a term $[[s]]$ such that $[[<Th,A,A'>t']] = [[s]]$ and $[[G, Th 2 :{-A'} |- s : B]]$.
  
  \item[ii.] If $[[G |- t : A]]$ and $[[G, x:A, G' |- t':B]]$, then there exists a term $[[s]]$ 
    such \\ that $[[ [t/x] A t']] = [[s]]$ and $[[G,G' |- s:B]]$.
  \end{itemize}
\end{lemma}
\begin{proof}
  This is a mutually inductive proof using the lexicographic combination
  $([[A]], f,[[t']])$ of our ordering on types,
  the natural number ordering where $f \in \{0,1\}$, and
  the strict subexpression ordering on terms. We first prove part one
  and then part two.  In both parts we case split on $[[t']]$.

  \noindent Part One.
  \vspace{-25px}
  \begin{changemargin}{10px}{5px}\noindent
  \begin{itemize}
  \item[Case.] Suppose $[[t']]$ is a variable $[[x]]$.  Then either there exists
    a term $[[a]]$ such that $[[(x,z,a)]] \in [[Th]]$ or not.  Suppose so. Then 
    $[[<Th,A,A'>x]] = [[\y:A->A'.(z (y a))]]$ where $[[y]]$ is fresh in $[[x]]$, $[[z]]$ and $[[a]]$.
    Now suppose there does not exist any term $[[a]]$ or $[[z]]$ such that $[[(x,z,a)]] \in [[Th]]$.  Then
    $[[<Th,A,A'>x]] = [[x]]$. Typing clearly holds, because if $[[(x,z,a)]] \in [[Th]]$ then 
    $[[B]] \equiv [[{-(A->A')}]]$ and we know $[[G,Th 2:{-A'} |- \y:A->A'.(z (y a)) : B]]$ or
    $[[x]] \not\in [[Th 1]]$ then it must be the case that $[[x:B]] \in [[G]]$, hence,
    by assumption and weakening for typing $[[G,Th 2:{-A'} |- x : B]]$.

  \item[Case.] It must be the case that $[[B]] \equiv [[B1 -> B2]]$ for some types $[[B1]]$ and
    $[[B2]]$.  Suppose $[[t']] \equiv [[\y:B1.t'1]]$. Then 
    $[[<Th,A,A'>t']] = [[<Th ,A,A'>(\y:B1.t'1)]] = [[\y:B1.<Th,A,A'>t'1]]$.  Now sense
    $([[A]],1,[[t']]) > ([[A]],1,[[t'1]])$ we may apply the induction hypothesis to obtain
    that there exists a term $[[s]]$ such that $[[<Th,A,A'>t'1]] = [[s]]$, and $[[G,Th 2:{-A'},y:B1 |- s : B2]]$.
    Thus, by definition and the typing rule for $\lambda$-abstractions we obtain $[[<Th,A,A'>t']] = [[\y:B1.s]]$
    and $[[G,Th 2:{-A'} |- \y:B1.s:B1 -> B2]]$.

  \item[Case.] Suppose $[[t']] \equiv [[\m y:{-B}.t'1]]$. Similar to the previous case.

  \item[Case.] Suppose $[[t']] \equiv [[t'1 t'2]]$. We have two cases to consider.
    \begin{itemize}
    \item[Case.] Suppose $[[t'1]] \equiv [[x]]$ for some variable $[[x]]$.  In each
      case $[[B]] \equiv [[_|_]]$.
      \begin{itemize}
      \item[Case.] Suppose $[[t'2]] \equiv [[\y:A.t''2]]$, for some $[[y]]$ and $[[t''2]]$,
        $[[(x,z,t)]] \in [[Th]]$. Since 
        $(A,1,[[t']]) > (A,1,[[t''2]])$ and the typing assumptions
        hold by inversion we can apply the induction
        hypothesis to obtain $[[<Th,A,A'>t''2]] = [[s]]$ for some term $[[s]]$ and
        $[[G,Th 2:{-A'},y:A |- s : A']]$.
        Furthermore, sense $(A,1,t') > (A,0,s)$, the previous typing condition and
        the typing assumptions we also know from the induction hypothesis that 
        $[[ [t/y] A s = s' ]]$ for some term $[[s']]$ and
        $[[G,Th 2:{-A'} |- s' : A']]$. Finally, by definition we know 
        $[[<Th, A, A'>t']] = [[z ([t/y] A s)]] = [[z s']]$ and 
        by using the application typing rule that $[[G,Th 2:{-A'} |- z s' : B]]$.

      \item[Case.] Suppose $[[t'2]] \equiv [[\m y:{-(A->A')}.t''2]]$, for some $[[y]]$ and 
        $[[t''2]]$, $[[(x,z,t)]] \in [[Th]]$. Since $(A,1,t') > (A,1,[[t''2]])$ we know from the 
        induction hypothesis that $[[<Th,(y,z2,t),A,A'>t''2]] = [[s]]$ for some fresh variable $[[z]]$
        and term $[[s]]$, and $[[G,Th 2:{-A'},z2:{-A'} |- s : _|_ ]]$.
        Finally, $[[<Th, A,A'>t']] = [[z (\m z2:{-A'}.s)]]$ by definition, and by
        using the application typing rule 
        $[[G,Th 2:{-A'} |- z (\m z2:{-A'}.s) : B ]]$.

      \item[Case.] Suppose $[[t'2]]$ is not an abstraction, and $[[(x,z,t)]] \in [[Th]]$.  
        Since $(A,1,t') > (A,1,[[t'2]])$ we know from the 
        induction hypothesis that $[[<Th,A,A'>t'2]] = [[s]]$ for some term $[[s]]$
        and $[[G,Th 2:{-A'} |- s : A -> A' ]]$.  Finally,
        $[[<Th, A,A'>t']] = [[z s]]$ by definition, and by
        using the application typing rule 
        $[[G,Th 2:{-A'} |- z s : B ]]$.

      \item[Case.] Suppose $[[(x,z,t'')]] \not \in [[Th]]$ for any term $[[t'']]$ and $[[z]]$.  
        Since $(A,1,t') > (A,1,[[t'2]])$ we know from the 
        induction hypothesis that $[[<Th,A,A'>t'2]] = [[s]]$ for some term $[[s]]$
        and $[[G,Th 2:{-A'} |- s : A -> A' ]]$.  Finally,
        $[[<Th, A,A'>t']] = [[x s]]$ by definition, and by
        using the application typing rule 
        $[[G,Th 2:{-A'} |- x s : B ]]$.
      \end{itemize}
    \item[Case.] Suppose $[[t'1]]$ is not a variable.  This case follows easily from
      the induction hypothesis.
    \end{itemize}

  \end{itemize}
  \end{changemargin}
  
  \noindent Part two.
  \vspace{-25px}
  \begin{changemargin}{10px}{5px}\noindent
  \begin{itemize}
  \item[Case.] Suppose $[[t']]$ is either $[[x]]$ or a variable $[[y]]$ distinct from $[[x]]$.  
    Trivial in both cases.

  \item[Case.] Suppose $[[t']] \equiv [[\ y:A1.t'1]]$.  By inversion 
    we know there exists a type $[[A2]]$ such that
    $[[G,x:A,G',y:A1 |- t'1:A2]]$.
    We also know that $[[t'_1]]$ is a strict subexpression of $[[t']]$, hence we can apply the second part of the
    induction hypothesis to obtain
    $[[ [t/x] A t'1]] = [[s1]]$ and $[[G,G',y:A1 |- s1:A2]]$
    for some term $[[s1]]$.  By the definition of the hereditary substitution function 
    \begin{center}
      \begin{math}
        \begin{array}{lll}
          [[ [t/x] A t']] & = & [[\ y:A1.[t/x] A t'_1]] \\
          & = & [[\y:A1.s1]].
        \end{array}
      \end{math}
    \end{center}
    It suffices to show that $[[G,G' |- \y:A1.s1:A1 -> A2]]$.  
    By simply applying the typing rule $\LamDeltadrulename{\normalsize Lam}$ using
    $[[G,G',y:A1 |- s1:A2]]$ we obtain $[[G,G' |- \y:A1.s1:A1 -> A2]]$.
    
  \item[Case.] Suppose $[[t']] \equiv [[\m y:{-B}.t'1]]$.  Similar to the previous case.

  \item[Case.] Suppose $[[t']] \equiv [[t'1 t'2]]$.  By inversion we know
    $[[G, x:A, G' |- t'1 : B' -> B]]$ and
    $[[G, x:A, G' |- t'2 : B']]$ for some type $[[B']]$.
    Clearly, $[[t'1]]$ and $[[t'2]]$ are strict subexpressions of $[[t']]$.  Thus, by the second part of the
    induction hypothesis there exists terms $[[s1]]$ and $[[s2]]$ such that $[[ [t/x] A t'1]] = [[s1]]$ and
    $[[ [t/x] A t'2]] = [[s2]]$, and $[[G, G' |- s1 : B' -> B']]$ and
    $[[G, G' |- s2 : B']]$.  We case split on whether or not $[[s1]]$ is a $\lambda$-abstraction or
    a $\Delta$-abstraction and $[[t'1]]$ is not, or $[[s1]]$ and $[[t'1]]$ are both a $\lambda$-abstraction or
    a $\Delta$-abstraction.
    We only consider the non-trivial cases when $[[s1]] \equiv [[\y:B'.s'1]]$ and $[[t'1]]$ is not a 
    $\lambda$-abstraction, and $[[s1]] \equiv [[\m y:{-(B'->B)}.s'1]]$ and $[[t'1]]$ is not a 
    $\Delta$-abstraction.  Consider the former.
    
    Now by Lemma~\ref{lemma:ctype_props} it is the case that 
    there exists a $[[B'']]$ such that $[[ctype A (x,t'1)]] = [[B'']]$, 
    $[[B'']] \equiv [[B' -> B]]$, and $[[B]]$ is a subexpression of $[[A]]$, hence
    $[[A]] > [[B']]$.  By the definition of the hereditary substitution function
    $[[ [t/x] A (t'1 t'2)]] = [[ [s2/y] B' s'1]]$. Therefore, by the induction hypothesis there exists a 
    term $[[s]]$ such that $[[ [s2/y] A s'1]] = [[s]]$ and $[[G,G' |- s:B]]$.
    
    At this point consider when $[[s1]] \equiv [[\m y:{-(B'->B)}.s'1]]$ and $[[t'1]]$ is not a 
    $\Delta$-abstraction.  Again, by Lemma~\ref{lemma:ctype_props} it is the case that 
    there exists a $[[B'']]$ such that $[[ctype A (x,t'1)]] = [[B'']]$, $[[B'']] \equiv [[B' -> B]]$ and
    $[[B' -> B]]$ is a subexpression of $[[A]]$.  Hence, $[[A]] > [[B']]$. Let $[[r]]$ be a fresh variable of type $[[{-B}]]$.    
    Then by the induction hypothesis, there exists a term $s''$, such that, $[[<(y,r,s2),B',B>s'1]] = s''$ and 
    $[[G,r:{-B} |- s'' : _|_]]$.  Therefore, $[[ [t/x] A (t'1 t'2)]] = [[\m r:{-B}.<(y,r,s2),B',B>s'1]] = [[\m r:{-B}.s'']]$,
    and by the $\Delta$-abstraction typing rule $[[G |- \m r:{-B}.s'' : B]]$.
  \end{itemize}
  \end{changemargin}
\end{proof}
The next property shows that the hereditary substitution function is
normality preserving. The proof of normality preservation depends on
the following auxiliary result.
\begin{lemma}
  \label{lemma:ssub_var_head}
  For any $[[Th]]$, $[[A]]$ and $[[A']]$, if $[[n1 n2]]$ is normal then 
  $[[head (<Th,A,A'>(n1 n2))]]$ is a variable.
\end{lemma}
\begin{proof}
  This is a proof by induction on the form of $[[n1 n2]]$.
In every case where $[[n1]]$ is a variable and $([[n1]],[[z]],[[t]]) \in [[Th]]$ for some 
term $[[t]]$ and variable $[[z]]$, we know by definition
that $[[<Th,A,A'>(n1 n2)]] = [[z t2]]$ for some variable $[[z]]$ 
and term $[[t2]]$.  In the case where $[[n1]]$ is a variable and 
$([[n1]],[[z]],[[t]]) \not\in [[Th]]$ for some 
term $[[t]]$ and variable $[[z]]$, we know by definition
that $[[<Th,A,A'>(n1 n2)]] = [[(<Th,A,A'>n1) (<Th,A,A'>n2)]]$.  Now
by hypothesis and definition $[[<Th,A,A'>n1]] = [[n1]]$.  Thus,
$[[(<Th,A,A'>n1) (<Th,A,A'>n2)]] = [[n1 (<Th,A,A'>n2)]]$ and we know $[[n1]]$ is
a variable.  The final case is when $[[n1]]$ is not a variable.  Then it must
be the case that $[[n1]]$ is a normal application.  So by the induction hypothesis
$[[head (<Th,A,A'>n1)]]$ is a variable.  Therefore,
$[[head (<Th,A,A'>(n1 n2))]] = [[head ((<Th,A,A'> n1) (<Th,A,A'> n2))]] = [[head (<Th,A,A'> n1)]]$ is a variable.
\end{proof}
\begin{lemma}[Normality Preservation]
  \label{lemma:normality_preservation}
  \begin{itemize}
  \item[i.] If $[[norm (Th 3)]]$, $[[G |- Th 3 : A]]$ and $[[G, Th 1:{-(A->A')} |- n' : B]]$, 
    then there exists a normal form $[[m]]$ such that $[[<Th,A,A'>n']] = [[m]]$.
    
  \item[ii.] If $[[G |- n:A]]$ and $[[G, x:A, G' |- n':B]]$ then there exists a term $[[m]]$ 
    such that $[[ [n/x] A n']] = [[m]]$. 
  \end{itemize} 
\end{lemma}
\begin{proof}
  This is a mutually inductive proof using the lexicographic combination\\
  $([[A]], f,[[n']])$ of our ordering on types,
  the natural number ordering where $f \in \{0,1\}$, and
  the strict subexpression ordering on terms. We first prove part one
  and then part two.  In both parts we case split on $[[n']]$.
  
  \noindent Part One.
  \vspace{-25px}
  \begin{changemargin}{10px}{5px}\noindent
  \begin{itemize}
  \item[Case.] Suppose $[[n']]$ is a variable $[[x]]$.  Then either there exists
    a normal form $[[m]]$ and variable $[[z]]$, such that, $[[(x,z,m)]] \in [[Th]]$ or not.  
    Suppose so. Then 
    $[[<Th,A,A'>x]] = [[\y:A->A'.(z (y m))]]$ where $[[y]]$ is fresh in $[[x]]$, $[[z]]$ and 
    $[[m]]$.  Clearly, $[[\y:A->A'.(z (y m))]]$ is normal.
    Now suppose there does not exist any term $[[m]]$ or $[[z]]$ such that 
    $[[(x,z,m)]] \in [[Th]]$.  Then $[[<Th,A,A'>x]] = [[x]]$ which is clearly normal.

  \item[Case.] Suppose $[[n']] \equiv [[\y:B1.n'1]]$.  Then 
    $[[<Th,A,A'>n']] = [[<Th ,A,A'>(\y:B1.n'1)]] = [[\y:B1.<Th,A,A'>n'1]]$.  Now sense
    $([[A]],1,[[n']]) > ([[A]],1,[[n'1]])$ we may apply the induction hypothesis to obtain
    that there exists a term $[[m]]$ such that $[[<Th,A,A'>n'1]] = [[m]]$.  
    Thus, by definition we obtain $[[<Th,A,A'>n']] = [[\y:B1.m]]$.

  \item[Case.] Suppose $[[n']] \equiv [[\m y:{-B}.n'1]]$. Similar to the previous case.

  \item[Case.] Suppose $[[n']] \equiv [[n'1 n'2]]$. We have two cases to consider.
    \begin{itemize}
    \item[Case.] Suppose $[[n'1]] \equiv [[x]]$ for some variable $[[x]]$.     
      \begin{itemize}
      \item[Case.] Suppose $[[n'2]] \equiv [[\y:A.n''2]]$, for some $[[y]]$ and $[[n''2]]$,
        $[[(x,z,n)]] \in [[Th]]$. Since 
        $(A,1,[[n']]) > (A,1,[[n''2]])$ and the typing assumptions
        hold by inversion we can apply the induction
        hypothesis to obtain $[[<Th,A,A'>n''2]] = [[m]]$ for some term $[[m]]$.
        We know from Lemma~\ref{lemma:normality_preservation} that 
        $[[G,Th 2:{-A'},y:A |- m : A']]$.
        Furthermore, sense $(A,1,n') > (A,0,m)$, the previous typing condition and
        the typing assumptions we also know from the induction hypothesis that 
        $[[ [t/y] A m = m' ]]$ for some term $[[m']]$. Finally, by definition we know 
        $[[<Th, A, A'>n']] = [[z ([n/y] A m)]] = [[z m']]$.  It is easy to see that
        $[[z m']]$ is normal.      

      \item[Case.] Suppose $[[n'2]] \equiv [[\m y:{-(A->A')}.n''2]]$, for some $[[y]]$ and 
        $[[n''2]]$, $[[(x,z,n)]] \in [[Th]]$. Since $(A,1,n') > (A,1,[[n''2]])$ we know from the 
        induction hypothesis that $[[<Th,(y,z2,n),A,A'>n''2]] = [[m]]$ for some normal form 
        $[[m]]$, and $[[G,Th 2:{-A'},z2:{-A'} |- m : _|_ ]]$.
        Finally, $[[<Th, A,A'>n']] = [[z (\m z2:{-A'}.m)]]$ by definition.

      \item[Case.] Suppose $[[n'2]]$ is not an abstraction, and $[[(x,z,n)]] \in [[Th]]$.  
        Since $(A,1,n') > (A,1,[[n'2]])$ we know from the 
        induction hypothesis that $[[<Th,A,A'>n'2]] = [[m]]$ for some normal form $[[m]]$.  Finally,
        $[[<Th, A,A'>n']] = [[z m]]$ by definition.      

      \item[Case.] Suppose $[[(x,z,n'')]] \not \in [[Th]]$ for any term $[[n'']]$ and $[[z]]$.  
        Since $(A,1,n') > (A,1,[[n'2]])$ we know from the 
        induction hypothesis that $[[<Th,A,A'>n'2]] = [[m]]$ for some term $[[m]]$.  Finally,
        $[[<Th, A,A'>n']] = [[x m]]$ by definition.      
      \end{itemize}
    \item[Case.] Suppose $[[n'1]]$ is not a variable.  This case follows easily from
      the induction hypothesis and Lemma~\ref{lemma:ssub_var_head}.
    \end{itemize}
  \end{itemize}
  \end{changemargin}
  
  \noindent Part two.
  \vspace{-25px}
  \begin{changemargin}{10px}{5px}\noindent
  \begin{itemize}
  \item[Case.] Suppose $[[n']]$ is either $[[x]]$ or a variable $[[y]]$ distinct from $[[x]]$.  
    Trivial in both cases.

  \item[Case.] Suppose $[[n']] \equiv [[\ y:B1.n'1]]$.  
    We also know that $[[n'_1]]$ is a strict subexpression of $[[n']]$, hence we can apply the second part of the
    induction hypothesis to obtain
    $[[ [n/x] A n'1]] = [[m1]]$ for some normal form $[[m1]]$.  
    By the definition of the hereditary substitution function 
    \begin{center}
      \begin{math}
        \begin{array}{lll}
          [[ [n/x] A n']] & = & [[\ y:A1.[n/x] A n'_1]] \\
          & = & [[\y:B1.m1]].
        \end{array}
      \end{math}
    \end{center}
    Clearly, $[[\y:B1.m1]]$ is normal.
    
  \item[Case.] Suppose $[[n']] \equiv [[\m y:{-B}.n'1]]$.  Similar to the previous case.

  \item[Case.] Suppose $[[t']] \equiv [[t'1 t'2]]$.
    Clearly, $[[n'1]]$ and $[[n'2]]$ are strict subexpressions of $[[n']]$.  Thus, by the 
    induction hypothesis there exists normal forms $[[n1]]$ and $[[n2]]$ such that 
    $[[ [n/x] A n'1]] = [[m1]]$ and
    $[[ [n/x] A n'2]] = [[m2]]$.
    We case split on whether or not $[[m1]]$ is a $\lambda$-abstraction or
    a $\Delta$-abstraction and $[[n'1]]$ is not, or $[[m1]]$ and $[[n'1]]$ are both a 
    $\lambda$-abstraction or a $\Delta$-abstraction.  
    We only consider the non-trivial cases when $[[m1]] \equiv [[\y:B'.m'1]]$ and 
    $[[n'1]]$ is not a 
    $\lambda$-abstraction, and $[[m1]] \equiv [[\m y:{-(B'->B)}.m'1]]$ and $[[n'1]]$ is not a 
    $\Delta$-abstraction.  Consider the former.
   
    Now by Lemma~\ref{lemma:ctype_props} it is the case that 
    there exists a $[[B'']]$ such that $[[ctype A (x,t'1)]] = [[B'']]$, 
    $[[B'']] \equiv [[B' -> B]]$, and $[[B]]$ is a subexpression of $[[A]]$, hence
    $[[A]] > [[B']]$.  By the definition of the hereditary substitution function
    $[[ [n/x] A (n'1 n'2)]] = [[ [m2/y] B' m'1]]$. Therefore, by the induction hypothesis there 
    exists a 
    normal form $[[m]]$ such that $[[ [m2/y] A m'1]] = [[m]]$.
    
    At this point consider when $[[m1]] \equiv [[\m y:{-(B'->B)}.m'1]]$ and $[[n'1]]$ is not a 
    $\Delta$-abstraction.  Again, by Lemma~\ref{lemma:ctype_props} it is the case that 
    there exists a $[[B'']]$ such that $[[ctype A (x,t'1)]] = [[B'']]$, $[[B'']] \equiv [[B' -> B]]$ and
    $[[B' -> B]]$ is a subexpression of $[[A]]$.  Hence, $[[A]] > [[B']]$. Let $[[r]]$ be a fresh variable of type $[[{-B}]]$.    
    Then by the induction hypothesis, there exists a term $m''$, such that, $[[<(y,r,m2),B',B>m'1]] = m''$ and 
    Therefore, $[[ [n/x] A (n'1 n'2)]] = [[\m r:{-B}.<(y,r,m2),B',B>m'1]] = [[\m r:{-B}.m'']]$.
  \end{itemize}
  \end{changemargin}
\end{proof}
\noindent
The final correctness property of the hereditary substitution function is
soundness with respect to reduction.  We need one last piece of notation.
Suppose $[[Th]] = ([[x1]],[[z1]],[[t1]]),\ldots,([[xi]],[[zi]],[[ti]])$ for
some natural number $[[i]]$.
Then $[[lift Th A A' t']] =^{def} [ [[\y:A -> A'.(zi (y ti))]]/[[xi]] ](\cdots([ [[\y:A -> A'.(z1 (y t1))]]/[[x1]] ][[t1]])\cdots)$.

\begin{lemma}[Soundness with Respect to Reduction]
  \label{lemma:soundness_reduction}  
  \begin{itemize}
  \item[i.] If $[[G |- Th 3 : A]]$ and $[[G, Th 1:{-(A->A')} |- t' : B]]$, then
    $[[lift Th A A' t' ~*> <Th,A,A'>t']]$.
  
  \item[ii.] If $[[G |- t : A]]$ and $[[G, x:A, G' |- t':B]]$ then 
    $[[ [t/x]t' ~*> [t/x] A t']]$.   
  \end{itemize}  
\end{lemma}
\begin{proof}
  This is a mutually inductive proof using the lexicographic combination\\
  $([[A]], f,[[t']])$ of our ordering on types,
  the natural number ordering where $f \in \{0,1\}$, and
  the strict subexpression ordering on terms. We first prove part one
  and then part two.  In both parts we case split on $[[t']]$.

  \noindent Part One.
  \vspace{-25px}
  \begin{changemargin}{10px}{5px}\noindent
  \begin{itemize}
  \item[Case.] Suppose $[[t']]$ is a variable $[[x]]$.  Then either there exists
    a term $[[a]]$ such that $[[(x,z,a)]] \in [[Th]]$ or not.  Suppose so. 
    Then by definition we know $[[lift Th A A' x]] = [[\y:A->A'.(z (y a))]]$,
    for some fresh variable $[[y]]$.  Now $[[<Th,A,A'>x]] = [[\y:A->A'.(z (y a))]]$,
    where we choose the same $[[y]]$.  Thus, $[[lift Th A A' x ~*> <Th,A,A'>x]]$.
    Now suppose there does not exist any term $[[a]]$ or $[[z]]$ such that $[[(x,z,a)]] \in [[Th]]$.
    Then $[[<Th,A,A'>x]] = [[lift Th A A' x]] = x$. Thus, $[[lift Th A A' x ~*> <Th,A,A'>x]]$.

  \item[Case.] Suppose $[[t']] \equiv [[\y:B1.t'1]]$. This case follows from the
    induction hypothesis.

  \item[Case.] Suppose $[[t']] \equiv [[\m y:{-B}.t'1]]$. Similar to the previous case.

  \item[Case.] Suppose $[[t']] \equiv [[t'1 t'2]]$. We have two cases to consider.
    \begin{itemize}
    \item[Case.] Suppose $[[t'1]] \equiv [[x]]$ for some variable $[[x]]$. 
      \begin{itemize}
      \item[Case.] Suppose $[[t'2]] \equiv [[\y:A.t''2]]$, for some $[[y]]$ and $[[t''2]]$,
        $[[(x,z,t)]] \in [[Th]]$. 
        Now
        \begin{center}
          \begin{math}
            \begin{array}{lll}
              [[lift Th A A' (x (\y:A.t''2))]] \\
              \,\,\,= [[(\y:A->A'.(z (y t))) (\y:A.(lift Th A A' t''2))]]\\
              \,\,\,\redto [[z ((\y:A.(lift Th A A' t''2)) t)]]\\
              \,\,\,\redto [[z ([t/y](lift Th A A' t''2))]]\\
            \end{array}
          \end{math}
        \end{center}
        Since 
        $(A,1,[[t']]) > (A,1,[[t''2]])$ we can apply the induction
        hypothesis to obtain $[[lift Th A A' t''2 ~*> <Th,A,A'>t''2]]$. Hence,
        \begin{center}
          \begin{math}
            \begin{array}{lll}
              [[z ([t/y](lift Th A A' t''2))]] & \redto^* & [[z ([t/y](<Th,A,A'>t''2))]]\\
            \end{array}
          \end{math}
        \end{center}
        Furthermore, sense $(A,1,t') > (A,0,[[<Th,A,A'>t''2]])$, we also know from the induction hypothesis that 
        \begin{center}
          \begin{math}
            \begin{array}{lll}
              [[z ([t/y](<Th,A,A'>t''2))]] & \redto^* & [[z ([t/y] A (<Th,A,A'>t''2))]]\\
              & =        & [[<Th,A,A'>t']]\\
            \end{array}
          \end{math}
        \end{center}

      \item[Case.] Suppose $[[t'2]] \equiv [[\m y':{-(A->A')}.t''2]]$, for some $[[y]]$ and 
        $[[t''2]]$, $[[(x,z,t)]] \in [[Th]]$. 
        Now using a fresh variable $[[z2]]$ we know 
        \begin{center}
          \small
          \begin{math}
            \begin{array}{lll}
              [[lift Th A A' (x (\m y':{-(A->A')}.t''2))]] \\
              \ \ \ \ = [[(\y:A->A'.(z (y t))) (\m y':{-(A->A')}.(lift Th A A' t''2))]]\\ 
              \ \ \ \ \redto [[z ((\m y':{-(A->A')}.(lift Th A A' t''2)) t)]] \\
              \ \ \ \ \redto [[z (\m z2:{-A'}.([\y:A->A'.(z2 (y t))/y'](lift Th A A' t''2)))]] \\
              \ \ \ \ = [[z (\m z2:{-A'}.(lift Th,(y',z2,t) A A' t''2))]]\\
            \end{array}
          \end{math}
        \end{center}
        Since $(A,1,t') > (A,1,[[t''2]])$ we know from the 
        induction hypothesis that $[[lift Th,(y',z2,t) A A' t''2 ~*> <Th,(y',z2,t),A,A'>t''2]]$.
        Thus, 
        \begin{center}
          \small
          \begin{math}
            \begin{array}{lll}
              [[z (\m z2:{-A'}.(lift Th,(y',z2,t) A A' t''2))]] & \redto^* & [[z (\m z2:{-A'}.(<Th,(y',z2,t),A,A'>t''2))]]\\
              & = & [[<Th,A,A'>t']].
            \end{array}
          \end{math}
        \end{center}
        
      \item[Case.] Suppose $[[t'2]]$ is not an abstraction, and $[[(x,z,t)]] \in [[Th]]$. 
        Since $([[A]],1,[[t']]) > (A,1,[[t'2]])$ we know from the 
        induction hypothesis that $[[lift Th A A' t'2 ~*> <Th,A,A'>t'2]]$.  Thus,
        \begin{center}
          \begin{math}
            \begin{array}{lll}
              [[lift Th A A' t']] & = & [[lift Th A A' (x t'2)]]\\ 
              & = & [[z (lift Th A A' t'2)]]\\ 
              & \redto^* & [[z (<Th,A,A'>t'2)]]\\
              & = & [[<Th,A,A'> (x t'2)]] = [[<Th,A,A'>t']].
            \end{array}
          \end{math}
        \end{center}

      \item[Case.] Suppose $[[(x,z,t'')]] \not \in [[Th]]$ for any term $[[t'']]$ and $[[z]]$.  
        Since $(A,1,t') > (A,1,[[t'2]])$ we know from the 
        induction hypothesis that $[[lift Th A A' t'2 ~*> <Th,A,A'>t'2]]$.  Thus,
        \begin{center}
          \begin{math}
            \begin{array}{lll}
              [[lift Th A A' t']] & = & [[lift Th A A' (x t'2)]]\\ 
              & = & [[x (lift Th A A' t'2)]]\\ 
              & \redto^* & [[x (<Th,A,A'>t'2)]]\\
              & = & [[<Th,A,A'> (x t'2)]] = [[<Th,A,A'>t']].
            \end{array}
          \end{math}
        \end{center}
      \end{itemize}
    \item[Case.] Suppose $[[t'1]]$ is not a variable.  This case follows easily from
      the induction hypothesis.
    \end{itemize}
  \end{itemize}
  \end{changemargin}

  \noindent Part two
  \vspace{-25px}
  \begin{changemargin}{10px}{5px}\noindent
  \begin{itemize}
  \item[Case.] Suppose $[[t']]$ is a variable $[[x]]$ or $[[y]]$ distinct from $[[x]]$.  
    Trivial in both cases.
    
  \item[Case.] Suppose $[[t']] \equiv [[\y:B1.s]]$.  Then
    $[[ [t/x] (\ y:B1.s) = \y:B1.([t/x] s)]]$. 
    Now $[[s]]$ is a strict subexpression of $[[t']]$ so we can apply the second part of the induction hypothesis to obtain 
    $[[ [t/x]s ~*> [t/x] A s]]$.  At this point we can see that since 
    $[[\y:B1.[t/x]s == [t/x](\ y:B1.s)]]$ we may
    conclude that $[[\ y:B1.[t/x]s ~*> \y:B1.[t/x] A s]]$.

  \item[Case.] Suppose $[[t']] \equiv [[\m y:{-B}.s]]$.  Similar to the previous case.

  \item[Case.] Suppose $[[t' == t'1 t'2]]$.  By Lemma~\ref{lemma:totality_and_type_preservation}
    there exists terms $[[s1]]$ and $[[s2]]$
    such that $[[ [t/x] A t'1 = s1]]$ and $[[ [t/x] A t'2 = s2]]$.  Since
    $[[t'1]]$ and $[[t'2]]$ are strict subexpressions of $[[t']]$ we can apply the second part of the induction hypothesis to obtain
    $[[ [t/x]t'1 ~*> s1]]$ and $[[ [t/x]t'2 ~*> s2]]$.  Now we case
    split on whether or not $[[s1]]$ is a $\lambda$-abstraction and $[[t'1]]$ is not,
    a $\Delta$-abstraction and $[[t'1]]$ is not, or $[[s1]]$ is not a $\lambda$-abstraction or a $\Delta$-abstraction.  If
    $[[s1]]$ is not a $\lambda$-abstraction or a $\Delta$-abstraction then 
    $[[ [t/x] A t' = ([t/x] A t'1) ([t/x] A t'2) == s1 s2]]$. Thus, by two applications of the
    induction hypothesis, $[[ [t/x]t' ~*> [t/x] A t']]$, because $[[ [t/x]t' = ([t/x] t'1) ([t/x] t'2)]]$.
    
    Suppose $[[s1 == \ y:B'.s'1]]$ and $[[t'1]]$ is not a $\lambda$-abstraction.  
    By Lemma~\ref{lemma:ctype_props} there exists a type $[[B'']]$ such that
    $[[ctype A (x,t'1)]] = [[B'']]$, $[[B'' == B' -> B]]$, and $[[B'']]$ is a subexpression
    of $[[A]]$.  Then by the definition of the hereditary substitution function 
    $[[ [t/x] A (t'1 t'2) = [s2/y] B' s'1]]$.
    Now we know $[[A > B']]$ so we can apply the second part of the induction hypothesis to obtain 
    $[[ [s2/y] s'1 ~*> [s2/y] B' s'1]]$. By knowing that 
    $[[((\ y:B'.s'1) s2) ~> ([s2/y] s'1)]]$ and
    by the previous fact we know $[[(\y:B'.s'1) s2 ~*> [s2/y] B' s'1]]$.
    We now make use of the well known result of full $\beta$-reduction.  The
    result is stated as
    \begin{center}
      \begin{math}
        $$\mprset{flushleft}
        \inferrule* [right=] {
          [[a ~*> a']]
          \\\\
          [[b ~*> b']]
          \\
          [[a' b' ~*> c]]
        }{[[a b ~*> c]]}
      \end{math}
    \end{center}
    \vspace{-5px}
    where $[[a]]$, $[[a']]$, $[[b]]$, $[[b']]$, and $[[c]]$ are all terms.  We apply this
    result by instantiating $[[a]]$, $[[a']]$, $[[b]]$, $[[b']]$, and $[[c]]$ with
    $[[ [t/x] t'1]]$, $[[s1]]$, $[[ [t/x] t'2]]$, $[[s2]]$, and $[[ [s2/y] B' s'1]]$ 
    respectively.  Therefore, $[[ [t/x](t'1 t'2) ~*> [s2/y] B' s'1]]$.        
    
    Suppose $[[s1 == \m y:{-(B'->B)}.s'1]]$ and $[[t'1]]$ is not a $\Delta$-abstraction.
    By Lemma~\ref{lemma:ctype_props} there exists a type $[[B'']]$ such that
    $[[ctype A (x,t'1)]] = [[B'']]$, $[[B'' == B' -> B]]$, and $[[B'']]$ is a subexpression
    of $[[A]]$.  Then by the definition of the hereditary substitution function
    $[[ [t/x] A (t'1 t'2) = \m z:{-B}.<(y,z,s2),B',B>s'1]]$, where $[[z]]$ is fresh variable.
    Now
    \begin{center}
      \begin{math}
        \begin{array}{lll}
          [[ [t/x] (t'1 t'2)]] & = & [[([t/x] t'1) ([t/x] t'2)]]\\
          & \redto^* & [[s1 s2]]\\
          & \equiv & [[(\m y:{-(B'->B)}.s'1) s2]]\\
          & \redto & [[\m z:{-B}.[\y':B'->B.(z (y' s2))/y]s'1]]\\
          & = & [[\m z:{-B}.(lift (y,z,s2) B' B s'1)]]
        \end{array}
      \end{math}
    \end{center}
    It suffices to show that $[[\m z:{-B}.(lift (y,z,s2) B' B s'1) ~*> \m z:{-B}.<(y,z,s2),B',B>s'1]]$,
    but this follows from the induction hypothesis, because $([[A]],0,[[t']]) > ([[B']],1,[[s'1]])$.
  \end{itemize} 
  \end{changemargin}
\end{proof}
Using these properties it is now possible to conclude normalization for the $\lambda\Delta$-calculus.
% subsection properties_of_the_hereditary_substitution_function (end)
% section the_hereditary_substitution_function_for_the_ld-calculus (end)

\section{Concluding Normalization}
\label{sec:concluding_normalization}
We now define the interpretation $\interp{[[T]]}_[[G]]$ of types
$[[T]]$ in typing context $[[G]]$.  This is in fact the same
interpretation of types that was used to show normalization using
hereditary substitution of the various extensions of SSF.
\begin{definition}
  \label{def:semantics}
  The interpretation of types $\interp{[[T]]}_[[G]]$ is defined by:
  \begin{center}
    \begin{math}
      \begin{array}{lll}
        [[n]] \in \interp{[[T]]}_[[G]] & \iff & [[G |- n : T]]
      \end{array}
    \end{math}
  \end{center}
  We extend this definition to non-normal terms $t$ in the following way:
  \begin{center}
    \begin{math}
      \begin{array}{lll}
        [[t]] \in \interp{[[T]]}_[[G]] & \iff & \exists [[n]].[[t]] \normto [[n]] \in \interp{[[T]]}_[[G]]
      \end{array}
    \end{math}
  \end{center}
\end{definition}
Finally, we have the main lemma hereditary substitution for the
interpretation of types.
\begin{lemma}[Hereditary Substitution for the Interpretation of Types]
  \label{lemma:substitution_for_the_interpretation_of_types}
  If $[[n]] \in \interp{[[T]]}_[[G]]$ and $[[n']] \in \interp{[[T']]}_{[[G,x:T,G']]}$, then
  $[[ [n/x] T n']] \in \interp{[[T']]}_{[[G,G']]}$.
\end{lemma}
\begin{proof}
  We know by Lemma~\ref{lemma:totality_and_type_preservation} that there exists a term $[[s]]$ such that
  $[[ [n/x] T n' = s]]$ and $[[G,G' |- s:T']]$, and by Lemma~\ref{lemma:normality_preservation} $[[s]]$ is
  normal.  Therefore, $[[s]] \in \interp{[[T']]}_{[[G,G']]}$.
\end{proof}
Using the previous lemma and the properties of the hereditary
substitution function we can now prove type soundness.
\begin{thm}[Type Soundness]
  \label{theorem:type_soundness}
  If $[[G |- t : T]]$ then $[[t]] \in \interp{[[T]]}_[[G]]$.
\end{thm}
\newpage
\begin{proof}
  This is a proof by induction on the assumed typing derivation.
  \vspace{-25px}
  \begin{changemargin}{10px}{5px}\noindent
  \begin{itemize}
  \item[Case.] \ \\
    \begin{center}
      $\LamDeltadruleAx{}$
    \end{center}
    Trivial.
    
  \item[Case.] \ \\
    \begin{center}
      $\LamDeltadruleLam{}$
    \end{center}
    By the induction hypothesis $[[t]] \in \interp{[[B]]}_{[[G,x:A]]}$.  By the definition of the
    interpretation of types $[[t]] \normto [[n]] \in \interp{[[B]]}_{[[G,x:A]]}$ and 
    $[[G, x:A |- n:B]]$.  Thus, by applying the $\lambda$-abstraction type-checking
    rule, $[[G |- \x:A.n:A -> B]]$, hence by the definition of the 
    interpretation of types $[[\x:A.n]] \in \interp{[[A -> B]]}_[[G]]$.  Therefore,
    $[[\x:A.t]] \normto [[\x:A.n]] \in \interp{[[A -> B]]}_[[G]]$.
    
  \item[Case.] \ \\
    \begin{center}
      $\LamDeltadruleDelta{}$
    \end{center}
    Similar to the previous case.
    
  \item[Case.] \ \\
    \begin{center}
      $\LamDeltadruleApp{}$
    \end{center}
    
    By the induction hypothesis we know $[[t1]] \in \interp{[[A -> B]]}_[[G]]$ and $[[t2]] \in \interp{[[A]]}_[[G]]$.
    So by the definition of the interpretation of types we know there exists normal forms $[[n1]]$ and $[[n2]]$
    such that $[[t1 ~*> n1]] \in \interp{[[A -> B]]}_[[G]]$ and $[[t2 ~*> n2]] \in \interp{[[A]]}_[[G]]$. Assume $[[y]]$ is a fresh
    variable in $[[n1]]$ and $[[n2]]$ of type $[[A]]$.    
    Then by hereditary 
    substitution for the interpretation of types (Lemma~\ref{lemma:substitution_for_the_interpretation_of_types}) 
    $[[ [n1/y] A (y n2)]] \in \interp{[[B]]}_[[G]]$.  
    It suffices to show that $[[t1 t2 ~*> [n1/y] A (y n2)]]$.  This is an easy consequence of soundness with respect
    to reduction (Lemma~\ref{lemma:soundness_reduction}), that is, $[[t1 t2 ~*> n1 n2]] = [[ [n1/y](y n2)]]$ 
    and by soundness with respect to reduction $[[ [n1/y](y n2) ~*> [n1/y] A (y n2)]]$.  Therefore, 
    $[[t1 t2]] \in \interp{[[B]]}_[[G]]$.  
  \end{itemize}
  \end{changemargin}
\end{proof}
\noindent
Finally, we conclude normalization for the $\lambda\Delta$-calculus using hereditary substitution.
\begin{corollary}[Normalization]
  \label{corollary:normalization}
  If $[[G |- t : T]]$ then there exists a term $[[n]]$ such that $[[t]] \normto [[n]]$.
\end{corollary}
% section concluding_normalization (end)

\section{Related Work}
\label{sec:related_work}
We first compare the proof method normalization using hereditary
substitution with other known proof methods.  The
$\lambda\Delta$-calculus could have been proven weakly and strongly
normalizing by translation to $\lambda\mu$-calculus.  It is true that
this is not as complicated as the proof method here, but a proof by
translation does not yield a direct proof.  

A direct proof of weak and strong normalization could have been given
using the Tait-Girard reducibility method.  However, we claim that the
proof method used here is less complicated.  The statement of the type
soundness theorem is qualitatively less complex due to the fact that
there is no need to universally quantify over the set of well-formed
substitutions.  We are able to prove type soundness on open terms
directly.  Additionally, the formalization of normalization using
hereditary substitution does not require recursive types to define the
semantics of types which are required when formalizing a proof using
reducibility.  

R. David and K. Nour give a short proof of normalization of the
$\lambda\Delta$-calculus in \cite{David:2003}.  There they use a
rather complicated lexicographic combination to give a completely
arithmetical proof of strong normalization.  While they show strong
normalization their proof method is comparable to using hereditary
substitution.  As we mentioned in the introduction hereditary
substitution is the constructive content of normalization proofs using
the lexicographic combination of an ordering on types and the strict
subexpression ordering on terms.  It is currently unknown if
hereditary substitution can be extended to show strong normalization,
but we conjecture that the constructive content of the proof of Lemma
3..6 in David and Nour's work would yield a hereditary substitution
like function.  Furthermore, for simply typed theories we believe it
is enough to show weak normalization and never need to show strong
normalization.  It is well-known due to the work of G. Barthe et
al. in \cite{Barthe:2001} that for the entire left hand side of the
$\lambda$-cube weak normalization implies strong normalization.  We
conjecture that this result would extend to the left hand side of the
classical $\lambda$-cube given in \cite{Barthe:1997}.  Thus, showing
normalization using hereditary substitution is less complicated than
the work of David and Nour's.

Similar to the work of David and Nour is the work of F. Joachimski and
R. Matthes.  In \cite{Joachimski:1999} they prove weak and strong
normalization of various simply typed theories.  The proof method used
is induction on various lexicographic combinations similar to
hereditary substitution.  After proving weak normalization of each
type theory they extract the constructive content of the proof
yielding a normalization function which depends on a substitution
function similar to the hereditary substitution function.  In contrast
once hereditary substitution is defined for a type theory we can easily
define a normalization function. Note that the following function is 
the computational content of the type-soundness theorem 
(Theorem~\ref{theorem:type_soundness}).
\begin{definition}
  \label{def:norm_fun_hs}
  We define a normalization function for the $\lambda\Delta$-calculus using hereditary
  substitution as follows:
  \vspace{-10px}
  \begin{center}
    \begin{itemize}
    \item[] $[[ norm_fun x ]] = [[x]]$\\
    \item[] $[[norm_fun (\x:A.t)]] = [[\x:A.(norm_fun t)]]$\\
    \item[] $[[norm_fun (\m x:A.t)]] = [[\m x:A.(norm_fun t)]]$\\
    \item[] $[[norm_fun (t1 t2)]] = [[ [n1/r] A (r n2)]]$\\
      \begin{tabular}{lll}
      & Where $[[norm_fun t1]] = [[n1]]$, $[[norm_fun t2]] = [[n2]]$, $[[A]]$ is
      the type of $[[t1]]$, and $[[r]]$ is fresh in $[[t1]]$\\
      &  and $[[t2]]$.\\
    \end{tabular}
    \end{itemize}
  \end{center}
\end{definition}

This function is similar to the normalization functions in Joachimski 
and Matthes' work.  We could use the above normalization function to
decide $\beta\eta$-equality for the $\lambda\Delta$-calculus.  Indeed
this one of the main application of hereditary substitution.

A. Abel in 2006 shows how to implement a normalizer using sized
heterogeneous types which is a function similar to the hereditary
substitution function in \cite{Abel:2006}.  He then uses hereditary
substitution to prove normalization of the type level of a type theory
with higher-order subtyping in \cite{Abel:2008}.  This results in a
purely syntactic metatheory.  C. Keller and T. Altenkirch recently
implemented hereditary substitution as a normalization function for
the simply typed $\lambda$-calculus in Agda \cite{Keller:2010}.  Their
results show that hereditary substitution can be used to decide $\beta
\eta$-equality.  They found hereditary substitution to be convenient
to use in a total type theory, because it can be implemented without a
termination proof.  This is because the hereditary-substitution
function can be recognized as structurally recursive, and hence
accepted directly by Agda's termination checker.
% section defining_a_normalization_function (end)

%%% Local Variables: 
%%% mode: latex
%%% reftex-default-bibliography: ("/Users/hde/thesis/paper/thesis.bib")
%%% TeX-master: "/Users/hde/thesis/paper/thesis.tex"
%%% End: 