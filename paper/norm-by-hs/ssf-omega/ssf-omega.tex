\section{Normalization of Stratified System $\Fw$}
\label{subsec:normalization_stratified_system_fw}
In this section we do something a bit different from the previous
sections, we show that if we combine the type theories STLC and SSF
into a stratified version of Girard's system $\Fw$, then we can use
the previous normalization proofs for STLC and SSF to conclude
normalization of this new language.  We call this new type theory
SS$\Fw$, and we define it in the next section.

\subsection{The Language}
\label{subsec:the_language}
Girard defined a well known type-theory where the types did not only
consist of type variables, function types, and universal
quantification, but also contained $\lambda$-abstractions and
application.  This type theory is called System $\Fw$
\cite{barendregt92} (See Section~\ref{sec:moderen_type_theory}). These
new $\lambda$-abstractions represent functions from types to types.
System $\Fw$ is an impredicative theory, but this adds a lot of
complexity to the language, so we restrict the theory just as we did
for SSF to obtain a less expressive theory called Stratified System
$\Fw$ (SS$\Fw$). The following definition defines the syntax for
SS$\Fw$.
\begin{definition}
  \label{def:syntax_ssfw}
  The syntax for terms, types, and kinds:
  \begin{center}
    \begin{tabular}{lll}
      $K$ & $:=$ & $K \to K$ $|$ $*_0$ $|$ $*_1$ $|$ $\ldots$\\
      $\phi$ & $:=$ & $X$   $|$ $\phi \rightarrow \phi$ $|$ $\forall X:K.\phi$ $|$ $\lambda X:*_l.\phi$ $|$ $\phi\ \phi$\\
      $t$ & $:=$ & $x$   $|$ $\lambda x:\phi.t$   $|$ $t\ t$ $|$ $\Lambda X:K.t$ $|$ $t[\phi]$\\
    \end{tabular}
  \end{center}
\end{definition}
\noindent We define the reduction relation for SS$\Fw$ next.
\begin{definition}
  \label{def:reduction_rules_ssf}
  Full $\beta$-reduction for SS$\Fw$:
  \begin{center}
    \begin{tabular}{lll}
      $(\Lambda X:K.t)[\phi]$     & $\rightsquigarrow$ & $[\phi/X]t$\\
      $(\lambda x:\phi.t)\ t'$      & $\rightsquigarrow$ & $[t'/x]t$\\
      $(\lambda X:*_l.\phi)\ \phi'$ & $\rightsquigarrow$ & $[\phi'/x]\phi$
    \end{tabular}
  \end{center}
\end{definition}

We can see that this system is almost the same as SSF except we have
added essentially STLC to the type level of the theory, and we now
allow quantification over predicates.  That is, take STLC where the
terms are types and the types are kinds, and this is essentially what
we have added to the type level of SSF.  This allows us to compute
types, so in terms of programming we are able to write more generic
programs, where the type of the program may need to be computed.  Next
we define the kinding and typing rules for this system.  The rules for
well-formed contexts do not change from SSF
(Figure~\ref{fig:well-formed_ssf}), so we do not define them here.

\begin{definition}
  \label{def:l_kinding_rules_ssfw}
  The kind assignment rules for SS$\Fw$ are defined as follows:
  \begin{center}
    \begin{mathpar}
      \inferrule* [right=] {
        \Gamma \vdash \phi_1 : *_p
        \\
        \Gamma \vdash \phi_2 : *_q
      }{\Gamma \vdash \phi_1 \rightarrow \phi_2 : *_{max(p,q)}}
      \and
      \inferrule* [right=] {
        \Gamma,X : *_q \vdash \phi : *_p
      }{\Gamma \vdash \forall X:*_q.\phi : *_{max(p,q)+1}}
      \and
      \inferrule* [right=] {
        \Gamma(X) = *_p
        \\
        p \leq q
        \\
        \Gamma\ Ok
      }{\Gamma \vdash X : *_q}
      \and
      \inferrule* [right=] {
        \Gamma,X:K_1 \vdash \phi:K_2
      }{\Gamma \vdash \lambda X:K_1.\phi:K_1 \to K_2}
      \and
      \inferrule* [right=] {
        \Gamma \vdash \phi_1:K_1 \to K_2
        \\
        \Gamma \vdash \phi_2:K_1
      }{\Gamma \vdash \phi_1\ \phi_2:K_2}
    \end{mathpar}
  \end{center}
\end{definition}
\noindent
To obtain SS$\Fw$ from SSF all we have to do is add the kinding rules
for type-level $\lambda$-abstractions and type-level application.  The
typing rules are the exactly same as SSF.  We state them next.
\begin{definition}
  \label{fig:typing_rules_ssfw}
  Type assignment rules for SS$\Fw$:
  \begin{center}
    \begin{mathpar}
      \inferrule* [right=] {
        \Gamma(x) = \phi
        \\
        \Gamma\ Ok
      }{\Gamma \vdash x : \phi}
      \and
      \inferrule* [right=] {
        \Gamma,x : \phi_1 \vdash t : \phi_2
      }{\Gamma \vdash \lambda x : \phi_1.t : \phi_1 \rightarrow \phi_2}
      \and
      \inferrule* [right=] {
        \Gamma \vdash t_1 : \phi_1 \rightarrow \phi_2 
        \\
        \Gamma \vdash t_2 : \phi_1
      }{\Gamma \vdash t_1\ t_2 : \phi_2}
      \and
      \inferrule* [right=] {
        \Gamma, X : *_p \vdash t : \phi
      }{\Gamma \vdash \Lambda X:*_p.t:\forall X : *_p.\phi}
      \and
      \inferrule* [right=] {
        \Gamma \vdash t:\forall X:*_l.\phi_1
        \\
        \Gamma \vdash \phi_2:*_l
      }{\Gamma \vdash t[\phi_2]: [\phi_2/X]\phi_1}
      \end{mathpar} 
    \end{center}
\end{definition}  
\noindent
All the basic syntactic lemmas proved for SSF in
Section~\ref{subsec:properties_of_the_hereditary_substitution_function}
hold for SS$\Fw$ so we do not state them here.
% subsection the_language (end)

\subsection{Concluding Normalization}
\label{subsec:concluding_normalization_ssfw}
Concluding normalization of SS$\Fw$ is a little bit tricky, because
the type level no longer consists of just constant types.  We conduct
the proof similarly to how H. Barendregt proves normalization of
System $\Fw$ in \cite{barendregt92}, except we use hereditary
substitution, where he uses a technique called reducibility.  Since
types contain computation we first must prove normalization of the
type level, and then using this result we prove normalization for the
term (program) level.  

Now the type level is exactly STLC with constructors for function
types and universal types.  So all we have to do is define the
hereditary substitution function, then use the results from
Section~\ref{sec:hereditary_substitution_for_stlc} to
obtain normalization.  Similarly, the term level of SS$\Fw$ is exactly
SSF so we do not need to define the hereditary substitution for
SS$\Fw$.  We can just use the results from
Section~\ref{subsec:normalization_stratified_system_f} to conclude
normalization.  

First, we have to define the construct kind function.  Note that
$ckind_K$ is the exact same as $ctype_\phi$ except at the type level.
Now all the properties of $ctype_\phi$ are also properties of
$ckind_K$ so we do not prove them again here.  The construct type
function is exactly the same as SSF, so we do not redefine it here.
Also the kind $K$ in $ckind_K$ is called the cut kind, and is used in
the exact same way as the cut type $\phi$.
\begin{definition}
  \label{def:ctype_stlc}
  The $ckind_K$ function is defined with respect to a fixed kind $K$
  and has two arguments, a free variable $X$, and a type $\phi$ where $X$
  may be free in $\phi$.  We define $ckind_K$ by induction on the form of $\phi$.
\begin{itemize}
\item[] $ckind_K(X,X) = K$
  \item[] $ckind_K(X,\phi_1\ \phi_2) = K'$
  \item[] \ \ \ \ Where $ckind_K(X,\phi_1) = K'' \to K'$.
  \end{itemize}
\end{definition}
\noindent
At this point we define the type-level hereditary substitution function.
\begin{definition}
  \label{def:hereditary_substitution_function__type_level_ssfw}
  The following defines the hereditary substitution function for the
  type level of SS$\Fw$.  
  \begin{itemize}
  \item[] $\{\phi/X\}^K X = \phi$
  \item[] $\{\phi/X\}^K Y = Y$
  \hscase{ Where $Y$ is a variable distinct from $X$.}
  \item[] $\{\phi/X\}^K (\phi_1 \to \phi_2) = (\{\phi/X\}^K \phi_1) \to (\{\phi/X\}^K \phi_2)$
  \item[] $\{\phi/X\}^K (\forall Y:*_l.\phi') = \forall Y:*_l.\{\phi/X\}^K \phi'$
  \item[] $\{\phi/X\}^K (\lambda Y:K_1.\phi') = \lambda Y:K_1.(\{\phi / X\}^K \phi')$
  \item[] $\{\phi/X\}^K (\phi_1\ \phi_2) = (\{\phi/X\}^K \phi_1)\ (\{\phi/X\}^K \phi_2)$
  \hscase{
    Where $(\{\phi/X\}^K \phi_1)$ is not a $\lambda$-abstraction, or both $(\{\phi/X\}^K \phi_1)$
    and $\phi_1$ are $\lambda$-abstractions.
  } 
  \item[] $\{\phi/X\}^{K} (\phi_1\ \phi_2) = \{(\{\phi/X\}^{K} \phi_2)/y\}^{K''} \phi'_1$
    \hscase{
      Where $(\{\phi/X\}^{K} \phi_1) \equiv \lambda Y:K''.\phi'_1$ 
      for some $Y$, $\phi'_1$, and $K''$ and \\ $ckind_K(X,\phi_1) = K'' \to K'$.
    }
  \end{itemize}
\end{definition}
Now all the properties of the hereditary substitution function from
Section~\ref{sec:hereditary_substitution_for_stlc}
are easily modified to obtain the exact same results for the function
just defined.  The statements of the properties do not change, except
where there are terms we use types, and where there are types we use
kinds, so we do not repeat the properties here.  

% \begin{lemma}[Properties of $ctype_\phi$]
%   \label{lemma:ctype_props_tl_ssfw}
%   \begin{itemize}
%   \item[]
%   \item[i.] If $ckind_K(X,\phi) = K'$ then $head(\phi) = X$ and $K'$ 
%     is a subexpression of $K$.
    
%   \item[ii.] If $\Gamma,X:K,\Gamma' \vdash \phi:K'$ and $ckind_K(X,\phi) = K''$ then
%     $K' \equiv K''$.
    
%   \item[iii.] If $\Gamma,X:K,\Gamma' \vdash \phi_1\ \phi_2:K'$, $\Gamma \vdash \phi:K$,
%     $[\phi/X]^K \phi_1 = \lambda Y:K_1.\psi$, and $\phi_1$ is not then there exists a kind
%     $\psi'$ such that $ckind_K(X,\phi_1) = \psi'$.
%   \end{itemize}
% \end{lemma}
% \begin{proof}
%   We prove part one first. This is a proof by induction on the structure of $\phi$.

% \begin{itemize}
% \item[Case.] Suppose $\phi \equiv X$.  Then $ckind_K(X,X) = K$.  Clearly,
%   $head(X) = X$ and $K$ is a subexpression of itself.
  
% \item[Case.] Suppose $\phi \equiv \phi_1\ \phi_2$.  Then $ckind_K(X,\phi_1\ \phi_2) = K''$
%   when $ckind_K(X,\phi_1) = K' \to K''$.  Now $\phi > \phi_1$ so by the induciton
%   hypothesis $head(\phi_1) = X$ and $K' \to K''$ is a subexpression of $K$.
%   Therefore, $head(\phi_1\ \phi_2) = X$ and certainly $K''$ is a subexpression of $K$.
% \end{itemize}

% \ \\
% We now prove part two.  This is also a proof by induction on the structure of $\phi$.

% \begin{itemize}
% \item[Case.] Suppose $\phi \equiv X$.  Then $ckind_K(X,X) = K$.  Clearly,
%   $K \equiv K$.
  
% \item[Case.] Suppose $\phi \equiv \phi_1\ \phi_2$.  Then $ckind_K(X,\phi_1\ \phi_2) = K_2$
%   when $ckind_K(X,\phi_1) = K_1 \to K_2$.  By inversion on the assumed kinding
%   derivation we know there exists a kind $K''$ such that $\Gamma,X:K,\Gamma' \vdash \phi_1:K'' \to K'$.
%   Now $\phi > \phi_1$ so by the induciton hypothesis $K_1 \to K_2 \equiv K'' \to K'$.
%   Therefore, $K_1 \equiv K''$ and $K_2 \equiv K'$.

% \ \\
% Next we prove part three.  This is a proof by induction on the structure of $\phi_1\ \phi_2$.

% \ \\
% The only possiblities for the form of $\phi_1$ is $X$, $\hat{\phi}_1\ \hat{\phi}_2$.  All other 
% forms would not result in $\{\phi/X\}^K \phi_1$ being a $\lambda$-abstraction and $\phi_1$ not.
% If $\phi_1 \equiv X$ then there exist a kind $K''$ such that $K \equiv K'' \to K'$ and
% $ckind_K(X,X\ \phi_2) = K'$ when $ckind_K(X,X) = K \equiv K'' \to K'$ in this case.  We know
% $K''$ to exist by inversion on $\Gamma,X:K,\Gamma' \vdash \phi_1\ \phi_2:K'$.

% \ \\
% Now suppose $\phi_1 \equiv (\hat{\phi}_1\ \hat{\phi}_2)$.  Now knowing $\phi'_1$ to not be a $\lambda$-abstraction
% implies that $\hat{\phi}_1$ is also not a $\lambda$-abstraction or $\{\phi/X\}^K \phi_1$ would be an application
% instead of a $\lambda$-abstraction.  So it must be the case that $\{\phi/X\}^K \hat{\phi}_1$ is a $\lambda$-abstraction
% and $\hat{\phi}_1$ is not.  Since $\phi_1\ \phi_2 > \phi_1$ we can apply the induction hypothesis to obtain there exists
% a kind $K'''$ such that $ckind_K(X,\hat{\phi}_1) = K'''$.  
% Now by inversion on $\Gamma,X:K,\Gamma' \vdash \phi_1\ \phi_2:K'$ we know there exists a kind $K''$ such that
% $\Gamma,X:K,\Gamma' \vdash \phi_1:K'' \to K'$.  We know $\phi_1 \equiv (\hat{\phi}_1\ \hat{\phi}_2)$ so by inversion on
% $\Gamma,X:K,\Gamma' \vdash \phi_1:K'' \to K'$ we know there exists a kind $\hat{K}$ such that
% $\Gamma,X:K,\Gamma' \vdash \hat{\phi}_1:\hat{K} \to (K'' \to K')$.
% By part two of Lemma~\ref{lemma:ctype_props_tl_ssfw} we know $K''' \equiv \hat{K} \to (K'' \to K')$ and
% $ckind_K(X,\phi_1) = ckind_K(X,\hat{\phi}_1\ \hat{\phi}_2) = K'' \to K'$ 
% when $ckind_K(X,\hat{\phi}_1) = \hat{K} \to (K'' \to K')$, because we know $ckind_K(X,\hat{\phi}_1) = K'''$.
% \end{itemize}
% \end{proof}

% \begin{lemma}[Total and Type Preserving]
%   \label{lemma:total_tl_ssfw}
%   Suppose $\Gamma \vdash \phi : K$ and $\Gamma, X:K, \Gamma' \vdash \phi':K'$. Then
%   there exists a type $\phi''$ such that $\{\phi/X\}^K \phi' = \phi''$ and $\Gamma,\Gamma' \vdash \phi'':K'$.
% \end{lemma}
% \begin{proof}
%   This is a proof by induction on the lexicorgraphic combination $(K,
%   \phi')$ of the strict subexpression ordering.  We case split on $\phi'$.
  
%   \begin{itemize}
%   \item[Case.] Suppose $\phi'$ is either $X$ or a variable $Y$ distinct from $X$.  
%     Trivial in both cases.

%   \item[Case.] Suppose $\phi' \equiv \phi'_1 \to \phi'_2$.  Trivial.
    
%   \item[Case.] Suppse $\phi' \equiv \forall X:*_l.\phi''$.  Trivial.
    
%   \item[Case.] Suppose $\phi' \equiv \lambda Y:K_1.\phi'_1$.  By inversion on the
%     kinding judgement we know $\Gamma,X:K,\Gamma',Y:K_1 \vdash \phi'_1:K_2$.
%     We also know $\phi' > \phi'_1$, hence we can apply the induction hypothesis to obtain
%     $\{\phi/X\}^K \phi'_1 = \hat{\phi}'_1$ and $\Gamma,\Gamma',Y:K_1 \vdash \hat{\phi}'_1:K_2$
%     for some type $\hat{\phi}'_1$.  By the definition of the hereditary substitution function 
%     $\{\phi/X\}^K \phi' = \lambda Y:K_1.\{\phi/X\}^K \phi'_1 = \lambda Y:K_1.\hat{\phi}'_1$.  It suffices
%     to show that $\Gamma,\Gamma' \vdash \lambda Y:K_1.\hat{\phi}'_1:K_1 \to K_2$.  
%     By simply applying the $\lambda$-abstraction kinding rule using
%     $\Gamma,\Gamma',Y:K_1 \vdash \hat{\phi}:K_2$ we obtain 
%     $\Gamma,\Gamma' \vdash \lambda Y:K_1.\hat{\phi}'_1:K_1 \to K_2$.
    
%   \item[Case.] Suppose $\phi' \equiv \phi'_1\ \phi'_2$.  By inversion we know
%     $\Gamma, X:K, \Gamma' \vdash \phi'_1 : K'' \to K'$ and
%     $\Gamma, X:K, \Gamma' \vdash \phi'_2 : K''$ for some types $K'$ and $K''$.
%     Clearly, $\phi' > \phi'_i$ for $i \in \{1,2\}$.  Thus, by the induction hypothesis
%     there exists types $\psi_1$ and $\psi_2$ such that $\{\phi/X\}^K \phi'_i = \psi_i$,
%     $\Gamma, \Gamma' \vdash \psi_1 : K'' \to K'$ and
%     $\Gamma, \Gamma' \vdash \psi_2 : K''$ for
%     $i \in \{1,2\}$.  We case split on whether or not $\psi_1$ is a $\lambda$-abstraction,
%     and $\phi'_1$ is not, or $ckind_K(X,\phi'_1)$ is undefined.  
%     We only consider the non-trivial case when 
%     $\psi_1 \equiv \lambda Y:K''.\psi'_1$ and $\phi'_1$ is not a $\lambda$-abstraction.
    
%     Now by Lemma~\ref{lemma:ctype_props_tl_ssfw} it is the case that 
%     there exists a $\hat{K}$ such that $ckind_K(X,\phi'_1) = \hat{K}$, 
%     $\hat{K} \equiv K'' \to K'$, and $\hat{K}$ is a subexpression of $K$, hence
%     $K >_{\Gamma,\Gamma'} K''$.
%     Then $\{\phi/X\}^K (\phi'_1\ \phi'_2) = \{\psi_2/Y\}^{K''} \psi'_1$.  
%     Therefore, by the induction hypothesis there exists a 
%     term $\psi$ such that $\{\psi_2/Y\}^{K''} \psi'_1 = \psi$ and $\Gamma,\Gamma' \vdash \psi:K''$.
%   \end{itemize}
% \end{proof}

% \begin{lemma}[Normality Preserving]
%   \label{corollary:normalization_preserving_tl_ssfw}
%   If $\Gamma \vdash \phi:K$ and $\Gamma, X:K \vdash \phi':\phi'$, where $\phi$ and $\phi'$ are normal, then there exists a normal type $\phi''$ 
%   such that $\{\phi/X\}^K \phi' = \phi''$.
% \end{lemma}
% \begin{proof}
%   By Lemma~\ref{lemma:total_tl_ssfw} we know there exists a type $\phi''$ such that $\{\phi/X\}^K \phi' = \phi''$ and by 
%   Lemma~\ref{lemma:redex_preserving_tl_ssfw} 
%   $|rset(\phi', \phi)| \geq |rset(\{\phi/X\}^K \phi')|$.  Hence, $|rset(\phi', \phi)| \geq |rset(\phi'')|$, but
%   $|rset(\phi', \phi)| = 0$.  Therefore, $|rset(\phi'')| = 0$ which implies $\phi''$ is normal.
% \end{proof}

% \begin{lemma}[Soundness with Respect to Reduction]
%   \label{lemma:soundness_reduction_tl_ssfw}
%   If $\Gamma \vdash \phi : K$ and $\Gamma, X:K, \Gamma' \vdash \phi':K'$ then
%   $\{\phi/X\}\phi' \redto^* \{\phi/X\}^K \phi'$.
% \end{lemma}
% \begin{proof}
%   This is a proof by induction on the lexicorgraphic combination
%   $(K, \phi')$ of the strict subexpression ordering.
%   We case split on the structure of $\phi'$.  When applying
%   the induction hypothesis we must show that the input terms to the
%   substitution and the hereditary substitution function are kindable.
%   We do not explicitly state kinding results that are simple
%   conseqences of inversion.

%   \begin{itemize}
%   \item[Case.] Suppose $\phi'$ is a variable $X$ or $Y$ distinct from $X$.  
%     Trivial in both cases.
  
%   \item[Case.] Suppose $\phi' \equiv \lambda Y:K'.\hat{\phi}$.  Then
%     $\{\phi/X\}^K (\lambda Y:K'.\hat{\phi}) = \lambda Y:K'.(\{\phi/X\}^K \hat{\phi})$. 
%     Now $\phi' > \hat{\phi}$ so we can apply the induction hypothesis to obtain 
%     $\{\phi/X\}\hat{\phi} \redto^* \{\phi/X\}^K \hat{\phi}$.  At this point we can see that since 
%     $\lambda Y:K'.\{\phi/X\}\hat{\phi} \equiv \{\phi/X\}(\lambda Y:K'.\hat{\phi})$ and we may
%     conclude that $\lambda Y:K'.\{\phi/X\}\hat{\phi} \redto^* \lambda Y:K'.\{\phi/X\}^K \hat{\phi}$.

%   \item[Case.] Suppose $\phi' \equiv \phi'_1 \to \phi'_2$.  Trivial.
    
%   \item[Case.] Suppse $\phi' \equiv \forall X:*_l.\phi''$.  Trivial.
    
%   \item[Case.] Suppose $\phi' \equiv \phi'_1\ \phi'_2$.  By
%     Lemma~\ref{lemma:total_tl_ssfw} there exists terms $\hat{\phi}'_1$ and
%     $\hat{\phi}'_2$ such that $\{\phi/X\}^K \phi'_1 = \hat{\phi}'_1$ and
%     $\{\phi/X\}^K \phi'_2 = \hat{\phi}'_2$.  Since $\phi' > \phi'_1$ and $\phi' > \phi'_2$
%     we can apply the induction hypothesis to obtain $\{\phi/X\}\phi'_1 \redto^*
%     \hat{\phi}'_1$ and $\{\phi/X\}\phi'_2 \redto^* \hat{\phi}'_2$.  Now we case split
%     on whether or not $\hat{\phi}'_1$ is a $\lambda$-abstraction and $\phi'_1$
%     is not or $ckind_K(X,\phi'_1)$ is undefined.  If
%     $ckind_K(X,\phi'_1)$ is undefined or $\hat{\phi}'_1$ is not a
%     $\lambda$-abstraction then $\{\phi/X\}^K \phi' =
%     (\{\phi/X\}^K \phi'_1)\ (\{\phi/X\}^\phi \phi'_2) \equiv
%     \hat{\phi}'_1\ \hat{\phi}'_2$. Thus, $\{\phi/X\}\phi' \redto^* \{\phi/X\}^K \phi'$,
%     because $\{\phi/X\}\phi' = (\{\phi/X\} \phi'_1)\ (\{\phi/X\} \phi'_2)$.  So suppose
%     $\hat{\phi}'_1 \equiv \lambda Y:K'.\hat{\phi}''_1$ and $\phi'_1$ is not a
%     $\lambda$-abstraction.  By Lemma~\ref{lemma:ctype_props_tl_ssfw} there
%     exists a kind $\hat{K}$ such that $ckind_K(X,\phi'_1) = \hat{K}$, $\hat{K}
%     \equiv K'' \to K'$, and $\hat{K}$ is a subexpression of $K$,
%     where by inversion on $\Gamma,X:K,\Gamma' \vdash \phi':K'$ there
%     exists a kind $K''$ such that $\Gamma,X:K,\Gamma' \vdash
%     \phi'_1:K'' \to K'$.  Then by the definiton of the hereditary
%     substitution function $\{\phi/X\}^K (\phi'_1\ \phi'_2) =
%     \{\hat{\phi}'_2/Y\}^{K'} \hat{\phi}''_1$.  Now we know $K
%     > K'$ so we can apply the induction hypothesis
%     to obtain $\{\hat{\phi}'_2/Y\} \hat{\phi}''_1 \redto^*
%     \{\hat{\phi}'_2/Y\}^{K'} \hat{\phi}''_1$.  Now by knowing that $(\lambda
%     Y:K'.\hat{\phi}''_1)\ \phi'_2 \redto \{\hat{\phi}'_2/Y\} \hat{\phi}''_1$ and by
%     the previous fact we know $(\lambda Y:K'.\hat{\phi}''_1)\ \phi'_2
%     \redto^* \{\hat{\phi}'_2/Y\}^{K'} \hat{\phi}''_1$.  We now make use of
%     the well known result of full $\beta$-reduction.  The result is
%     stated as
%     \begin{center}
%       \begin{math}
%         $$\mprset{flushleft}
%         \inferrule* [right=] {
%           a \redto^* a'
%           \\\\
%           b \redto^* b'
%           \\
%           a'\ b' \redto^* c
%         }{a\ b \redto^* c}
%       \end{math}
%     \end{center}
%     where $a$, $a'$, $b$, $b'$, and $c$ are all terms.  We apply this
%     result by instantiating $a$, $a'$, $b$, $b'$, and $c$ with
%     $[\phi/X] \phi'_1$, $\hat{\phi}'_1$, $[\phi/X] \phi'_2$, $\hat{\phi}'_2$, and $\{\hat{\phi}'_2/Y\}^{K'} \hat{\phi}''_1$ 
%     respectively.  Therefore, $[\phi/X](\phi'_1\ \phi'_2) \redto^* \{\hat{\phi}'_2/Y\}^{K'} \hat{\phi}''_1$.    
%   \end{itemize}
% \end{proof}

% \noindent
% We redefine the $rset$ function for types as follows:
% \begin{center}
%     \begin{itemize}
%     \item[] $rset(X) = \emptyset$\\
%     \item[] $rset(\lambda X:K.\phi) = rset(\phi)$\\
%     \item[] $rset(\phi_1 \to \phi_2) = rset(\phi_1,\phi_2)$\\
%     \item[] $rset(\forall X:K.\phi) = rset(\phi)$\\
%     \item[] $rset(\phi_1\ \phi_2)$\\
%       \begin{math}
%         \begin{array}{lll}
%           = & rset(\phi_1, \phi_2) & \text{if } \phi_1 \text{ is not a } \lambda \text{-abstraction.}\\
%           = & \{\phi_1\ \phi_2\} \cup rset(\phi'_1, \phi_2)\ & \text{if } \phi_1 \equiv \lambda X:K.\phi'_1.\\
%         \end{array}
%       \end{math}
%     \end{itemize}
%   \end{center}

% \begin{lemma}[Redex Preserving]
%   \label{lemma:redex_preserving_tl_ssfw}
%   If $\Gamma \vdash \phi : K$, $\Gamma, X:K, \Gamma' \vdash \phi':K'$
%   then  $|rset(\phi', \phi)| \geq |rset(\{\phi/X\}^K \phi')|$.
% \end{lemma}
% \begin{proof}
%   This is a proof by induction on the lexicorgraphic combination
%   $(K, \phi')$ of the strict subexpression ordering.
%   We case split on the structure of $\phi'$.  
  
%   \begin{itemize}
%   \item[Case.] Let $\phi' \equiv X$ or $\phi' \equiv Y$ where $Y$ is distinct from $X$.  Trivial. 

%   \item[Case.] Suppose $\phi' \equiv \phi'_1 \to \phi'_2$.  Trivial.
    
%   \item[Case.] Suppse $\phi' \equiv \forall X:*_l.\phi''$.  Trivial.
  
%   \item[Case.] Let $\phi' \equiv \lambda X:K_1.\phi''$.  Then $\{\phi/X\}^K \phi' \equiv \lambda X:K_1.\{\phi/X\}^K \phi''$.
%     Now 
%     \begin{center}
%       \begin{math}
%         \begin{array}{lll}
%           rset(\lambda X:K_1.\phi'', \phi) & = & rset(\lambda X:K_1.\phi'') \cup rset(\phi)\\
%           & = & rset(\phi'') \cup rset(\phi)\\
%           & = & rset(\phi'', \phi).\\
%         \end{array}
%       \end{math}
%     \end{center}
%     We know that $\phi' > \phi''$ by the strict subexpression ordering, hence by the induction hypothesis
%     $|rset(\phi'', \phi)| \geq |rset(\{\phi/X\}^K \phi'')|$ which implies $|rset(\phi', \phi)| \geq |rset(\{\phi/X\}^K \phi')|$.
    
%   \item[Case.] Let $\phi' \equiv \phi'_1\ \phi'_2$.  First consider when $\phi_1'$ is not a $\lambda$-abstraction. Then
%     \begin{center}
%       $rset(\phi'_1\ \phi'_2, \phi) = rset(\phi'_1, \phi'_2, \phi)$
%     \end{center}  
%     Clearly,  $\phi' > \phi'_i$ for $i \in \{1,2\}$, hence, by the induction hypothesis $|rset(\phi'_i,\phi)| \geq |rset(\{\phi/X\}^K \phi'_i)|$.  
%     We have two cases to consider.  That is whether or not $\{\phi/X\}^K \phi'_1$ is a $\lambda$-abstraction or
%     $ckind_K(X,\phi'_1)$ is undefined.  
%     Suppose the former.
%     Then by Lemma~\ref{lemma:ctype_props_tl_ssfw} $ctype_\phi(X,\phi'_1) = \hat{K}$ for some kind $\hat{K}$ and by inversion on 
%     $\Gamma,X:K,\Gamma' \vdash \phi'_1\ \phi'_2:K'$
%     there exists a kind $K''$ such that $\Gamma,X:K,\Gamma' \vdash \phi_1:K'' \to K'$.  Again, by Lemma~\ref{lemma:ctype_props_tl_ssfw}
%     $\hat{K} \equiv K'' \to K'$. Thus, $ckind_K(X,\phi'_1) = K'' \to K'$ and $K'' \to K'$ is a subexpression of $K$.
%     So by the definition of the hereditary substitution function $\{\phi/X\}^K \phi'_1\ \phi'_2 = \{(\{\phi/X\}^K \phi'_2)/Y\}^{K''} \phi''_1$, where
%     $\{\phi/X\}^K \phi'_1 = \lambda Y:K''.\phi''_1$.  Hence,
%     \begin{center}
%       \begin{math}
%         |rset(\{\phi/X\}^K \phi'_1\ \phi'_2)| = |rset(\{(\{\phi/X\}^K \phi'_2)/Y\}^{K''} \phi''_1)|.
%       \end{math}
%     \end{center}
%     Now $K > K''$ so by the induction hypothesis 
%     \begin{center}
%       \begin{math}
%         \begin{array}{lll}
%           |rset(\{(\{\phi/X\}^K \phi'_2)/Y\}^{K''} \phi''_1)| & \leq & |rset(\{\phi/X\}^K \phi'_2, \phi''_1)|\\
%           & \leq & |rset(\phi'_2, \phi''_1, \phi)|\\
%           & = & |rset(\phi'_2, \{\phi/X\}^K \phi'_1, \phi)|\\
%           & \leq & |rset(\phi'_2, \phi'_1, \phi)|\\
%           & = & |rset(\phi'_1, \phi'_2, \phi)|.\\
%         \end{array}
%       \end{math}
%     \end{center}
    
%     \ \\
%     \noindent
%     Suppose $\{\phi/X\}^K \phi'_1$ is not a $\lambda$-abstractions or $ckind_K(X,\phi'_1)$ is undefined.  Then
%     \begin{center}
%       \begin{math}
%         \begin{array}{lll}
%           rset(\{\phi/X\}^K (\phi'_1\ \phi'_2)) & = & rset(\{\phi/X\}^K \phi'_1\ \{\phi/X\}^K \phi'_2)\\
%           & = & rset(\{\phi/X\}^K \phi'_1, \{\phi/X\}^K \phi'_2).\\
%           & \leq & rset(\phi'_1, \phi'_2, \phi).\\
%         \end{array}
%       \end{math}
%     \end{center}
  
%     \ \\
%     Next suppose $\phi'_1 \equiv \lambda Y:K_1.\phi''_1$.  Then 
%     \begin{center}
%       \begin{math}
%         \begin{array}{lll}
%           rset((\lambda Y:K_1.\phi''_1)\ \phi'_2, \phi) & = & \{ (\lambda Y:K_1.\phi''_1)\ \phi'_2\} \cup rset(\phi''_1, \phi'_2, \phi).
%         \end{array}
%       \end{math}
%     \end{center}
%     By the definition of the hereditary substitution function,
%     \begin{center}
%       \begin{math}
%         \begin{array}{lll}
%           rset(\{\phi/X\}^K (\lambda Y:K_1.\phi''_1)\ \phi'_2) & = & rset(\{\phi/X\}^K (\lambda Y:K_1.\phi''_1)\ \{\phi/X\}^K \phi'_2)\\
%           & = & rset((\lambda Y:K_1.\{\phi/X\}^K \phi''_1)\ \{\phi/X\}^K \phi'_2)\\
%           & = & \{(\lambda Y:K_1.\{\phi/X\}^K \phi''_1)\ \{\phi/X\}^K \phi'_2\} \cup 
%           rset(\{\phi/X\}^K \phi''_1) \cup rset(\{\phi/X\}^K \phi'_2).\\
%         \end{array}
%       \end{math}
%     \end{center}
%     Since $\phi' > \phi''_1$ and $\phi' > \phi'_2$ we can apply the induction hypothesis to obtain,
%     $|rset(\phi''_1, \phi)| \geq |rset(\{\phi/X\}^K \phi''_1)|$ and $|rset(\phi'_2,\phi)| \geq |rset(\{\phi/X\}^K \phi'_2)|$.  Therefore, \\
%     $|\{ (\lambda Y:K_1.\phi''_1)\ \phi'_2\}\ \cup\ rset(\phi''_1,\phi)\ \cup\ rset(\phi'_2,\phi)| \geq $ 
%     $|\{(\lambda Y:K_1.\{\phi/X\}^K \phi''_1)\ \{\phi/X\}^K \phi'_2\}\ \cup\ rset(\{\phi/X\}^K \phi''_1)\ \cup\ rset(\{\phi/X\}^K \phi'_2)|$.
%   \end{itemize}
% \end{proof}

We now prove normalization for the type-level of SS$\Fw$.  The
following defines the interpretation of types and the main
substitution lemma.

\begin{definition}
  \label{def:interpretation_of_types_ssfw}
  First we define when a normal form is a member of the interpretation of type $\phi$ in context $\Gamma$
  \begin{center}
    \begin{math}
      \psi \in \interp{K}_\Gamma \iff \Gamma \vdash \phi:K,
  \end{math}
  \end{center}
  and this definition is extended to non-normal forms in the following way
  \begin{center}
    \begin{math}
      \phi \in \interp{K}_\Gamma \iff \phi \normto \psi \in \interp{K}_\Gamma,
  \end{math}
  \end{center}
 where we use $\psi$ to denote a normal form at the type level.
\end{definition}

\noindent 
Next we show that the definition of the interpretation of types is closed under
hereditary substitutions.  

\begin{lemma}[Substitution for the Interpretation of Types]
  If $\phi' \in \interp{K'}_{\Gamma,X:K,\Gamma'}$, $\phi \in \interp{K}_\Gamma$, then 
  $\{\phi/X\}^K \phi' \in \interp{K'}_{\Gamma,\Gamma'}$.
  
  \label{lemma:interpretation_of_types_closed_substitution_ssfw}
\end{lemma}
\begin{proof}
  This proof is similar to the proof of the same result for STLC
  (Lemma~\ref{lemma:interpretation_of_types_closed_substitution_stlc}).
\end{proof}
\noindent
Finally, by the definition of the interpretation of types the following result implies that the type-level
of SS$\Fw$ is normalizing.
\begin{thm}[Type Soundness]
  If $\Gamma \vdash \phi:K$ then $\phi \in \interp{K}_\Gamma$.
  \label{thm:soundness_ssfw}
\end{thm}
\begin{proof}
  This proof is similar to the proof of the type soundness theorem in
  the normalization proof for STLC (Theorem~\ref{thm:soundness_stlc}).
\end{proof}
\begin{corollary}[Type-level Normalization]
  \label{coro:normalization_type_level_ssfw}
  If $\Gamma \vdash \phi:K$, then there exists a normal form $\psi$,
  such that $\phi \normto \psi$.
\end{corollary}

Now we have to use the fact that we know the type level of SS$\Fw$ is
normalizing to conclude normalization of the term level.  First we
define the interpretation of types for the term level.

\begin{definition}
  \label{def:interpretation_of_types_stlc}
  We define when a normal form is a member of the interpretation of
  normal type $\psi$ in context $\Gamma$ as follows:
  \begin{center}
    \begin{math}
    n \in \interp{\psi}_\Gamma \iff \Gamma \vdash n:\psi,
  \end{math}
  \end{center}
  and this definition is extended to non-normal forms in the following way
  \begin{center}
    \begin{math}
      t \in \interp{\psi}_\Gamma \iff t \normto n \in \interp{\psi}_\Gamma.
  \end{math}
  \end{center}
\end{definition}

\noindent 
Next we show that the definition of the interpretation of types is closed under
hereditary substitutions.  

\begin{lemma}[Substitution for the Interpretation of Types]
  If $n' \in \interp{\psi'}_{\Gamma,x:\phi,\Gamma'}$, $n \in \interp{\psi}_\Gamma$, then 
  $[n/x]^\psi n' \in \interp{\psi'}_{\Gamma,\Gamma'}$.
  
  \label{lemma:interpretation_of_types_closed_substitution_stlc}
\end{lemma}
\begin{proof}
  This proof is similar to the proof for SSF
  (Lemma~\ref{lemma:interpretation_of_types_closed_substitution_ssf}).
\end{proof}
At this point we must show that all terms of arbitrary type are
normalizing. We do this in two steps, first, we show that for all
typeable terms (where their types are normal) are normalizing.  Then
knowing this, we show that all typeable terms (where their types are
arbitrary) are normalizing. The former is established next.
\begin{thm}[Type Soundness (Normal Types)]
  If $\Gamma \vdash t:\psi$ and $\psi$ is normal, then $t \in \interp{\psi}_\Gamma$.
  \label{thm:soundness__normal_ssfw}
\end{thm}
\begin{proof}
  This proof is equivalent to the proof for standard SSF
  (Theorem~\ref{thm:soundness_ssf}).
\end{proof}
\noindent
The following lemmas are used in the proof of the general type
soundness theorem, which shows that all typeable terms normalize.
\begin{lemma}[Preservation of Types]
  \label{lemma:preservation_for_kinding_ssfw}
  \begin{itemize}
  \item[i.] If $(\Gamma,x:\phi,\Gamma')\ Ok$ and $\phi \redto \phi'$, then $(\Gamma,x:\phi',\Gamma')\ Ok$.
  \item[ii.] If $\Gamma \vdash \phi:K$ and $\phi \redto \phi'$, then $\Gamma \vdash \phi':K$.
  \end{itemize}
\end{lemma}
\begin{proof}
  We prove part one and two by mutual induction starting with part 1.
\begin{itemize}
\item[Case.] \ \\
  \begin{center}
    \begin{math}
      $$\mprset{flushleft}
      \inferrule* [right=] {
        \   
      }{\cdot\ Ok}
    \end{math}
  \end{center}
  Trivial.

\item[Case.]\ \\
  \begin{center}
    \begin{math}
      $$\mprset{flushleft}
      \inferrule* [right=] {
        (\Gamma,x:\phi,\Gamma')\ Ok
      }{(\Gamma,x:\phi,\Gamma',X:*_p)\ Ok}
    \end{math}
  \end{center}
  A simple application of the inductive hypothesis yeilds the result.

\item[Case.]\ \\
  \begin{center}
    \begin{math}
      $$\mprset{flushleft}
      \inferrule* [right=] {
        \Gamma,x:\phi,\Gamma' \vdash \psi:*_p
        \\
        (\Gamma,x:\phi,\Gamma')\ Ok
      }{(\Gamma,x :\phi,\Gamma',y:\psi)\ Ok}
    \end{math} 
  \end{center}
\end{itemize}
By weakening for kinding, $\Gamma,\Gamma' \vdash \psi:*_p$, and then, by strengthening for kinding, $\Gamma,x:\phi',\Gamma' \vdash \psi:*_p$.
The inductive hypothesis on $(\Gamma,x:\phi,\Gamma')\ Ok$ yields $(\Gamma,x:\phi',\Gamma')\ Ok$, thus by applying the above rule
$(\Gamma,x :\phi',\Gamma',y:\psi)\ Ok$.

\ \\
\noindent
Next we prove part two.
\begin{itemize}
\item[Case.] \ \\
  \begin{center}
    \begin{math}
      $$\mprset{flushleft}
      \inferrule* [right=] {
        \Gamma \vdash \phi_1 : *_p
        \\
        \Gamma \vdash \phi_2 : *_q
      }{\Gamma \vdash \phi_1 \rightarrow \phi_2 : *_{max(p,q)}}
    \end{math}
  \end{center}
  By two applications of the inductive hypothesis $\Gamma \vdash \phi'_2 : *_q$ and
  $\Gamma \vdash \phi'_1 : *_p$, where $\phi_1 \redto \phi'_1$ and $\phi_2 \redto \phi'_2$, because
  $\phi \equiv (\phi_1 \to \phi_2) \redto \phi' \equiv (\phi'_1 \to \phi'_2)$.  Now by applying
  the above rule $\Gamma \vdash \phi'_1 \to \phi'_2 : *_{max(p,q)}$. 

\item[Case.]\ \\
  \begin{center}
    \begin{math}
      $$\mprset{flushleft}
      \inferrule* [right=] {
        \Gamma,X : *_q \vdash \phi : *_p
      }{\Gamma \vdash \forall X:*_q.\phi : *_{max(p,q)+1}}
    \end{math}
  \end{center}
  We know by assumption that $\phi \equiv (\forall X:*_q.\phi) \redto \phi' \equiv (\forall X:*_q.\phi')$.  Hence,
  $\phi \redto \phi'$ so by the inductive hypothesis $\Gamma,X:*_q \vdash \phi' : *_q$.  Finally, by applying the
  above rule $\Gamma \vdash \forall X:*_q.\phi':*_{max(p,q)+1}$.

\item[Case.]\ \\
  \begin{center}
    \begin{math}
      $$\mprset{flushleft}
      \inferrule* [right=] {
        \Gamma(X) = *_p
        \\
        p \leq q
        \\
        \Gamma\ Ok
      }{\Gamma \vdash X : *_q}
    \end{math} 
  \end{center}
  Trivial, because $X$ is a normal form.
\end{itemize}
\end{proof}
\begin{lemma}
  \label{lemma:preservation_2}
  If $\Gamma \vdash t:\phi$ and $\phi \redto \phi'$ then there exists a $\Gamma'$ such that $\Gamma' \vdash t:\phi'$.
\end{lemma}
\begin{proof}
  This is a proof by induction on the assumed typing derivation.

\begin{itemize}
\item[Case.]\ \\
  \begin{center}
    \begin{math}
      $$\mprset{flushleft}
      \inferrule* [right=] {
        \Gamma(x) = \phi
        \\
        \Gamma\ Ok
      }{\Gamma \vdash x : \phi}
    \end{math}  
  \end{center}
  Suppose there exists a type $\phi'$ such that $\phi \redto \phi'$.  It suffcies to show that
  $\Gamma \vdash x:\phi'$.  We know it must be the case that $\Gamma \equiv (\Gamma'',x:\phi,\Gamma''')$
  and by Lemma~\ref{lemma:preservation_for_kinding_ssfw} $(\Gamma'',x:\phi',\Gamma''')\ Ok$.  Now clearly,
  $\Gamma'',x:\phi',\Gamma''' \vdash x:\phi'$.  Take, $\Gamma'',x:\phi',\Gamma'''$ for $\Gamma'$. 

\item[Case.]\ \\
  \begin{center}
    \begin{math}
      $$\mprset{flushleft}
      \inferrule* [right=] {
        \Gamma,x : \phi_1 \vdash t : \phi_2
      }{\Gamma \vdash \lambda x : \phi_1.t : \phi_1 \rightarrow \phi_2}
    \end{math} 
  \end{center}
  First, by assumption $\phi \equiv \phi_1 \to \phi_2$ and $\phi_1 \to
  \phi_2 \redto \phi'$.  It must be the case that $\phi' \redto
  \phi'_1 \to \phi'_2$ some some types $\phi_1$ and $\phi_2$.  By the
  inductive hyothesis there exists a $\Gamma'$ such that $\Gamma'
  \vdash t:\phi'_2$. Lemma~\ref{lemma:kinding_ok_ssf} yeilds $(\Gamma,x:\phi_1)\ Ok$ which
  implies that $\Gamma,x:\phi \vdash \phi_1:K$ for some kind $K$.  Now
  by Lemma~\ref{lemma:preservation_for_kinding_ssfw} $\Gamma,x:\phi_1 \vdash
  \phi'_1:K$.  We obtain $\Gamma \vdash \phi'_1$ by strengthing for
  kinding and by weaking for kinding $\Gamma',x:\phi'_1 \vdash
  t:\phi'_2$.  Thus, by applying the $\lambda$-abstraction typing rule
  $\Gamma' \vdash \lambda x:\phi'_1.t:\phi'_1 \to \phi'_2$.

\item[Case.]\ \\
  \begin{center}
    \begin{math}
      $$\mprset{flushleft}
      \inferrule* [right=] {
        \Gamma \vdash t_1 : \phi_1 \rightarrow \phi_2 
        \\
        \Gamma \vdash t_2 : \phi_1
      }{\Gamma \vdash t_1\ t_2 : \phi_2}
    \end{math}
  \end{center}
  We know by assumption that $\phi \equiv \phi_2 \redto \phi'$ which implies that
  $\phi_1 \to \phi_2 \redto \phi_1 \to \phi'_2$.  By the inductive hypothesis
  there exists a $\Gamma' \vdash t_1 : \phi_1 \to \phi_2$.  Let $\Gamma'' \subseteq \Gamma$ contain
  only the free variables necessary to type $t_2$.  So $\Gamma \equiv \Gamma_1,\Gamma'',\Gamma_2$ and
  we know $\Gamma_1,\Gamma'',\Gamma_2 \vdash t_2:\phi_1$.  Clearly, if $\Gamma''$ contains only the
  variables needed to type $t_2$ then $\Gamma'' \vdash t_2:\phi_1$, because it must be the case that
  $\Gamma_1$ and $\Gamma_2$ are not needed for the typing of $t_2$.  Now by weakening for typing
  $\Gamma'',\Gamma' \vdash t_2:\phi_2$, which is equivalent to $\Gamma',\Gamma'' \vdash t_2:\phi_2$.
  Again by weakening for typing, $\Gamma',\Gamma'' \vdash t_1:\phi_1 \to \phi'_2$.  Thus,
  by applying the above typing rule $\Gamma',\Gamma'' \vdash t_1\ t_2:\phi'_2$.  
  
  \item[Case.]\ \\
    \begin{center}
      \begin{math}
        $$\mprset{flushleft}
        \inferrule* [right=] {
          \Gamma, X : *_p \vdash t : \phi
        }{\Gamma \vdash \Lambda X:*_p.t:\forall X : *_p.\phi}
      \end{math} 
    \end{center}
    Similar to the $\lambda$-abstraction case.

  \item[Case.]\ \\
    \begin{center}
      \begin{math}
        $$\mprset{flushleft}
        \inferrule* [right=] {
          \Gamma \vdash t:\forall X:*_l.\phi_1
          \\
          \Gamma \vdash \phi_2:*_l
        }{\Gamma \vdash t[\phi_2]: [\phi_2/X]\phi_1}
      \end{math}
    \end{center}
    Similar to the application case.
\end{itemize}
\end{proof}
\noindent
Finally, we conclude normalization.
\begin{thm}[Type Soundness]
  If $\Gamma \vdash t:\phi$ then $\phi \normto \psi$, and there exists a $\Gamma'$ such that $t \in \interp{\psi}_{\Gamma'}$.
  \label{thm:soundness_term_ssfw}
\end{thm}
\begin{proof}
  By regularity we know $\Gamma \vdash \phi:K$ for some kind $K$, and
  by Corollary~\ref{coro:normalization_type_level_ssfw} there exists a
  normal type $\psi$, such that $\phi \normto \psi$.  Finally, by
  Lemma~\ref{lemma:preservation_2}, there exists a $\Gamma'$, such
  that $\Gamma' \vdash t:\psi$.  Thus, by
  Theorem~\ref{thm:soundness__normal_ssfw} $t \in \interp{\psi}_{\Gamma'}$.
\end{proof}

\begin{corollary}[Normalization]
  If $\Gamma \vdash t:\phi$, then there exists a normal form $n$, such
  that $t \normto n$.
\end{corollary}
% subsection concluding_normalization (end)
% subsection normalization_stratified_system_fw (end)
% section normalization_by_hereditary_substitution (end)

%%% Local Variables: 
%%% mode: latex
%%% reftex-default-bibliography: ("/Users/hde/thesis/paper/thesis.bib")
%%% TeX-master: "/Users/hde/thesis/paper/thesis.tex"
%%% End: 
