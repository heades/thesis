\section{Normalization of Stratified System $\Fw$}
\label{subsec:normalization_stratified_system_fw}
In this section we do something a bit different from the previous
sections.  Here we show that if we combine the type theories STLC and
SSF that we can use the previous normalization results to conclude
normalization of this new language.  First we define the new language.

\subsection{The Language}
\label{subsec:the_language}
J. Girard defined a well known type-theory where the types not only
consist of type variables, function types, and universal
quantification, but also contain $\lambda$-abstractions and
application.  This type theory is called System $\Fw$
\cite{barendregt92}. Now these new $\lambda$-abstractions represent
functions from types to types.  Logically these correspond to
predicates which means this language corresponds to a restricted form
of second-order intuitionistic logic with only implication as a
primitive and no quantification over predicates.  System $\Fw$
contains impredicativity, but this adds a lot of complexity to the
language so we restrict the theory just as we did for SSF to obtain
a more less expressive theory called Stratified System $\Fw$ (SS$\Fw$).
Lets define the language.  The following definition defines the syntax
for SS$\Fw$.
\begin{definition}
  \label{def:syntax_ssfw}
  The syntax for terms, types, and kinds:
  \begin{center}
    \begin{tabular}{lll}
      $K$ & $:=$ & $K \to K$ $|$ $*_0$ $|$ $*_1$ $|$ $\ldots$\\
      $\phi$ & $:=$ & $X$   $|$ $\phi \rightarrow \phi$ $|$ $\forall X:K.\phi$ $|$ $\lambda X:*_l.\phi$ $|$ $\phi\ \phi$\\
      $t$ & $:=$ & $x$   $|$ $\lambda x:\phi.t$   $|$ $t\ t$ $|$ $\Lambda X:K.t$ $|$ $t[\phi]$\\
    \end{tabular}
  \end{center}
\end{definition}
\noindent Now we define our operational semantics for SS$\Fw$ in the following definition.
\begin{definition}
  \label{def:reduction_rules_ssf}
  Full $\beta$-reduction for SS$\Fw$:
  \begin{center}
    \begin{tabular}{lll}
      $(\Lambda X:K.t)[\phi]$     & $\rightsquigarrow$ & $[\phi/X]t$\\
      $(\lambda x:\phi.t)\ t'$      & $\rightsquigarrow$ & $[t'/x]t$\\
      $(\lambda X:*_l.\phi)\ \phi'$ & $\rightsquigarrow$ & $[\phi'/x]\phi$
    \end{tabular}
  \end{center}
\end{definition}

We can see that this system is almost the same as SSF except we have
added essentially STLC to the type level of the theory, and we now
allow quantification over predicates.  That is take
STLC where the terms are types and the types are kinds and that is
essentially what we have added to the type level of SSF to obtain
SS$\Fw$.  This allows us to compute types, so in terms of programming we
are able to write very generic programs where the type of the program
may need to be computed.  Next we define the kinding and typing rules for this
system.  Now the rules for well-formed contexts do not change from SSF so we
do not define them here.

\begin{definition}
  \label{def:l_kinding_rules_ssfw}
  The kind assignment rules for SS$\Fw$ are defined as follows:
  \begin{center}
    \begin{tabular}{cccc}
      \begin{math}
        $$\mprset{flushleft}
        \inferrule* [right=] {
          \Gamma \vdash \phi_1 : *_p
          \\
          \Gamma \vdash \phi_2 : *_q
        }{\Gamma \vdash \phi_1 \rightarrow \phi_2 : *_{max(p,q)}}
      \end{math}
      & 
      \begin{math}
        $$\mprset{flushleft}
        \inferrule* [right=] {
          \Gamma,X : *_q \vdash \phi : *_p
        }{\Gamma \vdash \forall X:*_q.\phi : *_{max(p,q)+1}}
      \end{math}
      &
      \begin{math}
        $$\mprset{flushleft}
        \inferrule* [right=] {
          \Gamma(X) = *_p
          \\
          p \leq q
          \\
          \Gamma\ Ok
        }{\Gamma \vdash X : *_q}
      \end{math} \\
      \\
      \begin{math}
        $$\mprset{flushleft}
        \inferrule* [right=] {
          \Gamma,X:K_1 \vdash \phi:K_2
        }{\Gamma \vdash \lambda X:K_1.\phi:K_1 \to K_2}
      \end{math}
      &
      \begin{math}
        $$\mprset{flushleft}
        \inferrule* [right=] {
          \Gamma \vdash \phi_1:K_1 \to K_2
          \\
          \Gamma \vdash \phi_2:K_1
        }{\Gamma \vdash \phi_1\ \phi_2:K_2}
      \end{math}
    \end{tabular}	
  \end{center}
\end{definition}
\noindent
To obtain SS$\Fw$ from SSF all we have to do is add the kinding rules
for type-level $\lambda$-abstractions and type-level application.  The
typing rules are the exactly same as SSF.  We state them next.
\begin{definition}
  \label{fig:typing_rules_ssfw}
  Type assignment rules for SS$\Fw$:
  \begin{center}
    \begin{tabular}{cccc}
      \begin{math}
        $$\mprset{flushleft}
        \inferrule* [right=] {
          \Gamma(x) = \phi
          \\
          \Gamma\ Ok
        }{\Gamma \vdash x : \phi}
      \end{math}  
      &
      \begin{math}
        $$\mprset{flushleft}
        \inferrule* [right=] {
          \Gamma,x : \phi_1 \vdash t : \phi_2
        }{\Gamma \vdash \lambda x : \phi_1.t : \phi_1 \rightarrow \phi_2}
      \end{math} 
      &
      \begin{math}
        $$\mprset{flushleft}
        \inferrule* [right=] {
          \Gamma \vdash t_1 : \phi_1 \rightarrow \phi_2 
          \\
          \Gamma \vdash t_2 : \phi_1
        }{\Gamma \vdash t_1\ t_2 : \phi_2}
      \end{math}  \\
      \\ 
      \begin{math}
        $$\mprset{flushleft}
        \inferrule* [right=] {
          \Gamma, X : *_p \vdash t : \phi
        }{\Gamma \vdash \Lambda X:*_p.t:\forall X : *_p.\phi}
      \end{math} 
      &
      \begin{math}
        $$\mprset{flushleft}
        \inferrule* [right=] {
          \Gamma \vdash t:\forall X:*_l.\phi_1
          \\
          \Gamma \vdash \phi_2:*_l
        }{\Gamma \vdash t[\phi_2]: [\phi_2/X]\phi_1}
      \end{math} 
      & \\
    \end{tabular}
  \end{center}
\end{definition}  
\noindent
Now all the basic syntactic lemmas proved for SSF in
Section~\ref{subsec:properties_of_the_hereditary_substitution_function}
hold for SS$\Fw$ so we do not state them here.
% subsection the_language (end)

\subsection{Concluding Normalization}
\label{subsec:concluding_normalization_ssfw}
Concluding normalization of SS$\Fw$ is a little bit tricky, because
the type level no longer consists of just constant types.  We conduct
the proof similarly to how H. Barendregt proves normalization of
System $\Fw$ in \cite{barendregt92} except we use hereditary
substitution where he uses a technique called reducibility.  Since
types contain computation we first must prove normalization of the
type level and then using this result we prove normalization for the
term (program) level.  

Now the type level is exactly STLC with constructors for function
types and universal types.  So all we have to do is define the
hereditary substitution function and then use the results from
Section~\ref{sec:normalization_the_simply_typed_lambda-calculus} to
obtain normalization.  Similarly, the term level of SS$\Fw$ is exactly
SSF so we do not need to define the hereditary substitution for
SS$\Fw$.  We can just use the results from
Section~\ref{subsec:normalization_stratified_system_f} to conclude
normalization.  Lets move onto defining the hereditary substitution
functions for the type level.  First we have to define the construct
kind function.  Note that $ckind_K$ is
the exact same as $ctype_\phi$ except at the type level.  Now all the
properties of $ctype_\phi$ are also properties of $ckind_K$ so we do
not prove them again here.  The construct type function is exactly the
same as SSF so we do not redefine it here.  Also the kind $K$ is
called the cut kind and is used in the exact same way as the cut type
$\phi$.

\begin{definition}
  \label{def:ctype_stlc}
  The $ckind_K$ function is defined with respect to a fixed kind $K$
  and has two arguments, a free variable $X$, and a type $\phi$ where $X$
  may be free in $\phi$.  We define $ckind_K$ by induction on the form of $\phi$.
\begin{itemize}
\item[] $ckind_K(X,X) = K$
  \item[]
  \item[] $ckind_K(X,\phi_1\ \phi_2) = K'$
  \item[] \ \ \ \ Where $ckind_K(X,\phi_1) = K'' \to K'$.
  \end{itemize}
\end{definition}
\noindent
Now we defined the type-level hereditary substitution function.
\begin{definition}
  \label{def:hereditary_substitution_function__type_level_ssfw}
  The following defines the hereditary substitution function for the type level of SS$\Fw$.  It
  is defined by recursion on the form of the term being substituted
  into and the cut kind $K$.
  \begin{itemize}
  \item[] $\{\phi/X\}^K X = \phi$
  \item[] $\{\phi/X\}^K Y = Y$
  \item[] \ \ \ \ Where $Y$ is a variable distinct from $X$.
  \item[] $\{\phi/X\}^K (\phi_1 \to \phi_2) = (\{\phi/X\}^K \phi_1) \to (\{\phi/X\}^K \phi_2)$
  \item[] $\{\phi/X\}^K (\forall Y:*_l.\phi') = \forall Y:*_l.\{\phi/X\}^K \phi'$
  \item[] $\{\phi/X\}^K (\lambda Y:K_1.\phi') = \lambda Y:K_1.(\{\phi / X\}^K \phi')$
  \item[] $\{\phi/X\}^K (\phi_1\ \phi_2) = (\{\phi/X\}^K \phi_1)\ (\{\phi/X\}^K \phi_2)$
  \item[] \ \ \ \ Where $(\{\phi/X\}^K \phi_1)$ is not a $\lambda$-abstraction, or both $(\{\phi/X\}^K \phi_1)$
    and $\phi_1$ are $\lambda$-abstractions,
  \item[] \ \ \ \ or $ckind_K(X,\phi_1)$ is undefined.
  \item[] $\{\phi/X\}^{K} (\phi_1\ \phi_2) = \{(\{\phi/X\}^{K} \phi_2)/y\}^{K''} \phi'_1$
  \item[] \ \ \ \ Where $(\{\phi/X\}^{K} \phi_1) \equiv \lambda Y:K''.\phi'_1$ 
    for some $Y$, $\phi'_1$, and $K''$ and $ckind_K(X,\phi_1) = K'' \to K'$.
  \end{itemize}
\end{definition}
Now all the properties of the hereditary substitution function from
Section~\ref{subsec:properties_of_the_hereditary_substitution_function}
are easily modified to obtain the exact same results for the function
just defined.  The statements of the properties do not change except
where there are terms we use types and where there are types we use
kinds so we do not repeat the properties here.  We now prove
normalization for the type-level of SS$\Fw$.  The following defines
the interpretation of types and the main substitution lemma.

\begin{definition}
  \label{def:interpretation_of_types_ssfw}
  First we define when a normal form is a member of the interpretation of type $\phi$ in context $\Gamma$
  \begin{center}
    \begin{math}
      \psi \in \interp{K}_\Gamma \iff \Gamma \vdash \phi:K,
  \end{math}
  \end{center}
  and this definition is extended to non-normal forms in the following way
  \begin{center}
    \begin{math}
      \phi \in \interp{K}_\Gamma \iff \phi \normto \psi \in \interp{K}_\Gamma,
  \end{math}
  \end{center}
 where we use $\psi$ to denote a normal form at the type level.
\end{definition}

\noindent 
Next we show that the definition of the interpretation of types is closed under
hereditary substitutions.  

\begin{lemma}[Substitution for the Interpretation of Types]
  If $\phi' \in \interp{K'}_{\Gamma,X:*_l,\Gamma'}$, $\phi \in \interp{K}_\Gamma$, then 
  $\{\phi/X\}^K \phi' \in \interp{K'}_{\Gamma,\Gamma'}$.
  
  \label{lemma:interpretation_of_types_closed_substitution_ssfw}
\end{lemma}

\noindent
Finally, by the definition of the interpretation of types the following result implies that the type-level
of SS$\Fw$ is normalizing.
\begin{thm}[Type Soundness]
  If $\Gamma \vdash \phi:K$ then $\phi \in \interp{K}_\Gamma$.
  \label{thm:soundness_ssfw}
\end{thm}

\begin{corollary}[Normalization]
  \label{coro:normalization_type_level_ssfw}
  If $\Gamma \vdash \phi:K$ then $\phi \normto \psi$.
\end{corollary}

Now we have to use the fact that we know the type level of SS$\Fw$ is
normalizing to conclude normalization of the term level.  First we
define the interpretation of types for the term level.

\begin{definition}
  \label{def:interpretation_of_types_stlc}
  First we define when a normal form is a member of the interpretation of normal type $\phi$ in context $\Gamma$
  \begin{center}
    \begin{math}
    n \in \interp{\phi}_\Gamma \iff \Gamma \vdash n:\phi,
  \end{math}
  \end{center}
  and this definition is extended to non-normal forms in the following way
  \begin{center}
    \begin{math}
    t \in \interp{\phi}_\Gamma \iff t \normto n \in \interp{\phi}_\Gamma.
  \end{math}
  \end{center}
\end{definition}

\noindent 
Next we show that the definition of the interpretation of types is closed under
hereditary substitutions.  

\begin{lemma}[Substitution for the Interpretation of Types]
  If $n' \in \interp{\phi'}_{\Gamma,x:\phi,\Gamma'}$, $n \in \interp{\phi}_\Gamma$, then 
  $[n/x]^\phi n' \in \interp{\phi'}_{\Gamma,\Gamma'}$.
  
  \label{lemma:interpretation_of_types_closed_substitution_stlc}
\end{lemma}

At this point we have to show that for any term $t$ of some type
$\phi$ no matter if $\phi$ is normal or not then $t$ reduces to a
normal form. We have to do this in two steps.  We first show that for
all typeable terms where their types are normal they reduces to a
normal form.  Then knowing this we show that for all typeable terms
where their types are not necessarily normal they reduce to a normal
form.  We first show the former.
\begin{thm}[Type Soundness Normal Types]
  If $\Gamma \vdash t:\phi$ and $\phi$ is normal then $t \in \interp{\phi}_\Gamma$.
  \label{thm:soundness__normal_ssfw}
\end{thm}
\noindent
The following lemmas are used in the proof of the type soundness lemma
which shows that all typeable terms are in the interpretation of the
normal form of their type.
\begin{lemma}[Preservation of Types]
  \label{lemma:preservation_for_kinding_ssfw}
  \begin{itemize}
  \item[]
  \item[i.] If $(\Gamma,x:\phi,\Gamma')\ Ok$ and $\phi \redto \phi'$ then $(\Gamma,x:\phi',\Gamma')\ Ok$.
  \item[ii.] If $\Gamma \vdash \phi:K$ and $\phi \redto \phi'$ then there exists a $\Gamma'$ such that $\Gamma' \vdash \phi':K$.
  \end{itemize}
\end{lemma}

\begin{lemma}
  \label{lemma:preservation_2}
  If $\Gamma \vdash t:\phi$ and $\phi \redto \phi'$ then there exists a $\Gamma'$ such that $\Gamma' \vdash t:\phi'$.
\end{lemma}

\noindent
Finally, we conclude normalization by showing type soundness.
\begin{thm}[Type Soundness]
  If $\Gamma \vdash t:\phi$ then $\phi \normto \psi$, and there exists a $\Gamma'$ such that $t \in \interp{\psi}_{\Gamma'}$.
  \label{thm:soundness_term_ssfw}
\end{thm}
\begin{proof}
  By regularity we know $\Gamma \vdash \phi:K$ for some kind $K$ and by Corollary~\ref{coro:normalization_type_level_ssfw}
  there exists a normal type $\psi$ such that $\phi \normto \psi$.  Finally,
  by Lemma~\ref{lemma:preservation_2} there exists a $\Gamma'$ such that $\Gamma' \vdash t:\psi$.  Thus, by Theorem~\ref{thm:soundness__normal_ssfw} 
  $t \in \interp{\psi}_{\Gamma'}$.
\end{proof}

\begin{corollary}[Normalization]
  If $\Gamma \vdash t:\phi$ then $t \normto n$.
\end{corollary}
% subsection concluding_normalization (end)
% subsection normalization_stratified_system_fw (end)
% section normalization_by_hereditary_substitution (end)