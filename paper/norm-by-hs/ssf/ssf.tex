\section{Normalization of Stratified System F (SSF)}
\label{subsec:normalization_stratified_system_f}
In \cite{Eades:2010} Harley Eades and Aaron Stump show that SSF is
normalizing using a proof method which uses the hereditary
substitution function implicitly.  We find it apparent that the
hereditary substitution technique we have introduced in
Section~\ref{sec:hereditary_substitution} is easier to understand,
use, and is more informative -- with respect to the hereditary
substitution function and its interaction with the type theory -- than
the implicit version of the proof method.  So in this section we
reprove normalization of SSF using the hereditary function explicitly.
This will set the stage for the later chapters which extend SSF with
various features and then reprove normalization using hereditary
substitution.  For a brief introduction to SSF and its history see
Section~\ref{sec:moderen_type_theory}.

Stratified System F, consists of types which are stratified into
levels (or ranks) based on type-quantification.  The types that belong
to level zero have no type-quantification, the types at level one only
quantify over types of level zero, and the types at level $n$ quantify
over the types of level $n-1$.  Stratifying System F into levels
prevents impredicativity.  That is a type $\forall X.T'$ is no
longer allowed quantify over itself.  This restriction is similar to
Russell's simple theory of types.

First, we briefly reintroduce SSF for the readers convenience.  The
syntax for Stratified System F can be found in the next definition
followed by the definition of the reduction rules stated as rewrite
rules where we omit the congruence rules.
\begin{definition}
  \label{def:syntax_ssf}
  The syntax for terms, types, and kinds:
  \begin{center}
    \begin{tabular}{lll}
      $K$ & $:=$ & $*_0$ $|$ $*_1$             $|$ $\ldots$\\
      $T$ & $:=$ & $X$   $|$ $T \rightarrow T$ $|$ $\forall X:K.T$\\
      $t$ & $:=$ & $x$   $|$ $\lambda x:T.t$   $|$ $t\ t$ $|$ $\Lambda X:K.t$ $|$ $t[T]$\\
    \end{tabular}
  \end{center}
\end{definition}

\begin{definition}
  \label{def:reduction_rules_ssf}
  Full $\beta$-reduction for SSF:
  \begin{center}   
    \begin{tabular}{rll}
      $(\Lambda X:*_p.t)[T]$ & $\rightsquigarrow$ & $[T/X]t$\\
      $(\lambda x:T.t)t'$    & $\rightsquigarrow$ & $[t'/x]t$
    \end{tabular}
  \end{center}
\end{definition}
\noindent
Both the kinding and typing relations depend on well-formed contexts
which is defined next.
\begin{definition}
  \label{fig:well-formed_ssf}
  Context well-formedness rules:
  \begin{mathpar}    
      \inferrule* [right=] {
        \   
      }{\cdot\ Ok}      
      \and
      \inferrule* [right=] {
        \Gamma\ Ok
      }{\Gamma,X:*_p\ Ok}
      \and
      \inferrule* [right=] {
        \Gamma \vdash T:*_p
        \\
        \Gamma\ Ok
      }{\Gamma,x :T\ Ok}
      \end{mathpar}   
\end{definition}
\noindent
As stated before we use kinds to denote the level of a type.  The
following defines kinding relation:
\begin{definition}
  \label{fig:l_kinding_rules}
  The kind assignment rules for SSF are defined as follows:
  \begin{mathpar}
    \inferrule* [right=] {
      \Gamma \vdash T_1 : *_p
      \\
      \Gamma \vdash T_2 : *_q
    }{\Gamma \vdash T_1 \rightarrow T_2 : *_{max(p,q)}}
    \and
    \inferrule* [right=] {
      \Gamma,X : *_q \vdash T : *_p
    }{\Gamma \vdash \forall X:*_q.T : *_{max(p,q)+1}}
    \and
    \inferrule* [right=] {
      \Gamma(X) = *_p
      \\
      p \leq q
      \\
      \Gamma\ Ok
    }{\Gamma \vdash X : *_q}
  \end{mathpar} 
\end{definition}
\noindent
These kind assignement rules are slightly different then the level
assignment rules defined by Leivant.  The following rule:
\[
   \inferrule* [right=] {
     \Gamma,X : *_q \vdash T : *_p
   }{\Gamma \vdash \forall X:*_q.T : *_{max(p,q)+1}}
\]
was orignially defined to be the following by Leivant:
\[
   \inferrule* [right=] {
     \Gamma,X : *_q \vdash T : *_p
   }{\Gamma \vdash \forall X:*_q.T : *_{max(p+1,q)}}
\]
We conjecture that this modification does not hinder the expressive
power of the type theory, meaning, any typeable term of Leivant's
system is typeable in ours, but at potentially higher level.  However,
we do not prove this here.  The following lemma shows that all
kindable types are kindable with respect to a well-formed context.
\begin{lemma}
  If $\Gamma \vdash T:*_p$ then $\Gamma\ Ok$.
  \label{lemma:kinding_ok_ssf}
\end{lemma}
\begin{proof}
  This is a proof by structural induction on the kinding derivation of $\Gamma \vdash T:*_p$.
  \begin{itemize}
  \item[Case.]\ \\
    \begin{center}
      \begin{math}
        $$\mprset{flushleft}
        \inferrule* [right=] {
          \Gamma(X) = *_p
          \\
          p \leq q
          \\
          \Gamma\ Ok
        }{\Gamma \vdash X : *_q}
      \end{math}
    \end{center}
    By inversion of the kind-checking rule $\Gamma\ Ok$.

  \item[Case.]\ \\
    \begin{center}
      \begin{math}
        $$\mprset{flushleft}
        \inferrule* [right=] {
          \Gamma \vdash T_1 : *_p
          \\
          \Gamma \vdash T_2 : *_q
        }{\Gamma \vdash T_1 \rightarrow T_2 : *_{max(p,q)}}
      \end{math}
    \end{center}
    By the induction hypothesis, $\Gamma \vdash T_1:*_p$ and
    $\Gamma \vdash T_2:*_q$ both imply $\Gamma\ Ok$.  Since the
    arrow-type kind-checking rule does not modify $\Gamma$ in anyway
    $\Gamma$ will remain $Ok$.

  \item[Case.]\ \\
    \begin{center}
      \begin{math}
        $$\mprset{flushleft}
        \inferrule* [right=] {
          \Gamma,X : *_q \vdash T : *_p
        }{\Gamma \vdash \forall X:*_q.T : *_{max(p,q)+1}}
      \end{math}
    \end{center}
    By the induction hypothesis $\Gamma,X:*_p\ Ok$, and by inversion of the type-variable
    well-formed contexts rule $\Gamma\ Ok$.
  \end{itemize}
\end{proof}
\noindent
The previous lemma ensures that if a type is kindable then we could
not have used a ``bogus'' context where we assume something we are not
allowed to assume.  Now we define the typing relation:
\begin{definition}
  \label{fig:typing_rules_ssf}
  Type assignment rules for SSF:
    \begin{mathpar}
      \inferrule* [right=] {
        \Gamma(x) = T
        \\
        \Gamma\ Ok
      }{\Gamma \vdash x : T}
      \and
      \inferrule* [right=] {
        \Gamma,x : T_1 \vdash t : T_2
      }{\Gamma \vdash \lambda x : T_1.t : T_1 \rightarrow T_2}
      \and
      \inferrule* [right=] {
        \Gamma \vdash t_1 : T_1 \rightarrow T_2 
        \\
        \Gamma \vdash t_2 : T_1
      }{\Gamma \vdash t_1\ t_2 : T_2}
      \and
      \inferrule* [right=] {
        \Gamma, X : *_p \vdash t : T
      }{\Gamma \vdash \Lambda X:*_p.t:\forall X : *_p.T}
      \and
      \inferrule* [right=] {
        \Gamma \vdash t:\forall X:*_l.T_1
        \\
        \Gamma \vdash T_2:*_l
      }{\Gamma \vdash t[T_2]: [T_2/X]T_1}
    \end{mathpar} 
\end{definition}  
\noindent
The type assignment rules depend on the kinding relation defined
above.  Note that the level of the type $T$ and the level of the
type variable $X$ in the type application rule must be the same.
% subsection the_language (end)

\subsection{Basic Syntactic Lemmas}
\label{subsec:basic_syntactic_lemmas}
We now state several results about the kinding relation. All of these
are just basic results needed by the proofs of the key results
later. The reader may wish to just quickly read through them.  We
simply list them with their proofs.  Briefly,
Lemma~\ref{lemma:substitution_for_kinding_ssf} is used in the proof of
Substitution for Typing (Lemma~\ref{lemma:type_sub_ssf}),
Lemma~\ref{lemma:context_strengthening_for_kinding_ssf} is used in the
main substitution lemma
(Lemma~\ref{lemma:interpretation_of_types_closed_substitution_ssf}),
and Lemma~\ref{lemma:context_weakening_for_kinding_ssf} is used in the
proof of Context Weakening for the Interpretation of Types
(Lemma~\ref{lemma:context_weakening_interpretations_ssf}).

\begin{lemma}[Level Weakening for Kinding]
  If $\Gamma \vdash T:*_r$ and $r < s$ then $\Gamma \vdash T:*_s$.
  \label{lemma:level_weakening_for_kinding_ssf}
\end{lemma}
\begin{proof}
  We show level weakening for kinding by structural induction on the kinding derivation 
  of $T:*_r$.
  \begin{itemize}
  \item[Case.]\ \\
    \begin{center}
      \begin{math}
        $$\mprset{flushleft}
        \inferrule* [right=] {
          \Gamma(X) = *_p
          \\
          p \leq q
          \\
          \Gamma\ Ok
        }{\Gamma \vdash X : *_q}
      \end{math}
    \end{center}
    By assumption we know $q < s$, hence by reapplying the rule and transitivity we 
    obtain $\Gamma \vdash X:*_s$.
    
  \item[Case.]\ \\
    \begin{center}
      \begin{math}
        $$\mprset{flushleft}
        \inferrule* [right=] {
          \Gamma \vdash T_1 : *_p
          \\
          \Gamma \vdash T_2 : *_q
        }{\Gamma \vdash T_1 \rightarrow T_2 : *_{max(p,q)}}
      \end{math}
    \end{center}
    By the induction hypothesis $\Gamma \vdash T_1 : *_s$ and 
    $\Gamma \vdash T_2 : *_s$ for some arbitrary $s > max(p,q)$.  Therefore, by 
    reapplying the rule we obtain $\Gamma \vdash T_1 \rightarrow T_2 : *_s$.
    
  \item[Case.]\ \\
    \begin{center}
      \begin{math}
        $$\mprset{flushleft}
        \inferrule* [right=] {
          \Gamma,X : *_q \vdash T' : *_p
        }{\Gamma \vdash \forall X:*_q.T' : *_{max(p,q)+1}}
      \end{math}
    \end{center}
    We know by assumption that $max(p,q) + 1 < s$ which implies that $max(p,q) < s - 1$.  
    Now by the induction hypothesis $\Gamma,X:*_q \vdash T':*_{s-1}$.  Lastly, we 
    reapply the rule and obtain $\Gamma \vdash \forall X:*_q.T' : *_{s}$.
  \end{itemize}
\end{proof}

\begin{lemma}[Substitution for Kinding, Context-Ok]
  Suppose $\Gamma \vdash T':*_p$.  If $\Gamma,X:*_p,\Gamma' \vdash T:*_q$ 
  with a derivation of depth $d$, then $\Gamma,[T'/X]\Gamma' \vdash [T'/X]T:*_q$
  with a derivation of depth $d$.
  Also, if $\Gamma,X:*_p,\Gamma'\ Ok$ with a derivation of depth $d$, then 
  $\Gamma,[T'/X]\Gamma'\ Ok$ with a derivation of depth $d$.
  \label{lemma:substitution_for_kinding_ssf}
\end{lemma}
\begin{proof}
  This is a prove by induction on $d$.  We prove the first implication first, and then the 
  second, doing a case analysis for each implication on the form of the derivation whose depth 
  is being considered. 

  \begin{itemize}
  \item[Case.]\ \\
    \begin{center}
      \begin{math}
        $$\mprset{flushleft}
        \inferrule* [right=] {
          (\Gamma,X:*_p,\Gamma')(Y) = *_r
          \\
          r \leq s
          \\
          \Gamma,X:*_p,\Gamma'\ Ok
        }{\Gamma,X:*_p,\Gamma' \vdash Y : *_s}
      \end{math}
    \end{center}
    By assumption $\Gamma \vdash T':*_p$.  We must consider whether or not 
    $X \equiv Y$.  If $X \equiv Y$ then $[T'/X]Y \equiv T'$, $r = p$, and $q = s$; 
    this conclusion is equivalent to $\Gamma,[T'/X]\Gamma' \vdash T':*_q$ and
    by the induction hypothesis $\Gamma,[T'/X]\Gamma'\ Ok$.  
    If $X \not \equiv Y$ then $[T'/X]Y \equiv Y$ and 
    by the induction hypothesis $\Gamma,[T'/X]\Gamma'\ Ok$, hence,
    $\Gamma,[T'/X]\Gamma' \vdash Y:*_q$.\\
    
  \item[Case.]\ \\
    \begin{center}
      \begin{math}
        $$\mprset{flushleft}
        \inferrule* [right=] {
          \Gamma,X:*_p,\Gamma' \vdash T_1 : *_r
          \\
          \Gamma,X:*_p,\Gamma' \vdash T_2 : *_s
        }{\Gamma,X:*_p,\Gamma' \vdash T_1 \rightarrow T_2 : *_{max(r,s)}}
      \end{math}
    \end{center}
    Here $q = max(r,s)$ and by the induction hypothesis $\Gamma,[T'/X]\Gamma' \vdash $
    $[T'/X]T_1:*_r$ and $\Gamma,[T'/X]\Gamma' \vdash [T'/X]T_2:*_s$.
    We can now reapply the rule to get 
    $\Gamma,[T'/X]\Gamma' \vdash [T'/X](T_1 \rightarrow T_2) : *_q$.\\
    
  \item[Case.]\ \\
    \begin{center}
      \begin{math}
        $$\mprset{flushleft}
        \inferrule* [right=] {
          \Gamma,X : *_q,\Gamma',Y:*_r \vdash T : *_s
        }{\Gamma,X:*_p,\Gamma' \vdash \forall Y:*_r.T : *_{max(r,s)+1}}
      \end{math}
    \end{center}
    Here $q = max(r,s) + 1$ and by the induction hypothesis 
    $\Gamma,[T'/X]\Gamma',Y:*_r \vdash [T'/X]T:*_s$.  We can reapply this rule to get 
    $\Gamma,[T'/X]\Gamma' \vdash [T'/X] \forall Y:*_r.T:*_q$.
  \end{itemize}

  \noindent We now show the second implication.
  The case were $d = 0$ cannot arise, since it requires the context to be empty.
  Suppose $d = n + 1$.  We do a case analysis on the last rule applied in the derivation of 
  $\Gamma,X:*_p,\Gamma'$.
  \begin{itemize}
  \item[Case.] Suppose $\Gamma' = \Gamma'',Y:*_q$.\ \\
    \begin{center}
      \begin{math}
        $$\mprset{flushleft}
        \inferrule* [right=] {
          \Gamma,X:*_p,\Gamma''\ Ok
        }{\Gamma,X:*_p,\Gamma'',Y:*_q\ Ok}
      \end{math}
    \end{center}
    By the induction hypothesis, $\Gamma,[T'/X]\Gamma''\ Ok$.  Now, by reapplying
    the rule above $\Gamma,[T'/X]\Gamma'',Y:*_q\ Ok$, hence $\Gamma,[T'/X]\Gamma'\ Ok$, 
    since $X \not \equiv Y$.
  \item[Case.]Suppose $\Gamma' = \Gamma'',y:T$.\ \\
    \begin{center}
      \begin{math}
        $$\mprset{flushleft}
        \inferrule* [right=] {
          \Gamma,X:*_p,\Gamma'' \vdash T:*_q
          \\
          \Gamma,X:*_p,\Gamma''\ Ok
        }{\Gamma,X:*_p,\Gamma'',y :T\ Ok}
      \end{math} 
    \end{center}
    By the induction hypothesis, $\Gamma',[T'/X]\Gamma'' \vdash [T'/X]T:*_q$ and 
    $\Gamma',[T'/X]\Gamma''\ Ok$.  Thus,
    by reapplying the rule above $\Gamma,[T'/X]\Gamma'',x:[T'/X]T\ Ok$, therefore,\\
    $\Gamma,[T'/X]\Gamma'\ Ok$.
  \end{itemize}
\end{proof}

\begin{lemma}[Context Strengthening for Kinding, Context-Ok]
  If $\Gamma,x:T',\Gamma' \vdash T:*_p$ with a derivation of depth $d$, then 
  $\Gamma,\Gamma' \vdash T:*_p$ with a derivation of depth $d$.  Also, if 
  $\Gamma,x:T,\Gamma'\ Ok$ with a 
  derivation of depth $d$, then $\Gamma,\Gamma'\ Ok$ with a derivation of depth $d$.
  \label{lemma:context_strengthening_for_kinding_ssf}
\end{lemma}
\begin{proof}
  This is a prove by induction on $d$.  We prove the first implication first, and then the 
  second, doing a case analysis for each implication on the form of the derivation whose depth 
  is being considered.

  \begin{itemize}
  \item[Case.]\ \\
    \begin{center}
      \begin{math}
        $$\mprset{flushleft}
        \inferrule* [right=] {
          (\Gamma,x:T',\Gamma')(X) = *_p
          \\
          p \leq q
          \\
          \Gamma,x:T',\Gamma'\ Ok
        }{\Gamma,x:T',\Gamma' \vdash X : *_q}
      \end{math}
    \end{center}
    
    By the second implication of the induction hypothesis, $\Gamma,\Gamma'\ Ok$. Also, 
    $(\Gamma,\Gamma')(X) = *_p$.  Now by reapplying the rule above, 
    $\Gamma,\Gamma' \vdash X:*_q$.
    
  \item[Case.]\ \\
    \begin{center}
      \begin{math}
        $$\mprset{flushleft}
        \inferrule* [right=] {
          \Gamma,x:T',\Gamma' \vdash T_1 : *_p
          \\
          \Gamma,x:T',\Gamma' \vdash T_2 : *_q
        }{\Gamma,x:T',\Gamma' \vdash T_1 \rightarrow T_2 : *_{max(p,q)}}
      \end{math}
    \end{center}
    
    By the first implication of the induction hypothesis, $\Gamma,\Gamma' \vdash T_1:*_p$ and 
    $\Gamma,\Gamma' \vdash T_2:*_q$.  By reapplying the rule above we get, 
    $\Gamma,\Gamma' \vdash T_1 \rightarrow T_2:*_{max(p,q)}$.
    
  \item[Case.]\ \\
    \begin{center}
      \begin{math}
        $$\mprset{flushleft}
        \inferrule* [right=] {
          \Gamma,x:T,\Gamma',Y:*_q \vdash T : *_p
        }{\Gamma,x:T',\Gamma' \vdash \forall Y:*_q.T : *_{max(p,q)+1}}
      \end{math}
    \end{center}
    By the first implication of the induction hypothesis, $\Gamma,\Gamma',Y:*_q \vdash T:*_p$.
    By reapplying the rule we get, $\Gamma,\Gamma' \vdash \forall Y:*_q.T:*_{max(p,q)+1}$.
  \end{itemize}

  \noindent We now prove the second implication.
  The case where $d = 0$ cannot arise, since it requires the context to be empty.
  Suppose $d = n + 1$.  We do a case analysis on the last rule applied in the derivation of
  $\Gamma,x:T,\Gamma'\ Ok$.
  \begin{itemize}
  \item[Case.]  Suppose $\Gamma' = \Gamma'',Y:*_l$.  Then the last rule of the derivation of
    $\Gamma,x:T,\Gamma'\ Ok$ is as follows.
    \begin{center}
      \begin{math}
        $$\mprset{flushleft}
        \inferrule* [right=] {
          \Gamma,x:T,\Gamma''\ Ok
        }{\Gamma,x:T,\Gamma'',Y:*_l\ Ok}
      \end{math}
    \end{center}
    By the second implication of the induction hypothesis, $\Gamma,\Gamma''\ Ok$.  Now 
    reapplying the rule we get, $\Gamma,\Gamma'',Y:*_l\ Ok$, which is equivalent to 
    $\Gamma,\Gamma'\ Ok$.
    
  \item[Case.]  Suppose $\Gamma' = \Gamma'',y:T'$.  Then the last rule of the derivation of
    $\Gamma,x:T,\Gamma'\ Ok$ is as follows.
    \begin{center}
      \begin{math}
        $$\mprset{flushleft}
        \inferrule* [right=] {
          \Gamma,x:T,\Gamma'' \vdash T':*_p
          \\
          \Gamma,x:T,\Gamma''\ Ok
        }{\Gamma,x:T,\Gamma'',y :T'\ Ok}
      \end{math} 
    \end{center}
    By the first implication of the induction hypothesis, $\Gamma,\Gamma'' \vdash T':*_p$ and 
    by the second, $\Gamma,\Gamma''\ Ok$.
    Therefore, by reapplying the rule above, $\Gamma,\Gamma'',y:T'\ Ok$, which is equivalent 
    to $\Gamma,\Gamma'\ Ok$.
  \end{itemize}
\end{proof}

\begin{lemma}[Context Weakening for Kinding]
  If $\Gamma,\Gamma'',\Gamma'\ Ok$, and $\Gamma,\Gamma' \vdash
  T:*_p$, then $\Gamma,\Gamma'',\Gamma' \vdash T:*_p$.
  \label{lemma:context_weakening_for_kinding_ssf}
\end{lemma}
\begin{proof}
  This is a proof by structural induction on the kinding derivation of 
$\Gamma,\Gamma' \vdash T:*_p$.
\begin{itemize}
\item[Case.]\ \\
  \begin{center}
    \begin{math}
      $$\mprset{flushleft}
      \inferrule* [right=] {
        (\Gamma,\Gamma')(X) = *_p
	\\
	p \leq q
	\\
	\Gamma,\Gamma'\ Ok
      }{\Gamma,\Gamma' \vdash X : *_q}
    \end{math}
  \end{center}
  If $(\Gamma,\Gamma')(X) = *_p$ then $(\Gamma,\Gamma'',\Gamma')(X) = *_p$, hence, by 
  reapplying the type-variable kind-checking rule, $\Gamma,\Gamma'',\Gamma' \vdash T:*_p$.
\item[Case.]\ \\
  \begin{center}
    \begin{math}
      $$\mprset{flushleft}
      \inferrule* [right=] {
        \Gamma,\Gamma' \vdash T_1 : *_p
	\\
	\Gamma,\Gamma' \vdash T_2 : *_q
      }{\Gamma,\Gamma' \vdash T_1 \rightarrow T_2 : *_{max(p,q)}}
    \end{math}
  \end{center}
  By the induction hypothesis $\Gamma,\Gamma'',\Gamma' \vdash T_1:*_p$ and 
  $\Gamma,\Gamma'',\Gamma' \vdash T_2:*_q$,
  hence, by reapplying the arrow-type kind-checking rule 
  $\Gamma,\Gamma'',\Gamma'' \vdash T_1 \rightarrow T_2:*_{max(p,q)}$.

\item[Case.]\ \\
  \begin{center}
    \begin{math}
      $$\mprset{flushleft}
      \inferrule* [right=] {
        \Gamma,\Gamma',X : *_q \vdash T' : *_p
      }{\Gamma,\Gamma' \vdash \forall X:*_q.T' : *_{max(p,q)+1}}
    \end{math}
  \end{center}
  By the induction hypothesis $\Gamma,\Gamma'',\Gamma',X:*_p \vdash T:*_q$, hence, by 
  reapplying the forall-type
  kind-checking rule $\Gamma,\Gamma'',\Gamma' \vdash \forall X:*_p.T:*_{max(p,q)+1}$.
\end{itemize}
\end{proof}

\begin{lemma}[Regularity]
  If $\Gamma \vdash t:T$ then $\Gamma \vdash T:*_p$ for some $p$.
  \label{lemma:regularity_ssf}
\end{lemma}
\begin{proof}
  This proof is by structural induction on the derivation of $\Gamma \vdash t:T$.
\begin{itemize}
\item[Case.] \ \\
  \begin{center}
    \begin{math}
      $$\mprset{flushleft}
      \inferrule* [right=] {
        \Gamma(x) = T
        \\
        \Gamma\ Ok
      }{\Gamma \vdash x : T}
    \end{math}  
  \end{center}
  By the definition of well-formedness contexts $\Gamma \vdash T:*_p$ for some $p$.
  
\item[Case.] \ \\
  \begin{center}
    \begin{math}
      $$\mprset{flushleft}
      \inferrule* [right=] {
        \Gamma,x : T_1 \vdash t : T_2
      }{\Gamma \vdash \lambda x : T_1.t : T_1 \rightarrow T_2}
    \end{math}
  \end{center}
  By the induction hypothesis $\Gamma \vdash T_1:*_p$,
  $\Gamma,x:T_1 \vdash T_2:*_q$ and by Lemma~\ref{lemma:context_strengthening_for_kinding_ssf},
  $\Gamma \vdash T_2:*_q$.
  By applying the arrow-type kind-checking rule we get 
  $\Gamma \vdash T_1 \rightarrow T_2:*_{max(p,q)}$.
  
\item[Case.] \ \\
  \begin{center}
    \begin{math}
      $$\mprset{flushleft}
      \inferrule* [right=] {
        \Gamma \vdash t_1 : T_1 \rightarrow T_2 
        \\
        \Gamma \vdash t_2 : T_1
      }{\Gamma \vdash t_1\ t_2 : T_2}
    \end{math}
  \end{center}
  By the induction hypothesis $\Gamma \vdash T_1 \rightarrow T_2:*r$ and
  $\Gamma \vdash T_1:*_p$.  By inversion of the arrow-type kind-checking rule 
  $r = max(p,q)$, for some $q$, which implies $\Gamma \vdash T_2:*_q$.
  
\item[Case.] \ \\
  \begin{center}
    \begin{math}
      $$\mprset{flushleft}
      \inferrule* [right=] {
        \Gamma, X : *_p \vdash t : T
      }{\Gamma \vdash \Lambda X:*_p.t:\forall X : *_q.T}
    \end{math}
  \end{center}
  By the induction hypothesis $\Gamma,X:*_q \vdash T:*_p$.  By applying
  the forall-type kind-checking rule $\Gamma \vdash \forall X.T:*_{max(p,q)+1}$.
  
\item[Case.] \ \\
  \begin{center}
    \begin{math}
      $$\mprset{flushleft}
      \inferrule* [right=] {
        \Gamma \vdash t:\forall X:*_p.T_1
        \\
        \Gamma \vdash T_2:*_p
      }{\Gamma \vdash t[T_2]: [T_2/X]T_1}
    \end{math}
  \end{center}
  By assumption $\Gamma \vdash T_2:*_r$.  By the induction hypothesis 
  $\Gamma \vdash \forall X:*_p.T_1:*_s$ and by inversion of the forall-type
  kind-checking rule $r = max(p,q)+1$, for some $q$, which implies 
  $\Gamma,X:*_p \vdash T_1:*_q$.  Now, by Lemma~\ref{lemma:substitution_for_kinding_ssf},
  $\Gamma \vdash [T_2/X]T_1:*_q$.
\end{itemize}
\end{proof}
\noindent
We have stated all the basic lemmas we will need.  We now proceed to
the proof of normalization for SSF using hereditary substitution.
% subsection basic_syntactic_lemmas (end)

\subsection{Well-Founded Ordering on Types}
\label{subsec:well-founded_ordering_on_types_ssf}
The following definition defines a well-founded ordering on the types
of SSF.  It consists of essentially the strict-subexpression ordering
with an additional case for universal types.  For the case of
universal types the ordering states that a they are always larger than
their instantiation.  Now this seems odd, because syntactically the
instantiation could have increased, but it turns out that the level of
the type actually decreases.  That is, we know the level of the
universal type is larger than the level of the instantiation.
\begin{definition}
  The ordering $>_\Gamma$ is defined as the least relation satisfying the universal closures of 
  the following formulas:
  \begin{center}
    \begin{tabular}{lll}
      \begin{tabular}{lll}
        $T_1 \rightarrow T_2$ & $>_\Gamma$ & $T_1$\\
        $T_1 \rightarrow T_2$ & $>_\Gamma$ & $T_2$\\
        $\forall X:*_l.T$        & $>_\Gamma$ & $[T'/X]T$ where 
        $\Gamma \vdash T':*_l$.\\
      \end{tabular}
    \end{tabular}
  \end{center}
  \label{def:ordering_ssf}
\end{definition}

\noindent
We need transitivity in a number of places so we state that next.

\begin{lemma}[Transitivity of $>_\Gamma$]
  Let $T$, $T'$, and $T''$ be kindable types.  If $T >_\Gamma T'$ and 
  $T' >_\Gamma T''$ then $T >_\Gamma T''$.
  \label{lemma:transitivity_ssf}
\end{lemma}
\begin{proof}
  Suppose $T >_\Gamma T'$ and $T' >_\Gamma T''$.  If 
  $T \equiv T_1 \rightarrow T_2$ then,
  $T'$ must be a subexpression of $T$.  Now if $T' \equiv T'_1 \rightarrow T'_2$ 
  then,
  $T''$ must be a subexpression of $T'$, which implies that $T''$ is a subexpression of 
  $T$.  Thus, $T >_\Gamma T''$.
  If $T' \equiv \forall X:*_l.T'_1$ then, there exists a type $T'_2$ where, 
  $\Gamma \vdash T'_2:*_l$, such that, 
  $T'' \equiv [T'_2/X]T'_1$.  The level of $T'$ is $max(l,l')+1$, where $l'$ is the 
  level of $T'_1$, the level of 
  $T''$ is $max(l,l')$, and the level of $T$ is $max(max(l,l')+1,p)$, where $p$ is the 
  level of the type, which is, the second subexpression of $T$.
  Clearly, $max(max(l,l')+1,p) \geq max(l,l')$, thus, $T >_\Gamma T''$.

  If $T \equiv \forall X:*_l.T_1$, then $T' \equiv [T_2/X]T_1$ for some type 
  $T_2$, where $\Gamma \vdash T_2:*_l$.  If
  $[T_2/X]T_1 \equiv T'_1 \rightarrow T'_2$ then the level of $T'$ is $max(p,q)$, where 
  $p$ is the level
  of $T'_1$ and $q$ is the level of $T'_2$.  Now $T''$ must be a subexpression of 
  $T'$, hence the level of $T''$ is either $p$ or $q$.  Now, since
  the level of $T$ is greater than the level of $T'$ and we know, $max(p,q)$ is greater 
  than both $p$ and $q$ then $T >_\Gamma T''$.  If 
  $[T_2/X]T_1 \equiv \forall Y:*_{l'}.T'_1$, then $T'' \equiv [T'_2/X]T'_1$.  
  Now if $p$ is the level of $T_1$, then the level of $T$ is
  $max(l,p)+1$ and the level of $T'$ must be $max(l,p)$ since we know the level of $T'$ is 
  greater than the level of $T''$ then clearly, the level of 
  $T$ is greater than the level of $T''$.  Thus, $T >_\Gamma T''$.
\end{proof}
\noindent
To prove that the ordering on types ($>_\Gamma$) is well founded we need
a function which computes the depth of a type.  We will use this in a
lexicographic ordering in the proof of
Lemma~\ref{lemma:well-founded_measure_ssf} and is vital to showing
that our ordering on types is well founded.

\begin{definition}
  The depth of a type $T$ is defined as follows:
  \begin{center}
    \begin{tabular}{lll}
      $depth(X)$                  & $=$ & $1$\\
      $depth(T \to T')$     & $=$ & $depth(T) + depth(T')$\\
      $depth(\forall X:*_l.T)$ & $=$ & $depth(T) + 1$\\
    \end{tabular}
  \end{center}
\end{definition}

We define the following metric $(l,d)$ in lexicographic combination,
where $l$ is the level of a type $T$ and $d$ is the depth of
$T$.  The following lemma shows that if $T >_\Gamma T'$ then
$(l,d) > (l',d')$.  We will use this lemma to show well-foundedness of
the ordering on types $>_\Gamma$.  

\begin{lemma}[Well-Founded Measure]
  \label{lemma:well-founded_measure_ssf}
  If $T >_\Gamma T'$ then $(l,d) > (l',d')$, where $\Gamma \vdash T:*_l$, 
  $depth(T) = d$,  $\Gamma \vdash T:*_{l'}$, and $depth(T') = d'$.
\end{lemma}
\begin{proof}
  Assume $T >_\Gamma T'$ for some types $T$ and $T'$.  We case split on
  the form of $T$.  Clearly, $T$ is not a type variable.
  \begin{itemize}
  \item[Case.]  Suppose $T \equiv T_1 \to T_2$.  Then $T'$ must be of the 
    form $T_1$ or $T_2$.  In both
    cases we have two cases to
    consider; either $T$ and $T'$ have the same level or they do not.  Consider the
    first form and suppose they have the same level.  Then it is clear that 
    $depth(T) > depth(T')$.  Now consider the latter form and suppose
    $T$ and $T'$ have the same level.  Then, clearly, $depth(T) > depth(T')$.  
    In either form if the level of $T$ and $T'$ are different, then the level of 
    $T$ is larger than the level of $T'$.  In all cases $(l,d) > (l',d')$.
    
  \item[Case.] Suppose $T \equiv \forall X:*_l.T_1$.  Then $T'$ must be of the
    form $[T_2/X]T_1$ for some type $\Gamma \vdash T_2:*_l$.  It is obvious that
    the level of $T$ is always larger than the level of $T'$.  Hence, $(l,d) > (l',d')$.
  \end{itemize}
\end{proof}
\noindent We now have the desired results to prove that the ordering $>_\Gamma$ is 
well-founded.

\begin{thm}[Well-Founded Ordering]
  The ordering $>_\Gamma$ is well-founded on types $T$ such that 
  $\Gamma \vdash T:*_l$ for some $l$.
  \label{thm:well-founded_ordering_ssf}
\end{thm}
\begin{proof}
  If there exists a infinite decreasing sequence using our ordering on types,
  then there is an infinite decreasing sequence using our measure by 
  Lemma~\ref{lemma:well-founded_measure_ssf}, but that is impossible.
\end{proof}
% subsection well-founded_ordering_on_types (end)

\subsection{The Hereditary Substitution Function}
\label{subsec:the_hereditary_substitution_function}
The definition of the hereditary substitution function is a basic
extension of hereditary substitution function for STLC.  Before
defining the hereditary substitution function we first define the
construct type function for SSF.  This function is now defined for
three different types of input: term variables, term applications, and
type applications.
\begin{definition}
  \label{def:ctype_function_ssf}
  The construct type function for SSF is defined as follows:
  \begin{itemize}
  \item[] $ctype_T(x,x) = T$
  \item[] $ctype_T(x,t_1\ t_2) = T''$\\
    \begin{tabular}{lll}
      & Where $ctype_T(x,t_1) = T' \to T''$.
    \end{tabular}    
  \item[] $ctype_T(x,t[T']) = [T'/X]T''$\\
    \begin{tabular}{lll}
      & Where $ctype_T(x,t) = \forall X:*_l.T''$.
    \end{tabular}    
  \end{itemize}
\end{definition}

\noindent
Finally, we can define the hereditary substitution function for SSF.
\begin{definition}
  \label{def:hereditary_substitution_ssf}
  We define the hereditary substitution function for SSF as follows:
  \begin{itemize}
  \item[] $[t/x]^T x = t$
  \item[] $[t/x]^T y = y$\\
    \begin{tabular}{lll}
      & Where $y$ is a variable distinct from $x$.\\
    \end{tabular}
  \item[] $[t/x]^T (\lambda y:T'.t') = \lambda y:T'.([t/x]^T t')$
  \item[] $[t/x]^T (\Lambda X:*_l.t') = \Lambda X:*_l.([t/x]^T t')$
  \item[] $[t/x]^T (t_1\ t_2) = ([t/x]^T t_1)\ ([t/x]^T t_2)$\\
    \begin{tabular}{lll}
      & Where $([t/x]^T t_1)$ is not a $\lambda$-abstraction, or both $([t/x]^T t_1)$ and $t_1$\\
      & are $\lambda$-abstractions.
    \end{tabular}
  \item[] $[t/x]^{T} (t_1\ t_2) = [([t/x]^{T} t_2)/y]^{T''} s'_1$\\
    \begin{tabular}{lll}
      & Where $([t/x]^{T} t_1) \equiv \lambda y:T''.s'_1$ for some $y$, $s'_1$, and $T''$ \\
      & and $ctype_T(x,t_1) = T'' \to T'$.
    \end{tabular}
  \item[] $[t/x]^T (t'[T']) = ([t/x]^T t')[T']$\\
    \begin{tabular}{lll}
      & Where $[t/x]^T t'$ is not a type abstraction or $t'$ and $[t/x]^T t'$ \\
      & are type abstractions.
    \end{tabular}
    \item[] $[t/x]^{T} (t'[T']) = [T'/X]s'_1$\\
      \begin{tabular}{lll}
        & Where $[t/x]^{T} t' \equiv \Lambda X:*_l.s'_1$, for some $X$, $s'_1$ and $\Gamma \vdash T':*_q$, \\
        & such that, $q \leq l$ and $ctype_T(x,t') = \forall X:*_l.T''$.
      \end{tabular}
  \end{itemize}
\end{definition}

\noindent The next lemma states the familiar properties of the
construct type function.  The first property is slightly different
then the one defined for STLC.  The difference arises from the fact
that the ordering on types is not just the subexpression ordering, but
relies on the level of the type in the ordering on types.  So instead
of $T$ being a subexpression of the output of $ctype_T$ it will
be greater than or equal to the output of $ctype_T$.  The remainder
of the properties are as usual.

\begin{lemma}[Properties of $ctype_T$]
  \label{lemma:ctype_props_ssf}
  \begin{itemize}
  \item[i.] If $\Gamma,x:T,\Gamma' \vdash t:T'$ and $ctype_T(x,t) = T''$, then 
    $head(t) = x$, $T' \equiv T''$, and $T' \leq_{\Gamma,\Gamma'} T$.

  \item[ii.] If $\Gamma,x:T,\Gamma' \vdash t_1\ t_2:T'$, $\Gamma \vdash t:T$,
    $[t/x]^T t_1 = \lambda y:T_1.q$, and $t_1$ is not, then there exists a type
    $\psi$ such that $ctype_T(x,t_1) = \psi$.

  \item[iii.] If $\Gamma,x:T,\Gamma' \vdash t'[T'']:T'$, $\Gamma \vdash t:T$,
    $[t/x]^T t' = \Lambda X:*_l.t''$, and $t'$ is not, then there exists a type
    $\psi$ such that $ctype_T(x,t') = \psi$.
  \end{itemize}
\end{lemma}
\begin{proof}
  We prove part one first. This is a proof by induction on the structure of $t$.

\begin{itemize}
\item[Case.] Suppose $t \equiv x$.  Then $ctype_T(x,x) = T$.  Clearly,
  $head(x) = x$ and $T \equiv T$.
  
\item[Case.] Suppose $t \equiv t_1\ t_2$.  Then $ctype_T(x,t_1\ t_2) = T''$
  when $ctype_T(x,t_1) = T' \to T''$.  Now $t > t_1$ so by the induciton
  hypothesis $head(t_1) = x$ and $T' \to T'' \leq_{\Gamma,\Gamma'} T$.
  Therefore, $head(t_1\ t_2) = x$, and certainly, $T'' \leq_{\Gamma,\Gamma'} T$.
\end{itemize}

\ \\
Next we prove part two.  This is a proof by induction on the structure of $t_1\ t_2$.

\ \\
The only possiblities for the form of $t_1$ is $x$, $\hat{t}_1\ \hat{t}_2$, or $\hat{t}[T'']$.  All other 
forms would not result in $[t/x]^T t_1$ being a $\lambda$-abstraction and $t_1$ not.
If $t_1 \equiv x$ then there exist a type $T''$ such that $T \equiv T'' \to T'$ and
$ctype_T(x,x\ t_2) = T'$ when $ctype_T(x,x) = T \equiv T'' \to T'$ in this case.  We know
$T''$ to exist by inversion on $\Gamma,x:T,\Gamma' \vdash t_1\ t_2:T'$.

\ \\
Now suppose $t_1 \equiv (\hat{t}_1\ \hat{t}_2)$.  Now knowing $t'_1$ to not be a $\lambda$-abstraction
implies that $\hat{t}_1$ is also not a $\lambda$-abstraction or $[t/x]^T t_1$ would be an application
instead of a $\lambda$-abstraction.  So it must be the case that $[t/x]^T \hat{t}_1$ is a $\lambda$-abstraction
and $\hat{t}_1$ is not.  Since $t_1\ t_2 > t_1$ we can apply the induction hypothesis to obtain there exists
a type $\psi$ such that $ctype_T(x,\hat{t}_1) = \psi$.  
Now by inversion on $\Gamma,x:T,\Gamma' \vdash t_1\ t_2:T'$ we know there exists a type $T''$ such that
$\Gamma,x:T,\Gamma' \vdash t_1:T'' \to T'$.  We know $t_1 \equiv (\hat{t}_1\ \hat{t}_2)$ so by inversion on
$\Gamma,x:T,\Gamma' \vdash t_1:T'' \to T'$ we know there exists a type $\psi''$ such that
$\Gamma,x:T,\Gamma' \vdash \hat{t}_1:\psi'' \to (T'' \to T')$.
By part two of Lemma~\ref{lemma:ctype_props_ssf} we know $\psi \equiv \psi'' \to (T'' \to T')$ and
$ctype_T(x,t_1) = ctype_T(x,\hat{t}_1\ \hat{t}_2) = T'' \to T'$ 
when $ctype_T(x,\hat{t}_1) = \psi'' \to (T'' \to T')$, because we know $ctype_T(x,\hat{t}_1) = \psi$.

\ \\
The case where $t_1$ is a type application is similar to the previous case.

\ \\
The remaining parts of the lemma are similar to part two.
\end{proof}
% subsubsection the_hereditary_substitution_function (end)

\subsection{Properties of the Hereditary Substitution Function}
\label{sec:properties_of_the_hereditary_substitution_function_ssf}
We now define $rset$ as an extension of the same function for STLC by
adding type application redexes to the set of overall redexes
of a term.  It is defined in the following definition.  
\begin{definition}
  \label{def:rset_ssf}
  The following function constructs the set of redexes within a term:
  \begin{center}
    \begin{itemize}
    \item[] $rset(x) = \emptyset$\\
    \item[] $rset(\lambda x:T.t) = rset(t)$\\
    \item[] $rset(\Lambda X:*_l.t) = rset(t)$\\
    \item[] $rset(t_1\ t_2)$\\
      \begin{math}
        \begin{array}{lll}
          = & rset(t_1, t_2) & \text{if } t_1 \text{ is not a } \lambda \text{-abstraction.}\\
          = & \{t_1\ t_2\} \cup rset(t'_1, t_2)\ & \text{if } t_1 \equiv \lambda x:T.t'_1.\\
        \end{array}
      \end{math}
    \item[] $rset(t''[T''])$\\
      \begin{math}
        \begin{array}{lll}
          = & rset(t'') & \text{if } t'' \text{ is not a type abstraction.}\\
          = & \{t''[T'']\} \cup rset(t''') & \text{if } t'' \equiv \Lambda X:*_l.t'''.
        \end{array}
      \end{math}
    \end{itemize}
  \end{center}
  \ \\
  The extension of $rset$ to multiple arguments is defined as follows:
  \begin{center}
    $rset(t_1, \ldots, t_n) =^{def} rset(t_1) \cup \cdots \cup rset(t_n)$.
  \end{center}
\end{definition}
\noindent Next we state all the properties of the hereditary
substitution function.  They are equivalent to the properties stated
in
Section~\ref{subsec:properties_of_the_hereditary_substitution_function}
the only difference are their proofs.
\begin{lemma}[Total and Type Preserving]
  \label{lemma:total_ssf}
  Suppose $\Gamma \vdash t : T$ and $\Gamma, x:T, \Gamma' \vdash t':T'$. Then
  there exists a term $t''$, such that, $[t/x]^T t' = t''$ and $\Gamma,\Gamma' \vdash t'':T'$.
\end{lemma}
\begin{proof}
  This is a proof by induction on the lexicorgraphic combination $(T, t')$ of $>_{\Gamma,\Gamma'}$ and
the strict subexpression ordering.  We case split on $t'$.

\begin{itemize}
\item[Case.] Suppose $t'$ is either $x$ or a variable $y$ distinct from $x$.  
  Trivial in both cases.
  
\item[Case.] Suppose $t' \equiv \lambda y:T_1.t'_1$.  By inversion on the
  typing judgement we know $\Gamma,x:T,\Gamma',y:T_1 \vdash t'_1:T_2$.
  We also know $t' > t'_1$, hence we can apply the induction hypothesis to obtain
  $[t/x]^T t'_1 = \hat{t}'_1$ and $\Gamma,\Gamma',y:T_1 \vdash \hat{t}:T_2$
  for some term $\hat{t}'_1$.  By the definition of the hereditary substitution function 
  $[t/x]^T t' = \lambda y:T_1.[t/x]^T t'_1 = \lambda y:T_1.\hat{t}'_1$.  It suffices
  to show that $\Gamma,\Gamma' \vdash \lambda y:T_1.\hat{t}'_1:T_1 \to T_2$.  
  By simply applying the $\lambda$-abstraction typing rule using
  $\Gamma,\Gamma',y:T_1 \vdash \hat{t}:T_2$ we obtain 
  $\Gamma,\Gamma' \vdash \lambda y:T_1.\hat{t}'_1:T_1 \to T_2$.
  
\item[Case.] Suppose $t' \equiv \Lambda X:*_l.t'_1$.  Similar to the previous case.
  
\item[Case.] Suppose $t' \equiv t'_1\ t'_2$.  By inversion we know
  $\Gamma, x:T, \Gamma' \vdash t'_1 : T'' \to T'$ and
  $\Gamma, x:T, \Gamma' \vdash t'_2 : T''$ for some types $T'$ and $T''$.
  Clearly, $t' > t'_i$ for $i \in \{1,2\}$.  Thus, by the induction hypothesis
  there exists terms $m_1$ and $m_2$ such that $[t/x]^T t'_i = m_i$,
  $\Gamma, \Gamma' \vdash m_1 : T'' \to T'$ and
  $\Gamma, \Gamma' \vdash m_2 : T''$ for
  $i \in \{1,2\}$.  We case split on whether or not $m_1$ is a $\lambda$-abstraction
  and $t'_1$ is not, or $ctype_T(x,t'_1)$ is undefined.  
  We only consider the non-trivial cases when 
  $m_1 \equiv \lambda y:T''.m'_1$, $t'_1$ is not a $\lambda$-abstraction, and 
  $ctype_T(x,t'_1) = \psi'' \to \psi'$.  Suppose the former.  
  Now by Lemma~\ref{lemma:ctype_props_ssf} it is the case that 
  there exists a $\psi$ such that $ctype_T(x,t'_1) = \psi$, 
  $\psi \equiv T'' \to T'$, and $\psi \leq_{\Gamma,\Gamma'} T$, hence
  $T >_{\Gamma,\Gamma'} T''$.
  Then $[t/x]^T (t'_1\ t'_2) = [m_2/y]^{\psi''} m'_1$.  
  Therefore, by the induction hypothesis there exists a 
  term $m$ such that $[m_2/y]^{T''} m'_1 = m$ and $\Gamma,\Gamma' \vdash m:T''$.
  
\item[Case.] Suppose $t' \equiv t'_1[T'']$. Similar to the previous
  case.  
\end{itemize}
\end{proof}

\begin{lemma}[Redex Preserving]
  \label{lemma:redex_preserving_ssf}
  If $\Gamma \vdash t : T$, $\Gamma, x:T, \Gamma' \vdash t':T'$, then
  $|rset(t', t)| \geq |rset([t/x]^T t')|$.
\end{lemma}
\begin{proof}
  This is a proof by induction on the lexicorgraphic combination
$(T, t')$ of $>_{\Gamma,\Gamma'}$ and the strict subexpression ordering.
We case split on the structure of $t'$.  
\begin{itemize}
\item[Case.] Let $t' \equiv x$ or $t' \equiv y$ where $y$ is distinct from $x$.  Trivial. 
  
\item[Case.] Let $t' \equiv \lambda x:T_1.t''$.  Then $[t/x]^T t' \equiv \lambda x:T_1.[t/x]^T t''$.
  Now 
  \begin{center}
    \begin{math}
      \begin{array}{lll}
        rset(\lambda x:T_1.t'', t) & = & rset(\lambda x:T_1.t'') \cup rset(t)\\
        & = & rset(t'') \cup rset(t)\\
        & = & rset(t'', t).\\
      \end{array}
    \end{math}
  \end{center}
  We know that $t' > t''$ by the strict subexpression ordering, hence by the induction hypothesis
  $|rset(t'', t)| \geq_{\Gamma,\Gamma'} |rset([t/x]^T t'')|$ which implies $|rset(t', t)| \geq |rset([t/x]^T t')|$.
  
\item[Case.] Let $t' \equiv \Lambda X:*_l.t''$.  Similar to the previous case.
  
\item[Case.] Let $t' \equiv inl(t'')$. We know $rset(t', t) = rset(t'', t)$.  Since $t' > t''$ we can apply
  the induction hypothesis to obtain $|rset(t'', t)| \geq |rset([t/x]^T t'')|$.  This implies
  $|rset(t', t)| \geq_{\Gamma,\Gamma'} |rset([t/x]^T t')|$.
  
\item[Case.] Let $t' \equiv inr(t'')$. Similar to the previous case.
  
\item[Case.] Let $t' \equiv t'_1\ t'_2$.  First consider when $t'_1$ is not a $\lambda$-abstraction. Then
  \begin{center}
    $rset(t'_1\ t'_2, t) = rset(t'_1, t'_2, t)$
  \end{center}  
  Clearly,  $t' > t'_i$ for $i \in \{1,2\}$, hence, by the induction hypothesis $|rset(t'_i,t)| \geq |rset([t/x]^T t'_i)|$.  
  We have two cases to consider.  That is whether or not $[t/x]^T t'_1$ is a $\lambda$-abstraction or not.  Suppose so.
  Then by Lemma~\ref{lemma:ctype_props_ssf} $ctype_T(x.t'_1) = \psi$ and by inversion on $\Gamma,x:T,\Gamma' \vdash t'_1\ t'_2:T'$
  there exists a type $T''$ such that $\Gamma,x:T,\Gamma' \vdash t_1:T'' \to T'$.  Again, by Lemma~\ref{lemma:ctype_props_ssf}
  $\psi \equiv T'' \to T'$. Thus, $ctype_T(x,t'_1) = T'' \to T'$ and $T'' \to T'$ is a subexpression of $T$.
  So by the definition of the hereditary substitution function $[t/x]^T t'_1\ t'_2 = [([t/x]^T t'_2)/y]^{T''} t''_1$, where
  $[t/x]^T t'_1 = \lambda y:T''.t''_1$.  Hence,
  \begin{center}
    \begin{math}
      |rset([t/x]^T t'_1\ t'_2)| = |rset([([t/x]^T t'_2)/y]^{T''} t''_1)|.
    \end{math}
  \end{center}
  Now $T >_{\Gamma,\Gamma'} T''$ so by the induction hypothesis 
  \begin{center}
    \begin{math}
      \begin{array}{lll}
        |rset([([t/x]^T t'_2)/y]^{T''} t''_1)| & \leq & |rset([t/x]^T t'_2, t''_1)|\\
        & \leq & |rset(t'_2, t''_1, t)|\\
        & = & |rset(t'_2, [t/x]^T t'_1, t)|\\
        & \leq & |rset(t'_2, t'_1, t)|\\
        & = & |rset(t'_1, t'_2, t)|.\\
      \end{array}
    \end{math}
  \end{center}
  
  \ \\
  \noindent
  Suppose $[t/x]^T t'_1$ is not a $\lambda$-abstractions or $ctype_T(x,t'_1)$ is undefined.  Then
  \begin{center}
    \begin{math}
      \begin{array}{lll}
        rset([t/x]^T (t'_1\ t'_2)) & = & rset([t/x]^T t'_1\ [t/x]^T t'_2)\\
        & = & rset([t/x]^T t'_1, [t/x]^T t'_2).\\
        & \geq & rset(t'_1, t'_2, t).\\
      \end{array}
    \end{math}
  \end{center}
  
  \ \\
  Next suppose $t'_1 \equiv \lambda y:T_1.t''_1$.  Then 
  \begin{center}
    \begin{math}
      \begin{array}{lll}
        rset((\lambda y:T_1.t''_1)\ t'_2, t) & = & \{ (\lambda y:T_1.t''_1)\ t'_2\} \cup rset(t''_1, t'_2, t).
      \end{array}
    \end{math}
  \end{center}
  By the definition of the hereditary substitution function,
  \begin{center}
    \begin{math}
      \begin{array}{lll}
        rset([t/x]^T (\lambda y:T_1.t''_1)\ t'_2) & = & rset([t/x]^T (\lambda y:T_1.t''_1)\ [t/x]^T t'_2)\\
        & = & rset((\lambda y:T_1.[t/x]^T t''_1)\ [t/x]^T t'_2)\\
        & = & \{(\lambda y:T_1.[t/x]^T t''_1)\ [t/x]^T t'_2\} \cup 
        rset([t/x]^T t''_1) \cup rset([t/x]^T t'_2).\\
        
      \end{array}
    \end{math}
  \end{center}
  Since $t' > t''_1$ and $t' > t'_2$ we can apply the induction hypothesis to obtain,
  $|rset(t''_1, t)| \geq |rset([t/x]^T t''_1)|$ and $|rset(t'_2,t)| \geq |rset([t/x]^T t'_2)|$.  Therefore, \\
  $|\{ (\lambda y:T_1.t''_1)\ t'_2\}\ \cup\ rset(t''_1,t)\ \cup\ rset(t'_2,t)| \geq $ 
  $|\{(\lambda y:T_1.[t/x]^T t''_1)\ [t/x]^T t'_2\}\ \cup\ rset([t/x]^T t''_1)\ \cup\ rset([t/x]^T t'_2)|$.
  
\item[Case.] Suppose $t' \equiv t'_1[T'']$.  It suffices to show that 
  $|rset(t,t')| \geq |rset([t/x]^T t')|$.  Now 
  \begin{center}
    \begin{math}
      \begin{array}{lll}
        |rset(t,t')| & = & |rset(t,t'_1[T''])|\\
        & = & |rset(t) \cup rset(t'_1[T''])|\\
        & = & |rset(t) \cup rset(t'_1)|\\
        & = & |rset(t,t'_1)|.
      \end{array}
    \end{math}
  \end{center}
  and
  \begin{center}
    $|rset[t/x]^T t')| = |rset([t/x]^T(t'_1[T'']))|$.
  \end{center}
  We have several cases to consider.  Suppose $t'_1$ and $[t/x]^T t'_1$ are not type abstractions.
  Then 
  \begin{center}
    \begin{math}
      \begin{array}{lll}
        |rset([t/x]^T(t'_1[T'']))| & = & |rset(([t/x]^T t'_1)[T''])|\\
        & = & |rset([t/x]^T t'_1)|.
      \end{array}
    \end{math}
  \end{center}
  We can see that $t' > t'_1$ so by the induction hypothesis 
  \begin{center}
    \begin{math}
      \begin{array}{lll}
        |rset([t/x]^T t'_1)| & \leq & |rset(t,t'_1)|\\
        & = & |rset(t,t')|.
      \end{array}
    \end{math}
  \end{center}
  
  \ \\
  Suppose $t'_1 \equiv \Lambda X:*_l.t''_1$.  Then 
  \begin{center}
    \begin{math}
      \begin{array}{lll}
        |rset(t,t')| & = & |rset(t, t'_1[T''])|\\
        & = & |\{t'_1[T'']\} \cup rset(t,t''_1)|
      \end{array}
    \end{math}
  \end{center}
  and
  \begin{center}
    \begin{math}
      \begin{array}{lll}
        |rset([t/x]^T t')| & = & |rset([t/x]^T(t'_1[T''])|\\
        & = & |rset((\Lambda X:*_l.[t/x]^T t''_1)[T''])|\\
        & = & |\{(\Lambda X:*_l.[t/x]^T t''_1)[T'']\} \cup rset([t/x]^T t''_1)|.
      \end{array}
    \end{math}
  \end{center}
  Again, $t' > t'_1$ so by the induciton hypothesis $|rset([t/x]^T t''_1)| \leq |rset(t,t''_1)$.  Thus,
  $|rset(t,t')| \geq |rset([t/x]^T t')|$.
  
  \ \\
  Suppose $t'_1$ is not a type abstraction, but $[t/x]^T t'_1 \equiv \lambda X:*_l.t''_1$.  Then
  \begin{center}
    \begin{math}
      \begin{array}{lll}
        |rset([t/x]^T t')| & = & |rset([T''/X]t''_1)|\\
        & = & |rset(t''_1)|
      \end{array}
    \end{math}
  \end{center}
  and
  \begin{center}
    \begin{math}
      \begin{array}{lll}
        |rset(t',t)| & = & |rset(t'_1[T''], t)|\\
        & = & |rset(t'_1,t)|.
      \end{array}
    \end{math}
  \end{center}
  Since $t' > t'_1$ we can apply the induction hypthothesis to obtain
  \begin{center}
    \begin{math}
      \begin{array}{lll}
        |rset([t/x]^T t'_1)| = |rset(t''_1)\\ 
        \leq |rset(t'_1,t)|.\\
        
      \end{array}
    \end{math}
  \end{center}
  Therefore, $|rset([t/x]^T t')| \leq |rset(t',t)|$.
  
\item[Case.] Let $t' \equiv t'_1\ t'_2$.  First consider when $t_1'$ is not a $\lambda$-abstraction. Then
  \begin{center}
    $rset(t'_1\ t'_2, t) = rset(t'_1, t'_2, t)$
  \end{center}  
  Clearly,  $t' > t'_i$ for $i \in \{1,2\}$, hence, by the induction hypothesis $|rset(t'_i,t)| \geq |rset([t/x]^T t'_i)|$.  
  We have three cases to consider.  That is whether or not $[t/x]^T t'_1$ is a $\lambda$-abstraction and $t'_1$ is not, or
  $ctype_T(x,t'_1)$ is undefined.  
  Suppose $t'_1$ is a $\lambda$-abstraction.
  Then by Lemma~\ref{lemma:ctype_props_ssf} $ctype_T(x.t'_1) = \psi$ and by inversion on $\Gamma,x:T,\Gamma' \vdash t'_1\ t'_2:T'$
  there exists a type $T''$ such that $\Gamma,x:T,\Gamma' \vdash t_1:T'' \to T'$.  Again, by Lemma~\ref{lemma:ctype_props_ssf}
  $\psi \equiv T'' \to T'$. Thus, $ctype_T(x,t'_1) = T'' \to T'$ and $T'' \to T'$ is a subexpression of $T$.
  So by the definition of the hereditary substitution function $[t/x]^T t'_1\ t'_2 = [([t/x]^T t'_2)/y]^{T''} t''_1$, where
  $[t/x]^T t'_1 = \lambda y:T''.t''_1$.  Hence,
  \begin{center}
    \begin{math}
      |rset([t/x]^T t'_1\ t'_2)| = |rset([([t/x]^T t'_2)/y]^{T''} t''_1)|.
    \end{math}
  \end{center}
  Now $T >_{\Gamma,\Gamma'} T''$ so by the induction hypothesis 
  \begin{center}
    \begin{math}
      \begin{array}{lll}
        |rset([([t/x]^T t'_2)/y]^{T''} t''_1)| & \leq & |rset([t/x]^T t'_2, t''_1)|\\
        & \leq & |rset(t'_2, t''_1, t)|\\
        & = & |rset(t'_2, [t/x]^T t'_1, t)|\\
        & \leq & |rset(t'_2, t'_1, t)|\\
        & = & |rset(t'_1, t'_2, t)|.\\
      \end{array}
    \end{math}
  \end{center}

  \ \\
  \noindent
  Suppose $[t/x]^T t'_1$ is not a $\lambda$-abstractions or $ctype_T(x,t'_1)$ is undefined.  Then
  \begin{center}
    \begin{math}
      \begin{array}{lll}
        rset([t/x]^T (t'_1\ t'_2)) & = & rset([t/x]^T t'_1\ [t/x]^T t'_2)\\
        & = & rset([t/x]^T t'_1, [t/x]^T t'_2).\\
        & \leq & rset(t'_1, t'_2, t).\\
      \end{array}
    \end{math}
  \end{center}
  
  \ \\
  Next suppose $t'_1 \equiv \lambda y:T_1.t''_1$.  Then 
  \begin{center}
    \begin{math}
      \begin{array}{lll}
        rset((\lambda y:T_1.t''_1)\ t'_2, t) & = & \{ (\lambda y:T_1.t''_1)\ t'_2\} \cup rset(t''_1, t'_2, t).
      \end{array}
    \end{math}
  \end{center}
  By the definition of the hereditary substitution function,
  \begin{center}
    \begin{math}
      \begin{array}{lll}
        rset([t/x]^T (\lambda y:T_1.t''_1)\ t'_2) & = & rset([t/x]^T (\lambda y:T_1.t''_1)\ [t/x]^T t'_2)\\
        & = & rset((\lambda y:T_1.[t/x]^T t''_1)\ [t/x]^T t'_2)\\
        & = & \{(\lambda y:T_1.[t/x]^T t''_1)\ [t/x]^T t'_2\} \cup 
        rset([t/x]^T t''_1) \cup rset([t/x]^T t'_2).\\
        
      \end{array}
    \end{math}
  \end{center}
  Since $t' > t''_1$ and $t' > t'_2$ we can apply the induction hypothesis to obtain,
  $|rset(t''_1, t)| \geq |rset([t/x]^T t''_1)|$ and $|rset(t'_2,t)| \geq |rset([t/x]^T t'_2)|$.  Therefore, \\
  $|\{ (\lambda y:T_1.t''_1)\ t'_2\}\ \cup\ rset(t''_1,t)\ \cup\ rset(t'_2,t)| \geq $ 
  $|\{(\lambda y:T_1.[t/x]^T t''_1)\ [t/x]^T t'_2\}\ \cup\ rset([t/x]^T t''_1)\ \cup\ rset([t/x]^T t'_2)|$.
\end{itemize}
\end{proof}

\begin{lemma}[Normality Preserving]
  \label{corollary:normalization_preserving_ssf}
  If $\Gamma \vdash n:T$ and $\Gamma, x:T \vdash n':T'$ then there exists a normal term $n''$ 
  such that $[n/x]^T n' = n''$.
\end{lemma}
\begin{proof}
  By Lemma~\ref{lemma:total_ssf} we know there exists a term $n''$ such that $[n/x]^T n' = n''$ and by 
Lemma~\ref{lemma:redex_preserving_ssf} 
$|rset(n', n)| \geq |rset([n/x]^T n')|$.  Hence, $|rset(n', n)| \geq |rset(n'')|$, but
$|rset(n', n)| = 0$.  Therefore, $|rset(n'')| = 0$ which implies $n''$ has no redexes.  
\end{proof}

\begin{lemma}[Soundness with Respect to Reduction]
  \label{lemma:soundness_reduction_ssf}
  If $\Gamma \vdash t : T$ and $\Gamma, x:T, \Gamma' \vdash t':T'$ then
  $[t/x]t' \redto^* [t/x]^T t'$.
\end{lemma}
\begin{proof}
  This is a proof by induction on the lexicorgraphic combination
$(T, t')$ of $>_{\Gamma,\Gamma'}$ and the strict subexpression ordering.
We case split on the structure of $t'$.  When applying
the induction hypothesis we must show that the input terms to the
substitution and the hereditary substitution functions are typeable.
We do not explicitly state typing results that are simple
conseqences of inversion.

\begin{itemize}
\item[Case.] Suppose $t'$ is a variable $x$ or $y$ distinct from $x$.  
  Trivial in both cases.
  
\item[Case.] Suppose $t' \equiv \lambda y:T'.\hat{t}$.  Then
  $[t/x]^T (\lambda y:T'.\hat{t}) = \lambda y:T'.([t/x]^T \hat{t})$. 
  Now $t' > \hat{t}$ so we can apply the induction hypothesis to obtain 
  $[t/x]\hat{t} \redto^* [t/x]^T \hat{t}$.  At this point we can see that since 
  $\lambda y:T'.[t/x]\hat{t} \equiv [t/x](\lambda y:T'.\hat{t})$ and we may
  conclude that $\lambda y:T'.[t/x]\hat{t} \redto^* \lambda y:T'.[t/x]^T \hat{t}$.
  
\item[Case.] Suppose $t' \equiv \Lambda X:*_l.\hat{t}$.  Similar to the previous case.  

\item[Case.] Suppose $t' \equiv t'_1\ t'_2$.  By Lemma~\ref{lemma:total_ssf}
  there exists terms $\hat{t}'_1$ and $\hat{t}'_2$
  such that $[t/x]^T t'_1 = \hat{t}'_1$ and $[t/x]^T t'_2 = \hat{t}'_2$.  Since
  $t' > t'_1$ and $t' > t'_2$ we can apply the induction hypothesis to obtain
  $[t/x]t'_1 \redto^* \hat{t}'_1$ and $[t/x]t'_2 \redto^* \hat{t}'_2$.  Now we case
  split on whether or not $\hat{t}'_1$ is a $\lambda$-abstraction and $t'_1$ is not, or $ctype_T(x,t'_1)$ is undefined. If
  $ctype_T(x,t'_1)$ is undefined or $\hat{t}'_1$ is not a $\lambda$-abstraction then 
  $[t/x]^T t' = ([t/x]^T t'_1)\ ([t/x]^T t'_2) \equiv \hat{t}'_1\ \hat{t}'_2$. Thus,
  $[t/x]t' \redto^* [t/x]^T t'$, because $[t/x]t' = ([t/x] t'_1)\ ([t/x] t'_2)$.  So suppose 
  $\hat{t}'_1 \equiv \lambda y:T'.\hat{t}''_1$ and $t'_1$ is not a $\lambda$-abstraction.  
  By Lemma~\ref{lemma:ctype_props_ssf} there exists a type $\psi$ such that
  $ctype_T(x,t'_1) = \psi$, $\psi \equiv T'' \to T'$, and $\psi$ is a subexpression
  of $T$, where by inversion on $\Gamma,x:T,\Gamma' \vdash t':T'$ there exists a type
  $T''$ such that $\Gamma,x:T,\Gamma' \vdash t'_1:T'' \to T'$.  
  Then by the definiton of the hereditary substitution function $[t/x]^T (t'_1\ t'_2) = 
  [\hat{t}'_2/y]^{T'} \hat{t}''_1$.
  Now we know $T >_{\Gamma,\Gamma'} T'$ so 
  we can apply the induction hypothesis to obtain 
  $[\hat{t}'_2/y] \hat{t}''_1 \redto^* [\hat{t}'_2/y]^{T'} \hat{t}''_1$.  Now by knowing that 
  $(\lambda y:T'.\hat{t}''_1)\ t'_2 \redto [\hat{t}'_2/y] \hat{t}''_1$ and
  by the previous fact we know $(\lambda y:T'.\hat{t}''_1)\ t'_2 \redto^* [\hat{t}'_2/y]^{T'} \hat{t}''_1$.
  We now make use of the well known result of full $\beta$-reduction.  The
  result is stated as
  \begin{center}
    \begin{math}
      $$\mprset{flushleft}
      \inferrule* [right=] {
        a \redto^* a'
        \\\\
        b \redto^* b'
        \\
        a'\ b' \redto^* c
      }{a\ b \redto^* c}
    \end{math}
  \end{center}
  where $a$, $a'$, $b$, $b'$, and $c$ are all terms.  We apply this
  result by instantiating $a$, $a'$, $b$, $b'$, and $c$ with
  $[t/x] t'_1$, $\hat{t}'_1$, $[t/x] t'_2$, $\hat{t}'_2$, and $[\hat{t}'_2/y]^{T'} \hat{t}''_1$ 
  respectively.  Therefore, $[t/x](t'_1\ t'_2) \redto^* [\hat{t}'_2/y]^{T'} \hat{t}''_1$.    
  
\item[Case.] Suppose $t' \equiv t'_1[T'']$.
  Since $t' > t'_1$ we can apply the induction hypothesis to
  obtain $[t/x] t'_1 \redto^* [t'/x]^T t'_1$.  We case split on whether or not $[t'/x]^T t'_1$ is
  a type abstraction and $t'_1$ is not.  The case where it is not is trivial so we only consider
  the case where $[t'/x]^T t'_1 \equiv \Lambda X:*_l.s'$.  Then 
  $[t'/x]^T t'  = [T'/X]s'$.  Now we have $[t/x] t'_1 \redto^* [t'/x]^T t'_1$ and
  $[t/x](t'_1[T]) \equiv ([t/x]t'_1)[T] \redto^* ([t'/x]^T t'_1)[T] \redto [T/X]s'$.  Thus,
  $[t/x]t' \redto^* [t'/x]^T t'$.  
\end{itemize}
\end{proof}
% subsection properties_of_the_hereditary_substitution_function_ssf (end)

\subsection{Substitution for the Interpretation of Types.}
\label{subsec:substitution_for_the_interpretation_of_types_ssf}
The definition of the interpretation of types is identical to the
definition for STLC (Definition~\ref{def:interpretation_of_types_stlc}),
so we do not repeat it here.  Before concluding normalization we state
the main substitution lemma for the interpretation of types.
\begin{lemma}[Substitution for the Interpretation of Types]
  If $n' \in \interp{T'}_{\Gamma,x:T,\Gamma'}$, $n \in \interp{T}_\Gamma$, then 
  $[n/x]^T n' \in \interp{T'}_{\Gamma,\Gamma'}$.
  
  \label{lemma:interpretation_of_types_closed_substitution_ssf}
\end{lemma}
\begin{proof}
  By Lemma~\ref{lemma:total_ssf} we know there exists a term $\hat{n}$ 
  such that $[n/x]^T n' = \hat{n}$ and $\Gamma,\Gamma' \vdash \hat{n}:T'$ and by 
  Lemma~\ref{corollary:normalization_preserving_ssf} $\hat{n}$ is normal.  Therefore,
  $[n/x]^T n' = \hat{n} \in \interp{T'}_{\Gamma,\Gamma'}$.
\end{proof}

Before moving on to proving soundness of typing and concluding
normalization we need a couple of results about the interpretation of
types: context weakening and type substitution.  They both are used in
the proof of the type soundness theorem
(Theorem~\ref{thm:soundness_ssf}).

\begin{lemma}[Context Weakening for Interpretations of Types]
  If $\Gamma,\Gamma',\Gamma''\ Ok$ and $n \in \interp{T}_{\Gamma,\Gamma''}$, then 
  $n \in \interp{T}_{\Gamma,\Gamma',\Gamma''}$.
  \label{lemma:context_weakening_interpretations_ssf}
\end{lemma}
\begin{proof}
  This proof is by structural induction on $n$.
\begin{itemize}
\item[Case.]  Let $n \equiv x$.  By the definition of the interpretation of types 
  $\Gamma(x) = T$.  Clearly,
  $(\Gamma,\Gamma')(x) = T$, and Lemma~\ref{lemma:context_weakening_for_kinding_ssf} gives 
  us $\Gamma,\Gamma' \vdash T:*_p$ hence, $x \in \interp{T}_{\Gamma,\Gamma'}$.
  
\item[Case.]  Let $n \equiv \lambda x:T_1.n'$.  By the definition of the interpretation of 
  types, there exists a type $T_2$, such that $T = T_1 \rightarrow T_2$, and 
  $n' \in \interp{T_2}_{\Gamma,x:T_1}$.  By
  the induction hypothesis, $n' \in \interp{T_2}_{\Gamma,\Gamma',x:T_1,}$, and by the 
  definition of the interpretation of types 
  $\lambda x:T_1.n' \in \interp{T_1 \rightarrow T_2}_{\Gamma,\Gamma'}$.
  
\item[Case.]  Let $n \equiv n_1\ n_2$.  By the definition of the interpretation of types, there 
  exists a type $T_1$, such that 
  $n_1 \in \interp{T_1 \rightarrow T_2}_\Gamma$, and $n_2 \in \interp{T_2}_\Gamma$.  By 
  the induction hypothesis,
  $n_1 \in \interp{T_1 \rightarrow T_2}_{\Gamma,\Gamma'}$, and 
  $n_2 \in \interp{T_2}_{\Gamma,\Gamma'}$.  Thus, by
  the definition of the interpretation of types $n_1n_2 \in \interp{T_2}_{\Gamma,\Gamma'}$.
  
\item[Case.]  Let $n \equiv \Lambda X:*_p.n'$.  By the definition of the interpretation of 
  types, there exists a type $T'$, such that
  $n' \in \interp{T'}_{\Gamma,X:*_p}$, and by the induction hypothesis 
  $n' \in \interp{T'}_{\Gamma,X:*_p,\Gamma'}$.  By the
  definition of the interpretation of types 
  $\Lambda X:*_p.n' \in \interp{\forall X:*_p.T'}_{\Gamma,\Gamma'}$.
  
\item[Case.]  Let $n \equiv n'[T']$.  By the definition of the interpretation of types, 
  there exists a type $T''$ and $l$, 
  such that $T = [T'/X]T''$, $\Gamma \vdash T':*_l$, and 
  $n' \in \interp{\forall X:*_l.T''}_{\Gamma}$.  By the induction
  hypothesis $n' \in \interp{\forall X:*_l.T''}_{\Gamma,\Gamma'}$.  We know, 
  $\Gamma \vdash T':*_k$, for some $k \leq l$, so by
  Lemma~\ref{lemma:context_weakening_for_kinding_ssf}, $\Gamma,\Gamma' \vdash T':*_k$, 
  hence 
  $\Gamma \vdash T':*_l$. Thus, $n[T'] \in
  \interp{[T'/X]T''}_{\Gamma,\Gamma'}$.
\end{itemize}
\end{proof}
\begin{lemma}[Type Substitution for the Interpretation of Types]
  If $n \in \interp{T'}_{\Gamma,X:*_l,\Gamma'}$ and 
  $\Gamma \vdash T:*_l$, then 
  $[T/X]n \in \interp{[T/X]T'}_{\Gamma,[T/X]\Gamma'}$.
  \label{lemma:type_sub_ssf}
\end{lemma}
\begin{proof}
  This proof is by structural induction on $n$.
\begin{itemize}
\item[Case.] $n$ is a variable $y$.  Clearly, $[T/X]n \equiv $
  $[T/X]y = y \in \interp{T'}_{\Gamma,X:*_l,\Gamma'}$, and\\
  $(\Gamma,[T/X]\Gamma')(y) = [T/X]T'$. Also,
  we have $\Gamma,[T/X]\Gamma' \vdash [T/X]T':*_p$ for some $p$, by 
  Lemma~\ref{lemma:substitution_for_kinding_ssf}. Hence,
  by the definition of the interpretation of types, 
  $y \in \interp{[T/X]T'}_{\Gamma,[T/X]\Gamma'}$.
  
\item[Case.] Let $n \equiv \lambda y:\psi.n'$.  By the definition of the
  interpretation of types $T' \equiv \psi \rightarrow \psi'$.  
  By the induction hypothesis 
  $[T/X]n' \in \interp{[T/X]\psi'}_{\Gamma,\Gamma',y:[T/X]\psi}$. 
  Again by the definition of the interpretation of types
  $\lambda y:[T/X]\psi.[T/X]n' \equiv $
  $[T/X](\lambda y:\psi.n') \in $
  $\interp{[T/X]T'}_{\Gamma,[T/X]\Gamma'}.$
  
\item[Case.]  Let $n \equiv n_1\ n_2$.  By the definition of the 
  interpretation of types $T' \equiv \psi$, 
  $n_1 \in \interp{\psi' \rightarrow \psi}_{\Gamma,X:*_q,\Gamma'}$, and
  $n_2 \in \interp{\psi'}_{\Gamma,X:*_q,\Gamma'}$.  By the induction hypothesis 
  $[T/X]n_1 \in $
  $\interp{[T/X](\psi' \rightarrow \psi)}_{\Gamma,[T/X]\Gamma'}$ and
  $[T/X]n_2 \in \interp{[T/X]\psi'}_{\Gamma,[T/X]\Gamma'}$.  Now by
  the definition of the interpretation of types 
  $([T/X]n_1)([T/X]n_2) \in \interp{[T/X]\psi}_{\Gamma,[T/X]\Gamma'}$,
  since $[T/X]n_1$, cannot be a $\lambda$-abstraction.
  
\item[Case.]  Let $n \equiv \Lambda Y:*_q.n'$.  By the definition of the
  interpretation of types $T' = \forall Y:*_q.\psi$ and 
  $n' \in \interp{\psi}_{\Gamma,X:*_l,\Gamma',Y:*_q}$.  By the induction 
  hypothesis \\
  $[T/X]n' \in \interp{[T/X]\psi}_{\Gamma,[T/X]\Gamma',Y:*_q}$ and by
  the definition of the interpretation of types 
  $\Lambda Y:*_q.[T/X]n' \in 
  \interp{\forall Y:*_q.[T/X]\psi}_{\Gamma,[T/X]\Gamma'}$ which is 
  equivalent to
  $[T/X](\Lambda Y:*_q.n') \in $
  $\interp{[T/X](\forall Y:*_q.\psi)}_{\Gamma,[T/X]\Gamma'}$.
  
\item[Case.]  Let $n \equiv n'[\psi]$.  By the definition of the
  interpretation of types $T' = [\psi/Y]\psi'$, for some $Y$, $\psi$, and 
  there exists a $q$ such that $\Gamma,X:*_l,\Gamma' \vdash \psi:*_q$, and 
  $n' \in \interp{\forall Y:*_q.\psi'}_{\Gamma,X:*_l,\Gamma'}$.  By the 
  induction hypothesis 
  $[T/X]n' \in \interp{[T/X](\forall Y:*_q.\psi')}_{\Gamma,[T/X]\Gamma'}$.
  Therefore, by the definition of the interpretation of types\\
  $([T/X]n')[\psi] \in$
  $ \interp{[\psi/Y]([T/X]\psi')}_{\Gamma,[T/X]\Gamma'}$, which is equivalent
  to \\
  $[T/X](n'[\psi]) \in \interp{[T/X]([\psi/Y]\psi')}_{\Gamma,[T/X]\Gamma'}$.
\end{itemize}
\end{proof}
% subsubsection substitution_for_the_interpretation_of_types_ssf (end)

\subsection{Concluding Normalization.}
\label{subsec:soundness_of_typing_ssf}
We are now ready to present our main result.  The next theorem shows
that the type assignment rules are sound with respect to the
interpretation of types.  

\begin{thm}[Type Soundness]
  If $\Gamma \vdash t:T$ then $t \in \interp{T}_\Gamma$.
  \label{thm:soundness_ssf}
\end{thm}
\begin{proof}
  This is a proof by induction on the structure of the typing derivation of $t$.

\begin{itemize}
\item[Case.]\ \\
  \begin{center}
    \begin{math}
      $$\mprset{flushleft}
      \inferrule* [right=] {
        \Gamma(x) = T
        \\
        \Gamma\ Ok
      }{\Gamma \vdash x:T}
    \end{math}
  \end{center}
  By regularity $\Gamma \vdash T:*_l$ for some $l$, hence $\interp{T}_\Gamma$ is nonempty.
  Clearly, $x \in \interp{T}_\Gamma$ by the definition of the interpretation of types.
  
\item[Case.]\ \\
  \begin{center}
    \begin{math}
      $$\mprset{flushleft}
      \inferrule* [right=] {
        \Gamma,x:T_1 \vdash t:T_2
      }{\Gamma \vdash \lambda x:T_1.t : T_1 \rightarrow T_2}
    \end{math}
  \end{center}
  By the induction hypothesis $t \in
  \interp{T_2}_{\Gamma,x:T_1}$ and by the definition of the
  interpretation of types $t \normto n \in $ 
  $\interp{T_2}_{\Gamma,x:T_1}$ and $\Gamma, x:T_1 \vdash
  n:T_2$.  Thus, by applying the $\lambda$-abstraction type-checking
  rule, $\Gamma \vdash \lambda x:T_1.n:\Pi x:T_1.T_2$ so 
  by the definition of the interpretation of types $\lambda x:T_1.n
  \in \interp{T_1 \rightarrow T_2}_\Gamma$.  Thus, according to the
  definition of the interpretation of types $\lambda x:T_1.t
  \normto \lambda x:T_1.n \in \interp{T_1 \rightarrow T_2}_\Gamma$.

\item[Case.]\ \\
  \begin{center}
    \begin{math}
      $$\mprset{flushleft}
      \inferrule* [right=] {
        \Gamma \vdash t_1 : T_2 \rightarrow T_1 
        \\
        \Gamma \vdash t_2 : T_2
      }{\Gamma \vdash t_1\ t_2 : T_1}
    \end{math}
  \end{center}
  By the induction hypothesis $t_1 \normto n_1 \in $
  $\interp{T_2 \rightarrow T_1}_\Gamma$,
  $t_2 \normto n_2 \in \interp{T_2}_\Gamma$, $\Gamma \vdash T_2 \rightarrow T_1:*_p$, 
  and $\Gamma \vdash T_2:*_q$.  Inversion on the arrow-type kind-checking rule yields, 
  $\Gamma \vdash T_1:*_r$, and by
  Lemma~\ref{lemma:context_weakening_for_kinding_ssf}, 
  $\Gamma,x:T_2,\Gamma' \vdash T_1:*_r$.

  \ \\
  Now we know from above that $n_1 \in \interp{T_2 \rightarrow T_1}_\Gamma$ and
  $n_2 \in \interp{T_2}_\Gamma$, hence $\Gamma \vdash n_1:T_2 \to T_1$ and
  $\Gamma \vdash n_2:T_2$.  It suffices to show that $n_1\ n_2 \in \interp{T_2}_\Gamma$.
  Clearly, $n_1\ n_2 = [n_1/z](z\ n_2)$ for some variable $z \not \in FV(n_1,n_2)$.  
  Lemma~\ref{lemma:total_ssf}, Lemma~\ref{lemma:soundness_reduction_ssf}, 
  and Lemma~\ref{corollary:normalization_preserving_ssf} allow us to conclude that 
  $[n_1/z](z\ n_2) \redto^* [n_1/z]^{T_2 \to T_1}(z\ n_2)$, $\Gamma \vdash [n_1/z]^{T_2 \to T_1}(z\ n_2):T_2$,
  and $[n_1/z]^{T_2 \to T_1}(z\ n_2)$ is normal.  Thus, 
  $t_1\ t_2 \redto^* n_1\ n_2 = [n_1/z](z\ n_2) \normto [n_1/z]^{T_2 \to T_1}(z\ n_2) \in \interp{T_2}_\Gamma$.
  
\item[Case.]\ \\
  \begin{center}
    \begin{math}
      $$\mprset{flushleft}
      \inferrule* [right=] {
        \Gamma, X : *_p \vdash t : T
      }{\Gamma \vdash \Lambda X:*_p.t:\forall X:*_p.T}
    \end{math}
  \end{center}
  By the induction hypothesis and definition of the interpretation of types 
  $t \in \interp{T}_{\Gamma,X:*_p}$, $t \normto n \in \interp{T}_{\Gamma,X:*_p}$ and 
  $\Lambda X:*_p.n \in \interp{T}_{\Gamma}$.  Again, by definition of the interpretation 
  of types $\Lambda X:*_p.t \normto \Lambda X:*_p.n \in \interp{T}_{\Gamma}$.

\item[Case.]\ \\
  \begin{center}
    \begin{math}
      $$\mprset{flushleft}
      \inferrule* [right=] {
        \Gamma \vdash t:\forall X:*_l.T_1
        \\
        \Gamma \vdash T_2:*_l
      }{\Gamma \vdash t[T_2]: [T_2/X]T_1}
    \end{math}
  \end{center}
  By the induction hypothesis $t \in \interp{\forall X:*_l.T_1}_\Gamma$ and by the 
  definition of the interpretation of types we know 
  $t \normto n \in \interp{\forall X:*_l.T_1}_\Gamma$.  We case
  split on whether or not $n$ is a type abstraction. If not then again, by the 
  definition of the interpretation of types 
  $n[T_2] \in \interp{[T_2/X]T_1}_\Gamma$, therefore 
  $t \in \interp{[T_2/X]T_1}_\Gamma$.  Suppose $n \equiv \Lambda X:*_l.n'$.  Then 
  $t[T_2] \redto^* (\Lambda X:*_l.n')[T_2] \redto [T_2/X]n'$.  By the definition 
  of the interpretation of types $n' \in \interp{T_1}_{\Gamma,X:*_l}$. Therefore, by
  Lemma~\ref{lemma:type_sub_ssf} $[T_2/X]n' \in \interp{[T_2/X]T_1}_{\Gamma}$.
\end{itemize}
\end{proof}
\noindent Therefore, we conclude normalization of SSF.
\begin{corollary}[Normalization]
  If $\Gamma \vdash t:T$, then there exists a normalform $n$, such
  that $t \normto n$.
\end{corollary}
% subsubsection concluding_normalization_ssf (end)
% subsection normalization_stratified_system_f (end)

