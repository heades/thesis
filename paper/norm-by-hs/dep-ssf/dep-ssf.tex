\section{Dependent Stratified System $\F$ (DSS$\F$)}
\label{sec:dependent_ssf_with_type_equality}
DSS$\F$ is SSF extended with dependent types and equations between
terms.  Equations between terms are an important concept in
Martin-L\"of type theory~\cite{hofmann+94,nps90}, and play a central
role also in dependently typed programming languages, such as the
second author's \textsc{Guru} language~\cite{stump+09}.  The syntax
and reduction rules are defined in Fig.~\ref{fig:syntax_ssfe}.  The
kind-assignment rules are defined in
Fig.~\ref{fig:kinding_rules_ssfe}.  One thing to note regarding the
kind-assignment rules is that the level of an equation is the same
level as the type of the terms in the equation.  The terms used in an
equation must have the same type.  Finally, the type-assignment rules
are defined in Fig.~\ref{fig:typing_rules_ssfe}.  Note that $t_1 \join t_2$ denotes
there exists a term $t$ such that $t_1 \redto^* t$ and $t_2 \redto^* t$.

\begin{figure}[t]
  \begin{center}
    \begin{tabular}{lll}
      \begin{tabular}{lll}
        $t$ & $:=$ & $x$   $|$ $\lambda x:\phi.t$   $|$ $t\ t$ $|$  
        $\Lambda X:K.t$ $|$ $t[\phi]$ $|$ $join$\\
        $\phi$ & $:=$ & $X$   $|$ $\Pi x:\phi.\phi$ $|$ $\forall X:K.\phi$ $|$ $t = t$\\
        $K$ & $:=$ & $*_0$ $|$ $*_1$ $|$ $\ldots$
      \end{tabular}
      \\ \\
      \begin{tabular}{lll}
        $(\Lambda X:*_p.t)[\phi]$ & $\rightsquigarrow$ & $[\phi/X]t$\\
        $(\lambda x:\phi.t)t'$    & $\rightsquigarrow$ & $[t'/x]t$\\
      \end{tabular}
    \end{tabular}
    \end{center}
  \caption[]{Syntax of Terms, Types, and Kinds and Reduction Rules for DSS$\F$}
  \label{fig:syntax_ssfe}
  \label{fig:reduction_rules_ssfe}
\end{figure}

\begin{figure}[t]
  \begin{center}
    \setlength{\tabcolsep}{1pt}
    \begin{tabular}{cccc}
      \begin{math}
        $$\mprset{flushleft}
        \inferrule* [right=] {
          \Gamma(X) = *_p
          \\\\
          \Gamma\ Ok
          \\
          p \leq q
        }{\Gamma \vdash X : *_q}
      \end{math}
      &
      \begin{math}
        $$\mprset{flushleft}
        \inferrule* [right=] {
          \Gamma \vdash \phi_1 : *_p
          \\\\
          \Gamma,x:\phi_1 \vdash \phi_2 : *_q
        }{\Gamma \vdash \Pi x:\phi_1.\phi_2 : *_{max(p,q)}}
      \end{math}
      &
      \begin{math}
        $$\mprset{flushleft}
        \inferrule* [right=] {
          \Gamma,X : *_q \vdash \phi : *_p
        }{\Gamma \vdash \forall X:*_q.\phi : *_{max(p,q)+1}}
      \end{math}
      &
      \begin{math}
        $$\mprset{flushleft}
        \inferrule* [right=] {
	  \Gamma \vdash t_1:\phi
          \\\\
	  \Gamma \vdash t_2:\phi
          \\
          \Gamma \vdash \phi:*_p
        }{\Gamma \vdash t_1 = t_2 : *_p}
      \end{math}
    \end{tabular}	
    \caption[]{DSS$\F$ Kinding Rules}
    \label{fig:kinding_rules_ssfe}
  \end{center}
\end{figure}

\begin{figure}[t]
  \begin{center}
    \setlength{\tabcolsep}{1pt}
    \begin{tabular}{cccc}
      \begin{math}
        $$\mprset{flushleft}
        \inferrule* [right=] {
          \Gamma\ Ok
          \\\\
          \Gamma(x) = \phi
        }{\Gamma \vdash x : \phi}
      \end{math}  
      &
      \begin{math}
        $$\mprset{flushleft}
        \inferrule* [right=] {
          \Gamma,x : \phi_1 \vdash t : \phi_2
        }{\Gamma \vdash \lambda x : \phi_1.t : \Pi x:\phi_1.\phi_2}
      \end{math} 
      &
      \begin{math}
        $$\mprset{flushleft}
        \inferrule* [right=] {
          \Gamma \vdash t_2 : \phi_1
          \\\\
          \Gamma \vdash t_1 : \Pi x:\phi_1.\phi_2 
        }{\Gamma \vdash t_1\ t_2 : [t_2/x]\phi_2}
      \end{math}  
      \\ \\
      \begin{math}
        $$\mprset{flushleft}
        \inferrule* [right=] {
          \Gamma, X : *_l \vdash t : \phi
        }{\Gamma \vdash \Lambda X:*_l.t:\forall X : *_l.\phi}
      \end{math} 
      &
      \begin{math}
        $$\mprset{flushleft}
        \inferrule* [right=] {
          \Gamma \vdash t:\forall X:*_l.\phi_1
          \\\\
          \Gamma \vdash \phi_2:*_l
        }{\Gamma \vdash t[\phi_2]: [\phi_2/X]\phi_1}
      \end{math} 
      & 
      \begin{math}
        $$\mprset{flushleft}
        \inferrule* [right=] {
          t_1 \join t_2
          \\
	  \Gamma \vdash t_1:\phi
          \\\\
          \Gamma\ Ok
          \\\
          \ \Gamma \vdash t_2:\phi
        }{\Gamma \vdash join : t_1 = t_2}
      \end{math}
      \\ \\
      \begin{math}
        $$\mprset{flushleft}
        \inferrule* [right=] {
          \Gamma \vdash t_0:t_1 = t_2
          \\\\
          \Gamma \vdash t:[t_1/x]\phi
        }{\Gamma \vdash t:[t_2/x]\phi}
      \end{math}\\
    \end{tabular}
    
    \caption[]{DSS$\F$ Type-Assignment Rules}
    \label{fig:typing_rules_ssfe}
  \end{center}
\end{figure}

\begin{figure}
  \begin{center}
    \begin{tabular}{lll}
      \begin{math}
        $$\mprset{flushleft}
        \inferrule* [right=$\text{TE}_{\text{Q}_1}$] {
          \Gamma \vdash p:t_1 = t_2
        }{\Gamma \vdash [t_1/x]\phi \approx [t_2/x]\phi}
      \end{math}
      &
      \begin{math}
        $$\mprset{flushleft}
        \inferrule* [right=$\text{TE}_{\text{Q}_2}$] {
          \Gamma \vdash [t_1/x]\phi \approx [t_1/x]\phi'
          \\
          \Gamma \vdash p:t_1 = t_2
        }{\Gamma \vdash [t_1/x]\phi \approx [t_2/x]\phi'}
      \end{math}
    \end{tabular}
  \end{center}
  \caption{DSS$\F$ Type Syntactic Equality}
  \label{fig:type_equality_ssfe}
\end{figure}

\subsection{Syntactic Inversion}
\label{subsec:basic_syntactic_lemmas_ssfe}
This section covers syntactic inversion of the typing relation.  In
Sect.~\ref{subsec:concluding_normalization_ssfe} we define the
interpretation of types and semantic inversion must hold for this
definition.  Since the interpretation of types is simply the
restriction of the typing relation to normal forms syntactic inversion
implies semantic inversion.  Hence, we only need to show syntactic
inversion.  Syntactic inversion depends on a judgment called type
syntactic equality.  It is defined in
Fig.~\ref{fig:type_equality_ssfe}.  We show that syntactic
conversion holds for syntactic type equality in
Lemma~\ref{lemma:type_syntactic_conversion_ssfe}.  We only state the
syntactic inversion lemma for the typing relation, because syntactic
inversion for the kinding relation is trivial.  Note that we use
syntactic inversion for kinding throughout the paper without
indication.  First we define some convenient syntax used in the
statement of the following lemma.  We write $\exists
(a_1,a_2,\ldots,a_n)$ for $\exists a_1.\exists a_2\ldots\exists a_n$.
Note that the proofs of the following lemmas depend on a number of
other auxiliary lemmas.  They can also be found in the companion
report.

The first lemma is type syntactic conversion which states that if a
term $t$ inhabits a type $\phi$ then it inhabits all types equivalent
to $\phi$.  Following this is injectivity of $\Pi$-types which is
needed in the proof of syntactic inversion.  Finally, we conclude
syntactic inversion.
\begin{lemma}[Type Syntactic Conversion]
  \label{lemma:type_syntactic_conversion_ssfe}
  If $\Gamma \vdash t:\phi$ and $\Gamma \vdash \phi \approx \phi'$ then $\Gamma \vdash t:\phi'$.
\end{lemma}

\begin{lemma}[Injectivity of $\Pi$-Types for Type Equality]
  \label{lemma:injectivity_of_pi-types_for_type_equality_ssfe}
  If $\Gamma \vdash \Pi y:\phi_1.\phi_2 \approx \Pi y:\phi'_1.\phi'_2$ then
  $\Gamma \vdash \phi_1 \approx \phi'_1$ and $\Gamma,y:\phi_1 \vdash \phi_2 \approx \phi'_2$.
\end{lemma}

\begin{lemma}[Syntactic Inversion]
  \label{lemma:inversion_ssfe}
  \begin{itemize}\itemsep2pt
  \item[]
  \item[i.] If $\Gamma \vdash \lambda x:\phi_1.t:\phi$
    then $\exists \phi_2.\ \Gamma,x:\phi_1 \vdash t:\phi_2 \land 
    \Gamma \vdash \Pi x:\phi_1.\phi_2 \approx \phi$.

  \item[ii.] If $\Gamma \vdash t_1\ t_2:\phi$ then $\exists (x,\phi_1, \phi_2).$\\
    $\Gamma \vdash t_1:\Pi x:\phi_1.\phi_2 \land
    \Gamma \vdash t_2:\phi_1 \land \Gamma \vdash \phi \approx
     [t_2/x]\phi_2$.

  \item[iii.] If $\Gamma \vdash \Lambda X:*_l.t:\phi$
    then $\exists \phi'.\ \Gamma,X:*_l \vdash t:\phi'\ \land $
    $\Gamma \vdash \phi \approx \forall X:*_l.\phi'$.

  \item[iv.] If $\Gamma \vdash t[\phi_2]:\phi$ then 
    $\exists (\phi_1, \phi_2)$. \\
    $\Gamma \vdash t:\forall X:*_l.\phi_1$ $\land$
    $\Gamma \vdash \phi_2:*_l$ $\land$ 
    $\Gamma \vdash \phi \approx [\phi_2/X]\phi_1$.

  \item[v.] If $\Gamma \vdash join:\phi$ then
    $\exists (t_1,t_2,\phi')$.\\
    $t_1 \join t_2$ $\land$
    $\Gamma \vdash t_1:\phi'$ $\land$
    $\Gamma \vdash t_2:\phi'$ $\land$
    $\Gamma \vdash \phi \approx t_1 = t_2$ $\land$
    $\Gamma\ Ok$.
  \end{itemize}
\end{lemma}
% subsection basic_syntactic_lemmas (end)

\subsection{The Ordering on Types and The Hereditary Substitution Function}
\label{sec:the_hereditary_substitution_function_ssfe}
The ordering on types is defined as follows:
\begin{definition}
  The ordering $>_\Gamma$ is defined as the least relation satisfying the universal closure
  of the following formulas:
  \begin{center}
    \begin{tabular}{lll}
      $\Pi x:\phi_1.\phi_2$ & $>_\Gamma$ & $\phi_1$\\
      $\Pi x:\phi_1.\phi_2$ & $>_{\Gamma}$ & $[t/x]\phi_2$, where $\Gamma \vdash t:\phi_1$.\\
      $\forall X:*_l.\phi$  & $>_\Gamma$ & $[\phi'/X]\phi$, where $\Gamma \vdash \phi':*_l$.\\
    \end{tabular}
  \end{center}
  \label{def:ordering_ssfe}
\end{definition}
Just as we seen before this is the ordering used in the proofs of the
properties of the hereditary substitution function which will be
defined next.  As one might have expected this is a well-founded
ordering.
\begin{thm}[Well-Founded Ordering]
  The ordering $>_\Gamma$ is well-founded on types $\phi$ such that 
  $\Gamma \vdash \phi:*_l$ for some $l$.
  \label{thm:well-founded_ordering_ssfe}
\end{thm}
The type-syntactic-equality relation respects this ordering.
\begin{lemma}[]
  \label{lemma:typeq_ordering_ssfe}
  If $\Gamma \vdash \phi' \approx \phi''$ and $\phi >_{\Gamma} \phi'$ then $\phi >_\Gamma \phi''$.
\end{lemma}
The following lemma is used in the proof of totality of the hereditary substitution function.
\begin{lemma}[]
  \label{lemma:A_prop_ssfe}
  If $\Gamma \vdash \psi \approx \Pi y:\phi_1.\phi_2$ and $\psi$ is a subexpression of $\phi''$ then
  $\phi'' >_\Gamma \phi_1$ and $\phi'' >_{\Gamma,y:\phi_1} \phi_2$.
\end{lemma}

We now define the hereditary substitution function for DSS$\F$.  The
function is fully defined in
Fig.~\ref{fig:hereditary_substitution_function_ssfe}.  We do not
repeat the definition of $ctype_\phi$ for DSS$\F$, because it is
exactly the same as the previous system. The definition is rather
straight forward.  Due to space constraints and the fact that all of
the properties of the hereditary substitution function are so similar
to the previous type theory they can be found in the companion report.
\begin{figure}[t]
  \small
  \begin{itemize}
  \item[] $[t/x]^\phi x = t$

  \item[] $[t/x]^\phi y = y$\\
    \begin{tabular}{lll}
      & Where $y$ is a variable distinct from $x$.\\
    \end{tabular}

  \item[] $[t/x]^\phi join = join$

  \item[] $[t/x]^\phi (\lambda y:\phi'.t') = \lambda y:\phi'.([t/x]^\phi t')$

  \item[] $[t/x]^\phi (\Lambda X:*_l.t') = \Lambda X:*_l.([t/x]^\phi t')$

  \item[] $[t/x]^\phi (t_1\ t_2) = ([t/x]^\phi t_1)\ ([t/x]^\phi t_2)$\\
    \begin{tabular}{lll}
      & Where $([t/x]^\phi t_1)$ is not a $\lambda$-abstraction, 
      $([t/x]^\phi t_1)$ and $t_1$ are $\lambda$-abstractions,\\ 
      & or $ctype_\phi(x,t_1)$ is undefined.
    \end{tabular}

  \item[] $[t/x]^{\phi} (t_1\ t_2) = [([t/x]^{\phi} t_2)/y]^{\phi''} s'_1$\\
    \begin{tabular}{lll}
      & Where $([t/x]^{\phi} t_1) \equiv \lambda y:\phi''.s'_1$ 
        for some $y$ and $s'_1$ and $t_1$ is not a $\lambda$-abstraction, and \\
      & $ctype_\phi(x,t_1) = \Pi y:\phi''.\phi'$.\\
    \end{tabular}

  \item[] $[t/x]^\phi (t'[\phi']) = ([t/x]^\phi t')[\phi']$\\
    \begin{tabular}{lll}
      & Where $[t/x]^\phi t'$ is not a type abstraction or
      $t'$ and $[t/x]^\phi t'$ are type abstractions.
    \end{tabular}

    \item[] $[t/x]^{\phi} (t'[\phi']) = [\phi'/X]s'_1$\\
      \begin{tabular}{lll}
        & Where $[t/x]^{\phi} t' \equiv \Lambda X:*_l.s'_1$,
        for some $X$, $s'_1$ and $\Gamma \vdash \phi':*_q$, such that, $q \leq l$ and\\
	& $t'$ is not a type abstraction.
      \end{tabular}
  \end{itemize}
  \caption{Hereditary Substitution Function for Stratified System $\F$}
  \label{fig:hereditary_substitution_function_ssfe}
\end{figure}
% subsection the_hereditary_substitution_function (end)

\subsection{Concluding Normalization}
\label{subsec:concluding_normalization_ssfe}
We are now ready to conlcude normalization.  We will accomplish this
by first defining the interpretation of types, then proving the
interpretation of types are closed under hereditary substitutions, and
finally proving type soundness.  The interpretation of types are
defined exactly the same way as they were for SS$\Fp$.  We do not
repeat that definition here.  Just as in the previous type system
semantic inversion must hold for the previous definition.  This is
implied by syntactic inversion, which we proved in the previous
section.  The next two lemmas are used in the proof of the type
soundness theorem.

\begin{corollary}[Type Substitution for the Interpretation of Types]
  If $n \in \interp{\phi'}_{\Gamma,X:*_l,\Gamma'}$ and 
  $\Gamma \vdash \phi:*_l$ then 
  $[\phi/X]n \in \interp{[\phi/X]\phi'}_{\Gamma,[\phi/X]\Gamma'}$.
  \label{lemma:type_sub_ssfe}
\end{corollary}

\begin{lemma}[Semantic Equality]
  \label{lemma:equality_of_interpretation_of_types_ssfe}
  If $\Gamma \vdash p:t_1 = t_2$ then $\interp{[t_1/x]\phi}_\Gamma = \interp{[t_2/x]\phi}_\Gamma$.
\end{lemma}
We now conclude type soundness for SS$\F$, and hence, normalization.  

\begin{lemma}[Hereditary Substitution for the Interpretation of Types]
  If $n' \in \interp{\phi'}_{\Gamma,x:\phi,\Gamma'}$, $n \in \interp{\phi}_\Gamma$,
  then $[n/x]^\phi n' \in \interp{[n/x]\phi'}_{\Gamma,[n/x]\Gamma'}$.
  \label{lemma:interpretation_of_types_closed_substitution_ssfe}
\end{lemma}

\begin{thm}[Type Soundness]
  If $\Gamma \vdash t:\phi$ then $t \in \interp{\phi}_\Gamma$.
  \label{thm:soundness_ssfe}
\end{thm}

\begin{corollary}[Normalization]
  If $\Gamma \vdash t:\phi$ then $t \normto n$.
\end{corollary}
% subsection concluding_normalization_ssfp (end)
% section dependent_ssf_with_type_equality_(dss$\f$) (end)