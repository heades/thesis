\section{Dependent Stratified System $\F$ (DSS$\F$)}
\label{sec:dependent_ssf_with_type_equality}
DSS$\F$ is SSF extended with dependent types and equations between
terms.  Equations between terms are an important concept in
Martin-L\"of type theory~\cite{hofmann+94,nps90}, and play a central
role also in dependently typed programming languages, such as the
\textsc{Guru} language~\cite{stump+09}, $\Sep$
(Section~\ref{chap:separation_of_proof_from_program}), the freedom of
speech language (Section~\ref{chap:freedom_of_speech}), Coq
\cite{CoqRefMan:2008}, and many more.  The syntax and reduction rules
are defined in Fig.~\ref{fig:syntax_ssfe}.  The kind-assignment rules
are defined in Fig.~\ref{fig:kinding_rules_ssfe}.  One thing to note
regarding the kind-assignment rules is that the level of an equation
is the same level as the type of the terms in the equation.  The terms
used in an equation must have the same type -- also known as
homogenous equality.  Finally, the type-assignment rules are defined
in Fig.~\ref{fig:typing_rules_ssfe}.  Note that $t_1 \join t_2$
denotes there exists a term $t$ such that $t_1 \redto^* t$ and $t_2
\redto^* t$.

\begin{figure}[t]
  \begin{center}
    \begin{tabular}{lll}
      \begin{tabular}{lll}
        $t$ & $:=$ & $x$   $|$ $\lambda x:\phi.t$   $|$ $t\ t$ $|$  
        $\Lambda X:K.t$ $|$ $t[\phi]$ $|$ $join$\\
        $\phi$ & $:=$ & $X$   $|$ $\Pi x:\phi.\phi$ $|$ $\forall X:K.\phi$ $|$ $t = t$\\
        $K$ & $:=$ & $*_0$ $|$ $*_1$ $|$ $\ldots$
      \end{tabular}
      \\ \\
      \begin{tabular}{lll}
        $(\Lambda X:*_p.t)[\phi]$ & $\rightsquigarrow$ & $[\phi/X]t$\\
        $(\lambda x:\phi.t)t'$    & $\rightsquigarrow$ & $[t'/x]t$\\
      \end{tabular}
    \end{tabular}
    \end{center}
  \caption[]{Syntax of Terms, Types, and Kinds and Reduction Rules for DSS$\F$}
  \label{fig:syntax_ssfe}
  \label{fig:reduction_rules_ssfe}
\end{figure}

\begin{figure}[t]
  \begin{center}
    \setlength{\tabcolsep}{1pt}
    \begin{tabular}{cccc}
      \begin{math}
        $$\mprset{flushleft}
        \inferrule* [right=] {
          \Gamma(X) = *_p
          \\\\
          \Gamma\ Ok
          \\
          p \leq q
        }{\Gamma \vdash X : *_q}
      \end{math}
      &
      \begin{math}
        $$\mprset{flushleft}
        \inferrule* [right=] {
          \Gamma \vdash \phi_1 : *_p
          \\\\
          \Gamma,x:\phi_1 \vdash \phi_2 : *_q
        }{\Gamma \vdash \Pi x:\phi_1.\phi_2 : *_{max(p,q)}}
      \end{math}
      &
      \begin{math}
        $$\mprset{flushleft}
        \inferrule* [right=] {
          \Gamma,X : *_q \vdash \phi : *_p
        }{\Gamma \vdash \forall X:*_q.\phi : *_{max(p,q)+1}}
      \end{math}
      &
      \begin{math}
        $$\mprset{flushleft}
        \inferrule* [right=] {
	  \Gamma \vdash t_1:\phi
          \\\\
	  \Gamma \vdash t_2:\phi
          \\
          \Gamma \vdash \phi:*_p
        }{\Gamma \vdash t_1 = t_2 : *_p}
      \end{math}
    \end{tabular}	
    \caption[]{DSS$\F$ Kinding Rules}
    \label{fig:kinding_rules_ssfe}
  \end{center}
\end{figure}

\begin{figure}[t]
  \begin{center}
    \setlength{\tabcolsep}{1pt}
    \begin{tabular}{cccc}
      \begin{math}
        $$\mprset{flushleft}
        \inferrule* [right=] {
          \Gamma\ Ok
          \\\\
          \Gamma(x) = \phi
        }{\Gamma \vdash x : \phi}
      \end{math}  
      &
      \begin{math}
        $$\mprset{flushleft}
        \inferrule* [right=] {
          \Gamma,x : \phi_1 \vdash t : \phi_2
        }{\Gamma \vdash \lambda x : \phi_1.t : \Pi x:\phi_1.\phi_2}
      \end{math} 
      &
      \begin{math}
        $$\mprset{flushleft}
        \inferrule* [right=] {
          \Gamma \vdash t_2 : \phi_1
          \\\\
          \Gamma \vdash t_1 : \Pi x:\phi_1.\phi_2 
        }{\Gamma \vdash t_1\ t_2 : [t_2/x]\phi_2}
      \end{math}  
      \\ \\
      \begin{math}
        $$\mprset{flushleft}
        \inferrule* [right=] {
          \Gamma, X : *_l \vdash t : \phi
        }{\Gamma \vdash \Lambda X:*_l.t:\forall X : *_l.\phi}
      \end{math} 
      &
      \begin{math}
        $$\mprset{flushleft}
        \inferrule* [right=] {
          \Gamma \vdash t:\forall X:*_l.\phi_1
          \\\\
          \Gamma \vdash \phi_2:*_l
        }{\Gamma \vdash t[\phi_2]: [\phi_2/X]\phi_1}
      \end{math} 
      & 
      \begin{math}
        $$\mprset{flushleft}
        \inferrule* [right=] {
          t_1 \join t_2
          \\
	  \Gamma \vdash t_1:\phi
          \\\\
          \Gamma\ Ok
          \\\
          \ \Gamma \vdash t_2:\phi
        }{\Gamma \vdash join : t_1 = t_2}
      \end{math}
      \\ \\
      \begin{math}
        $$\mprset{flushleft}
        \inferrule* [right=] {
          \Gamma \vdash t_0:t_1 = t_2
          \\\\
          \Gamma \vdash t:[t_1/x]\phi
        }{\Gamma \vdash t:[t_2/x]\phi}
      \end{math}\\
    \end{tabular}
    
    \caption[]{DSS$\F$ Type-Assignment Rules}
    \label{fig:typing_rules_ssfe}
  \end{center}
\end{figure}

\begin{figure}
  \begin{center}
    \begin{tabular}{lll}
      \begin{math}
        $$\mprset{flushleft}
        \inferrule* [right=$\text{TE}_{\text{Q}_1}$] {
          \Gamma \vdash p:t_1 = t_2
        }{\Gamma \vdash [t_1/x]\phi \approx [t_2/x]\phi}
      \end{math}
      &
      \begin{math}
        $$\mprset{flushleft}
        \inferrule* [right=$\text{TE}_{\text{Q}_2}$] {
          \Gamma \vdash [t_1/x]\phi \approx [t_1/x]\phi'
          \\
          \Gamma \vdash p:t_1 = t_2
        }{\Gamma \vdash [t_1/x]\phi \approx [t_2/x]\phi'}
      \end{math}
    \end{tabular}
  \end{center}
  \caption{DSS$\F$ Type Syntactic Equality}
  \label{fig:type_equality_ssfe}
\end{figure}

\subsection{Basic Syntactic Results}
\label{subsec:basic_syntactic_lemmas_ssfe}
In this section we give a number of basic results.  The reader may
wish to skip this section, and only refer to the results while reading
the rest of the chapter.  The most interesting results covered in this
section are the proof that type-syntactic equality is an equivalence
relation, and syntactic inversion of the typing relation.  The latter
depends on a judgment called type-syntactic equality.  It is defined
in Fig.~\ref{fig:type_equality_ssfe}.  We show that syntactic
conversion holds for syntactic-type equality in
Lemma~\ref{lemma:type_syntactic_conversion_ssfe}.  We only state the
syntactic inversion lemma for the typing relation, because syntactic
inversion for the kinding relation is trivial.  Note that we use
syntactic inversion for kinding throughout this chapter without
indication.  We now list several basic lemmata:
\begin{lemma}
  If $\Gamma \vdash \phi:*_p$ then $\Gamma\ Ok$.
  \label{lemma:kinding_ok_ssfe}
\end{lemma}
\begin{proof}
  Similar to the proof of the same lemma for SS$\Fp$.
\end{proof}
\begin{lemma}[Type Substitution for Kinding, Typing, and Context-Ok]
  Suppose $\Gamma \vdash \phi':*_p$.  Then
  \begin{itemize}
  \item[i.] if $\Gamma,X:*_p,\Gamma' \vdash \phi:*_q$ with a
    derivation of depth $d$, then $\Gamma,[\phi'/X]\Gamma' \vdash
    [\phi'/X]\phi:*_q$ with a derivation of depth $d$,
    
  \item[ii.] if $\Gamma, X:*_l,\Gamma' \vdash t:\phi$ with a
    derivation of depth $d$, then $\Gamma,[\phi'/X]\Gamma' \vdash
    [\phi'/X]t:[\phi'/X]\phi$ with a derivation of depth $d$, and
    
  \item[iii.] if $\Gamma,X:*_p,\Gamma'\ Ok$ with a derivation of depth $d$, then 
    $\Gamma,[\phi'/X]\Gamma'\ Ok$ with a derivation of depth $d$.
  \end{itemize}
  \label{lemma:substitution_for_kinding_ssfe}
\end{lemma}
\begin{proof}
  Similar to the proof of the same lemma for SS$\Fp$.
\end{proof}

\begin{lemma}[Term Substitution for Kinding, Typing, and Context-Ok]
  \label{lemma:term_substitution_for_kinding_context_ok_ssfe}
  Suppose $\Gamma \vdash t':\phi'$.  Then
  \begin{itemize}
  \item[i.] if $\Gamma,x:\phi',\Gamma' \vdash \phi:*_l$ with a
    derivation of depth $d$, then $\Gamma,[t'/x]\Gamma' \vdash
    [t'/x]\phi:*_l$ with a derivation of depth $d$,
  \item[ii.] if $\Gamma, x:\phi',\Gamma' \vdash t:\phi$ with a 
    derivation of depth $d$, then $\Gamma,[t'/x]\Gamma' \vdash [t'/x]t:[t'/x]\phi$ 
    with a derivation of depth $d$, and
  \item[iii.] if $\Gamma,x:\phi',\Gamma'\ Ok$ with a derivation of depth $d$, then 
    $\Gamma,[t'/x]\Gamma'\ Ok$ with a derivation of depth $d$.
  \end{itemize}
\end{lemma}
\begin{proof}
  All cases hold by straightforward induction on the $d$.
\end{proof}

\begin{lemma}[Context Weakening for Kinding and Typing]
  \label{lemma:context_weakening_for_kinding_and_typing_ssfe}
  Assume $\Gamma,\Gamma'',\Gamma'\ Ok$, $\Gamma,\Gamma' \vdash \phi:*_p$ and 
  $\Gamma,\Gamma' \vdash t:\phi$.  Then i. $\Gamma,\Gamma'',\Gamma' \vdash \phi:*_p$ and
  ii. $\Gamma,\Gamma'',\Gamma' \vdash t:\phi$. 
\end{lemma}
\begin{proof}
  Similar to the proof of the same lemma for SS$\Fp$.
\end{proof}

\begin{lemma}[Regularity]
  If $\Gamma \vdash t:\phi$ then $\Gamma \vdash \phi:*_p$ for some $p$.
  \label{lemma:regularity_ssfe}
\end{lemma}
\begin{proof}
  This holds by straightforward induction on the form of the assumed
  typing derivation.
\end{proof}
\noindent
At this point we show that type-syntactic equality is indeed an
equivalence relation, and that it respects substitution.  Each of
these may be used throughout the remainder of the chapter without
explicit mention.
\begin{lemma}[Transitivity of Type Equality]
  \label{lemma:transitivity_of_type_equality_ssfe}
  If $\Gamma \vdash \phi_1 \approx \phi_2$ and $\Gamma \vdash \phi_2 \approx \phi_3$ then
  $\Gamma \vdash \phi_1 \approx \phi_3$.
\end{lemma}
\begin{proof}
  This holds by straightforward induction on the form of the assumed
  type equality derivation.
\end{proof}
\begin{lemma}[Symmetry of Type Equality]
  \label{lemma:symmetry_of_type_equality}
  If $\Gamma \vdash \phi \approx \phi'$ then $\Gamma \vdash \phi' \approx \phi$.
\end{lemma}
\begin{proof}
  This holds by straightforward induction on the form of the assumed
  type equality derivation.
\end{proof}
\begin{lemma}[Substitution for Type Equality]
  \label{lemma:substitution_for_type_equality_ssfe}
  If $\Gamma,x:\phi,\Gamma' \vdash \phi' \approx \phi''$ and $\Gamma \vdash t:\phi$ then
  $\Gamma,[t/x]\Gamma' \vdash [t/x]\phi' \approx [t/x]\phi''$.
\end{lemma}
\begin{proof}
  This holds by straightforward induction on the form of the assumed
  type equality derivation.
\end{proof}
\begin{lemma}[]
  \label{lemma:pis_are_equal_to_pis_ssfe}
  If $\Gamma \vdash \phi \approx \Pi j:\phi_1.\phi_2$ then there exists a term $h$ and types $\phi'_1$ and $\phi'_2$
  such that $\phi \equiv \Pi h:\phi'_1.\phi'_2$.
\end{lemma}
\begin{proof}
  This is a proof by induction on the form of the assume type-equality derivation.

\begin{itemize}
\item[Case.]\ \\
  \begin{center}
    \begin{math}
      $$\mprset{flushleft}
      \inferrule* [right=$\text{TE}_{\text{Q}_1}$] {
        \Gamma \vdash p:t_1 = t_2
      }{\Gamma \vdash [t_1/x](\Pi j:\phi'_1.\phi'_2) \approx [t_2/x](\Pi j:\phi'_1.\phi'_2)}
    \end{math}
  \end{center}
  Trivial, because $\phi$ must also be a $\Pi$-type.
  
\item[Case.]\ \\
  \begin{center}
    \begin{math}
      $$\mprset{flushleft}
      \inferrule* [right=$\text{TE}_{\text{Q}_2}$] {
        \Gamma \vdash [t_1/x]\phi' \approx [t_1/x](\Pi j:\phi'_1.\phi'_2)
        \\
        \Gamma \vdash p:t_1 = t_2
      }{\Gamma \vdash [t_1/x]\phi' \approx [t_2/x](\Pi j:\phi'_1.\phi'_2)}
    \end{math}
  \end{center}
  By the induction hypothesis $\phi \equiv [t_1/x]\phi' \equiv \Pi h:\psi_1.\psi_2$ for some term
  $h$ and types $\psi_1$ and $\psi_2$.  
\end{itemize}
\end{proof}
\begin{lemma}[Type Equality Context Conversion]
  \label{lemma:type_equality_context_conversion}
  If $\Gamma,x:[\phi_1/X]\phi,\Gamma' \vdash t:\phi'$ and $\Gamma \vdash p:\phi_1 = \phi_2$ then
  $\Gamma,x:[\phi_2/X]\phi,\Gamma' \vdash t:\phi'$.
\end{lemma}
\begin{proof}
  This hold by straightforward induction on the assumed typing derivation.
\end{proof}
The next lemma is type syntactic conversion which states that if a
term $t$ inhabits a type $\phi$, then it inhabits all types equivalent
to $\phi$.  Following this is injectivity of $\Pi$-types which is
needed in the proof of syntactic inversion.  
\begin{lemma}[Type Syntactic Conversion]
  \label{lemma:type_syntactic_conversion_ssfe}
  If $\Gamma \vdash t:\phi$ and $\Gamma \vdash \phi \approx \phi'$ then $\Gamma \vdash t:\phi'$.
\end{lemma}
\begin{proof}
  If $\Gamma \vdash t:\phi$ and $\Gamma \vdash \phi \approx \phi'$ then we know several
  things: $\phi \equiv [\bar{t}/\bar{x}]\phi''$, $\phi' \equiv [\bar{t'}/\bar{x'}]\phi''$,
  $\Gamma \vdash \bar{p}:\bar{t} = \bar{t'}$, and $\Gamma \vdash t:[\bar{t}/\bar{x}]\phi''$ for
  some type $\phi''$.  Suppose each vector has $i$ elements.  Then by applying the conversion
  type-checking rule $i$ times with the appropriate proof from our vector of proofs we will obtain
  $\Gamma \vdash t:[\bar{t'}/\bar{x}]\phi''$.  This last result is exactly,
  $\Gamma \vdash t:\phi'$.
\end{proof}
\begin{lemma}[Injectivity of $\Pi$-Types for Type Equality]
  \label{lemma:injectivity_of_pi-types_for_type_equality_ssfe}
  If $\Gamma \vdash \Pi y:\phi_1.\phi_2 \approx \Pi y:\phi'_1.\phi'_2$ then
  $\Gamma \vdash \phi_1 \approx \phi'_1$ and $\Gamma,y:\phi_1 \vdash \phi_2 \approx \phi'_2$.
\end{lemma}
\begin{proof}
  This is a proof by induction on the form of the assumed typing derivation.
\begin{itemize}
\item[Case.]\ \\
  \begin{center}
    \begin{math}
      $$\mprset{flushleft}
      \inferrule* [right=$\text{TE}_{\text{Q}_1}$] {
        \Gamma \vdash p:t_1 = t_2
      }{\Gamma \vdash [t_1/x]\phi' \approx [t_2/x]\phi'}
    \end{math}
  \end{center}
  Trivial, becasue $\phi'$ and $\phi''$ only differ by
  terms, which do not affect the ordering of types.
  
\item[Case.]\ \\
  \begin{center}
    \begin{math}
      $$\mprset{flushleft}
      \inferrule* [right=$\text{TE}_{\text{Q}_2}$] {
        \Gamma \vdash [t_1/x]\phi' \approx [t_1/x]\phi''
        \\
        \Gamma \vdash p:t_1 = t_2
      }{\Gamma \vdash [t_1/x]\phi' \approx [t_2/x]\phi''}
    \end{math}
  \end{center}
  By the induction hypothesis $\phi >_\Gamma [t_1/x]\phi''$, which implies
  that $\phi >_\Gamma [t_2/x]\phi''$.
\end{itemize}
\end{proof}

\noindent
Finally, we prove syntactic inversion, but first we will need some
convenient syntax that is used in the statement of the following
lemma.  We write $\exists (a_1,a_2,\ldots,a_n)$ for $\exists
a_1.\exists a_2\ldots\exists a_n$.
\begin{lemma}[Syntactic Inversion]
  \label{lemma:inversion_ssfe}
  \begin{itemize}\itemsep2pt
  \item[i.] If $\Gamma \vdash \lambda x:\phi_1.t:\phi$
    then $\exists \phi_2.\ \Gamma,x:\phi_1 \vdash t:\phi_2 \land 
    \Gamma \vdash \Pi x:\phi_1.\phi_2 \approx \phi$.

  \item[ii.] If $\Gamma \vdash t_1\ t_2:\phi$ then $\exists (x,\phi_1, \phi_2).$\\
    $\Gamma \vdash t_1:\Pi x:\phi_1.\phi_2 \land
    \Gamma \vdash t_2:\phi_1 \land \Gamma \vdash \phi \approx
     [t_2/x]\phi_2$.

  \item[iii.] If $\Gamma \vdash \Lambda X:*_l.t:\phi$
    then $\exists \phi'.\ \Gamma,X:*_l \vdash t:\phi'\ \land $
    $\Gamma \vdash \phi \approx \forall X:*_l.\phi'$.

  \item[iv.] If $\Gamma \vdash t[\phi_2]:\phi$ then 
    $\exists (\phi_1, \phi_2)$. \\
    $\Gamma \vdash t:\forall X:*_l.\phi_1$ $\land$
    $\Gamma \vdash \phi_2:*_l$ $\land$ 
    $\Gamma \vdash \phi \approx [\phi_2/X]\phi_1$.

  \item[v.] If $\Gamma \vdash join:\phi$ then
    $\exists (t_1,t_2,\phi')$.\\
    $t_1 \join t_2$ $\land$
    $\Gamma \vdash t_1:\phi'$ $\land$
    $\Gamma \vdash t_2:\phi'$ $\land$
    $\Gamma \vdash \phi \approx t_1 = t_2$ $\land$
    $\Gamma\ Ok$.
  \end{itemize}
\end{lemma}
\begin{proof}
  We prove all cases by induction on the form of the typing relation.

\begin{itemize}
\item[Case.] Part i.\\
  \begin{itemize}
  \item[Case.] \ \\
    \begin{center}
      \begin{math}
        $$\mprset{flushleft}
        \inferrule* [right=] {
          \Gamma,x : \phi_1 \vdash t : \phi_2
        }{\Gamma \vdash \lambda x : \phi_1.t : \Pi x:\phi_1.\phi_2}
      \end{math}
    \end{center}
    Trivial.
  
  \item[Case.] \ \\
    \begin{center}
      \begin{math}
        $$\mprset{flushleft}
        \inferrule* [right=] {
          \Gamma \vdash p:t_1 = t_2
          \\
          \Gamma \vdash \lambda x : \phi_1.t:[t_1/y]\phi'
        }{\Gamma \vdash \lambda x : \phi_1.t:[t_2/y]\phi'}
      \end{math}
    \end{center}
    Here $\phi \equiv [t_2/y]\phi'$.  By the induction hypothesis, where $\phi$ is 
    $[t_1/y]\phi'$, there exists a type $\phi_2$, such that $\Gamma,x:\phi_1 \vdash t:\phi_2$ and
    $\Gamma \vdash \Pi x:\phi_1.\phi_2 \approx [t_1/y]\phi'$, which implies that 
    $\Gamma \vdash [t'_1/y](\Pi x:\phi'_1.\phi'_2) \approx [t_1/y]\phi'$ and
    $\Gamma \vdash p':t'_1 = t_1$ for some terms $t'_1$ and $p'$.  Hence, by $\text{TEq}_2$
    $\Gamma \vdash [t_1/y](\Pi x:\phi'_1.\phi'_2) \approx [t_1/y]\phi'$ and by applying the same
    rule a second time, except using the proof $\Gamma \vdash p:t_1 = t_2$ we obtain
    $\Gamma \vdash [t_1/y](\Pi x:\phi'_1.\phi'_2) \approx [t_2/y]\phi'$.  Finally, using 
    $\text{TEq}_2$ a third time using $\Gamma \vdash p':t'_1 = t_1$ we obtain
    $\Gamma \vdash [t'_1/y](\Pi x:\phi'_1.\phi'_2) \approx [t_2/y]\phi'$, which is equivalent to
    $\Gamma \vdash \Pi x:\phi_1.\phi_2 \approx [t_2/y]\phi'$.
  \end{itemize}
  
\item[Case.] Part ii.\\
  \begin{itemize}
  \item[Case.] \ \\
    \begin{center}
      \begin{math}
        $$\mprset{flushleft}
        \inferrule* [right=] {
          \Gamma \vdash t_1 : \Pi x:\phi_1.\phi_2 
          \\
          \Gamma \vdash t_2 : \phi_1
        }{\Gamma \vdash t_1\ t_2 : [t_2/x]\phi_2}
      \end{math}  
    \end{center}
    Trivial.
    
  \item[Case.] \ \\
    \begin{center}
      \begin{math}
        $$\mprset{flushleft}
        \inferrule* [right=] {
          \Gamma \vdash p:t_1 = t_2
          \\
          \Gamma \vdash t_1\ t_2:[t_1/y]\phi
        }{\Gamma \vdash t_1\ t_2:[t_2/y]\phi}
      \end{math}
    \end{center}
  \end{itemize}
  Similar to the previous case.

\item[Case.] Part iii.\\
  \begin{itemize}
  \item[Case.] \ \\
    \begin{center}
      \begin{math}
        $$\mprset{flushleft}
        \inferrule* [right=] {
          \Gamma, X : *_l \vdash t : \phi
        }{\Gamma \vdash \Lambda X:*_l.t:\forall X : *_l.\phi}
      \end{math}
    \end{center}
    Trivial.
    
  \item[Case.] \ \\
    \begin{center}
      \begin{math}
        $$\mprset{flushleft}
        \inferrule* [right=] {
          \Gamma \vdash p:t_1 = t_2
          \\
          \Gamma \vdash \Lambda X:*_l.t:[t_1/y]\phi''
        }{\Gamma \vdash \Lambda X:*_l.t:[t_2/y]\phi''}
      \end{math}
    \end{center}
    Similar to the previous case.
  \end{itemize}
  
\item[Case.] Part iv.\\
  \begin{itemize}
  \item[Case.] \ \\
    \begin{center}
      \begin{math}
        $$\mprset{flushleft}
        \inferrule* [right=] {
          \Gamma \vdash t:\forall X:*_l.\phi_1
          \\
          \Gamma \vdash \phi_2:*_l
        }{\Gamma \vdash t[\phi_2]: [\phi_2/X]\phi_1}
      \end{math} 
    \end{center}
    Trivial.

  \item[Case.] \ \\
    \begin{center}
      \begin{math}
        $$\mprset{flushleft}
        \inferrule* [right=] {
          \Gamma \vdash p:t_1 = t_2
          \\
          \Gamma \vdash t[\phi_2]:[t_1/y]\phi'
        }{\Gamma \vdash t[\phi_2]:[t_2/y]\phi'}
      \end{math}
    \end{center}
  \end{itemize}
  Similar to the previous case.

\item[Case.] Part v.\\
  \begin{itemize}
  \item[Case.] \ \\
    \begin{center}
      \begin{math}
        $$\mprset{flushleft}
        \inferrule* [right=] {
          t_1 \join t_2
	  \\
          \Gamma \vdash t_1:\phi
          \\
          \Gamma \vdash t_2:\phi
          \\
	  \Gamma\ Ok
        }{\Gamma \vdash join : t_1 = t_2}
      \end{math}
    \end{center}
    Trivial.
  
  \item[Case.] \ \\
    \begin{center}
      \begin{math}
        $$\mprset{flushleft}
        \inferrule* [right=] {
          \Gamma \vdash p:t'_1 = t'_2
          \\
          \Gamma \vdash join:[t'_1/y]\phi')
        }{\Gamma \vdash join:[t'_2/y]\phi'}
      \end{math}
    \end{center}
    Similar to the previous case.
  \end{itemize}
\end{itemize}
\end{proof}
% subsection basic_syntactic_lemmas (end)

\subsection{The Ordering on Types and The Hereditary Substitution Function}
\label{sec:the_hereditary_substitution_function_ssfe}
The ordering on types is defined as follows:
\begin{definition}
  The ordering $>_\Gamma$ is defined as the least relation satisfying the universal closure
  of the following formulas:
  \begin{center}
    \begin{tabular}{lll}
      $\Pi x:\phi_1.\phi_2$ & $>_\Gamma$ & $\phi_1$\\
      $\Pi x:\phi_1.\phi_2$ & $>_{\Gamma}$ & $[t/x]\phi_2$, where $\Gamma \vdash t:\phi_1$.\\
      $\forall X:*_l.\phi$  & $>_\Gamma$ & $[\phi'/X]\phi$, where $\Gamma \vdash \phi':*_l$.\\
    \end{tabular}
  \end{center}
  \label{def:ordering_ssfe}
\end{definition}
Just as we have seen before this is the ordering used in the proofs of
the properties of the hereditary substitution function, which state
next.  As one might have expected this is a well-founded ordering.
\begin{thm}[Well-Founded Ordering]
  The ordering $>_\Gamma$ is well-founded on types $\phi$ such that 
  $\Gamma \vdash \phi:*_l$ for some $l$.
  \label{thm:well-founded_ordering_ssfe}
\end{thm}
\begin{proof}
  This proof is similar to the same proof for SS$\Fp$.  It depends on
  the following function, and intermediate result.
  \begin{definition}
  The depth of a type $\phi$ is defined as follows:
  \begin{center}
    \begin{tabular}{lll}
      $depth(t)$                  & $=$ & $0$, where $t$ is any term.\\
      $depth(X)$                  & $=$ & $1$\\
      $depth(\Pi x:\phi.\phi')$   & $=$ & $depth(\phi) + depth(\phi')$\\
      $depth(\forall X:*_l.\phi)$ & $=$ & $depth(\phi) + 1$\\
    \end{tabular}
  \end{center}
\end{definition}
\noindent
We use the metric $(l,d)$ in lexicographic combination, where $l$ is the level of
a type $\phi$, and $d$ is the depth of $\phi$ in the proof of the next lemma.  

\begin{lemma}[Well-Founded Measure]
  \label{lemma:well-founded_measure_ssfe}
  If $\phi >_\Gamma \phi'$ then $(l,d) > (l',d')$, where $\Gamma \vdash \phi:*_l$, 
  $depth(\phi) = d$,  $\Gamma \vdash \phi:*_{l'}$, and $depth(\phi') = d'$.
\end{lemma}
\begin{proof}
  Similar to the proof of the same lemma for SS$\Fp$. 
\end{proof}
\end{proof}
\noindent
The type-syntactic-equality relation respects this ordering.
\begin{lemma}
  \label{lemma:typeq_ordering_ssfe}
  If $\Gamma \vdash \phi' \approx \phi''$ and $\phi >_{\Gamma} \phi'$ then $\phi >_\Gamma \phi''$.
\end{lemma}
\begin{proof}
  This is a proof by case analysis on the kinding derivation of 
$\Gamma \vdash \phi:*_p$, with a case analysis on the derivation of 
$\phi >_\Gamma \phi'$.\\

\begin{itemize}
\item[Case.]\ \\
  \begin{center}
    \begin{math}
      $$\mprset{flushleft}
      \inferrule* [right=] {
        \Gamma(X) = *_p
	\\
	p \leq q
	\\
	\Gamma\ Ok
      }{\Gamma \vdash X : *_q}
    \end{math}
  \end{center}
  This case cannot arise, because we do not have $X >_\Gamma \phi$ for any type 
  $\phi$.\\
  
\item[Case.]\ \\
  \begin{center}
    \begin{math}
      $$\mprset{flushleft}
      \inferrule* [right=] {
        \Gamma \vdash \phi_1 : *_p
	\\
	\Gamma \vdash \phi_2 : *_q
      }{\Gamma \vdash \phi_1 \rightarrow \phi_2 : *_{max(p,q)}}
    \end{math}
  \end{center}
  By analysis of the derivation of the assumed ordering statement, we must have 
  $\phi' \equiv \phi_1$ or $\phi' \equiv \phi_2$.  If $\phi' \equiv \phi_1$ and 
  $p \geq q$ then we have the required kind derivation for $\phi'$. If $p < q$ then by 
  level weakening $\Gamma \vdash \phi_1:*_q$, and we have the required kinding 
  derivation for $\phi'$.  The case for when $\phi' \equiv \phi_2$ is similar.\\
  
\item[Case.]\ \\
  \begin{center}
    \begin{math}
      $$\mprset{flushleft}
      \inferrule* [right=] {
        \Gamma \vdash \phi_1 : *_p
	\\
	\Gamma \vdash \phi_2 : *_q
      }{\Gamma \vdash \phi_1 + \phi_2 : *_{max(p,q)}}
    \end{math}
  \end{center}
  By analysis of the derivation of the assumed ordering statement, we must have 
  $\phi' \equiv \phi_1$ or $\phi' \equiv \phi_2$.  If $\phi' \equiv \phi_1$ and 
  $p \geq q$ then we have the required kind derivation for $\phi'$. If $p < q$ then by 
  level weakening $\Gamma \vdash \phi_1:*_q$, and we have the required kinding 
  derivation for $\phi'$.  The case for when $\phi' \equiv \phi_2$ is similar.\\
  
\item[Case.]\ \\
  \begin{center}
    \begin{math}
      $$\mprset{flushleft}
      \inferrule* [right=] {
        \Gamma,X : *_r \vdash \phi : *_s
      }{\Gamma \vdash \forall X:*_r.\phi : *_{max(r,s)+1}}
    \end{math}
  \end{center}
  By analysis of the derivation of the assumed ordering statement, we must have 
  $\phi' \equiv [\phi''/X]\phi$, for some type $\phi''$ with 
  $\Gamma \vdash \phi'':*_r$.  Let $t = max(r,s) + 1$. Clearly, 
  $s < t$, hence by level weakening $\Gamma,X:*_r \vdash \phi:*_t$ and by substitution 
  for kinding $\Gamma \vdash [\phi''/X]\phi:*_t$, and we have the required kinding 
  derivation for $\phi'$.  
\end{itemize}
\end{proof}
The following lemma is used in the proof of totality of the hereditary substitution function.
\begin{lemma}[Congruence of Products]
  \label{lemma:A_prop_ssfe}
  If $\Gamma \vdash \psi \approx \Pi y:\phi_1.\phi_2$ and $\psi$ is a subexpression of $\phi''$ then
  $\phi'' >_\Gamma \phi_1$ and $\phi'' >_{\Gamma,y:\phi_1} \phi_2$.
\end{lemma}
\begin{proof}
  The possible form for $\psi$ is only a $\Pi$-type.  So it suffices to show that
  if $\Gamma \vdash \Pi y:\phi'_1.\phi'_2 \approx \Pi y:\phi_1.\phi_2$ and $\psi$ is
  a subexpression of $\phi''$ then $\phi'' >_\Gamma \phi_1$ and $\phi'' >_{\Gamma,y:\phi_1} \phi_2$.

  \ \\
  It must be the case that $\phi'' >_\Gamma \phi'_1$ and $\phi'' >_{\Gamma,y:\phi_1} \phi'_2$, because
  $\phi'_1$ and $\phi'_2$ are both subexpressions of $\phi''$.  By injectivity of $\Pi$-types
  for typed equality we obtain $\Gamma \vdash \phi'_1 \approx \phi_1$ and $\Gamma,y:\phi_1 \vdash \phi'_2 \approx \phi_2$.
  Finally, by Lemma~\ref{lemma:typeq_ordering_ssfe} we know $\phi'' >_\Gamma \phi_1$ and $\phi'' >_{\Gamma,y:\phi_1} \phi_2$.
\end{proof}

The hereditary substitution function is fully defined in
Fig.~\ref{fig:hereditary_substitution_function_ssfe}.  We do not
repeat the definition of $ctype_\phi$ for DSS$\F$, because it is
exactly the same as the previous system. 
\begin{figure}[t]
  \small
  \begin{itemize}
  \item[] $[t/x]^\phi x = t$

  \item[] $[t/x]^\phi y = y$\\
    \begin{tabular}{lll}
      & Where $y$ is a variable distinct from $x$.\\
    \end{tabular}

  \item[] $[t/x]^\phi join = join$

  \item[] $[t/x]^\phi (\lambda y:\phi'.t') = \lambda y:\phi'.([t/x]^\phi t')$

  \item[] $[t/x]^\phi (\Lambda X:*_l.t') = \Lambda X:*_l.([t/x]^\phi t')$

  \item[] $[t/x]^\phi (t_1\ t_2) = ([t/x]^\phi t_1)\ ([t/x]^\phi t_2)$\\
    \begin{tabular}{lll}
      & Where $([t/x]^\phi t_1)$ is not a $\lambda$-abstraction, 
      $([t/x]^\phi t_1)$ and $t_1$ are $\lambda$-abstractions,\\ 
      & or $ctype_\phi(x,t_1)$ is undefined.
    \end{tabular}

  \item[] $[t/x]^{\phi} (t_1\ t_2) = [([t/x]^{\phi} t_2)/y]^{\phi''} s'_1$\\
    \begin{tabular}{lll}
      & Where $([t/x]^{\phi} t_1) \equiv \lambda y:\phi''.s'_1$ 
        for some $y$ and $s'_1$ and $t_1$ is not a $\lambda$-abstraction, and \\
      & $ctype_\phi(x,t_1) = \Pi y:\phi''.\phi'$.\\
    \end{tabular}

  \item[] $[t/x]^\phi (t'[\phi']) = ([t/x]^\phi t')[\phi']$\\
    \begin{tabular}{lll}
      & Where $[t/x]^\phi t'$ is not a type abstraction or
      $t'$ and $[t/x]^\phi t'$ are type abstractions.
    \end{tabular}

    \item[] $[t/x]^{\phi} (t'[\phi']) = [\phi'/X]s'_1$\\
      \begin{tabular}{lll}
        & Where $[t/x]^{\phi} t' \equiv \Lambda X:*_l.s'_1$,
        for some $X$, $s'_1$ and $\Gamma \vdash \phi':*_q$, such that, $q \leq l$ and\\
	& $t'$ is not a type abstraction.
      \end{tabular}
  \end{itemize}
  \caption{Hereditary Substitution Function for Stratified System $\F$}
  \label{fig:hereditary_substitution_function_ssfe}
\end{figure}
Next we have the properties of the hereditary substitution function.
All of the proofs are similar to the proofs for SS$\Fp$, so we omit
them here.
\begin{lemma}[Total, Type Preserving, and Sound with Respect to Reduction]
  \label{lemma:total_ssfe}
  Suppose $\Gamma \vdash t : \phi$ and $\Gamma, x:\phi, \Gamma' \vdash t':\phi'$.  Then
  \begin{itemize}
  \item[i.] there exists a term $t''$ and a type $\phi''$ such that $[t/x]^\phi t' = t''$, 
    $\Gamma,[t/x]\Gamma' \vdash t'':\phi''$, and $\Gamma,\Gamma' \vdash \phi'' \approx [t/x]\phi'$, and
  \item[ii.] $[t/x]t' \redto^* [t/x]^\phi t'$.
  \end{itemize}
\end{lemma}

\begin{corollary}
  \label{corollary:type_preserving_ssfe}
  Suppose $\Gamma \vdash t : \phi$ and $\Gamma, x:\phi, \Gamma' \vdash t':\phi'$. Then
  $\Gamma,[t/x]\Gamma' \vdash [t/x]^\phi t':[t/x]\phi'$.
\end{corollary}

\begin{lemma}[Redex Preserving]
  \label{lemma:redex_preserving_ssfe}
  If $\Gamma \vdash t : \phi$, $\Gamma, x:\phi, \Gamma' \vdash t':\phi'$
  then  $|rset(t', t)| \geq |rset([t/x]^\phi t')|$.
\end{lemma}
\begin{proof}
  Just as we have seen for the previous systems this proof depends on
  the following function:
  \begin{center}
    \begin{itemize}
    \item[] $rset(x) = \emptyset$\\
    \item[] $rset(join) = \emptyset$
    \item[] $rset(\lambda x:\phi.t) = rset(t)$\\
    \item[] $rset(\Lambda X:*_l.t) = rset(t)$\\
    \item[] $rset(t_1\ t_2)$\\
      \begin{math}
        \begin{array}{lll}
          = & rset(t_1, t_2) & \text{if } t_1 \text{ is not a } \lambda \text{-abstraction.}\\
          = & \{t_1\ t_2\} \cup rset(t'_1, t_2)\ & \text{if } t_1 \equiv \lambda x:\phi.t'_1.\\
        \end{array}
      \end{math}
    \item[] $rset(t''[\phi''])$\\
      \begin{math}
        \begin{array}{lll}
          = & rset(t'') & \text{if } t'' \text{ is not a type absraction.}\\
          = & \{t''[\phi'']\} \cup rset(t''') & \text{if } t'' \equiv \Lambda X:*_l.t'''.
        \end{array}
      \end{math}
    \end{itemize}
  \end{center}  
\end{proof}

\begin{lemma}[Normality Preserving]
  \label{corollary:normalization_preserving_ssfe}
  If $\Gamma \vdash n:\phi$ and $\Gamma, x:\phi' \vdash n':\phi'$ then there exists a normal term $n''$ 
  such that $[n/x]^\phi n' = n''$.
\end{lemma}
% subsection the_hereditary_substitution_function (end)

\subsection{Concluding Normalization}
\label{subsec:concluding_normalization_ssfe}
We are now ready to conlcude normalization.  The interpretation of
types are defined exactly the same way as they were for STLC
(Definition~\ref{def:interpretation_of_types_stlc}).  We do not repeat
that definition here.  The next two lemmas are used in the proof of
the type soundness theorem.

\begin{corollary}[Type Substitution for the Interpretation of Types]
  If $n \in \interp{\phi'}_{\Gamma,X:*_l,\Gamma'}$ and 
  $\Gamma \vdash \phi:*_l$ then 
  $[\phi/X]n \in \interp{[\phi/X]\phi'}_{\Gamma,[\phi/X]\Gamma'}$.
  \label{lemma:type_sub_ssfe}
\end{corollary}
\begin{proof}
  By the definition of the interpretation of types, $\Gamma,X:*_l,\Gamma' \vdash n:\phi'$ and
  by Lemma~\ref{lemma:substitution_for_kinding_ssfe}, 
  $\Gamma,[\phi/X]\Gamma' \vdash [\phi/X]n:[\phi/X]\phi'$.  Finally,
  by the definition of the interpretation of types, 
  $[\phi/X]n \in \interp{[\phi/X]\phi'}_{\Gamma,[\phi/X]\Gamma'}$.
\end{proof}
\begin{lemma}[Semantic Equality]
  \label{lemma:equality_of_interpretation_of_types_ssfe}
  If $\Gamma \vdash p:t_1 = t_2$ then $\interp{[t_1/x]\phi}_\Gamma = \interp{[t_2/x]\phi}_\Gamma$.
\end{lemma}
\begin{proof}
  We first prove the left to right containment.  Suppose $t \redto^* n
  \in \interp{[t_1/x]\phi}_\Gamma$. Then by the definition of the
  interpretation of types, $\Gamma \vdash n:[t_1/x]\phi$.  By assumption
  we know $\Gamma \vdash p:t_1 = t_2$, hence, by applying the conversion
  type-checking rule $\Gamma \vdash n:[t_2/x]\phi$.  Finally, by the
  definition of the interpretation of types, $n \in
  \interp{[t_2/x]\phi}_\Gamma$.  Therefore, $t \in
  \interp{[t_2/x]\phi}_\Gamma$.  The opposite direction is similar.
\end{proof}
\noindent
We now conclude type soundness for SS$\F$, and hence normalization.  

\begin{lemma}[Hereditary Substitution for the Interpretation of Types]
  If $n' \in \interp{\phi'}_{\Gamma,x:\phi,\Gamma'}$, $n \in \interp{\phi}_\Gamma$,
  then $[n/x]^\phi n' \in \interp{[n/x]\phi'}_{\Gamma,[n/x]\Gamma'}$.
  \label{lemma:interpretation_of_types_closed_substitution_ssfe}
\end{lemma}
\begin{proof}
  This proof is similar to the same proof for SS$\Fp$.
\end{proof}
\begin{thm}[Type Soundness]
  If $\Gamma \vdash t:\phi$ then $t \in \interp{\phi}_\Gamma$.
  \label{thm:soundness_ssfe}
\end{thm}
\begin{proof}
  This is a proof by induction on the structure of the typing derivation of $t$.
\begin{itemize}
\item[Case.]\ \\
  \begin{center}
    \begin{math}
      $$\mprset{flushleft}
      \inferrule* [right=] {
        \Gamma(x) = \phi 
        \\
        \Gamma\ Ok
      }{\Gamma \vdash x:\phi}
    \end{math}
  \end{center}
  By regularity $\Gamma \vdash \phi:*_l$ for some $l$, hence $\interp{\phi}_\Gamma$ is 
  nonempty.  Clearly, $x \in \interp{\phi}_\Gamma$ by the definition of the interpretation
  of types.
  
\item[Case.]\ \\
  \begin{center}
    \begin{math}
      $$\mprset{flushleft}
      \inferrule* [right=] {
        \Gamma,x:\phi_1 \vdash t:\phi_2
      }{\Gamma \vdash \lambda x:\phi_1.t : \Pi x:\phi_1.\phi_2}
    \end{math}
  \end{center}
  By the induction hypothesis and the definition of the interpretation of types 
  $t \in \interp{\phi_2}_{\Gamma,x:\phi_1}$, $t \normto n \in $
  $ \interp{\phi_2}_{\Gamma,x:\phi_1}$ and $\Gamma,x:\phi_1 \vdash n:\phi_2$.  Thus, by
  applying the $\lambda$-abstraction type-checking rule, 
  $\Gamma \vdash \lambda x:\phi_1.n:\Pi x:\phi_1.\phi_2$, hence by the definition of the 
  interpretation of types 
  $\lambda x:\phi_1.n \in \interp{\Pi x:\phi_1.\phi_2}_\Gamma$.  Since  
  $\lambda x:\phi_1.t \normto \lambda x:\phi_1.n \in $
  $\interp{\Pi x:\phi_1.\phi_2}_\Gamma$ we know by the definition of the interpretation of types
  $\lambda x:\phi_1.t \in \interp{\Pi x:\phi_1.\phi_2}_\Gamma$.

\item[Case.]\ \\
  \begin{center}
    \begin{math}
      $$\mprset{flushleft}
      \inferrule* [right=] {
        \Gamma \vdash t_1 : \Pi x:\phi_1.\phi_2 
        \\
        \Gamma \vdash t_2 : \phi_1 
      }{\Gamma \vdash t_1\ t_2 : [t_2/x]\phi_2}
    \end{math}
  \end{center}
  It suffices to show that there exists a normal term $n$ such that
  $t_1\ t_2 \normto n \in \interp{[t_2/x]\phi_2}_\Gamma$.
  By the induction hypothesis and the definition of the interpretation of
  types $t_1 \normto n_1 \in $ $\interp{\Pi x:\phi_1.\phi_2}_\Gamma$,  
  $\Gamma \vdash n_1:\Pi x:\phi_1.\phi_2$, $t_2 \normto n_2 \in \interp{\phi_1}_\Gamma$, 
  and $\Gamma \vdash n_2:\phi_1$.  Clearly,
  \begin{center}
    \begin{math}
      \begin{array}{lll}
        t_1\ t_2 & \redto^* & n_1\ n_2 \\
                 & =        & [n_1/z](z\ n_2),
      \end{array}
    \end{math}
  \end{center}
  for some fresh variable $z \not \in FV(n_1,n_2,\phi_1,\phi_2,x)$.  By Lemma~\ref{lemma:total_ssfe}, Lemma~\ref{corollary:normalization_preserving_ssfe},
  and Corollary~\ref{corollary:type_preserving_ssfe} $[n_1/x](z\ n_2) \redto^* [n_1/x]^{\phi_1} (z\ n_2)$, $[n_1/x]^{\phi_1} (z\ n_2)$ is normal, and
  $\Gamma \vdash [n_1/x]^{\phi_1} (z\ n_2):[n_2/x]\phi_2$.  Thus, $t_1\ t_2 \normto [n_1/x]^{\phi_1} (z\ n_2)$.  It suffices to show that 
  $\Gamma \vdash [n_1/x]^{\phi_1} (z\ n_2):[t_2/x]\phi_2$.  This is justified by the following typing derivation:
  \begin{center}
    \begin{math}
      $$\mprset{flushleft}
      \inferrule* [right=Conv] {
        $$\mprset{flushleft}
        \inferrule* [right=Join] {
          \Gamma \vdash t_2:\phi_1
          \\\\
          \Gamma \vdash n_2:\phi_1
          \\
          n_2 \join t_2
        }{\Gamma \vdash join:n_2 = t_2}
        \\
        \Gamma \vdash [n_1/x]^{\phi_1} (z\ n_2):[n_2/x]\phi_2
      }{\Gamma \vdash [n_1/x]^{\phi_1} (z\ n_2):[t_2/x]\phi_2}
    \end{math}
  \end{center}
  Therefore, $[n_1/x]^{\phi_1} (z\ n_2) \in \interp{[t_2/x]\phi_2}_\Gamma$ which implies that
  $t_1\ t_2 \in \interp{[t_2/x]\phi_2}_\Gamma$.
  
\item[Case.]\ \\
  \begin{center}
    \begin{math}
      $$\mprset{flushleft}
      \inferrule* [right=] {
        t_1 \downarrow t_2
	\\
	\Gamma\ Ok
      }{\Gamma \vdash join : t_1 = t_2}
    \end{math}
  \end{center}
  Clearly, $join \in \interp{t_1 = t_2}_\Gamma$ by the definition of the 
  interpretation of types.
  
\item[Case.]\ \\
  \begin{center}
    \begin{math}
      $$\mprset{flushleft}
      \inferrule* [right=] {
        \Gamma \vdash t_0:t_1 = t_2
        \\
        \Gamma \vdash t:[t_1/x]\phi
      }{\Gamma \vdash t:[t_2/x]\phi}
    \end{math}
  \end{center}
  By the induction hypothesis, $t \in \interp{[t_1/x]\phi}_\Gamma$.
  By the definition of the interpretation of types, $t \normto n \in
  \interp{[t_1/x]\phi}_\Gamma$.  By assumption we know, $\Gamma \vdash
  t_0:t_1 = t_2$. Thus, by
  Lemma~\ref{lemma:equality_of_interpretation_of_types_ssfe},
  $\interp{[t_1/x]\phi}_\Gamma = \interp{[t_2/x]\phi}_\Gamma$.
  Therefore, $n \in \interp{[t_2/x]\phi}_\Gamma$, hence, by the
  definition of the interpretation of types, $t \in
  \interp{[t_2/x]\phi}_\Gamma$.
  
\item[Case.]\ \\
  \begin{center}
    \begin{math}
      $$\mprset{flushleft}
      \inferrule* [right=] {
        \Gamma, X : *_p \vdash t : \phi
      }{\Gamma \vdash \Lambda X:*_p.t:\forall X:*_p.\phi}
    \end{math}
  \end{center}
  By the induction hypothesis, $t \in \interp{\phi}_{\Gamma,X:*_p}$, so
  by the definition of the interpretation of types, $t \normto n \in
  \interp{\phi}_{\Gamma,X:*_p}$ and $\Gamma,X:*_p \vdash n:\phi$.
  We can apply the $\Lambda$-abstraction type-checking rule to
  obtain $\Gamma \vdash \Lambda X:*_p.n:\forall X:*_p.\phi$, thus
  $\Lambda X:*_p.n \in \interp{\forall X:*_p.\phi}_{\Gamma}$.  Since
  $\Lambda X:*_p.t \normto \Lambda X:*_p.n$ by definition of the
  interpretation of types 
  $\Lambda X:*_p.t \in \interp{\forall X:*_p.\phi}_{\Gamma}$.

\item[Case.]\ \\
  \begin{center}
    \begin{math}
      $$\mprset{flushleft}
      \inferrule* [right=] {
        \Gamma \vdash t:\forall X:*_l.\phi_1
        \\
        \Gamma \vdash \phi_2:*_l
      }{\Gamma \vdash t[\phi_2]: [\phi_2/X]\phi_1}
    \end{math}
  \end{center}
  By the induction hypothesis $t \in \interp{\forall X:*_l.\phi_1}_\Gamma$, so by the 
  definition of the interpretation of types 
  $t \normto n \in \interp{\forall X:*_l.\phi_1}_\Gamma$ and
  $\Gamma \vdash n:\forall X:*_l.\phi_1$.  We do a case 
  split on whether or not $n$ is a $\Lambda$-abstraction. We can apply the
  type-instantiation type-checking rule to obtain 
  $\Gamma \vdash n[\phi_2]:[\phi_2/X]\phi_1$ and by the 
  definition of the interpretation of types 
  $n[\phi_2] \in \interp{[\phi_2/X]\phi_1}_\Gamma$. Therefore, 
  $t \in \interp{[\phi_2/X]\phi_1}_\Gamma$.  Suppose $n \equiv \Lambda X:*_l.n'$.  
  Then 
  $t[\phi_2] \rightsquigarrow^{*} (\Lambda X:*_l.n')[\phi_2] \rightsquigarrow $
  $ [\phi_2/X]n'$.  By inversion
  $n' \in \interp{\phi_1}_{\Gamma,X:*_l}$. Therefore, by
  Lemma~\ref{lemma:type_sub_ssfe} $[\phi_2/X]n' \in \interp{[\phi_2/X]\phi_1}_{\Gamma}$ and
  $t[\phi_2] \in \interp{[\phi_2/X]\phi_1}_{\Gamma}$, since $t[\phi_2] \join [\phi_2/X]n'$.
\end{itemize}
\end{proof}
\begin{corollary}[Normalization]
  If $\Gamma \vdash t:\phi$ then $t \normto n$.
\end{corollary}
% subsection concluding_normalization_ssfp (end)
% section dependent_ssf_with_type_equality_(dss$\f$) (end)

%%% Local Variables: 
%%% mode: latex
%%% reftex-default-bibliography: ("/Users/hde/thesis/paper/thesis.bib")
%%% TeX-master: "/Users/hde/thesis/paper/thesis.tex"
%%% End: 