\section{Dualized Intuitionistic Logic}
\label{sec:dualized_intuitionistic_logic}

\subsection{Consistency}
\label{subsec:consistency_dil}
In this section we prove consistency of DIL with respect to Rauszer's
Kripke semantics for BINT logic.  All of the results in this section
have been formalized in the Agda proof assistant\footnote{Agda source
  code is available at
  \url{https://github.com/heades/DIL-consistency}}.  We begin by first
defining a Kripke frame.

\begin{definition}
  \label{def:kripke_frame}
  A \textbf{Kripke frame} is a pair $(W, R)$ of a set of worlds $W$, and
  a preorder $R$ on $W$.  
\end{definition}
Then we extend the notion of a Kripke frame to include an evaluation for atomic
formulas resulting in a Kripke model.
\begin{definition}
  \label{def:kripke_model}
  A \textbf{Kripke model} is a tuple $(W, R, V)$, such that, $(W, R)$ is
  a Kripke frame, and $V$ is a binary monotone relation on $W$ and the
  set of atomic formulas of DIL.
\end{definition}
Now we can interpret formulas in a Kripke model as follows:
\begin{definition}
  \label{def:interpretation}
  The interpretation of the formulas of DIL in some Kripke model $(W, R, V)$
  is defined by recursion on the structure of the formula as follows:
  \begin{center}
    \begin{tabular}{lll}
      \begin{math}
        \begin{array}{rll}
          [[ [| < + > |] w]]   & = & [[True]]\\
          [[ [| < - > |] w]]   & = & [[False]]\\
          [[ [|a|] w]]         & = & [[V w a]]\\  
        \end{array}
      \end{math}
      &
      \begin{math}
        \begin{array}{rll}      
          [[ [| A /\+ B |] w]] & = & [[ [| A |] w]]  \land [[ [| B |] w]]\\    
          [[ [| A /\- B |] w]] & = & [[ [| A |] w]]  \lor [[ [| B |] w]]\\    
          [[ [| A ->+ B |] w]] & = & \forall w' \in W. [[R w w']] \to [[ [| A |] w']] \to [[ [| B |] w']]\\
          [[ [| A ->- B |] w]] & = & \exists w' \in W. [[R w' w]] \land \lnot [[ [| A |] w']] \land [[ [| B |] w']]\\
        \end{array}
      \end{math}
    \end{tabular}
  \end{center}
\end{definition}
The interpretation of formulas really highlights the fact that implication is dual to coimplication.  Monotonicity
holds for this interpretation.

\begin{lemma}[Monotonicity]
  \label{lemma:monotonicity}
  Suppose $(W, R, V)$ is a Kripke model, $[[A]]$ is some DIL formula, and $w,w' \in W$.
  Then $[[R w w']]$ and $[[ [| A |] w]]$ imply $[[ [| A |] w']]$.
\end{lemma}

At this point we must set up the mathematical machinery which allows
for the interpretation of sequents in a Kripke model.  This will
require the interpretation of graphs, and hence, nodes.  We interpret
nodes as worlds in the model using a function we call a node
interpreter.
\begin{definition}
  \label{def:node_interpreter}
  Suppose $(W, R, V)$ is a Kripke model and $S$ is a set of nodes of
  an abstract Kripke model $[[Gr]]$.  Then a \textbf{node interpreter} on
  $S$ is a function from $S$ to $W$.
\end{definition}
Now using the node interpreter we can interpret edges as statements
about the reachability relation in the model.  Thus, the
interpretation of a graph is just the conjunction of the
interpretation of its edges.
\begin{definition}
  \label{def:graph_interp}
  Suppose $(W, R, V)$ is a Kripke model, $[[Gr]]$ is an abstract Kripke
  model, and $[[N]]$ is a node interpreter on the set of nodes of $[[Gr]]$.
  Then the interpretation of
  $[[Gr]]$ in the Kripke model is defined by recursion on the structure
  of the graph as follows:
  \[
  \begin{array}{lll}
    [[ [| . |] N ]]            & = & [[True]]\\
    [[ [| n1 <=+ n2, Gr |] N]] & = & [[R (N n1) (N n2)]] \land [[ [| Gr |] N]]\\
    [[ [| n1 <=- n2, Gr |] N]] & = & [[R (N n2) (N n1)]] \land [[ [| Gr |] N]]\\
  \end{array}
  \]
\end{definition}
Now we can prove that if a particular reachability judgment holds,
then the interpretation of the nodes are reachable in the model.
\begin{lemma}[Reachability Interpretation]
  \label{lemma:reachability_interpretation}
  Suppose $(W, R, V)$ is a Kripke model, and $[[ [| Gr |] N]]$ for some abstract Kripke graph $[[Gr]]$. Then
  \begin{itemize}
  \item[i.] if $[[Gr |- n1 <=+ n2]]$, then $[[R (N n1) (N n2)]]$, and
  \item[ii.] if $[[Gr |- n1 <=- n2]]$, then $[[R (N n2) (N n1)]]$.
  \end{itemize}
\end{lemma}

We now have everything we need to interpret abstract Kripke
models. The final ingredient to the interpretation of sequents is the
interpretation of contexts.  
\begin{definition}
  \label{def:pol_interp}
  If $F$ is some meta-logical formula, we define $p\, F$ as follows:
  \[
  \begin{array}{lll}
    [[+]]\, F = F & \text{ and } &
    [[-]]\, F = \lnot F.\\
  \end{array}
  \]
\end{definition}
\begin{definition}
  \label{def:ctx_interp}
  Suppose $(W, R, V)$ is a Kripke model, $[[G]]$ is a context, 
  and $[[N]]$ is a node interpreter on the set of nodes in $[[G]]$.
  The interpretation of
  $[[G]]$ in the Kripke model is defined by recursion on the structure
  of the context as follows:
  \[
  \begin{array}{lll}
    [[ [| . |] N ]]        & = & [[True]]\\
    [[ [| p A @ n, G |] N]] & = & [[ p [| A |] (N n) ]] \land [[ [| G |] N]]\\
  \end{array}
  \]
\end{definition}
Combining these interpretations results in the following
definition of validity.
\begin{definition}
  \label{def:validity}
  Suppose $(W, R, V)$ is a Kripke model, $[[G]]$ is a context, 
  and $[[N]]$ is a node interpreter on the set of nodes in $[[G]]$.
  The interpretation of sequents is defined as follows:
  \[
  \begin{array}{lll}
    [[ [|Gr ; G |- p A @ n |] N ]] & = & \text{ if } [[ [| Gr |] N ]] \text{ and } [[ [| G |] N]], \text{ then } [[ p [| A |] (N n)]].
  \end{array}
  \]
\end{definition}
Notice that in the definition of validity the graph $[[Gr]]$ is
interpreted as a set of constraints imposed on the set of Kripke models, thus
reinforcing the fact that the graphs on sequents really are abstract
Kripke models.  Finally, using the previous definition of validity we
can prove consistency.
\begin{thm}[Consistency]
  \label{thm:consistency}
  Suppose $[[Gr ; G |-p A @ n]]$. 
  Then for any Kripke model $(W, R, V)$ and node interpreter $N$ on $|[[Gr]]|$, $[[ [|Gr ; G |- p A @ n |] N]]$.
\end{thm}
% subsection consistency (end)

\subsection{Completeness}
\label{subsec:completeness_dil}

% subsection completeness (end)

% section dualized_intuitionistic_logic (end)

\section{Dualized Type Theory}
\label{sec:dualized_type_theory}
We now present the basic metatheory of DTT, starting with
type preservation (proof omitted).  
\begin{lemma}[Type Preservation]
  \label{lemma:type_preservation}
  If $[[Gr ; H |- t : p A@n]]$, and $[[t]] \redto^* [[t']]$, then $[[Gr ; H |- t' : p A@n]]$.
\end{lemma}
\begin{proof}
  This holds by straightforward induction on the form of the assumed
  derivation $[[t]] \redto^* [[t']]$.
\end{proof}
\begin{figure}
    \begin{mathpar}
      \dttdruleClassAx{}     \and
      \dttdruleClassUnit{}   \and
      \dttdruleClassAnd{}    \and
      \dttdruleClassAndBar{} \and
      \dttdruleClassImp{}    \and 
      \dttdruleClassImpBar{} \and
      \dttdruleClassCut{}    
    \end{mathpar}
\caption{Classical typing of DTT terms}
\label{fig:classtp}
\end{figure}
A more substantial property is strong normalization of reduction for
typed terms.  To prove this result, we will prove a stronger property,
namely strong normalization for reduction of terms which are typable
using the system of classical typing rules in
Figure~\ref{fig:classtp}.  This is justified by the following easy
result (proof omitted), where $[[drop H]]$ just drops the world
annotations from assumptions in $\Gamma$:
\begin{thm}
\label{thm:inttoclass}
If $[[Gr ; H |- t : p A @ n]]$, then $[[ drop H |-c t : p A]]$
\end{thm}

Let $\SN$ be the set of terms which are strongly normalizing with
respect to the reduction relation.  Let \textit{Var} be the set of
term variables, and let us use $x$ and $y$ as metavariables for variables.  We
will prove strong normalization for classically typed terms using a
version of Krivine's classical realizability~\cite{krivine09}.  We
define three interpretations of types in Figure~\ref{fig:classreal}.
The definition is by mutual induction, and can easily be seen to
well-founded, as the definition of $\interp{A}^+$ invokes the
definition of $\interp{A}^-$ with the same type, which in turn invokes
the definition of $\interp{A}^{+c}$ with the same type; and the
definition of $\interp{A}^{+c}$ may invoke either of the other
definitions at a strictly smaller type.  The reader familiar with
such proofs will also recognize the debt owed to Girard~\cite{gtl90}.
All missing proofs below may be found in the companion report~\cite{Eades:2014}.

\begin{figure}
\[
\begin{array}{lll}
t \in \interp{A}^+ & \Leftrightarrow & \forall x \in \textit{Var}.\ \forall t'\in\interp{A}^-.\ [[ nu x . t * t' ]] \in \SN\\
t \in \interp{A}^- & \Leftrightarrow & \forall x \in \textit{Var}.\ \forall t'\in\interp{A}^{+c}.\ [[ nu x . t' * t ]] \in \SN\\
t \in \interp{[[< + >]]}^{+c} & \Leftrightarrow & t \in \textit{Var}\ \vee\ t \equiv [[triv]] \\
t \in \interp{[[< - >]]}^{+c} & \Leftrightarrow & t \in \textit{Var}\\
t \in \interp{[[A ->+ B]]}^{+c} & \Leftrightarrow & t \in \textit{Var}\ \vee\ \exists x, t'. t \equiv \lambda x.\, t'\ \wedge\ \forall t''\in\interp{A}^+.\ [t''/x] t'\in\interp{B}^+\\
t \in \interp{[[A ->- B]]}^{+c} & \Leftrightarrow & t \in \textit{Var}\ \vee\ \exists t_1\in\interp{A}^-, t_2\in\interp{B}^+.\ t \equiv [[ < t1 , t2 >]] \\
t \in \interp{[[A /\+ B]]}^{+c} & \Leftrightarrow & t \in \textit{Var}\ \vee\ \exists t_1\in\interp{A}^+, t_2\in\interp{B}^+.\ t \equiv [[ ( t1 , t2 )]] \\
t \in \interp{[[A1 /\- A2]]}^{+c} & \Leftrightarrow & t \in \textit{Var}\ \vee\ \exists d. \exists t'\in\interp{A_d}^+.\ t \equiv [[ inj d t']] 
\end{array}
\]
\caption{Interpretations of types}
\label{fig:classreal}
\end{figure}

\begin{lemma}[Step interpretations]
\label{lem:stepinterp}
If $t\in\interp{A}^+$ and $t\leadsto t'$, then
$t'\in\interp{A}^+$; and similiarly if $t\in\interp{A}^-$ or $t\in\interp{A}^{+c}$.
\end{lemma}
\begin{proof}
The proof is by a mutual well-founded induction.
Assume $t\in\interp{A}^+$ and $t\leadsto t'$.  We must show $t'\in\interp{A}^+$.
For this, it suffices to assume $y\in\textit{Var}$ and $t''\in\interp{A}^-$,
and show $[[nu y . t' * t'']]\in\SN$.  From the assumption that $t\in\interp{A}^+$,
we have 
\[
[[nu y . t * t'']] \in\SN
\]
which indeed implies that 
\[
[[nu y . t' * t'']] \in\SN
\]
A similar argument applies if $t\in\interp{A}^-$.  

For the last part of the lemma, assume $t\in\interp{A}^{+c}$ with
$t\leadsto t'$, and show $t'\in\interp{A}^{+c}$.  The only possible
cases are the following, where $t\not\in\textit{Vars}$.

If $A \equiv [[A1 ->+ A2]]$, then $t$ is of the form $\lambda x.t_a$
for some $x$ and $t_a$, where for all $t_b\in\interp{A_1}^+$, we have
$[t_b/x]t_a\in\interp{A_2}^+$.  Since $t\leadsto t'$, $t'$ must be
$\lambda x.t_a'$ for some $t_a'$ with $t_a\leadsto t_a'$.  It suffices
now to assume an arbitrary $t_b\in\interp{A_1}^+$, and show
$[t_b/x]t_a'\in\interp{A_2}^+$.  But $[t_b/x]t_a\leadsto [t_b/x]t_a'$
follows from $t_a\leadsto t_a'$, so by our IH, we have
$[t_b/x]t_a'\in\interp{A_2}^+$, as required.

If $A\equiv [[A1 ->- A2]]$, then $t$ is of the form $[[< t1 , t2 >]]$
for some $t_1\in\interp{A_1}^-$ and $t_2\in\interp{A_2}^+$; and
$t'\equiv [[< t1' , t2' >]]$ where either $t_1'\equiv t_1$ and $t_2\leadsto t_2'$
or else $t_1\leadsto t_1'$ and $t_2'\equiv t_2$.  Either way, we have
$t_1'\in\interp{A_1}^-$ and $t_2'\in\interp{A_2}^+$ by our IH, so
we have $[[< t1' , t2'>]]\in\interp{[[A1 ->- A2]]}^{+c}$ as required.

The other cases for $A\equiv [[A1 /\p A2]]$ are similar to the previous one.
\end{proof}

\begin{lemma}[SN interpretations]
\label{lem:sninterp}

\begin{tabular}{llllll}
1. & $\interp{A}^+ \subseteq \SN$ & \hspace{2cm} & 3. & $\interp{A}^- \subseteq \SN$\\
2. & $\textit{Vars}\subseteq \interp{A}^-$ & \hspace{2cm} & 4. & $\interp{A}^{+c} \subseteq \SN$
\end{tabular}
\end{lemma}
\begin{proof}
  For purposes of this proof and subsequent ones, define $\delta(t)$ to be
  the length of the longest reduction sequence from $t$ to a normal form, for
  $t\in\SN$.

  The proof of the lemma is by mutual well-founded induction on the
  pair $(A,n)$, where $n$ is the number of the proposition in the
  statement of the lemma; the well-founded ordering in question is the
  lexicographic combination of the structural ordering on types (for $A$) 
  and the ordering $3 < 2 < 1 < 4$ (for $n$).

  For proposition (1): assume $t\in\interp{A}^+$, and show $t\in\SN$.
  Let $x$ be a variable.  By IH(2), $x\in\interp{A}^-$, so by the
  definition of $\interp{A}^+$, we have
\[
  [[ nu x . t * x]] \in \SN
\]
  This implies $t\in\SN$.

  For proposition (2): assume $x\in\textit{Vars}$, and show
  $x\in\interp{A}^-$.  For the latter, it suffices to assume arbitrary
  $y\in\textit{Vars}$ and $t'\in\interp{A}^{+c}$, and show $[[nu y
  . t' * x]] \in \SN$.  We will prove this by inner induction on
  $\delta(t')$, which is defined by IH(3).  By the definition of
  $\interp{A}^{+c}$ for the various cases of $A$, we see that $[[nu y
  . t' * x]]$ cannot be a redex itself, as $t'$ cannot be a cut.  If
  $t'$ is a normal form we are done.  If $t\leadsto t''$, then we have
  $t''\in\interp{A}^{+c}$ by Lemma~\ref{lem:stepinterp}, and we may
  apply the inner induction hypothesis.

  For proposition (3): assume $t\in\interp{A}^-$, and show $t\in\SN$.
  By the definition of $\interp{A}^-$ and the fact that $\textit{Vars}\subseteq\interp{A}^{+c}$
  by definition of $\interp{A}^{+c}$, we have 
\[
  [[nu y . y * t]] \in \SN
\]
  This implies $t\in\SN$ as required.

  For proposition (4): assume $t\in\interp{A}^{+c}$, and consider the
  following cases.  If $t\in\textit{Vars}$ or $A\equiv[[< + >]]$, then
  $t$ is normal and the result is immediate.  So suppose $A \equiv
  [[A1 ->+ A2]]$.  Then $t\equiv \lambda x.t'$ for some $x$ and $t'$
  where for all $t''\in\interp{A_1}^+$, $[t''/x]t'\in\interp{A_2}^+$.
  By IH(2), the variable $x$ itself is in $\interp{A_1}^+$, so
  we know that $t'\equiv[x/x]t'\in\interp{A_2}^+$.  Then by IH(1)
  we have $t'\in\SN$, which implies $\lambda x.t'\in\SN$.  If $A\equiv [[A1 ->- A2]]$,
  then $t\equiv [[< t1 , t2>]]$ for some $t_1\in\interp{A_1}^-$ and
  $t_2\in\interp{A_2}^+$.  By IH(3) and IH(1), $t_1\in\SN$ and $t_2\in\SN$,
  which implies $[[< t1 , t2 >]]\in\SN$.  The cases for $A \equiv [[A1 /\p A2]]$
  are similar to this one.
\end{proof}  

\begin{definition}[Interpretation of contexts]
$\interp{\Gamma}$ is the set of substitutions $\sigma$ such that
for all $[[ x : p A]]\in\Gamma$, $\sigma(x)\in\interp{A}^p$.
\end{definition}

\begin{lemma}[Canonical positive is positive]
\label{lem:canonpos}
$\interp{A}^{+c}\subseteq\interp{A}^+$
\end{lemma}
\begin{proof}
Assume $t\in\interp{A}^{+c}$ and show $t\in\interp{A}^+$.
For the latter, assume arbitrary $x\in\textit{Vars}$ and $t'\in\interp{A}^-$,
and show $[[nu x . t * t']]\in\SN$.  This follows immediately
from the assumption that $t'\in\interp{A}^-$.
\end{proof}

\begin{thm}[Soundness]
\label{thm:sndinterp}
If $[[ J |-c t : p A]]$ then for all $\sigma\in\interp{\Gamma}$, $\sigma t\in\interp{A}^p$.
\end{thm}
\begin{proof}
  We consider one interesting case (see the companion report for all
  the cases).  Define $\delta(t)$ to be the length of the longest
  reduction sequence from $t$ to a normal form, for $t\in\SN$.

\begin{center}
\begin{math}
\inferrule* [right=\ifrName{ClassCut}] {[[ J , x : + A  |-c t1 : + B ]]  \qquad [[  J , x : + A  |-c t2 : - B ]]}{[[  J |-c nu x . t1 * t2 : - A ]]}
\end{math}
\end{center}

It suffices to consider arbitrary $y\in\textit{Vars}$ and
$t'\in\interp{A}^{+c}$, and show $[[nu y. t' * (nu x . sigma t1 *
sigma t2)]]\in\SN$.  By the IH and part 2 of Lemma~\ref{lem:sninterp},
we have $\sigma t_1\in\interp{B}^+$ and $\sigma t_2\in\interp{B}^-$,
which implies $\sigma t_1\in\SN$ and $\sigma t_2\in\SN$ by
Lemma~\ref{lem:sninterp} again.  We proceed by inner induction on
$\delta(t')+\delta(\sigma t_1)+\delta(\sigma t_2)$, using
Lemma~\ref{lem:stepinterp}, to show that all one-step successors of
$[[nu y. t' * (nu x . sigma t1 * sigma t2)]]$ are in $\SN$.  If it is
$t'$, $\sigma t_1$, or $\sigma t_2$ which steps, then the result holds
by inner IH.  The only other reduction possible is by
\dttdrulename{RBetaR}, since $t'$ cannot be a cut term by the
definition of $\interp{A}^{+c}$.  In this case, the IH gives us
$[t'/x]\sigma t_1\in\interp{B}^+$ and $[t'/x]\sigma
t_2\in\interp{B}^-$, and we then have $[[nu y . [t'/x]sigma t1 *
[t'/x]sigma t2]]\in\SN$ by the definition of $\interp{B}^+$.

The proof is by induction on the derivation of $[[ J |-c t : p A]]$.  We consider
the two possible polarities for the conclusion of the typing judgment separately.

\begin{itemize}
\item[Case.]\ 

\vspace{-.2cm}
\begin{center}
\begin{math}
\inferrule* [right=\ifrName{ClassAx}] {\ }{[[J , x : p A , J' |-c x : p A]]}
\end{math}
\end{center}

Since $\sigma\in\interp{[[J , x : p A, J']]}$, $\sigma(x)\in\interp{A}^p$ as required.

\item[Case.]\ 

\vspace{-.2cm}
\begin{center}
\begin{math}
\inferrule* [right=\ifrName{ClassUnit}] {\ }{[[J |-c triv : + < + >]]}
\end{math}
\end{center}

We have $[[triv]]\in\interp{[[<+>]]}^{+c}$ by definition.

\item[Case.]\ 

\vspace{-.2cm}
\begin{center}
\begin{math}
\inferrule* [right=\ifrName{ClassUnit}] {\ }{[[J |-c triv : - < - >]]}
\end{math}
\end{center}

To prove $[[triv]]\in\interp{[[<->]]}^{-}$, it suffices to assume
arbitrary $y\in\textit{Vars}$ and $t\in\interp{[[<->]]}^{+c}$, and
show $[[nu y . t * triv]]\in\SN$.  By definition of
$\interp{[[<->]]}^{+c}$, $t\in\textit{Vars}$, and then $[[nu y . t *
triv]]$ is in normal form.

\item[Case.]\ 

\vspace{-.2cm}
\begin{center}
\begin{math}
\inferrule* [right=\ifrName{ClassAnd}] {[[J |-c t1 : + A]] \qquad [[J |-c t2 : + B]]}{[[J |-c (t1,t2) : + A /\+ B]]}
\end{math}
\end{center}

By Lemma~\ref{lem:canonpos}, it suffices to show $(\sigma t_1,\sigma
t_2)\in\interp{[[A /\+ B]]}^{+c}$.  This follows directly from the
definition of $\interp{[[A /\+ B]]}^{+c}$, since the IH gives us
$\sigma t_1\in\interp{A}^+$ and $\sigma t_2\in\interp{B}^+$.

\item[Case.]\ 

\vspace{-.2cm}
\begin{center}
\begin{math}
\inferrule* [right=\ifrName{ClassAnd}] {[[J |-c t1 : - A1]] \qquad [[J |-c t2 : - A2]]}{[[J |-c (t1,t2) : - A1 /\- A2]]}
\end{math}
\end{center}

It suffices to assume arbitrary $y\in\textit{Vars}$ and
$t'\in\interp{[[A1 /\- A2]]}^{+c}$, and show $[[nu y . t' * (sigma t1
, sigma t2)]]\in\SN$.  If $t'\in\textit{Vars}$, then this follows by
Lemma~\ref{lem:sninterp} from the facts that $\sigma
t_1\in\interp{A_1}^+$ and $\sigma t_2\in\interp{A2}^+$, which we have
by the IH.  So suppose $t'$ is of the form $[[inj d t'']]$ for some
$d$ and some $t''\in\interp{A_d}^+$.  By the definition of $\SN$, it
suffices to show that all one-step successors $t_a$ of the term in
question are $\SN$.  The proof of this is by inner induction on
$\delta(t'') + \delta(\sigma t_1) + \delta(\sigma t_2)$, which exists
by Lemma~\ref{lem:sninterp}, using also Lemma~\ref{lem:stepinterp}.
Suppose that we step to $t_a$ by stepping $t''$, $\sigma t_1$, or
$\sigma t_2$.  Then the result holds by the inner IH.  So consider the
step
\[
[[nu y . inj d t'' * (sigma t1, sigma t2) ~> nu y . t'' * sigma h(t ! d)]]
\]
We then have $[[nu y . t'' * sigma h(t ! d)]]\in\SN$ from the facts
that $t''\in\interp{A_d}^+$ and $\sigma t_d\in\interp{A_d}^-$, by
the definition of $\interp{A_d}^+$.

\item[Case.]\ 

\vspace{-.2cm}
\begin{center}
\begin{math}
\inferrule* [right=\ifrName{ClassAndBar}] {[[J |-c t : + A ! d]]}{[[J |-c inj d t : + A1 /\- A2]]}
\end{math}
\end{center}
By Lemma~\ref{lem:canonpos}, it suffices to prove $[[inj d sigma
t]]\in\interp{[[ A1 /\- A2]]}^+$, but by the definition of $\interp{[[
  A1 /\- A2]]}^+$, this follows directly from $\sigma
t\in\interp{A_d}^+$, which we have by the IH.

\item[Case.]\ 

\vspace{-.2cm}
\begin{center}
\begin{math}
\inferrule* [right=\ifrName{ClassAndBar}] {[[J |-c t : - A ! d]]}{[[J |-c inj d t : - A1 /\+ A2]]}
\end{math}
\end{center}
To prove $[[inj d sigma t]]\in\interp{[[A1 /\+ A2]]}^-$, it suffices
to assume arbitrary $y\in\textit{Vars}$ and $t'\in\interp{[[A1 /\+
  A2]]}^{+c}$, and show $[[ nu y . t' * inj d sigma t]]\in\SN$.  If
$t'\in\textit{Vars}$, then this follows from the fact that $\sigma
t\in\SN$, which we have by Lemma~\ref{lem:sninterp} from $\sigma
t\in\interp{A_d}^-$ (which the IH gives us).  So suppose $t'$ is of
the form $(s_1,s_2)$ for some $s_1\in\interp{A_1}^+$ and
$s_2\in\interp{A_2}^+$.  It suffices to prove that all one-step
successors of the term in question are in $\SN$, as we did in a
previous case above.  Lemma~\ref{lem:sninterp} lets us proceed by
inner induction on $\delta(\sigma t) + \delta(s_1) + \delta(s_2)$,
using also Lemma~\ref{lem:stepinterp}.  If we step $\sigma t$, $s_1$
or $s_2$, then the result holds by inner IH.  Otherwise, we have the
step
\[
[[ nu y . (s1,s2) * inj d sigma t ~> nu y . s ! d * sigma t]]
\]
And this successor is in $\SN$ by the facts that $s_d\in\interp{A_d}^+$
and $\sigma t\in\interp{A_d}^-$, from the definition of $\interp{A_d}^+$.

\item[Case.]\ 

\vspace{-.2cm}
\begin{center}
\begin{math}
\inferrule* [right=\ifrName{ClassImp}] {[[J , x : + A |-c t : + B]]}{[[J |-c \ x . t : + A ->+ B]]}
\end{math}
\end{center}
By Lemma~\ref{lem:canonpos}, it suffices to assume arbitrary $y\in\textit{Vars}$ and $t'\in\interp{A}^+$,
and prove $[t'/x](\sigma t)\in\interp{B}^+$.  But this follows immediately from the IH, since
$[t'/x](\sigma t)\equiv (\sigma[x\mapsto t']) t$ and $\sigma[x\mapsto t]\in\interp{[[J , x : + A]]}$.

\item[Case.]\ 

\vspace{-.2cm}
\begin{center}
\begin{math}
\inferrule* [right=\ifrName{ClassImp}] {[[J , x : - A |-c t : - B]]}{[[J |-c \ x . t : - A ->- B]]}
\end{math}
\end{center}
It suffices to assume arbitrary $y\in\textit{Vars}$ and
$t'\in\interp{[[A ->- B]]}^{+c}$, and show $[[ nu y . t' * \ x . sigma
t]]\in\SN$.  Let us first observe that $[[sigma t]]\in\SN$, because by
the IH, for all $\sigma'\in\interp{[[J , x : - A]]}$, we have $\sigma'
t\in\interp{B}^-$, and $\interp{B}^-\subseteq\SN$ by
Lemma~\ref{lem:sninterp}.  We may instantiate this with
$\sigma[x\mapsto x]$, since by Lemma~\ref{lem:sninterp},
$x\in\interp{A}^-$.  Since $[[sigma t]]\in\SN$, we also have $[[\ x
. sigma t]]\in\SN$.  Now let us consider cases for the assumption
$t'\in\interp{[[A ->- B]]}^{+c}$.  If $t'\in\textit{Vars}$ then we
directly have $[[ nu y . t' * \ x . sigma t]]\in\SN$ from $[[\ x
. sigma t]]\in\SN$.  So assume $t'\equiv[[< t1 , t2>]]$ for some
$t_1\in\interp{A}^-$ and $t_2\in\interp{B}^+$.  By
Lemma~\ref{lem:sninterp} again, we may reason by inner induction on
$\delta(t_1)+\delta(t_2)+\delta(\sigma t)$ to show that all one-step
successors of $[[ nu y . < t1 , t2 > * \ x . sigma t]]$ are in $\SN$,
using also Lemma~\ref{lem:stepinterp}.  If $t_1$, $t_2$, or $\sigma t$
steps, then the result follows by the inner IH.  So suppose we have
the step
\[
[[ nu y . < t1 , t2 > * \ x . sigma t ~>  nu y . t2 * [t1 / x ] (sigma t) ]]
\]
Since $t_1\in\interp{A}^-$, the substitution $\sigma[x\mapsto t_1]$ is
in $\interp{[[J , x : - A]]}$.  So we may apply the IH to obtain $[t_1
/ x ] (\sigma t) \equiv \sigma[x\mapsto t_1]\in\interp{B}^-$.  Then
since $t_2\in\interp{B}^+$, we have $[[nu y . t2 * [t1 / x ] (sigma t)
]]$ by definition of $\interp{B}^+$.

\item[Case.]\ 

\vspace{-.2cm}
\begin{center}
\begin{math}
\inferrule* [right=\ifrName{ClassImpBar}] {[[J |-c t1 : - A]]  \qquad [[ J |-c t2 : + B ]]}{[[ J |-c < t1 , t2 > : + (A ->- B) ]]}
\end{math}
\end{center}
By Lemma~\ref{lem:canonpos}, as in previous cases of positive typing,
it suffices to prove $[[< sigma t1 , sigma t2>]]\in\interp{[[A ->-
  B]]}^{+c}$.  By the definition of $\interp{[[A ->- B]]}^{+c}$, this
follows directly from $\sigma t_1\in\interp{A}^-$ and $\sigma
t_2\in\interp{B}^+$, which we have by the IH.

\item[Case.]\ 

\vspace{-.2cm}
\begin{center}
\begin{math}
\inferrule* [right=\ifrName{ClassImpBar}] {[[J |-c t1 : + A]]  \qquad [[ J |-c t2 : - B ]]}{[[ J |-c < t1 , t2 > : - (A ->+ B) ]]}
\end{math}
\end{center}
It suffices to assume arbitrary $y\in\textit{Vars}$ and
$t'\in\interp{[[A ->+ B]]}^{+c}$, and show $[[nu y. t' * < sigma t1 ,
sigma t2 >]]\in\SN$.  By the IH, we have $\sigma t_1\in\interp{A}^+$
and $\sigma t_2\in\interp{B}^-$, and hence $\sigma t_1\in\SN$ and
$\sigma t_2\in\SN$ by Lemma~\ref{lem:sninterp}.  If
$t'\in\textit{Vars}$, then these facts are sufficient to show the term
in question is in $\SN$.  So suppose $t'\equiv \lambda x.t_3$, for
some $x\in\textit{Vars}$ and $t''$ such that for all
$t_4\in\interp{A}^+$, $[t_4/x]t_3\in\interp{B}^+$.  By similar
reasoning as in a previous case, we have $t_3\in\SN$.  So we may
proceed by inner induction on $\delta(t_1)+\delta(t_2)+\delta(t_3)$ to
show that all one-step successors of $[[nu y. \ x . t3 * < sigma t1 ,
sigma t2 >]]$ are in $\SN$, using also Lemma~\ref{lem:stepinterp}.  If
it is $t_3$, $\sigma t_1$, or $\sigma t_2$ which steps, then the
result follows by the inner IH.  So consider this step:
\[
[[nu y. \ x . t3 * < sigma t1 , sigma t2 >  ~>  nu y . [sigma t1 / x] t3 * sigma t2]]
\]
Since we have that $\sigma t_1\in\interp{A}^+$, the assumption about substitution
instances of $t_3$ gives us that $[\sigma t_1/x]t_3\in\interp{B}^+$, which is
then sufficient to conclude $[[nu y . [sigma t1 / x] t3 * sigma t2]]\in\SN$
by the definition of $\interp{B}^+$.

\item[Case.]\ 

\vspace{-.2cm}
\begin{center}
\begin{math}
\inferrule* [right=\ifrName{ClassCut}] {[[ J , x : - A  |-c t1 : + B ]]  \qquad [[  J , x : - A  |-c t2 : - B ]]}{[[  J |-c nu x . t1 * t2 : + A ]]}
\end{math}
\end{center}
It suffices to assume arbitrary $y\in\textit{Vars}$ and
$t'\in\interp{A}^-$, and show $[[nu y . (nu x . sigma t1 * sigma t2) *
t']]\in\SN$.  By the IH and part 2 of Lemma~\ref{lem:sninterp}, we
know that $\sigma t_1\in\interp{B}^+$ and $\sigma t_2\in\interp{B}^-$.
By Lemma~\ref{lem:sninterp} again, we have $t'\in\SN$, $\sigma
t_1\in\SN$, and $\sigma t_2\in\SN$.  So we may reason by induction on
$\delta(t')+\delta(\sigma t_1)+\delta(\sigma t_2)$ to show that all
one-step successors of $[[nu y . (nu x . sigma t1 * sigma t2) * t']]$
are in $\SN$, using also Lemma~\ref{lem:stepinterp}.  If it is $t'$,
$\sigma t_1$, or $\sigma t_2$ which steps, then the result follows by
the inner IH.  The only possible other reduction is by the
\dttdrulename{RBetaL} reduction rule (Figure~\ref{fig:dtt-red}).  And
then, since $t'\in\interp{A}^-$, we may apply the IH to conclude that
$[t'/x](\sigma t_1)\in\interp{B}^+$ and $[t'/x](\sigma t_2)\in\interp{B}^-$.
By the definition of $\in\interp{B}^+$, this suffices to prove
$[[nu y . [t'/x] sigma t1 * [t'/x]sigma t2]]\in\SN$, as required.

\item[Case.]\ 

\vspace{-.2cm}
\begin{center}
\begin{math}
\inferrule* [right=\ifrName{ClassCut}] {[[ J , x : - A  |-c t1 : + B ]]  \qquad [[  J , x : - A  |-c t2 : - B ]]}{[[  J |-c nu x . t1 * t2 : - A ]]}
\end{math}
\end{center}
It suffices to consider arbitrary $y\in\textit{Vars}$ and
$t'\in\interp{A}^{+c}$, and show $[[nu y. t' * (nu x . sigma t1 *
sigma t2)]]\in\SN$.  By the IH and part 2 of Lemma~\ref{lem:sninterp},
we have $\sigma t_1\in\interp{B}^+$ and $\sigma t_2\in\interp{B}^-$,
which implies $\sigma t_1\in\SN$ and $\sigma t_2\in\SN$ by
Lemma~\ref{lem:sninterp} again.  We proceed by inner induction on
$\delta(t')+\delta(\sigma t_1)+\delta(\sigma t_2)$, using
Lemma~\ref{lem:stepinterp}, to show that all one-step successors of
$[[nu y. t' * (nu x . sigma t1 * sigma t2)]]$ are in $\SN$.  If it is
$t'$, $\sigma t_1$, or $\sigma t_2$ which steps, then the result holds
by inner IH.  The only other reduction possible is by
\dttdrulename{RBetaR}, since $t'$ cannot be a cut term by the
definition of $\interp{A}^{+c}$.  In this case, the IH gives us
$[t'/x]\sigma t_1\in\interp{B}^+$ and $[t'/x]\sigma
t_2\in\interp{B}^-$, and we then have $[[nu y . [t'/x]sigma t1 *
[t'/x]sigma t2]]\in\SN$ by the definition of $\interp{B}^+$.
\end{itemize}
\end{proof}

\begin{corollary}[Strong Normalization]
  \label{thm:strong_normalization}
  If $[[Gr;H |- t : p A@n]]$, then $t \in \SN$.
\end{corollary}
\begin{proof} This follows easily by putting together Theorems~\ref{thm:inttoclass} and~\ref{thm:sndinterp}, with
Lemma~\ref{lem:sninterp}.
\end{proof}

\begin{corollary}[Cut Elimination]
If $[[Gr;H |- t : p A@n]]$, then there is normal $t'$ with
$t\leadsto^* t'$ and $t'$ containing only cut terms of the form
$[[nu x . y * t]]$ or $[[nu x . t * y]]$, for $y$ a variable.
\end{corollary}

\begin{lemma}[Local Confluence]
\label{lem:localconf}
The reduction relation of Figure~\ref{fig:dtt-red} is locally confluent.
\end{lemma}
\begin{proof} We may view the reduction rules as higher-order pattern
  rewrite rules.  It is easy to confirm that all critical pairs (e.g.,
  between \dttdrulename{RBetaR} and the rules \dttdrulename{RImp},
  \dttdrulename{RImpBar}, \dttdrulename{RAnd1},
  \dttdrulename{RAndBar1}, \dttdrulename{RAnd2}, and
  \dttdrulename{RAndBar2}) are joinable.  Local confluence then
  follows by the higher-order critical pair lemma~\cite{nipkow91}.
\end{proof}

\begin{thm}[Confluence for Typable Terms]
The reduction relation restricted to terms typable in DTT is confluent.
\end{thm}
\begin{proof} Suppose $[[ Gr ; H |- t : p A @ n]]$ for some $[[Gr]]$, $[[H]]$, $[[p]]$, and $[[A]]$.
By Lemma~\ref{lemma:type_preservation}, any reductions in the unrestricted reduction
relation from $t$ are also in the reduction relation restricted to typable terms.
The result now follows from Newman's Lemma, using Lemma~\ref{lem:localconf} and
Theorem~\ref{thm:strong_normalization}.
\end{proof}

\begin{comment}
In \cite{crolard01} Crolard introduced the formula 
$[[(A ->- <->) /\- A]]$.  This formula is equivalent to the
double-negation translation of the law-of-excluded middle into
intuitionistic logic.  Its inhabitant in DTT is 
$[[nu y.y * (inj 1 (\x.triv))]]$.  Note that the previous term is a
normal form, but contains a cut.  Thus, general canonicity fails for
DTT, but this is to be expected, because DIL 
does not have general cut elimination.  We conjecture that the cuts
present in normal forms should all be axiom cuts.  Now following Stump
in \cite{Stump:2014:RPD:2541568.2541575} we also conjecture that a
fragment of DTT can be specified where general canonicity holds.  We
leave this investigation for future work.  
% section metatheory_of_dtt (end)
\end{comment}

% section dualized_type_theory (end)
