Moses Ilyich Sch\"onfinkel initially invented combinatory logic as a
way of eliminating bound variables from logic.  He saw bound variables
as just some auxiliary syntax.  The combinatory logic can be
completely defined by two basic operators called $[[S]]$ and $[[K]]$.
The syntax and reduction rules for combinatory logic are defined by
Figure~\ref{fig:comb_syntax} and the typing rules in
Figure~\ref{fig:comb_typing}.  
\todo{Might need to explain the syntax of the judgments.}
This is one of the most beautiful and elegant languages every
conceived.  Later in the year 1927 the combinatory logic was
rediscovered by an american named Haskell Curry.  He was trying to
break the substitution process found in propositional logic down into
a basic algorithm.  To accomplish this he constructed an equivalent
system of cominbators and furthered his work -- even though he had
found that combintory logic was first discovered by Sch\"onfinkel --
on combinators in his thesis. In fact Curry's proof of completeness of
combinatory logic found in his thesis is one of the most efficent
proofs known \cite{Cardone:2006}.
\begin{figure}
  \begin{center}
    \begin{tabular}{lll}
      Syntax: & \\
      & 
      \begin{math}
        \begin{array}{lll}
          [[A]],[[B]],[[C]] & ::= & [[A -> B]]\\
          [[a]],[[b]],[[c]] & ::= & [[x]]\,|\,[[S]]\,|\,[[K]]\,|\,[[I]]\,|\,[[a b]]
        \end{array}
      \end{math}
      & \\
      Reduction Rules: & \\
      & 
      \begin{math}
        \begin{array}{lll}
          \CombdruleRedXXK{} & \CombdruleRedXXS{}
        \end{array}
      \end{math}
    \end{tabular}
  \end{center}
  
  \caption{Syntax and reduction rules for combinatory logic}
  \label{fig:comb_syntax}
\end{figure}

\begin{figure}
  \begin{center}
    \begin{math}
      \begin{array}{ccc}
        \CombdruleVar{} & \CombdruleS{}\\
        & \\
        \CombdruleK{}   & \CombdruleApp{}\\ 
        & \\
        \CombdruleI{}
      \end{array}
    \end{math}
  \end{center}

  \caption{Definitions and type-checking algorithm for combinatory logic}
  \label{fig:comb_typing}
\end{figure}

Much later in 1958 Curry found an amazing correspondence between the
axioms of the implicational fragement of intuitionistic propositional
logic and combinatory logic.  The axioms are defined in
Figure~\ref{fig:hilbert_system} \cite{Troelstra:2000}.  If one
compares the axioms with the types of the combinators $[[S]]$ and
$[[K]]$ one will find they correspond.  This is the first connection
between logic and type theory and the first peice of an isomorphism
today known as the Curry-Howard isomorphism, proposistions as types
and proofs as programs correspondence, or the Curry-Howard-Lambek
isomorphism \cite{Howard:1980}.  We will discuss the this
correspondence in detail in Section~\ref{sec:the_holy_trinity}.

\begin{figure}
  \begin{center}
    \begin{math}
      \begin{array}{lll}
        $$\mprset{flushleft}
        \inferrule* [right=] {
          \ 
        }{[[A -> (B -> A)]]}
        &
        $$\mprset{flushleft}
        \inferrule* [right=] {
          \ 
        }{[[(A -> (B -> C)) -> ((A -> B) -> (A ->C))]]}\\
        & \\
        $$\mprset{flushleft}
        \inferrule* [right=] {
         \  
        }{[[A]] \to \perp}
      \end{array}
    \end{math}
  \end{center}
  \caption{Implicational fragement of intuitionistic propositional logic}
  \label{fig:hilbert_system}
\end{figure}

%%% Local Variables: 
%%% mode: latex
%%% TeX-master: "paper"
%%% End:
