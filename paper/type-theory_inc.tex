% Theorems
\newcounter{thm}
%% Resets the thm counter at the start of each section.
\makeatletter
\@addtoreset{thm}{section}
\makeatother
%% This makes sure \ref has the section, subsection, and thm number.
\def\thethm{\thesubsection.\arabic{thm}}
%% This makes the figure numbering match the other environments.
\renewcommand{\thefigure}{\thesubsection.\arabic{figure}}

\newenvironment{thm}{
 \refstepcounter{thm} \par \addvspace{\baselineskip} \noindent \textbf{\underline{Theorem \thethm.}} \begin{em}
}{
  \end{em} \par \addvspace{\baselineskip} 
}
\newenvironment{lemma}{
  \refstepcounter{thm} \par \addvspace{\baselineskip} \noindent \textbf{\underline{Lemma \thethm.}} \begin{em}
}{
  \end{em} \par \addvspace{\baselineskip}
}
\newenvironment{proposition}{
  \refstepcounter{thm} \par \addvspace{\baselineskip} \noindent \textbf{\underline{Proposition \thethm.}} \begin{em}
}{
  \end{em} \par \addvspace{\baselineskip}
}
\newenvironment{corollary}{
  \refstepcounter{thm} \par \addvspace{\baselineskip} \noindent \textbf{\underline{Corollary \thethm.}} \begin{em}
}{
  \end{em} \par \addvspace{\baselineskip}
}
\newenvironment{definition}{
  \refstepcounter{thm} \par \addvspace{\baselineskip} \noindent \textbf{\underline{Definition \thethm.}} 
  \begin{em}
}{
  \end{em} \par \addvspace{\baselineskip} 
}
\newenvironment{example}{
  \refstepcounter{thm} \par \addvspace{\baselineskip} \noindent \textbf{\underline{Example \thethm.}} \begin{em}
}{
  \end{em} \par \addvspace{\baselineskip}
}
\newenvironment{openproblem}{
  \par \addvspace{\baselineskip} \noindent \textbf{\underline{Open Problem. \arabic{section}.\arabic{subsection}.\arabic{thm}.}} 
  \refstepcounter{thm} \begin{em}
}{
  \end{em} \par \addvspace{\baselineskip} 
}

% Commands that are useful for writing about type theory and programming language design.
\newcommand{\case}[4]{\text{case}\ #1\ \text{of}\ #2\text{.}#3\text{,}#2\text{.}#4}
\newcommand{\interp}[1]{[\negthinspace[#1]\negthinspace]}
\newcommand{\normto}[0]{\rightsquigarrow^{!}}
\newcommand{\join}[0]{\downarrow}
\newcommand{\redto}[0]{\rightsquigarrow}
\newcommand{\nat}[0]{\mathbb{N}}
\newcommand{\terms}[0]{\mathcal{T}}
\newcommand{\fun}[2]{\lambda #1.#2}
\newcommand{\CRI}[0]{\text{CR-Norm}}
\newcommand{\CRII}[0]{\text{CR-Pres}}
\newcommand{\CRIII}[0]{\text{CR-Prog}}
\newcommand{\subexp}[0]{\sqsubseteq}
\newcommand{\napprox}[2]{\lfloor #1 \rfloor_{#2}}
\newcommand{\interpset}{\mathcal{I}}
\newcommand{\powerset}[1]{\mathcal{P}(#1)}
\newcommand{\vinterp}[1]{\mathcal{V}[\negthinspace[#1]\negthinspace]}
\newcommand{\vbinterp}[2]{\bar{\mathcal{V}}_{#1}[\negthinspace[#2]\negthinspace]}
\newcommand{\ginterp}[1]{\mathcal{G}[\negthinspace[#1]\negthinspace]}
\newcommand{\dinterp}[1]{\mathcal{D}[\negthinspace[#1]\negthinspace]}
\newcommand{\tinterp}[1]{\mathcal{T}[\negthinspace[#1]\negthinspace]}
