\input{sep3-annotated-ott}
\input{sep3-unannotated-ott}

\renewcommand{\Sepdrulename}[1]{\scriptsize \textsc{#1}}
\renewcommand{\SepUdrulename}[1]{\scriptsize \textsc{#1}}

% See the seppp.tex file for reminders of the motivation of the
% design.

We first define some convenient notation. If $([[Ctor C]], [[t]],
[[M]]) \in [[D]]$ then $[[D]]_1([[Ctor C]]) = [[Ctor C]]$,
$[[D]]_2([[Ctor C]]) = [[t]]$, and $[[D]]_3([[Ctor C]]) = [[M]]$.

{\bf Judgement Descriptions}
\begin{center}
  \begin{tabular}{lll}
    $[[D , G |- M t tl : Ctor C x1 ep1 dots xn epn]]$\\
    &
    \begin{tabular}{lll}
      $[[M]]$  & List of datatype constructors.\\
      $[[t]]$  & Type of datatype being declared.\\
      $[[tl]]$ & Datatype constructor accumulator.\\
      $[[Ctor C x1 ep1 dots xn epn]]$ & The datatype being declared.\\
    \end{tabular}
    & \\
    $[[D , G |-PB R t1 t2 p l : P]]$\\
    &
    \begin{tabular}{lll}
      $[[R]]$  & List of case branches.\\
      $[[t1]]$ & The scrutiny of the case expression.\\
      $[[t2]]$ & The type of the scrutiny.\\
      $[[p]]$  & The proof that the scrutiny equals the pattern.\\
      $[[l]]$  & An accumulator.\\
      $[[P]]$  & The type of the case expression.\\
    \end{tabular}
    & \\
    $[[D , G |-TB H t1 t2 p l : t']]$\\
    &
    \begin{tabular}{lll}
      $[[H]]$  & List of case branches.\\
      $[[t1]]$ & The scrutiny of the case expression.\\
      $[[t2]]$ & The type of the scrutiny.\\
      $[[p]]$  & The proof that the scrutiny has a value.\\
      $[[l]]$  & An accumulator.\\
      $[[t']]$ & The type of the case expression.\\
    \end{tabular}
    & \\
    $[[D , G |- eqpf q ep : P]]$\\
    &
    \begin{tabular}{lll}
      $[[q]]$  & Either a proof of an equality or a trivial axiom of equality.\\
      $[[ep]]$ & The stage of $[[q]]$; $+$ if $[[q]]$ is a proof and $-$ otherwise.\\
    \end{tabular}
  \end{tabular}
\end{center}

