\input{sep3-annotated-ott}
\input{sep3-unannotated-ott}

\renewcommand{\Sepdrulename}[1]{\scriptsize \textsc{#1}}
\renewcommand{\SepUdrulename}[1]{\scriptsize \textsc{#1}}
\newcommand{\Sep}[0]{\text{Sep}^3}
% See the seppp.tex file for reminders of the motivation of the
% design.

\begin{figure}
  \begin{center}
    \small
    \begin{math}
      \begin{array}{llllllllllllllll}
        \text{(Programs) }    [[prog]] ::= [[pdef]] \mid [[prog pdef]]\\
        \text{(Definitions) } [[pdef]] ::= [[datatype_decl]] \mid [[predicateType x :: LK]] \mid
        [[predef x = P]] \mid \\
        [[theorem x :: P]] \mid [[prfdef x = p]] \mid [[type x :: t]] \mid [[termdef x = t]] \\
        \text{(Datatypes) }   [[datatype_decl]] ::= [[data Ctor C t where M]]\\
        \text{(Stages) }      [[ep]]   ::= [[+]] \mid [[-]]\\
        \text{(Classifiers) } [[ks]]   ::= [[L]] \mid [[P]]\\
        \text{(Argument Classifiers) } Arg, A ::= [[term]] \mid [[proof]] \mid LK\\
        \text{(Super Kinds) } [[L]] ::= [[Logical i]]\\
        \text{(Indicies) } [[i]] ::= 1 \mid 2 \mid 3 \mid \cdots\\
      \end{array}
    \end{math}
  \end{center}
  \caption{Syntax for Annotated $\Sep$}
  \label{fig:ann-sep3-syntax}
\end{figure}

We first define some convenient notation. If $([[Ctor C]], [[t]],
[[M]]) \in [[D]]$ then $[[D]]_1([[Ctor C]]) = [[Ctor C]]$,
$[[D]]_2([[Ctor C]]) = [[t]]$, and $[[D]]_3([[Ctor C]]) = [[M]]$.

{\bf Judgement Descriptions}
\begin{center}
  \begin{tabular}{lll}
    $[[D , G |- M t tl : Ctor C x1 ep1 dots xn epn]]$\\
    &
    \begin{tabular}{lll}
      $[[M]]$  & List of datatype constructors.\\
      $[[t]]$  & Type of datatype being declared.\\
      $[[tl]]$ & Datatype constructor accumulator.\\
      $[[Ctor C x1 ep1 dots xn epn]]$ & The datatype being declared.\\
    \end{tabular}
    & \\
    $[[D , G |-PB R t1 t2 p l : P]]$\\
    &
    \begin{tabular}{lll}
      $[[R]]$  & List of case branches.\\
      $[[t1]]$ & The scrutiny of the case expression.\\
      $[[t2]]$ & The type of the scrutiny.\\
      $[[p]]$  & The proof that the scrutiny equals the pattern.\\
      $[[l]]$  & An accumulator.\\
      $[[P]]$  & The type of the case expression.\\
    \end{tabular}
    & \\
    $[[D , G |-TB H t1 t2 p l : t']]$\\
    &
    \begin{tabular}{lll}
      $[[H]]$  & List of case branches.\\
      $[[t1]]$ & The scrutiny of the case expression.\\
      $[[t2]]$ & The type of the scrutiny.\\
      $[[p]]$  & The proof that the scrutiny has a value.\\
      $[[l]]$  & An accumulator.\\
      $[[t']]$ & The type of the case expression.\\
    \end{tabular}
    & \\
    $[[D , G |- eqpf q ep : P]]$\\
    &
    \begin{tabular}{lll}
      $[[q]]$  & Either a proof of an equality or a trivial axiom of equality.\\
      $[[ep]]$ & The stage of $[[q]]$; $+$ if $[[q]]$ is a proof and $-$ otherwise.\\
    \end{tabular}
  \end{tabular}
\end{center}

\label{sec:the_annotated_language}
\begin{definition}
  \label{def:asep3_togglepol}  
  \begin{center}
    \begin{math}
    \mathsf{togglePol} ([[t]], [[vl]]) = \left\{
    \begin{array}{ll}
      [[Pi - x : y.togglePol(t',vl)]]    & : [[t equiv Pi + x : y.t']] \land \exists [[A]].([[y]]:[[A]]) \in [[vl]]\\
      [[Pi ep x : A . togglePol(t',vl)]] & : \text{otherwise}
    \end{array}
    \right.
  \end{math}
  \end{center}
\end{definition}

\begin{definition}
  \label{def:eraser_function}
  The erasers are defined as follows:
  \begin{itemize}
  \item Super Kinds:\\
    \begin{math}
      \begin{array}{lll}
        |\mathsf{LogicalKind}_{[[i]]}| & = & \mathsf{LogicalKind}_{[[i]]}
      \end{array}
    \end{math}
  \item Logical Kinds:\\
    \begin{math}
      \begin{array}{lll}
        |[[x]]|                   & = & x\\
        |\mathsf{Formula}_{[[i]]}| & = & \mathsf{Formula}_{{[[i]]}}\\
        |[[Forall x : A . LK]]|   & = & \forall [[x]] : |[[A]]|.|[[LK]]|
      \end{array}
    \end{math}
    
  \item Predicates:\\
    \begin{math}
      \begin{array}{lll}
        |[[\ L x : A . P]]|       & = & \Lambda [[x]] . |[[P]]|\\
        |[[P a]]|                 & = & |P|\ |a|\\
        |[[Forall x : A . P]]|    & = & \forall [[x]]:|[[A]]|.|[[P]]|\\
        |[[t1 = t2]]|             & = & |[[t1]]| = |[[t2]]|\\
        |[[t !]]|                 & = & |[[t]]|!\\
        |[[let x = p in P]]|      & = & \mathsf{let}\,[[x]] = |[[p]]|\,in\,|[[P]]|\\
        |[[let x = P in P']]|     & = & \mathsf{let}\,[[x]] = |[[P]]|\,in\,|[[P']]|\\
        |[[let x = t [x'] in P]]| & = & \mathsf{let}\,[[x]] = |[[t]]|\,in\,|[[P]]|\\
        |[[P1 + P2]]|             & = & |[[P1]]| + |[[P2]]|\\
        |[[Exists x : A . P]]|    & = & \exists [[x]]:|[[A]]|.|[[P]]|\\
        |[[bot i]]|               & = & [[bot i]]\\
        |[[t < t']]|              & = & |[[t]]| < |[[t']]|
      \end{array}
    \end{math}
    
  \item Proofs:\\
    \begin{math}
      \begin{array}{lll}
        |[[x]]| & = & x\\
        |[[\ L x : A . p]]| & = & \Lambda [[x]].|[[p]]|\\
        |[[p p']]| & = & |[[p]]|\,|[[p']]|\\
        |[[p t]]| & = & |[[p]]|\,|[[t]]|\\
        |[[injl p with P]]| & = & \mathsf{injl}\,|[[p]]|\\
        |[[injr p with P]]| & = & \mathsf{injr}\,|[[p]]|\\
        |[[case p of x . p' , y . p'']]| & = & \mathsf{case}\,|[[p]]|\,\mathsf{of}\,[[x]] . |[[p']]| , [[y]] . |[[p'']]|\\
        |[[(a , p ) as P]]| & = & (|[[a]]|,|[[p]]|)\\
        |[[case p1 of ( x , y ) . p2]]| & = & \mathsf{case}\,|[[p1]]|\,\mathsf{of}\,( [[x]] , [[y]] ) . |[[p2]]|\\
        |[[let x = p' in p]]| & = & \mathsf{let}\,[[x]] = |[[p']]|\,in\,|[[p]]|\\
        |[[let x = P in p]]| & = & \mathsf{let}\,[[x]] = |[[P]]|\,in\,|[[p]]|\\
        |[[let x = t [x'] in p]]| & = & \mathsf{let}\,[[x]] = |[[t]]|\,in\,|[[p]]|\\
        |[[join t1 t2]]| & = & \mathsf{join}\\
        |[[conv p by q1 ... qn at x1 ... xn . P]]| & = & (\lambda x_1.\cdots.\lambda x_m.|[[p]]|)\,|q_1|\,\cdots\,|q_m|\\
        |[[predconv p P]]| & = & |[[p]]|\\
        |[[valax t]]| & = & \mathsf{valax}\\
        |[[case t [ x ] p of R]]| & = & \mathsf{case}\,|[[t]]|\,\mathsf{of}\,|[[R]]|\\
        |[[tcase t [ x ] of abort -> p1 | ! -> p2]]| & = & 
            \mathsf{tcase}\,|[[t]]|\,\mathsf{of}\,[[abort]] \to |[[p1]]|\,|\,! \to |[[p2]]|\\
        |[[ind f x : t , p1 . p2]]| & = & \mathsf{ind}\,f\,[[x]].|[[p2]]|\\
        |[[contra p1]]| & = & \mathsf{contra}\,|[[p1]]|\\
        |[[contraval p1 p2]]| & = & \mathsf{contraval}\,|[[p1]]|\,|[[p2]]|\\
      \end{array}
    \end{math}
    
  \item Proof branches ($[[R]]$):\\
    \begin{math}
      \begin{array}{lll}
        |[[Ctor C x1 ep1 dots xn epn => p | R]]| & = & |[[Ctor C x1 ep1 dots xn epn]]| \Rightarrow |[[p]]|\,|\,|[[R]]|\\
        |[[done]]| & = & done 
      \end{array}
    \end{math}

  \item Terms: \\
    \begin{math}
      \begin{array}{lll}
        |[[x]]| & = & x\\
        |\mathsf{Type}_{[[i]]}| & = & \mathsf{Type}_{[[i]]}\\
        |[[Pi + x : A . t]]| & = & \Pi [[x]]_{+} : |[[A]]|.|[[t]]|\\
        |[[Pi - x : A . t]]| & = & \Pi [[x]]_{-} : |[[A]]|.|[[t]]|\\
        |[[\ P + x : LK . t]]| & = & |[[t]]|\\
        |[[\ P - x : LK . t]]| & = & |[[t]]|\\
        |[[\ P + x : P . t]]| & = & |[[t]]|\\
        |[[\ P - x : P . t]]| & = & |[[t]]|\\
        |[[\ P + x : t' . t]]| & = & \lambda [[x]].|[[t]]|\\
        |[[\ P - x : t' . t]]| & = & |[[t]]|\\
        |[[conv t by q1 ... qn at x1 ... xm . t']]| & = & |[[t']]|\\
        |[[case t [ x ] of H]]| & = & \mathsf{case}\,|[[t]]|\,\mathsf{of}\,|[[H]]|\\
        |[[t t' +]]| & = & |[[t]]|\,|[[t']]|\\
        |[[t t' -]]| & = & |[[t]]|\\
        |[[t p +]]| & = & |[[t]]|\\
        |[[t p -]]| & = & |[[t]]|\\
        |[[t P +]]| & = & |[[t]]|\\
        |[[t P -]]| & = & |[[t]]|\\
        |[[let x = p in t]]|      & = & |[[ [p/x]t]]|\\
        |[[let x = P in t']]|     & = & |[[ [P/x]t']]|\\
        |[[let x = t [y] in t']]| & = & \mathsf{let}\,[[x]] = |[[t]]|\,in\,|[[t']]|\\
        |[[tcast t by p]]| & = & \mathsf{tcast}\,|[[t]]|\\
        |[[abort t]]| & = & \mathsf{abort}\\
        |[[rec f x : t1 . t2]]| & = & \mathsf{rec}\,[[f]]\,[[x]].|[[t2]]|\\
        |[[Ctor C ]]| & = & \Sepkw{C}\\
      \end{array}
    \end{math}
      
  \item Term branches ($[[H]]$):\\
    \begin{math}
      \begin{array}{lll}
        |[[Ctor C x1 ep1 dots xn epn => t | H]]| & = & |[[Ctor C x1 ep1 dots xn epn]]| \Rightarrow |[[t]]|\,|\,|[[H]]|\\
        |[[done]]| & = & done 
      \end{array}
    \end{math}

  \item Logical Contexts ($[[G]]$):\\
    \begin{math}
      \begin{array}{lll}
        |\cdot| = \cdot\\
        |[[G,x : g A]]| = |[[G]]|,[[x]]:^[[g]] |[[A]]|
      \end{array}
    \end{math}

  \item Programmatic Contexts ($[[G]]$):\\
    \begin{math}
      \begin{array}{lll}
        |\cdot| = \cdot\\
        |[[G, x : g t]]| = |[[G]]|,[[x]]:^[[g]] |[[t]]|\\
        |[[G, x : g P]]| = |[[G]]|\\
        |[[G, x : g LK]]| = |[[G]]|\\
      \end{array}
    \end{math}
   
  \item Constructor Sets ($[[l]]$):\\
    \begin{math}
      \begin{array}{lll}
        |[[empty]]|                 = \emptyset\\
        |\{ [[Ctor C]] : [[t']]\}|  = \{ [[Ctor C]]:|[[t]]| \}\\
        |[[l1]] \cap |[[l2]]|       = |[[l1]]| \cap |[[l2]]|\\
        |[[l1]] - [[l2]]|           = |[[l1]]| - |[[l2]]|\\
        |[[l1]] = [[l2]]|           = |[[l1]]| = |[[l2]]|\\\\
      \end{array}
    \end{math}

  \item Sets of Datatype Constructors ($[[M]]$):\\
    \begin{math}
      \begin{array}{lll}
        |\mathsf{nil}| = \mathsf{nil}\\
        |[[(Ctor C : t') :: M]]| = ([[Ctor C]] : |[[t']]|)::|[[M]]| \\
      \end{array}
    \end{math}
    
  \item Logical Signatures ($[[D]]$):\\
    \begin{math}
      \begin{array}{lll}
        |\cdot| = \cdot\\
        |[[D , ( C , t , M )]]| = |[[D]]|, ([[C]], |[[t]]|, |[[M]]|)\\
        |[[D , x = ( a , A )]]| = |[[D]]| , [[x]] = ( |[[a]]| , |[[A]]| )
      \end{array}
    \end{math}

  \item Programmatic Signatures ($[[D]]$):\\
    \begin{math}
      \begin{array}{lll}
        |\cdot| = \cdot\\
        |[[D , ( C , t , M )]]| = |[[D]]|, ([[C]], |[[t]]|, |[[M]]|)\\
        |[[D , x = ( P , LK )]]| = |[[D]]|\\
        |[[D , x = ( p , P )]]| = |[[D]]|\\
        |[[D , x = ( t , t' )]]| = |[[D]]| , [[x]] = ( |[[t]]| , |[[t']]| )
      \end{array}
    \end{math}
  \end{itemize}
\end{definition}
