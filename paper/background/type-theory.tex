Martin-L\"of is a Swedish mathematical logician and philosopher who
was interested in defining a constructive foundations of mathematics.
The foundation he defined he called Type Theory, but what is now
referred to as Martin-L\"of's Type Theory
\cite{Martin:1982,Martin:1984}.  It is considered the first full
dependent type theory.  Type Theory is defined by giving a syntax and
deriving three judgments.  Martin-L\"of placed particular attention
to judgments.  In Type Theory types can be considered as
specifications of programs, propositions, and sets.  Martin-L\"of then
stresses that one cannot know the meaning of a type without first
knowing what its canonical members are, knowing how to construct
larger members from the canonical members, and being able to tell when
two types are equal.  To describe this meaning he used judgments.  The
judgments are derived using inference rules just as we have seen, and
they tell us exactly which elements are canonical and which can be
constructed from smaller members.  There is also an equality judgment
which describes how to tell when two terms are equal.  Martin-L\"of's
Type Theory came in two flavors: intensional type theory and
extensional type theory.  The difference amounts to equality types and
whether the equality judgment is distinct from the propositional
equality or not.  The impact of intensional vs extensional is quite
profound.  The latter can be given a straightforward categorical
model, while the former cannot.  We first define a basic core of
Martin-L\"of's Type Theory and then we describe how to make it
intensional and then extensional.

The syntax of Martin-L\"of's Type Theory is defined in
Fig.~\ref{fig:lof_syntax}.
\begin{figure}
  \begin{center}
    \begin{math}
      \begin{array}{lll}
        [[S]] & ::= & \mathsf{Type}\,|\,\mathsf{True}\\
        [[T]], [[A]], [[B]], [[C]]        & ::= & [[X]]\,|\,[[1]]\,|\,[[0]]\,|\,[[A + B]]\,|\,[[( x : A ) -> B]]\,|\,[[{ x : A } -> B]]\\
        [[t]], [[s]], [[a]], [[b]], [[c]] & ::= & [[x]]\,|\,[[tt]]\,|\,[[\ x : A . t]]\,|\,[[t1 t2]]\,|\,[[( t1 , t2 )]]\,|\,[[case s of x , y . t]]\,|\,[[case s of x . t1 , y . t2]]\,|\,[[abort]]
      \end{array}
    \end{math}
  \end{center}
  \caption{The syntax of Martin-L\"of's Type Theory}
  \label{fig:lof_syntax}
\end{figure}
The language consists of sorts $[[S]]$ denoted $\mathsf{Type}$ and
$\mathsf{True}$. The sort $\mathsf{Type}$ is a type universe and has
as inhabitants types.  It is used to classify which things are valid
types.  The sort $\mathsf{True}$ will be used when treating
types as propositions to classify which formulas are true. The second
part of the language are types $[[T]]$.  Types consist of
propositional variables $[[X]]$, true or top $[[1]]$, false or bottom
$[[0]]$, sum types $[[A + B]]$ which correspond to constructive
disjunction, dependent products $[[(x : A) -> B]]$ which correspond to
function types, universal quantification, and implication, and
disjoint union $[[{ x : A} -> B]]$ which correspond to pairs,
constructive conjunction and existential quantification.  We can see
that dependent products and disjoint union bind terms in types, hence,
types do depend on terms.  The third and final part of the language
are terms.  We only comment on the term constructs we have not seen
before.  The term $[[tt]]$ is the inhabitant of $[[1]]$ and is called
unit.  We have a term which corresponds to a contradiction called
$[[abort]]$. Finally, we have two case constructs: $[[case s of
    x,y.t]]$ and $[[case s of x.t1,y.t2]]$.  The former is the
elimination form for disjoint union and says if $[[s]]$ is a pair then
substitute the first projection for $[[x]]$ in $[[t]]$ and the second
projection for $[[y]]$ in $[[t]]$.  Having the ability to project out
both pieces of a pair results in the disjoint union also called
$\Sigma$-types being strong.  A weak disjoint union type is one in
which only the first projection of a pair is allowed.  The second case
construct $[[case s of x.t1,y.t2]]$ is the elimination form for the
sum type.  This says that if $[[s]]$ is a term of type $[[A+B]]$, but
is an inhabitant of the type $[[A]]$ then substitute $[[a]]$ for
$[[x]]$ in $[[t1]]$, or if $[[s]]$ is an inhabitant of $[[B]]$
substitute it for $[[y]]$ in $[[t2]]$.  This we will see is the
elimination form for constructive disjunction.

In dependent type theory we replace arrow types $[[A -> B]]$ with
dependent product types $[[(x : A) -> B]]$, where $[[B]]$ is allowed
to depend on $[[x]]$.  It turns out that we can define arrow types as
$[[(x : A) -> B]]$ when $[[x]]$ is free in $[[B]]$; that is, $[[B]]$
does not depend on $[[x]]$.  We will often abbreviate this by $[[A ->
B]]$.  Recall that the arrow type corresponds to implication.  The
dependent product type also corresponds to universal quantification,
because it asserts for all terms of type $[[A]]$ we have $[[B]]$, or
for all proofs of the proposition $[[A]]$ we have $[[B]]$.
Additionally, in dependent type theory we replace cartesian product
$[[A x B]]$ by disjoint unions $[[{x : A} -> B]]$ where $[[B]]$ may
depend on $[[x]]$.  The inhabitants of this type are pairs $[[(a,b)]]$
where $[[b]]$ may depend on $[[a]]$.  Now simple pairs can be defined
just like arrow types are defined using product types.  The type $[[A
x B]]$ is defined by $[[{x : A} -> B]]$ where $[[B]]$ does not depend
on $[[x]]$.  Then $[[b]]$ in the pair $[[(a,b)]]$ does not depend on
$[[a]]$.  We can define projections for simple pairs as follows:
\begin{center}
  \begin{math}
    \begin{array}{lll}
      [[fst t]] & := & [[case t of x,y.x]]\\
      [[snd t]] & := & [[case t of x,y.y]].
    \end{array}
  \end{math}
\end{center}

The kinding rules are defined in Fig.~\ref{fig:lof_kinding}.
\begin{figure}
  \begin{center}
    \begin{mathpar}
        \TTdruleKXXBottom{}   \and
        \TTdruleKXXUnit{}     \and
        \TTdruleKXXExt{}      \and
        \TTdruleKXXProd{}     \and
        \TTdruleKXXPi{}       \and
        \TTdruleKXXArrow{}    \and
        \TTdruleKXXCoprod{}
    \end{mathpar}
  \end{center}
  \caption{Kinding for Martin-L\"of's Type Theory}
  \label{fig:lof_kinding}
\end{figure}
These rules derive the judgment $[[G |- T : Type]]$ which describes
all well-formed types -- inhabitants of $\mathsf{Type}$.  Now types
are also propositions of intuitionistic logic.  The judgment $[[G |- T
    : True]]$ describes which propositions are true constructively.
The rules deriving this judgment are in Fig.~\ref{fig:lof_validity}.
Note that while $[[0]]$ is a type, it is not a true proposition.  This
judgment validates the correspondence between types and propositions.
In fact we could have denoted $[[(x:A)->B]]$ as $\forall [[x]]:[[A]].[[B]]$
and $[[{x:A}->B]]$ as $\exists [[x]]:[[A]].[[B]]$. 
\begin{figure}
  \begin{center}
    \begin{mathpar}
        \TTdruleLXXTrue{}     \and
        \TTdruleLXXProd{}     \and
        \TTdruleLXXForalli{}  \and
        \TTdruleLXXForalle{}  \and  
        \TTdruleLXXImpi{}     \and
        \TTdruleLXXImpe{}     \and
        \TTdruleLXXOriOne{}   \and
        \TTdruleLXXOriTwo{}   \and 
        \TTdruleLXXOre{}      \and
        \TTdruleLXXExti{}     \and
        \TTdruleLXXExte{}
    \end{mathpar}
  \end{center}
  \caption{Validity for Martin-L\"of's Type Theory}
  \label{fig:lof_validity}
\end{figure}
The typing rules are defined in Fig.~\ref{fig:lof_typing}.
\begin{figure}
  \begin{center}
    \begin{mathpar}
        \TTdruleTXXUnit{}      \and
        \TTdruleTXXVar{}       \and
        \TTdruleTXXSum{}       \and
        \TTdruleTXXCaseOne{}   \and
        \TTdruleTXXProd{}      \and
        \TTdruleTXXProdOne{}   \and
        \TTdruleTXXProdTwo{}   \and
        \TTdruleTXXPi{}        \and
        \TTdruleTXXAppOne{}    \and
        \TTdruleTXXArrow{}     \and
        \TTdruleTXXAppTwo{}    \and
        \TTdruleTXXCoProdOne{} \and
        \TTdruleTXXCoProdTwo{} \and
        \TTdruleTXXCaseTwo{}   \and
        \TTdruleTXXAbort{}     \and
        \TTdruleTXXConv{}
    \end{mathpar}
  \end{center}
  \caption{Typing Rules for Martin L\"of's Type Theory}
  \label{fig:lof_typing}
\end{figure}
\begin{figure}
  \begin{center}
    \begin{mathpar}
        \TTdruleEqXXUnit{}      \and
        \TTdruleEqXXFst{}       \and
        \TTdruleEqXXSnd{}       \and
        \TTdruleEqXXBeta{}      \and
        \TTdruleEqXXEta{}       \and
        \TTdruleEqXXCaseOne{}   \and
        \TTdruleEqXXCaseTwo{}   \and
        \TTdruleEqXXCaseThree{} 
    \end{mathpar}
  \end{center}
  \caption{Equality for Martin-L\"of's Type Theory}
  \label{fig:lof_eq}
\end{figure}
We include the typing rules for the derived forms for arrow types
and cartesian products.  These can be derived as well.  The rules here
are straightforward, so we only comment on the elimination rule for sum
types.  The rule is defined as
\begin{center}
  \begin{mathpar}
    \TTdruleTXXCaseTwo{}.
  \end{mathpar}
\end{center}
We mentioned above that this rule corresponds to the elimination form
for constructive disjunction.  This rule tells us that to eliminate
$[[A]] \lor [[B]]$ we must assume $[[A]]$ and prove $[[C]]$ and then
assume $[[B]]$ and prove $[[C]]$, but this is exactly what the above
rule tells us.  The computational correspondence is that the case
construct gives us away to case split over terms of two types.

As it stands Martin-L\"of's Type Theory is a very powerful logic.  The
axiom of choice must be an axiom of set theory, because it cannot be
proven from the other axioms.  The axiom of choice states that the
cartesian product of a family of non-empty sets is non-empty.
Martin-L\"of showed in \cite{Martin:1984} that the axiom of choice can
be proven with just the theory we have defined thus far in this
section.  Thus, one could also prove the well-ordering theorem.  This
is good, because it shows that Type Theory is powerful enough to be a
candidate for a foundation of mathematics.  This is also good for
dependent type based verification, because we can formulate expressive
specifications of programs.

The final judgment of Martin-L\"of's Type Theory is the definitional
equality judgment. It has the form $[[G |- a = b : A]]$. The rules
deriving this judgment tell us when we can consider two terms as being
equal.  Then two types whose elements are equal based on this judgment
are equal.  The equality rules are defined in Fig.~\ref{fig:lof_eq} where
we leave the congruence rules implicit for presentation purposes.

These rules look very much like full $\beta$-reduction, but these are
equalities.  They are symmetric, transitive, and reflexive unlike
reduction which is not symmetric.  This is a definitional equality and
it can be used during type checking implicitly at will using the following rule:
\begin{center}
  \begin{math}
    \TTdruleTXXConv{}
  \end{math}
\end{center}
We now describe when Martin-L\"of's Type Theory is extensional or
intensional.

\textbf{Extensional Type Theory.} In extensional type theory our
equality judgment is not distinct from propositional equality.  To
make Martin-L\"of's type theory extensional we add the following
rules:
\begin{center}
  \begin{math}
    \begin{array}{lll}
      Kinding & Typing\\
        $$\mprset{flushleft}
        \inferrule* [right=] {
          [[G |- A : Type]]
          \\\\
          [[G |- a : A]]
          \\\\
          [[G |- b : A]]
        }{[[G |- Id A a b : Type]]}
        &
        $$\mprset{flushleft}
      \inferrule* [right=] {
        [[G |- a = b : A]]
      }{[[G |- tt : Id A a b]]}
      &
      $$\mprset{flushleft}
      \inferrule* [right=] {
        [[G |- t : Id A a b]]
      }{[[G |- a = b : Id A a b]]}
    \end{array}
  \end{math}
\end{center}
Using these rules we can prove all of the usual axioms of identity:
reflexivity, transitivity, and symmetry \cite{Martin:1984}.  Notice
that these rules collapse definitional equality into propositional equality. 
The right most typing rule is where extensional type theory gets its power.
This rule states that propositional equations can be used interchangeably anywhere. 
This power comes with sacrifice, some meta-theoretic properties one may wish to have like
termination of equality and decidability of type checking no longer hold
\cite{Streicher:1991,Streicher:1993}.

\textbf{Intensional Type Theory.} Now to make Martin-L\"of's Type Theory intensional we add the following rules:
\begin{center}
  \begin{tabular}{lll}
    \begin{math}
    \begin{array}{lll}
      \text{Kinding:} & \text{Typing:} \\
      $$\mprset{flushleft}
      \inferrule* [right=] {
        [[G |- a : A]]      
      }{[[G |- r(a) : Id A a a]]}
      &
      $$\mprset{flushleft}
      \inferrule* [right=] {
        [[G |- c : Id A a b]]
        \\
        [[G, x : A |- d : B(x,x,r(x))]]
        \\\\
        [[G, x:A,y:A,z:Id A x y |- B(x,y,z) : Type]]
      }{[[G |- J(d,a,b,c) : B(a,b,c)]]} \\              
    \end{array}          
  \end{math}\\
  & \\
  \begin{math}
    \begin{array}{lll}
      \text{Equality:} \\
      $$\mprset{flushleft}
      \inferrule* [right=] {
        [[G |- a : A]]
      }{[[G |- J(d,a,a,r(a)) = d a : B(a,a,r(a))]]}
    \end{array}
  \end{math}
  \end{tabular}  
\end{center}
As we can see here propositional equality is distinct from the
definitional equality judgment.  In the above rules $[[r(a)]]$ is the
constant denoting reflexivity, and $[[J(a,b,c,d)]]$ is
just an annotation on $[[d]]$ with all the elements of the equality.
Using these we can prove reflexivity, transitivity, and symmetry.  We
do not go into any more detail here between intensional and
extensional type theory, but a lot of research has gone into
understanding intensional type theory.  Models of intensional type
theory are more complex than extensional type theory.  Recently, there
has been an upsurge of interest in intensional type theory due to a
new model for type theory where types are interpreted as homotopies
\cite{Awodey:2010}.  See
\cite{Streicher:1991,Streicher:1993,Hofmann:1995,Hofmann:1998} for
more information.

We said at the beginning of this section that we would only define a
basic core of Martin-L\"of's type theory.  We have done both for
intensional Type Theory and extensional Type Theory, but Martin-L\"of
included a lot more than this in his classic paper \cite{Martin:1984}.
He included ways of defining finite types as well as arbitrary
infinite types called universes much like $\mathsf{Type}$.  He also
included rules for defining inductive types which in the design of
programming languages are very useful \cite{Dybjer:1997}.

The universe $\mathsf{Type}$ contains all well-formed types.  It is
quite natural to think of $\mathsf{Type}$ as a type itself.  This is
called the $\mathsf{Type}:\mathsf{Type}$ axiom.  In fact Martin-L\"of
did that in his original theory, but Girard was able to prove that
such an axiom destroys the consistency of the theory.  Girard was able
to define the Burali-Forti paradox in Type Theory with
$\mathsf{Type}:\mathsf{Type}$ \cite{Coquand:1986,Coquand:1994}.  Now
$\mathsf{Type}:\mathsf{Type}$ is inconsistent when the type theory
needs to correspond to logic, but if it is used purely for programming
it is a very nice feature.  It can be used in generic programming,
because it allows for impredicativity over terms as well as types.  In
Martin-L\"of's Type Theory without the $\mathsf{Type}:\mathsf{Type}$
axiom types are program specifications, hence, the theory can be seen
as a terminating functional programming
language \cite{Nordstrom:1990}.  
