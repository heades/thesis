The year 1980 was a wonderful year for type theory. Not only did Howard show that
there exists a correspondence between natural deduction style propositional logic
and type theory, but Joachim Lambek also showed that there is a correspondence
between type theory and cartesian closed categories \cite{Lambek:1980}.  In this
section we briefly outline how this is the case and give an interpretation of STLC
in a cartesian closed category.  Before we can interpret STLC we first summarize
some basic definitions of category theory.  We begin with the definition of a 
category.
\begin{definition}
  \label{def:category}
    A \emph{category} \index{Category} denoted $\mathcal{C}, \mathcal{D},\cdots$ is an
    abstract mathematical structure consisting of a set of objects $Obj$
    denoted $A,B,C,\cdots$ and a set of morphisms $Mor$ denoted
    $f,g,h,\cdots$.  Two functions assigning objects to morphisms called
    $src$ and $tar$. The function $src$ assigns a morphism its source
    object (domain object) while $tar$ assigns its target object (range
    object).  We denote this assignment as $f : A \to B$, where $src(f)
    = A$ and $tar(f) = B$. Now for each object $A \in Obj$ there exists
    a unique family of morphisms called identities denoted $id_A : A \to
    A$.  For any two morphisms $f : A \to B$ and $g : B \to C$ the
    composition of $f$ and $g$ must be a morphism $g \circ f : A \to C$.
  \ \\
  \noindent
  Morphisms must obey the following rules:
  \begin{center}
    \begin{math}
      \begin{array}{ccc}
        $$\mprset{flushleft}
        \inferrule* [right=] {
          f : A \to B
          \\
          id : B \to B
        }{id \circ f = f}
        &
        $$\mprset{flushleft}
        \inferrule* [right=] {
          f : A \to B
          \\
          id : A \to A
        }{f \circ id = f}\\
        &\\
        $$\mprset{flushleft}
        \inferrule* [right=] {
          c : C \to D
          \\
          b : B \to C
          \\
          a : A \to B
        }{(c \circ b) \circ a = c \circ (b \circ a)}
      \end{array}
    \end{math}
  \end{center}
\end{definition}
In order to interpret STLC we will need a category with some special
features.  The first of these is the final object. 
\begin{definition}
  \label{def:terminal_object}
    An object $1$ of a category $\mathcal{C}$ is the
    \emph{final}\index{Final/Terminal Object} object if and only if there exists
    exactly one morphism $\Diamond_A : A \to 1$ for every object $A$.
\end{definition}
\noindent
We will use the final object and finite products to interpret
typing contexts. Finite products are a generalization of the cartesian
product in set theory.
\begin{definition}
  \label{def:finite_products}
  \index{Finite Binary Product}
  An object of a category $\mathcal{C}$ denoted $A \times B$ is called a
  binary product of the objects $A$ and $B$ iff there exists morphisms
  $\pi_1 : A \times B \to A$ and $\pi_1 : A \times B \to B$ such that
  for any object $C$ and morphisms $f_1 : C \to A$ and $f_2 : C \to B$
  there exists a unique morphism $f: C \to A \times B$ such that
  the following diagram commutes (we denote the fact that $f$
  is unique by $!f$):
  \def\Assl{{\rm assl}}\def\Id{{\rm id}}
  \begin{diagram}
      &             & C            & &  \\
      & \ldTo{f_1}  & \dDashto{f!} & \rdTo{f_2} &  \\
    A & \lTo{\pi_1} & A \times B   & \rTo{\pi_1} & B.\\
  \end{diagram}
\end{definition}
The notion of a binary product can be extended in the straightforward
way to finite products of objects denoted $A_1 \times \cdots
\times A_n$ for some natural number $n$.  We will use finite products
to interpret typing contexts in the category.  We need one more categorical
structure to interpret STLC in a category.  We need a special object that can
be used to model implication or the arrow type.
\begin{definition}
  \label{def:exponentials} \index{Exponential Object}
  An \emph{exponential} of two objects $A$ and $B$ in a category
  $\mathcal{C}$ is an object $B^A$ and an arrow $\epsilon : B^A \times
  A \to B$ called the evaluator.  The evaluator must satisfy the
  universal property: for any object $A$ and arrow $f : A \times B \to
  C$, there is a unique arrow, $f^* : A \to C^B$ such that the
  following diagram commutes:
  \begin{diagram}
               &          &  C                       &                 &  \\
               & \ruTo{f} &                          & \luTo{\epsilon} &  \\
    A \times B &          & \rDashto{f^* \times id_B} &                & C^B \times B
  \end{diagram}
\end{definition}
We call ${f}^*$ the currying of $f$.  By universality of $\epsilon$
every binary morphism can be curried uniquely.  In the above
definition we are using $\_ \times \_$ as an endofunctor.  That is for
any morphisms $f : A \to C$ and $g : B \to D$ we obtain the morphism
$f \times g : A \times B \to C \times D$.  We say a category
$\mathcal{C}$ has all products and all exponentials if and only if for
any two objects in $\mathcal{C}$ the product of those two objects
exists in $\mathcal{C}$ and similarly for exponentials.
\begin{definition}
  \label{def:cartesian}
  A category $\mathcal{C}$ is \emph{cartesian closed}\index{Cartesian
    Closed Category} if and only if it has a terminal object $1$, all
  products, and all exponentials.
\end{definition}
\noindent
This is all the category theory we introduce in this thesis.  The
interested reader should
see \cite{Crole:1994,Gunter:1992,Lawvere:2009,Pierce:1991} for
excellent introductions to the subject.

We now have everything needed to interpret STLC as a category.  Our
interpretation follows that of \cite{Gunter:1992}.  The idea behind
the interpretation is to interpret types as objects and terms as
morphisms.  Now a term alone does not make up a morphism, because they
lack a source and a target object.  So instead we interpret only
typeable terms in a typing context.  That is we interpret triples
$\langle [[G]], [[t]], [[T]] \rangle$ where $[[G |- t : T]]$.
\begin{definition}
  \index{Categorical Interpretation of STLC}
  \label{def:cat_interp_stlc}
  Suppose $\mathcal{C}$ is a cartesian closed category.  Then we interpret
  STLC in the category $\mathcal{C}$ by first interpreting types as follows:
  \begin{center}
    \begin{math}
      \begin{array}{lll}
        \interp{[[X]]} & = & \hat{X}\\
        \interp{[[T1 -> T2]]} & = & \interp{[[T2]]}^{\interp{[[T1]]}}\\
      \end{array}
    \end{math}
  \end{center}
  Then typing contexts are interpreted in the following way:
  \begin{center}
    \begin{math}
      \begin{array}{lll}
        \interp{[[.]]}     & = & 1\\
        \interp{[[G,x:T]]} & = & \interp{[[G]]} \times \interp{[[T]]}
      \end{array}
    \end{math}
  \end{center}
  Finally, we interpret terms as follows:
  \begin{center}
    \begin{tabular}{lllllll}
      Variables:\\      
      \begin{math}
        \begin{array}{r}
          \interp{\langle [[G,xi:Ti]], [[xi]], [[Ti]] \rangle} = 
          \begin{diagram}
            (\interp{[[G]]}) \times \interp{[[T]]} & \rTo{\pi_i} & \interp{[[T]]}\\          
        \end{diagram}
        \end{array}
      \end{math}\\
      \\
      $\lambda$-Abstractions:\\
      \begin{math}
        \begin{array}{r}
          \interp{\langle [[G]],[[\x:T1.t]],[[T1 -> T2]] \rangle}   = 
        \begin{diagram}
          \interp{[[G]]} & \rTo{\interp{\langle [[G,x : T1]],[[t]],[[T2]] \rangle}^*} & \interp{[[T2]]}^{\interp{[[T1]]}}
        \end{diagram}
        \end{array}
      \end{math}\\
      \\
      Applications:\\
      \begin{math}
        \begin{array}{r}
          \interp{\langle [[G]],[[t1 t2]],[[T2]] \rangle}   = 
          \begin{diagram}
            \interp{[[G]]} & \rTo{\langle \interp{\langle [[G]],[[t1]],[[T1 -> T2]] \rangle},
              \interp{\langle [[G]],[[t2]],[[T1]] \rangle} \rangle} 
            & \interp{[[T2]]}^{\interp{[[T1]]}} \times \interp{[[T1]]}
            & \rTo{\epsilon} & \interp{[[T2]]}
          \end{diagram}
        \end{array}
      \end{math}
    \end{tabular}
  \end{center} 
\end{definition}
In the previous definition $\hat{X}$ is just an additional object of
the category.  It does not matter what we call it.  It does however
need to be unique.  This is how we interpret STLC as a cartesian
closed category.  Modeling other type theories with more advanced
features follows quite naturally.  It is not until we hit dependent
types where things change drastically.
