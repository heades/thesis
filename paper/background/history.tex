In this section we give a short history of type theory.  This history
will set the stage for the later development by illustrating the
reasons type theories exist and are important, and by giving some
definitions of well-known theories that make for good examples in
later sections.  We first start with the early days of type theory
between the years of 1900 and 1960 during the time of Bertrand Russell
and Alonzo Church.  They are as we consider them the founding fathers
of type theory.  % We then move to discuss more modern type theories.
% The time line depicted in Fig.~\ref{fig:timeline} gives a summary of
% the most significant breakthroughs in type theory.
The history given here is presented in chronological order.  This is
not to be considered a complete history, but rather a glimpse at the
highlights of the history of type theory.  This is the least amount of
history one must know to fully understand where we have been and where
this line of research may be heading.

% \begin{figure}
%   \begin{chronology}[10]{1900}{2000}{1ex}{10000px}
%     \event{1901}{{\tiny Russell's Paradox}}
    % \event[1908]{1920}{{\tiny Simple Type Theory}}
    % \event{1920}{{\tiny Sch\"onfinkel - Combinatory Logic}}
    % \event{1927}{{\tiny Curry - Rediscovered Combinatory Logic}}
    % \event{1932}{{\tiny BHK-Interpretation }}
    % \event{1934}{{\tiny Curry - Combinators and Intuitionistic Logic }}
    % \event{1940}{{\tiny Church's Simple Theory of Types}}
    % \event[1958]{1967}{{\tiny System T}}
    % \event{1967}{{\tiny Automath}}
    % \event{1965}{{\tiny Tait - Connection between Cut Elimination and $\beta$-Reduction}}
    % \event{1969}{{\tiny Howard - STLC and Natural Deduction}}
    % \event[1971]{1974}{{\tiny Type:Type, Girard's Paradox, and System F}}
    % \event[1975]{1995}{{\tiny Intuitionistic Type Theory, LCF}}
    % \event{1978}{{\tiny NuPRL}}
    % \event{1980}{{\tiny Lambek - STLC and Cartesian Closed Categories}}
    % \event{1988}{{\tiny The Calculus of Constructions }}
    % \event{1990}{{\tiny The Calculus of (Co)Inductive Constructions }}
    % \event{1992}{{\tiny Pure Type Systems and The $\lambda\mu$-Calculus}}
    % \event[1993]{1994}{{\tiny Edinburgh Logical Framework and $\lambda\Delta$-Calculus}}
%   \end{chronology}
%   \caption{Time line of the History of Type Theory}
%   \label{fig:timeline}
% \end{figure}

\section{The Early Days of Type Theory (1900 - 1960)}
\label{sec:the_early_days_of_type_theory}

In the early 1900's Bertrand Russell pointed out a paradox in naive
set theory.  The paradox states that if $H = \{ x\,|\,x \not \in x \}$
then $H \in H \iff H \not \in H$.  The problem Russell exploits is
that the comprehension axiom of naive set theory is allowed to use
impredicative-universal quantification. That is $x$ in the definition
of $H$ could be instantiated with $H$, because we are universally
quantifying over all sets.  Russell called this vicious circularity,
and he thought it made no sense at all. Russell plagued by this
paradox needed a way of eliminating it.  To avoid the paradox Russell,
as he described in letters to Gottlob Frege \cite{Hintikka:1995,
  Heijenoort:1967}, considers sets as having a certain level and such
sets may only contain objects of lower level.  Actually, in his
letters to Frege he gives a brief description of what came to be
called the ramified theory of types which is a generalization of the
type theory we describe here.  However, this less general type theory
is enough to avoid Russell's logical paradoxes.  These levels can be
considered as types of objects and so Russell's theory became known as
simple type theory.  Now what does such a theory look like?  Elliott
Mendleson gives a nice and simple definition of the simple type theory
in \cite{Mendelson:2009} and we summarize this in the following
definition.
\begin{definition}
  \label{def:simple_type_theory}
  Let $U$ denote the universe of sets. We divide $U$ as follows:
  \begin{itemize}
  \item $J^1$ is the collection of individuals (objects of type $0$).
  \item $J^{n+1}$ is the collection of objects of type $n$.
  \end{itemize}
\end{definition}
As mentioned above the simple type theory avoids Russell's paradox.
Lets consider how this is accomplished.  Take Russell's paradox
and add types to it following Def.~\ref{def:simple_type_theory}.  We
obtain if $H^n = \{ x^{n-1}\,|\,x^{n-1} \not \in x^{n-1} \}$ then $H^n
\in H^n \iff H^n \not \in H^n$.  We can easily see that this paradox is
false.  $H^n$ can only contain elements of type $n-1$ which excludes
$H^n$.

Russell's simple type theory reveals something beautiful.  It shows
that to enforce a particular property over a collection of objects we
can simply add types to the objects.  This is the common theme behind
all type theories.  The property Russell wished to enforce was
predicativity of naive set theory.  Throughout this thesis we will see
several different properties types can enforce.  While ramified type
theory and simple type theory are the first defined type theories they
however are not the formulation used throughout computer science.  The
most common formulations used are the varying formulations and
extensions of Alonzo Church's simply typed theory and Haskell Curry's
combinatory logic \cite{Church:1940,Cardone:2006} 
\footnote{While Church's simple type theory is the most common there
  are some other type theories that have become very common to use and
  extend.  To name a few: Thierry Coquand's Calculus of Constructions,
  Per Martin-L\"of's Type Theory, Michel Parigot's
  $\lambda\mu$-Calculus, and Philip Wadler's Dual Calculus.}.  

In 1932 Alonzo Church published a paper on a set of formal postulates
which he thought could be used to get around Russell's logical
paradoxes without the need for types \cite{Church:1933}.  In this
paper he defines what we now call the $\lambda$-calculus.  The
original $\lambda$-calculus consisted of variables, predicates denoted
$[[\x.t]]$, and predicate application denoted $[[t1 t2]]$.  See
Appendix~\ref{subsec:the_lambda-calculus} for a complete definition of
the $\lambda$-calculus.  It was not until Stephen Kleene and John Rosser were
able to show that the $\lambda$-calculus was inconsistent as a logic
when Church had to embrace types \cite{Kleene:1935}.  To over come the
logical paradoxes shown by Kleene and Rosser, Church, added types to
his $\lambda$-calculus to obtain the simply typed $\lambda$-calculus
\cite{Church:1940,Andrews:2009}.  In the next section we give a
complete definition of Church's simple type theory.  The reason we
postpone the definition of the simply typed $\lambda$-calculus is
because we provide a modern formulation of the theory.  So far we have
summarized the beginnings of type theory starting with Russell, Curry,
and Church.  Some really great references on this early history and more
can be found in \cite{Cardone:2006,Coquand:2010b,Barendregt:1992}.  We now
move on to modern type theory where we will cover a large part of type
theory as it stands today.
% section the_early_days_of_type_theory (end)

\section{Modern Type Theory (1961 - Present)}
\label{sec:moderen_type_theory}
In this section we take a journey through modern type theory by
presenting various important advances in the field.  We will provide
detailed definitions of each type theory considered.  The reader may
have noticed that the only definition of type theory we have provide
is that a type theory is any theory in which one must enforce a
property by organizing the objects of the theory into collections
based on a notion of type.  This is not at all a complete definition
and this section will serve as a guide to a more complete definition.
We do not give a complete general formal definition of a type theory,
but we hope that it is discernible from this survey.  The first type
theory we define is the modern formulation of the simply typed
$\lambda$-calculus.

\textbf{The simply typed $\lambda$-calculus.} \input{churchSTLC-inc}
\input{currySTLC-inc}

\textbf{G\"odel's system T.} \input{systemT-inc}

\textbf{Girard-Reynold's System F.} \input{systemF-inc}

\textbf{Stratified System F.} \input{SSF-inc}

\textbf{System $\text{F}^\omega$.} \input{systemFomega-inc}  
% section modern_type_theory (end)

%%% Local Variables: 
%%% mode: latex
%%% TeX-master: "paper"
%%% End:
