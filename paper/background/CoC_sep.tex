It turns out that CoC is really just an extension of system
$\Fw$\index{Calculus of Constructions (CoC)}.  We
do not have to define it using a collapsed syntax -- even though it is
prettier.  We call the extension of system $\Fw$ to CoC separated CoC
to distinguish it from the collapsed versions.  The syntax for
separated CoC is in Fig.~\ref{fig:CoC_sep_syntax}\footnote{We do not
  have a citation for where this formulation can be found.  It was
  learned by the author from Hugo Herbelin at the 2011 Oregon
  Programming Language Summer School.}.  
\begin{figure}
  \begin{center}
    \begin{math}
      \begin{array}{lll}
        [[S]] & ::= & *\\
        [[K]] & ::= & [[Type]]\,|\,[[( X : K ) -> K']]\,|\,[[( x : T ) -> K]]\\
        [[T]] & ::= & [[X]]\,|\,[[\ X : K . T]]\,|\,[[\ x : T1 . T2]]\,|\,[[T1 T2]]\,|\,[[T t]]\,|\,[[( X : K ) -> T]]\,|\,[[( x : T ) -> T']]\\
        [[t]] & ::= & [[x]]\,|\,[[\ x : T . t]]\,|\,[[\ X : K . t]]\,|\,[[t1 t2]]\,|\,[[t T]]\\
      \end{array}
    \end{math}
  \end{center}
  \caption{Syntax for the Separated Calculus of Constructions}
  \label{fig:CoC_sep_syntax}
\end{figure}
This formulation simply extends system $\Fw$ with dependency.
\begin{figure}
  \begin{center}
    \begin{mathpar}      
        \CoCSdruleSXXType{}    \and
        \CoCSdruleSXXProdOne{} \and
        \CoCSdruleSXXProdTwo{} 
    \end{mathpar}
  \end{center}
  \caption{Sorting Rules for the Separated Calculus of Constructions}
  \label{fig:CoC_sep_sorting}
\end{figure}
Due to the separation of the language we increase the number of
judgments.  We now have four judgments: sorting, kinding, typing, and
equality.  They are defined as one would expect.  We do not go into
much detail here. The sorting rules are defined in
Fig.~\ref{fig:CoC_sep_sorting}, the kinding rules are defined in
Fig.~\ref{fig:CoC_sep_kinding}, the typing rules in
Fig.~\ref{fig:CoC_sep_typing}, and finally the equality rules in
Fig.~\ref{fig:CoC_sep_red}.
\begin{figure}
  \begin{center}
    \begin{mathpar}
        \CoCSdruleKXXVar{}      \and
        \CoCSdruleKXXProdOne{}  \and
        \CoCSdruleKXXProdTwo{}  \and
        \CoCSdruleKXXLamOne{}   \and
        \CoCSdruleKXXLamTwo{}   \and
        \CoCSdruleKXXAppOne{}   \and
        \CoCSdruleKXXAppTwo{}
    \end{mathpar}
  \end{center}
  \caption{Kinding Rules for the Separated Calculus of Constructions}
  \label{fig:CoC_sep_kinding}
\end{figure}
\begin{figure}
  \begin{center}
    \begin{mathpar}
        \CoCSdruleVar{}     \and
        \CoCSdruleLam{}     \and
        \CoCSdruleApp{}     \and
        \CoCSdruleTypeAbs{} \and
        \CoCSdruleTypeApp{} \and
        \CoCSdruleConv{}
    \end{mathpar}
  \end{center}
  \caption{Typing Rules for the Separated Calculus of Constructions}
  \label{fig:CoC_sep_typing}
\end{figure}
\begin{figure}
  \begin{center}
    \begin{mathpar}
        \CoCSdruleRXXBetaOne{}      \and
        \CoCSdruleRXXBetaTwo{}      \and
        \CoCSdruleRXXBetaThree{}    \and
        \CoCSdruleRXXBetaFour{}     \and
        \CoCSdruleRXXLamOne{}       \and
        \CoCSdruleRXXLamTwo{}       \and
        \CoCSdruleRXXLamThree{}     \and
        \CoCSdruleRXXAppOne{}       \and
        \CoCSdruleRXXAppTwo{}       \and
        \CoCSdruleRXXAppThree{}     \and
        \CoCSdruleRXXAppFour{}      \and
        \CoCSdruleRXXTypeAppOne{}   \and
        \CoCSdruleRXXTypeAppTwo{}   \and
        \CoCSdruleRXXTypeAppThree{} 
    \end{mathpar}
  \end{center}
  \caption{The Equality for the Separated Calculus of Constructions}
  \label{fig:CoC_sep_red}
\end{figure}
We can see that this formulation makes sense from the PTS perspective,
because system $\Fw$ has the following set of rules $R = \{
(*,*),(\Box,*),(\Box,\Box)\}$ and we need to make it dependent.  That
is we need types to depend on terms.  So we add $(*,\Box)$ to $R$ and
we obtain CoC.  
