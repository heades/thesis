System T extended STLC with primitive recursion, but it is not really
that large of a leap forward, logically.  However, a large leap was
taken independently by a French logician named Jean-Yves Girard and an
American computer scientist named John Reynolds.  In 1971 Girard
published his thesis which included a number of advances in type
theory one of them being an extension of STLC with two new constructs
\cite{Girard:1971,Girard:1989,Barendregt:1992}.  In STLC we have term
variables and binders for them called $\lambda$-abstractions.  Girard
added type variables and binders for them.  This added the ability to
define a large class of truly universal functions. He named his theory
system F, and went on to show that it has a beautiful correspondence
with second order arithmetic \cite{Wadler:2007}.  He showed that
everything definable in second order arithmetic is also definable in
system F by defining a projection from system F into second order
arithmetic.  This shows that system F is a very powerful type theory
both computationally and as we will see later logically.  Later in
1974 Reynolds published a paper which contained a type theory
equivalent to Girard's system F \cite{Reynolds:1974,Reynolds:1998}.
Reynolds being in the field of programming languages was investigating
polymorphism.  That is the ability to define universal (or
generic\footnote{ Throughout this thesis we will use the term
  ``generic'' to mean that terms or programs are written with the most
  abstract type possible.  Try not to confuse this with generic
  programming in the sense used in the design of algorithms.})
functions within a programming language.  That is functions with
generic types which can be instantiated with other types. For example,
being able to write a generic fold operation which is polymorphic in
the type of data the list can hold.  In system T or STLC this was not
possible.  We would have to define a new fold for each type of list.
Reynolds also showed that system F is equivalent to second order
arithmetic, in a similar, although different, way Girard did
\cite{Wadler:2007}.

The syntax for terms, types, and the reduction rules are defined in
Fig.~\ref{fig:F_syntax} and the definition of the typing relation is
defined in Fig.~\ref{fig:F_typing}.
\begin{figure}
  \index{Girard/Reynolds System F}
  \begin{center}
    \begin{tabular}{lll}
      Syntax:
      \vspace{10px} \\
      \begin{math}
        \begin{array}{lll}
          [[T]] & ::= & [[X]]\,|\,[[T -> T']]\,|\,[[Forall X.T]]\\
          [[t]] & ::= & [[x]]\,|\,[[\x:T.t]]\,|\,[[\\X.t]]\,|\,[[t1 t2]]\,|\,[[t [T] ]]
        \end{array}
      \end{math}\\
      \\
      Full $\beta$-reduction: & \\      
      \begin{mathpar}
          \FdruleRXXBeta{}    \and  
          \FdruleRXXTypeRed{} \and
          \FdruleRXXLam{}     \and
          \FdruleRXXTypeAbs{} \and     
          \FdruleRXXAppOne{}  \and
          \FdruleRXXAppTwo{}  \and    
          \FdruleRXXTypeApp{}   
      \end{mathpar}
    \end{tabular}
  \end{center}

  \caption{Syntax and reduction rules for system F}
  \label{fig:F_syntax}
\end{figure}
\begin{figure}
  \begin{center}
    \begin{mathpar}
      \FdruleVar{}      \and
      \FdruleLam{}      \and
      \FdruleTypeAbs{}  \and
      \FdruleApp{}      \and
      \FdruleTypeApp{}  
    \end{mathpar}
  \end{center}
  \caption{Typing relation for the system F}
  \label{fig:F_typing}
\end{figure}
Similar to system T we can easily see that system F is an extension of
STLC.  Types now contain a new type $[[Forall X.T]]$ which binds the
type variable $[[X]]$ in the type $[[T]]$
\index{Polymorphism/Universal Types}.  This allows one to define more
universal types allowing for the definition of single functions that
can work on data of multiple different types.  Terms are extended with
two new terms the $[[\\X.t]]$ and $[[t [T] ]]$.  The former is the
introduction form for the $\forall$-type while the latter is the
elimination form for the $\forall$-type. The former binds the type
variable $[[X]]$ in $[[t]]$ similarly to the
$\lambda$-abstraction. The latter is read, ``instantiate the type of
term $[[t]]$ with the type $[[T]]$.''  The typing rules make this more
apparent.  The formulation of system F we present here is indeed
Church style so terms do contain type annotations.  We need a
reduction rule to eliminate the bound variable in $[[\\X.t]]$ with an
actual type much like application for $\lambda$-abstractions.  Hence,
we extended the reduction rules of STLC with a new rule
$\Fdrulename{R\_TypeRed}$ which does just that.  We next consider some
example functions in system F.

\begin{example}
  \label{ex:F_terms}
  Example functions with their types in system F:
  \begin{center}
    \begin{tabular}{lllllllllll}
      Identity: \\
      \begin{tabular}{rl}
        Type: & $[[Forall X.(X -> X)]]$\\
        Term: & $[[\\X.\x:X.x]]$\\      
      \end{tabular}
      \\
      Pairs: \\
      \begin{tabular}{rl}
        Type: & $[[Forall X.(Forall Y.(X -> (Y -> (PAIRTY X Y))))]]$\\
        Term: & $[[\\X.\\Y.\x:X.(\y:Y.\\Z.(\z:X->(Y->Z).((z x) y)))]]$\\      
      \end{tabular}
      \\
      First Projection: \\
      \begin{tabular}{rl}
        Type: & $[[Forall X.(Forall Y.((PAIRTY X Y) -> X))]]$\\
        Term: & $[[\\X.\\Y.(\p:PAIRTY X Y.((p [X]) (\x:X.\y:Y.x)))]]$\\      
      \end{tabular}
      \\
      Second Projection: \\
      \begin{tabular}{rl}
        Type: & $[[Forall X.(Forall Y.((PAIRTY X Y) -> Y))]]$\\
        Term: & $[[\\X.\\Y.(\p:PAIRTY X Y.((p [Y]) (\x:X.\y:Y.y)))]]$\\      
      \end{tabular}
      \\
      Natural Number $1$: \\
      \begin{tabular}{rl}
        Type: & $[[Forall X.((X->X)->(X->X))]]$\\
        Term: & $[[\\X.(\s:(X->X).(\z:X.(s z)))]]$\\      
      \end{tabular}
      \\
      Natural Number $2$: \\
      \begin{tabular}{rl}
        Type: & $[[Forall X.((X->X)->(X->X))]]$\\
        Term: & $[[\\X.(\s:(X->X).(\z:X.(s (s z))))]]$\\      
      \end{tabular}
      \\
      Natural Number $3$: \\
      \begin{tabular}{rl}
        Type: & $[[Forall X.((X->X)->(X->X))]]$\\
        Term: & $[[\\X.(\s:(X->X).(\z:X.(s (s (s z)))))]]$\\      
      \end{tabular}
      \\
      Natural Number $n$: \\
      \begin{tabular}{rl}
        Type: & $[[Forall X.((X->X)->(X->X))]]$\\
        Term: & $\Lambda [[X]].(\lambda [[s]]:([[X->X]]).(\lambda[[z]]:[[X]].([[s]]^n\,[[z]])))$\\
      \end{tabular}
    \end{tabular}
  \end{center}
\end{example}
Note that in the previous example we used the definition 
\begin{center}
  \begin{math}
    [[PAIRTY X Y]] \defeq [[Forall Z.((X -> (Y -> Z)) -> Z)]]
  \end{math}
\end{center}
for readability. We could have gone even further than natural numbers
and pairs by defining
addition, multiplication, exponentiation, and even primitive
recursion, but we leave those to the interested reader.  For more
examples, see \cite{Girard:1989}.  The encodings we use are the famous
Church encodings\index{Church Encodings} of pairs and natural numbers.  What is remarkable
about the encoding of natural numbers is that they act as function
iteration.  That is for any function $f$ from any type $[[X]]$ to
$[[X]]$ and value $v$ of type $[[X]]$ we have $n\,[X]\,f\,v \redto^* f^n
v$, where $n$ is the term $n$ in the above table.

There is one important property of $\Fdrulename{TypeApp}$ which the
reader should take notice of.  Notice that there are no restrictions
on what types $[[T]]$ ranges over.  That is there is nothing
preventing $[[T]]$ from being $[[Forall X.T']]$.  This property is
known as impredicativity\index{Impredicativity} and system F is an impredicative system.  The
reader may now be questioning whether or not this type theory is
terminating.  That is can we use impredicativity to obtain a looping
term?  The answer was settled negatively by Girard and we will see how
he proved this in Section~\ref{chap:metatheory_of_programming_languages}.
The possibility of writing a looping term in this theory depends on the
ability to be able find a closed inhabitant of the type 
$[[Forall X.X]]$.  We call a term closed if all of its variables are bound.
An inhabitant of a type $[[T]]$ is a term with type $[[T]]$.  Such a
term could be given the type $[[T1 -> T2]]$ and $[[T1]]$ which would
allow us to write a looping term.  However, it is impossible to define
a closed term of type $[[Forall X.X]]$.

%%% Local Variables: 
%%% mode: latex
%%% TeX-master: "paper"
%%% End:
