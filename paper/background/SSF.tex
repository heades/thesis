Russell called impredicativity vicious circularity and found it
appalling.  He actually took steps to remove it from his type theories
all together.  To remove impredicativity -- that is enforce
predicativity -- from his type theories he added a second level of
types which were used to organize the types of his theory.  This
organization made it impossible to instantiate a type with itself.
Predicative systems are less expressive than impredicative systems
\cite{Leivant:1991}.  This means that there are functions definable
in an impredicative theory which are not definable in its predicative
version.  In \cite{Leivant:1991,Danner:1999a} Daniel Leivant and
Norman Danner define and analyze a predicative version of
Reynolds-Girard's system F called Stratified System F (SSF).  They show
that SSF is substantially weaker than system F.  In fact we will
discuss the fact that SSF can be proven terminating by a much simpler
proof technique then system F suggesting that it is indeed weaker in
Section~\ref{sec:hereditary_substitution}.  The syntax and reduction
rules for SSF are defined in Fig.~\ref{fig:SSF_syntax}, kinding rules
in Fig.~\ref{fig:SSF_kinding}, and typing rules in
Fig.~\ref{fig:SSF_typing}.
\begin{figure}
  \begin{center}
    \begin{tabular}{lll}
      Syntax: 
      \vspace{10px} \\
      
      \begin{math}
        \begin{array}{lll}
              K & ::= & *_1\,|\,*_2\,|\,\ldots\\
          [[T]] & ::= & [[X]]\,|\,[[T -> T']]\,|\,[[Forall X:p.T]]\\
          [[t]] & ::= & [[x]]\,|\,[[\x:T.t]]\,|\,[[\\X:p.t]]\,|\,[[t1 t2]]\,|\,[[t [T] ]]
        \end{array}
      \end{math}\\
      \\
      Full $\beta$-reduction:\\
      \begin{mathpar}
          \SSFdruleRXXBeta{}     \and
          \SSFdruleRXXTypeRed{}  \and    
          \SSFdruleRXXLam{}      \and
          \SSFdruleRXXTypeAbs{}  \and    
          \SSFdruleRXXAppOne{}   \and
          \SSFdruleRXXAppTwo{}   \and
          \SSFdruleRXXTypeApp{}
      \end{mathpar}
    \end{tabular}
  \end{center}

  \caption{Syntax and reduction rules for SSF}
  \label{fig:SSF_syntax}
\end{figure}
\begin{figure}
  \begin{center}
    \begin{mathpar}
      \SSFdruleKXXVar{}    \and
      \SSFdruleKXXArrow{}  \and
      \SSFdruleKXXForall{} 
    \end{mathpar}
  \end{center}
  \caption{Kinding relation for the SSF}
  \label{fig:SSF_kinding}
\end{figure}
\begin{figure}
  \begin{center}
    \begin{mathpar}
        \SSFdruleVar{}     \and
        \SSFdruleLam{}     \and
        \SSFdruleApp{}     \and
        \SSFdruleTypeAbs{} \and
        \SSFdruleTypeApp{} 
    \end{mathpar}
  \end{center}
  \caption{Typing relation for the SSF}
  \label{fig:SSF_typing}
\end{figure}
The objective of SSF is to enforce the property of predicativity on
the types of system F.  To accomplish this Leivant took the same path
as Russell in that he added a second layer of typing to system F. This
second layer is known as the kind level.  Kinds are the types of
types.  The kinds of SSF are the elements of the syntactic category
$K$ in the syntax for SSF.  These are simply all the natural
numbers.  We call these type levels.  To stratify the types of system
F we use kinding rules to organize the types into levels making sure
that polymorphic types reside in a higher level than the types allowed
to instantiate these polymorphic types.  The kinding rules are pretty
straightforward. The one of interest is
\begin{center}
  \begin{math}
    \SSFdruleKXXForall{}.
  \end{math}
\end{center}
This is the rule which enforces predicativity. It does this by making
sure the level of $[[Forall X:p.T]]$ is at a larger level than
$[[X]]$.  This works, because all the types we instantiate this type
with must have the same level as $[[X]]$. We can easily see that
$[[p]] < max([[p]]+1,[[q]])$ for all $[[p]]$ and $[[q]]$.  Hence,
resulting in the enforcement of our desired property.

A understandable question one could ask at this point is, are
predicative theories expressive enough to capture advanced
mathematical reasoning, and real-world programming? Unfortunately
there is no correct answer at this time.  This is a debatable
question.  Some believe predicative systems are enough and that
impredicative systems are too paradoxical \cite{Feferman:2005}.  In
fact Hermann Weyl proposed a predicativist foundation of mathematics.
In his book \cite{Weyl:1918} he developed a predicative analysis using
stratification to enforce predicativity.  He goes on to show that a
substantial amount of mathematics can be done predictively.

I believe that impredicativity is not something that should be
abolished, but embraced.  It gives theories more expressive power in
an elegant way. This power comes at a cost that reasoning about
impredicative theories is more complex then predicative theories, but
this we think is to be expected.  However, we do believe that
impredicativity needs to be better understood.  At least in a
computational light.  
%%% Local Variables: 
%%% mode: latex
%%% TeX-master: "paper"
%%% End:
