\newcommand{\Fw}[0]{\text{F}^\omega} In Girard's thesis
\cite{Girard:1971} Girard extends the type language of system F with a
copy of STLC.  This type theory is called system $\Fw$.  The syntax
and reduction rules are in Fig.~\ref{fig:Fw_syntax}, the kinding rules
in Fig.~\ref{fig:Fw_kinding}, and the typing rules in
Fig.~\ref{fig:Fw_typing}.
\begin{figure}
  \index{System $\Fw$}
  \begin{center}
    \begin{tabular}{lll}
      Syntax: 
      \vspace{10px} \\
      \begin{math}
        \begin{array}{lll}
          [[K]] & ::= & [[Type]]\,|\,[[K -> K']]\\
          [[T]] & ::= & [[X]]\,|\,[[T -> T']]\,|\,[[Forall X:K.T]]\,|\,[[\X:K.T]]\,|\,[[T1 T2]]\\
          [[t]] & ::= & [[x]]\,|\,[[\x:T.t]]\,|\,[[\\X:K.t]]\,|\,[[t1 t2]]\,|\,[[t [T] ]]
        \end{array}
      \end{math} \\
      \\
      Full $\beta$-reduction:\\
      \begin{math}
        \begin{array}{cccc}
          \FwdruleRXXBeta{}       &       \FwdruleRXXTypeRed{}      \\
          & \\
          \FwdruleRXXLam{}          &    \FwdruleRXXTypeAbs{}      \\
          & \\
          \FwdruleRXXAppOne{}       &   \FwdruleRXXAppTwo{}       \\  
          & \\
          \FwdruleRXXTypeApp{}      &     \FwdruleTRXXTypeBeta{}    \\
          & \\
          \FwdruleTRXXTypeLam{}     &    \FwdruleTRXXTypeAppOne{}  \\
          & \\
          \FwdruleTRXXTypeAppTwo{}
        \end{array}
      \end{math}\\
      \\
      Type $\beta$-equality:\\
      \begin{math}
        \begin{array}{cccc}
          \FwdruleEqXXRefl{}       &
          \FwdruleEqXXSym{}      \\
          & \\
          \FwdruleEqXXTrans{}          &
          \FwdruleEqXXLam{}      \\
          & \\
          \FwdruleEqXXApp{}       &
          \FwdruleEqXXImp{}       \\
          & \\
          \FwdruleEqXXForall{}      &
          \FwdruleEqXXBeta{}    
        \end{array}
      \end{math}
    \end{tabular}
  \end{center}

  \caption{Syntax and reduction rules for system $\Fw$}
  \label{fig:Fw_syntax}
\end{figure}
\begin{figure}
  \begin{center}
    \begin{mathpar}
      \FwdruleKXXVar{}     \and
      \FwdruleKXXArrow{}   \and
      \FwdruleKXXForall{}  \and
      \FwdruleKXXLam{}     \and
      \FwdruleKXXApp{}     
    \end{mathpar}
  \end{center}
  \caption{Kinding rules of system $\Fw$}
  \label{fig:Fw_kinding}
\end{figure}
\begin{figure}
  \begin{center}
    \begin{mathpar}
      \FwdruleVar{}       \and
      \FwdruleLam{}       \and
      \FwdruleTypeAbs{}   \and
      \FwdruleApp{}       \and
      \FwdruleTypeApp{}   \and
      \FwdruleConv{}
    \end{mathpar}
  \end{center}
  \caption{Typing relation for the system $\Fw$}
  \label{fig:Fw_typing}
\end{figure}
There are two kinds denoted $[[Type]]$ and $[[K1 -> K2]]$.  The
formers inhabitants are well-formed types, while the latter's
inhabitants are type level functions whose inputs are types and
outputs are types.  There are only three forms of well-formed types:
variables, arrow types, and $\forall$-types.  The additional members
of the syntactic category for types are used to compute types.  These
are $\lambda$-abstractions denoted $[[\X:K.T]]$ and applications
denoted $[[T1 T2]]$.  Note that in general these are not types.  They
are type constructors.  However, applications may be considered a type
when $[[T1 T2]]$ has type $[[Type]]$, but this is not always the case,
because STLC allows for partial applications of functions.

The ability to compute types is known as type
computation\index{Type-level Computation}.  Type-level
computation adds a lot of power.  It can be used to write generic
function specifications.  We mentioned above that system F allows one
to write functions with more generic types which allows one to define
term level functions once and for all.  Type level computation
increases this ability.  In fact module systems can be encoded in
system $\Fw$ \cite{Shan:2006}.  There is one drawback though.  Since
terms are disjoint from types we obtain a lot of duplication.  For
example, we need two copies of the natural numbers: one at the type
level and one at the term level.  This is unfortunate.  A fix for this
problem is to unite the term and type level allowing for types to
depend on terms.  This is called dependent type theory and is the
subject of Section~\ref{chap:dependent_type_theory}.  Using dependent
types and type-level computation we could amongst other things define
and use only a single copy of the natural numbers.

Logically, through the computational trinity (see
Section~\ref{chap:the_three_perspectives}) system $\Fw$ corresponds to
higher-order logic, because we are able to define predicates of higher
type.  This is quite a large logical leap forward from System F which
corresponds to second order predicate logic.

Throughout this section we took a brief journey into modern type
theory.  We defined each of the most well-known type theories that
are at the heart of the vast majority of existing research in type
theory and foundations of functional programming languages.  This was
by no means a complete history, but whose aim was to give the reader a
nice introduction to the field.
%%% Local Variables: 
%%% mode: latex
%%% TeX-master: "paper"
%%% End:
