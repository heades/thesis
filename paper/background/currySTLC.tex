The Curry-style simply typed $\lambda$-calculus is exactly
Church-style simply type $\lambda$-calculus except there is no type
annotations on $\lambda$-abstractions.  That is we have $[[\x.t]]$
instead of $\lambda [[x]]:[[T]].[[t]]$ in the syntax for terms.  This
definition of STLC was an extension of Curry's work on combinator
logic. 

Now a large number of type theories can be either in Church-style or
in Curry-style.  The lack of typing annotations has a syntactic
benefit.  It prevents the programmer from having to fill in type
annotations when defining functions.  This can be very beneficial when
defining complicated functions.  Curry-style type theories also differ
semantically.  Church-style type theories contain annotations to
enforce the assignment of exactly one type to any given term.  Now
Curry-style type theories do not contain annotations, thus any given
term may have many different types.  Consider the identity function
$[[\x.x]]$ this function may have the type $\mathsf{Nat} \to
\mathsf{Nat}$, but it can also be given the type $\mathsf{Bool} \to
\mathsf{Bool}$. In fact, there are infinitely main types one can give
the Curry-style identity function.  We can also characterize this
semantic difference by what John Reynolds called intrinsic and
extrinsic meanings.  Church-style theories give an intrinsic meaning
to terms.  This means that typeable terms are the only terms assigned
a meaning.  Thus, the identity function $\lambda [[x]]:[[T]].[[x]]$
always has a meaning, because we can give it the type $[[T -> T]]$,
but the function $\lambda x:[[T]].[[x x]]$ has no meaning, because no
matter how hard we try we can never give the correct type annotation
$[[T]]$.  Now Curry-style theories give an extrinsic meaning to terms
which amounts to the same meaning we give un-type (or uni-typed) type
theories.  The identity function $[[\x.x]]$ can be assigned the
meaning that it is the identity function on the entire domain of
values, not just the typeable ones.  Note that we can give a
Curry-style type theory both an extrinsic and an intrinsic semantics,
but Church-style is always intrinsic \cite{Reynolds:2003}.  A last
remark is that type annotations can actually make giving an intrinsic
meaning difficult conducting the meta-theory of various expressive
type theories and programming languages, and thus removing
annotations, but still maintaining an intrinsic semantics may make
meta-theoretic reasoning less difficult.  This is the benefit of using
a type-annotation eraser function\footnote{This is often called ``type
  erasure'', but the erasure is the image of the type-annotation
  eraser function which is the result of applying the eraser, and
  hence not the process of erasing.} to translate a Church-style type
theory into a Curry-style type theory with an intrinsic semantics.
This has been very beneficial in the study of dependent type theories.

The syntax and reduction relation of the Curry-style STLC is defined
in Fig.~\ref{fig:cstlc_syntax} and the typing judgment is defined in
Fig.~\ref{fig:cstlc_typing}. We call the typing judgment defined here
an implicit typing paradigm.  The fact that it is implicit shows up in
the application typing rule $\CSTLCdrulename{App}$:
\begin{center}
  \begin{math}
    \CSTLCdruleApp{}
  \end{math}
\end{center}
Recall that these rules are read bottom up.  Until now we have
considered the typing judgment as simply a checking procedure with the
type as one of the inputs, but often this judgment is defined so that
the type is computed and becomes an output.  In theories like this the
above rule causes some trouble.  The type $[[T1]]$ is left implicit
that is by looking at only the conclusion of the rule one cannot tell
what the value of $[[T1]]$ must be.  This problem also exists for the
typing rule for $\lambda$-abstractions.  This is, however, not a
problem in Church style STLC because that type is annotated on
functions.  This suggest that for some Curry style type theories type
construction is undecidable.  Not all type theories have a Church
style and a Curry style formulations.  Thierry Coquand's Calculus of
Constructions is an example of a type theory that is in the style of
Church, but it is also unclear how to define a Curry style version.
It is also unclear how to define a Church style version of the type
theory of intersection types \cite{Barendregt:1992}.

\begin{figure*}
  \begin{center}
    \begin{tabular}{lll}
      Syntax: 
      \vspace{10px} \\
      \begin{math}
        \begin{array}{lll}
          [[T]] & ::= & [[X]]\,|\,[[T -> T']]\\
          [[t]] & ::= & [[x]]\,|\,[[\x.t]]\,|\,[[t1 t2]]
        \end{array}
      \end{math}\\
      & \\
      Full $\beta$-reduction: \\
      \begin{mathpar}        
        \CSTLCdruleRXXBeta{}    \and
        \CSTLCdruleRXXLam{}     \and  
        \CSTLCdruleRXXAppOne{}  \and
        \CSTLCdruleRXXAppTwo{}
      \end{mathpar}
    \end{tabular}
  \end{center}

  \caption{Syntax and reduction rules for the Curry-style simply-typed $\lambda$-calculus}
  \label{fig:cstlc_syntax}
\end{figure*}

\begin{figure}
  \begin{center}
    \begin{mathpar}      
      \CSTLCdruleVar{} \and
      \CSTLCdruleLam{} \and
      \CSTLCdruleApp{}
    \end{mathpar}
  \end{center}
  \caption{Typing relation for the Curry-style simply typed $\lambda$-calculus}
  \label{fig:cstlc_typing}
\end{figure}

%%% Local Variables: 
%%% mode: latex
%%% TeX-master: "paper"
%%% End:
