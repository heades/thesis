An entire class of type theories called Pure Type Systems\index{Pure
  Type System} may be
expressed by a very simple core type theory, a set of type
universes\index{Type Universe} called sorts, a set of axioms, and a set of rules.  The rules specify
how the sorts are to be used, and govern what dependencies are allowed
in the type theory.  There is a special class of eight pure type
systems with only two sorts called $\Box$ and $*$\footnote{ It is also
standard to called these $\mathsf{Type}$ and $Prop$ respectively.
$\mathsf{Type}$ is the same as we have seen above and $\mathsf{Prop}$
classifies logical propositions.} called the $\lambda$-cube \cite{Barendregt:1992}.  The
following expresses the language of this class of types theories.
\begin{definition}
  \label{def:pst_syntax}
  The language of the $\lambda$-cube:
  \begin{center}
    \begin{math}
      \begin{array}{lll}
        [[t]],[[a]],[[b]] & ::= & 
            \Box\,|\,*\,|\,c\,|\,[[x]]\,|\,[[t1 t2]]\,|\,[[\x:t1.t2]]\,|\,[[(x:t1)->t2]]\\
      \end{array}
    \end{math}
  \end{center}
\end{definition}
\noindent
Notice in the previous definition that terms and types are members of
the same language. They are not separated into two syntactic
categories.  This is one of the beauties of pure type systems.  They
have a really clean syntax, but this beauty comes with a cost.  Some
collapsed type theories are very hard to reason about.

A pure type system is defined as a triple $(S, A, R)$, where $S$ is a
set of sorts and is a subset of the constants of the language, $A$
is a set of axioms, and $R$ is a set of rules.  In the $\lambda$-cube
$\{*, \Box\} \subseteq S$, $A = \{(*, \Box)\}$, and $R$ varies depending on
the system. The axioms stipulate which sorts the constants of the
language have.  In the $\lambda$-cube\index{$\lambda$-Cube} there are at least two constants
$\Box$ and $*$.  The set of rules are subsets of the set 
$\{(*,*),(*,\Box),(\Box,*),(\Box,\Box)\}$. These rules represent four forms of dependencies:
\begin{center}
  \begin{tabular}{lll}
    i.   & terms depend on terms: & $(*,*)$\\
    ii.  & terms depend on types: & $(\Box,*)$\\
    iii. & types depend on terms: & $(*,\Box)$\\
    iv.  & types depend on types: & $(\Box,\Box)$
  \end{tabular}  
\end{center}
In the $\lambda$-cube terms always depend on terms, hence $(*,*) \in R$ 
for any system.  For example, $[[(\x:t.a) b]]$ is a term depending
on a term and $\lambda [[x]]:[[Type]].[[b]]$ where $[[b]]$ is a type
is a type depending on a type.  An example of a term depending on a
type is the $\Lambda$-abstraction of system F.  Finally, an example of
a type depending on a term is the product type of Martin-L\"of's Type
Theory\index{Dependent Type}.  Now using the notion of dependency we define the core set of
inference rules in the next definition.
\begin{definition}
  \label{def:pst_rules}
  Given a system of the $\lambda$-cube $(S, A, R)$ the inference rules
  are defined as follows:
  \vspace{-20px}
  \begin{center}
    \begin{mathpar}

      \inferrule* [right=\ifrName{Beta}] {
        \ 
      }{[[(\x:t.b) a]] \redto [[ [a/x]b]]}

      \and

        \inferrule* [right=\ifrName{Axioms}] {
          (c,s) \in A
        }{[[.]] \vdash c : s}
        
        \and

        \inferrule* [right=\ifrName{Var}] {
          [[G]] \vdash [[a]] : s          
        }{[[G, x: a |- x : a]]}
        
        \and
        
        \inferrule* [right=\ifrName{Weakening}] {
          [[G |- a : b]]
          \\\\
          [[G]] \vdash [[a']] : s
          \\
          [[(x:a') \in G]]
        }{[[G,x:a' |- a:b]]}
        
        \and
                
        \inferrule* [right=\ifrName{App}] {
          [[G |- a' : b]]
          \\\\
          [[G |- a : (x : b) -> b']]
        }{[[G |- a a' : [a'/x]b']]}
        
      \end{mathpar}
      \begin{math}
        \begin{array}{cccc}
          \inferrule* [right=\ifrName{Conv}] {
          [[G |- a : b]]
          \\\\
          [[G]] \vdash [[b']]:s
          \\
          [[b]] \equiv_\beta [[b']]
        }{[[G |- a : b']]}
        &
        \inferrule* [right=\ifrName{Pi}] {                 
          [[G]] \vdash [[a]]:s_1
          \\\\
          [[G,x:a]] \vdash [[b]]:s_2
          \\
          (s_1, s_2) \in R
        }{[[G]] \vdash [[(x:a) -> b]]:s_2}\\
        & \\
        \inferrule* [right=\ifrName{Lam}] {
          [[G,x:a |- b : b']]
          \\
          [[G]] \vdash [[(x:a) -> b']] : s
        }{[[G |- \x:a.b:(x : a) -> b']]}
        \end{array}
      \end{math}      
  \end{center}
\end{definition}
In the previous definition $s$ ranges over $S$, and we left out the
congruence rules for reduction to make the definition more compact.
However, either they need to be added or evaluation contexts do, for a
full treatment of reduction.  It turns out that this is all we need to
define every intuitionistic type theory we have defined in this
part of the thesis including a few others we have not defined. However,
Martin-L\"of's Type Theory is not definable as a pure type system.
Taking the set $R$ to be $\{ (*,*) \}$ results in STLC.  System F
results from taking the set $R = \{ (*,*),(\Box,*) \}$. System $\Fw$
is definable by the set $R = \{(*,*),(\Box,*),(\Box,\Box)\}$.

A good question now to ask is what type theory results from adding all
possible rules to $R$?  That is what type theory is defined by $R = \{(*,*), (*, \Box), (\Box,*), (\Box,\Box)\}$?  This type theory is
clearly a dependent type theory and is called the Calculus of
Constructions (CoC).  It was first defined by Thierry Coquand in
\cite{Coquand:1988}.  It is the most powerful of all the eight pure type
system in the $\lambda$-cube.  We have seen one formulation of CoC as
a pure type system, but we give one more.  

%%% Local Variables: 
%%% mode: latex
%%% TeX-master: ../thesis.tex
%%% End: 
