There are three formulations of the simply typed $\lambda$-calculus.
The first one is called Church style
\cite{Girard:1989,Barendregt:1992,Church:1940}, the second is called
Curry style \cite{Barendregt:1992,Reynolds:1998}, and the third is in
the form of a pure type system.  We introduce pure type systems in
Section~\ref{sec:the_calculus_of_constructions}.  We define the first
and the second formulations here beginning with the first.  We will
first define the type theories and then we will comment on the
differences between the two theories.  The first step in defining a
type theory is to define its language or syntax.  Following the syntax
are several judgments assigning some meaning to the language.  A
judgment is a statement about the object language derived from a set
of inference rules.  In the following type theories we will derive two
judgments: the reduction relation and the type-assignment relation.
The syntax and reduction relation of the Church-style simply typed
$\lambda$-calculus (STLC) is defined in Figure~\ref{fig:chstlc_syntax}
where $[[t]]$ ranges over syntactic expressions called terms and
$[[T]]$ ranges over syntactic expressions called types. Terms consist
of variables $[[x]]$, unary functions $[[\x:T.t]]$ (called
$\lambda$-abstractions) where $[[x]]$ is bound in $[[t]]$, and
function application denoted $[[t1 t2]]$.  Now types are variables
$[[X]]$ (we use variables as base types or constants just to indicate
that we may have any number of constants), and function types denoted
$[[T1 -> T2]]$ where we call $[[T1]]$ the domain type and $[[T2]]$ the
range type.  Note that if we remove the syntax for types from
Figure~\ref{fig:chstlc_syntax} then we would obtain the (untyped)
$\lambda$-calculus.  The syntax defines what language is associated
with the type theory.  Additionally, the reduction rules describe how
to compute with the terms.  The $\CHSTLCdrulename{Beta}$ rule says
that if a $\lambda$-abstraction $[[\x:T.t]]$ is applied to some term
$[[t']]$, then that term may be reduced to the term resulting from
substituting $[[t']]$ for $[[x]]$ in $[[t]]$ which is the English
interpretation for $[[ [t'/x]t]]$. We call $[[ [t'/x] t]]$ the capture
avoiding substitution function. It is a meta-level function. That is,
it is not part of the object language. In STLC the types and terms are
disjoint, but in type theories the types are used to enforce
particular properties on the terms.  To enforce these properties we
need a method for assigning types to terms.  This is the job of what
we will call the typing judgment, type-checking judgment, or
type-assignment judgment\footnote{Throughout the literature one may
  find the typing judgment being called the typing algorithm,
  type-checking algorithm, or type-assignment algorithm.  However,
  this is a particular case where the rules deriving the typing
  judgment are algorithmic in the sense that when deriving conclusions
  from the inference rules deriving the judgment there is always a
  deterministic choice on how to proceed.}.  A judgment is a statement
about the object language derived from a set of inference rules.  The
typing judgment for STLC is defined in Figure~\ref{fig:chstlc_typing}.
The typing judgment depends on a typing context $[[G]]$ which for now
can be considered as a list of ordered pairs consisting of a variable
and a type.  This list is used to keep track of the types of the free
variables in a term.  The grammar for context is as follows:
\begin{center}
  \begin{math}
    \begin{array}{lll}
      [[G]] & ::= & [[.]]\,|\,[[x:T]]\,|\,[[G1,G2]]
    \end{array}
  \end{math}
\end{center}
Here the empty context is denoted $[[.]]$ and context extension is denoted $[[G1,G2]]$.

\begin{figure}
  \begin{center}
    \begin{tabular}{lll}
      Syntax: 
      \vspace{10px} \\
      \begin{math}
        \begin{array}{lll}
          [[T]] & ::= & [[X]]\,|\,[[T -> T']]\\
          [[t]] & ::= & [[x]]\,|\,[[\x:T.t]]\,|\,[[t1 t2]]
        \end{array}
      \end{math}\\
      & \\
      Full $\beta$-reduction: \\
      \begin{mathpar}
        \CHSTLCdruleRXXBeta{}   \and
        \CHSTLCdruleRXXLam{}    \and
        \CHSTLCdruleRXXAppOne{} \and
        \CHSTLCdruleRXXAppTwo{}
      \end{mathpar}
    \end{tabular}
  \end{center}

  \caption{Syntax and reduction rules for the Church-style simply-typed $\lambda$-calculus}
  \label{fig:chstlc_syntax}
\end{figure}

\begin{figure*}
  \begin{mathpar}
    \CHSTLCdruleVar{} \and
    \CHSTLCdruleLam{} \and
    \CHSTLCdruleApp{}
  \end{mathpar}
  
  \caption{Typing Relation for the Church-style simply typed $\lambda$-calculus}
  \label{fig:chstlc_typing}
\end{figure*}

The inference rules deriving the typing judgment are used to determine
if a term has a particular type.  That is the term
$[[t]]$ has type $[[T]]$ in context $[[G]]$ if there is a derivation
with conclusion $[[G |- t : T]]$ and beginning with axioms.
Derivations are constructed in a goal directed fashion.  We first
match our desired conclusion with a rule that matches its pattern and
then derives its premises bottom up.  To illustrate this consider the
following example.

\begin{example}
  \label{ex:chstlc_derivation}
  Suppose $[[G == x:T1 -> T2,y:T1]]$.  Then we apply each rule starting with its conclusion:
  \begin{center}
    \begin{math}
      $$\mprset{flushleft}
      \inferrule* [right=$\CHSTLCdrulename{Lam}$] {
        $$\mprset{flushleft}
        \inferrule* [right=$\CHSTLCdrulename{Lam}$] {
          $$\mprset{flushleft}
          \inferrule* [right=$\CHSTLCdrulename{App}$] {
            $$\mprset{flushleft}
            \inferrule* [right=$\CHSTLCdrulename{Var}$] {
              \ 
            }{[[G |- x : T1 -> T2]]}
            \\
            $$\mprset{flushleft}
            \inferrule* [right=$\CHSTLCdrulename{Var}$] {
              \ 
            }{[[G |- y : T1]]}
          }{[[x:T1 -> T2,y:T1 |- x y : T2]]}
        }{[[x:T1 -> T2 |- \y:T1.(x y):T1 -> T2]]}
      }{[[. |- \x:T1->T2.\y:T1.(x y) : ((T1 -> T2) -> T1) -> T2]]}
    \end{math}
  \end{center}
\end{example}

%%% Local Variables: 
%%% mode: latex
%%% TeX-master: "paper"
%%% End:
