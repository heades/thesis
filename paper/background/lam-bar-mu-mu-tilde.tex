Pierre-Louis Curien and Hugo Herbelin put these dualities to work in a
very computational way.  They used these dualities to show that the
call-by-value reduction strategy (CBV) is dual to the call-by-name
reduction strategy (CBN).  To do this they crafted an extension of the
$\lambda\mu$-calculus formalized in such a way that the symmetries are
explicit \cite{Curien:2000}.  They are not the first to attempt this.
Andrzej Filinski to our knowledge was the first to investigate
dualities with respect to programming languages in his masters thesis
\cite{Filinski:1989}.  It is there he investigates the dualities in a
categorical setting.  Advancing on this early work Peter Selinger gave
a categorical semantics to the $\lambda\mu$-calculus and then used
these semantics to show that CBV is dual to CBN \cite{Selinger:2001}.
However, Selinger's work did not provide an involution of duality.  In
\cite{Selinger:2001} Selinger defines a new class of categories called
control categories.  These provide a model for control operators.  He
takes the usual cartesian closed category and enriches it with a new
functor modeling classical disjunction.  While this is beautiful work
we do not go into the details here.

Following Filinski and Selinger are the work of Curien and Herbelin,
while following them is the work of Philip Wadler.  Below we discuss
Curien and Herbelin's work then Wadler's.  Before going into their work
we first define call-by-value and call-by-name reduction.

The call-by-value reduction strategy\index{Call-by-value Reduction (CBV)} is a restriction of full
$\beta$-reduction.  It is defined for the $\lambda\mu$-calculus
as follows.  We first extend the language of the $\lambda\mu$-calculus
by defining two new syntactic categories called values
and evaluation contexts.
\begin{center}
  \begin{math}
    \begin{array}{lll}
      v ::= x\,|\,\lambda x.t\,|\, \mu \alpha.t\\
      E ::= \Box\,|\,E\,t\,|\,v\,E\,|\,[\alpha]E\\
    \end{array}
  \end{math}
\end{center}
Values are the well-formed results of computations.  In the
$\lambda\mu$-calculus we only consider variables,
$\lambda$-abstractions, and $\mu$-abstractions as values.  The
evaluation contexts are defined by $E$.  They give the locations of
reduction and reduction order.  They tell us that one may reduce the
head of an application at any moment, but only reduce the tail of an
application if and only if the head has been reduced to a value.  This
is called left-to-right CBV and is defined next.
\begin{definition}
  \label{def:lamu_cbv}
  CBV is defined by the following rules:
  \vspace{-30px}
  \begin{center}
    \begin{mathpar}      
      \inferrule* [right=\ifrName{Beta}] {
        \ 
      }{(\lambda x.t)\,v \redto [v/x]t} 
      \and
      \inferrule* [right=\ifrName{Struct}] {
        \ 
      }{(\mu \alpha.s)\,v \redto [v/^* \alpha]s}
      \and      
      \inferrule* [right=\ifrName{Naming}] {
        \ 
      }{[\alpha](\mu \beta.s) \redto [\alpha/\beta]s}
      \and
      \inferrule* [right=\ifrName{Context}] {
        t \redto t'
      }{E[t] \redto E[t']}
    \end{mathpar}
  \end{center}
\end{definition}
\noindent
A similar definition can be given for right-to-left CBV, but we do not
give it here.  

CBN can now be defined.  We use the same definition of values as for CBV,
but we redefine the evaluation contexts.
\begin{center}
  \begin{math}
    \begin{array}{lll}
      E ::= \Box\,|\,E\,t\,|\,[\alpha]E\,|\,\mu \alpha.E\\
    \end{array}
  \end{math}
\end{center}
\begin{definition}
  \label{def:lamu_cbn}
  CBN is defined by the following rules:
  \vspace{-37px}
  \begin{center}
    \begin{mathpar}
        \inferrule* [right=\ifrName{Beta}] {
          \ 
        }{(\lambda x.t)\,t' \redto [t/x]t}
        \and
        \inferrule* [right=\ifrName{Struct}] {
          \ 
        }{(\mu \alpha.s)\,t \redto [t/^* \alpha]s}
        \and
        \inferrule* [right=\ifrName{Naming}] {
          \ 
        }{[\alpha](\mu \beta.s) \redto [\alpha/\beta]s}
        \and
        \inferrule* [right=\ifrName{Context}] {
          t \redto t'
        }{E[t] \redto E[t']}
    \end{mathpar}
  \end{center}
\end{definition}
\noindent
The difference between CBN and CBV is that in CBN no reduction takes
place within the argument to a function\index{Call-by-name Reduction (CBN)}.  Instead we wait and reduce
the argument if it is needed within a function.  If the argument is
never used it is never reduced.  CBN in general is less efficient than
CBV, but it can terminate more often than CBV.  If the argument to a
function is divergent then CBV will never terminate, because it must
reduce the argument to a value, but CBN may terminate if the argument
is never used, because arguments are not reduced.

At this point we would like to give some intuition of why CBV is dual
to CBN.  We reformulate an explanation due to Curien and Herbelin in
\cite{Curien:2000}.  To understand the relationship between CBN and
CBV we encode CBV on top of CBN using a new term construct and
reduction rule.  It is well-known how to encode CBN on top of CBV, but
encoding CBV on top of CBN illustrates their relationship between each
other.  Suppose we extend the language of the CBN
$\lambda\mu$-calculus with the following term:
\begin{center}
  \begin{math}
    t ::= \ldots\,|\,\textsf{let}\,x = E\,\textsf{in}\,t
  \end{math}
\end{center}
This extends the language to allow for terms to contain their evaluation
contexts.  Then we add the following reduction rule:
\begin{center}
  \begin{math}
    $$\mprset{flushleft}
    \inferrule* [right=\ifrName{LetCtx}] {
      \ 
    }{v\,(\textsf{let}\,x = \Box\,\textsf{in}\,t) \redto [v/x]t}
  \end{math}
\end{center}
Using this new term and reduction rule we can now encode CBV on top
of CBN.  That is a CBV redex is defined in the following way:
\begin{center}
  \begin{math}
    (\lambda x.t)_{CBV}\,t' := t'\,(\textsf{let}\,y = \Box\,\textsf{in}\,(\lambda x.t) y)
  \end{math}
\end{center}
Now consider the following redex:
\begin{center}
  \begin{math}
    (\mu x.s)\,(\textsf{let}\,x = \Box\,\textsf{in}\,t)
  \end{math}
\end{center}
We can reduce the previous term by first reducing the $\mu$-redex, but
we can also start by reducing the let-redex, because
$\mu$-abstractions are values. However, the two reducts obtained from
doing these reductions are not always joinable.  This forms a critical
pair and shows an overlap between the $\LBMMTdrulename{\normalsize LetCtx}$ rule
and the $\mu$-reduction rule.  This can be overcome by giving priority
to one or the other redex.  Now if we give priority to $\mu$-redexes
over all other redexes then it turns out that the reduction strategy
will be all CBN, but if we choose to give priority to the let-redexes
over all other redexes then all terms containing let-redexes will be
reduced using CBV, because the let-expression forces the term we are
binding to $x$ to be a value.  What does this have to do with duality?
Well the let-expression we added to the $\lambda\mu$-calculus is
actually the dual to the $\mu$-abstraction.  To paraphrase Curien and
Herbelin \cite{Curien:2000}:
\begin{quote}
  \emph{The CBV discipline manipulates input in the same way as the
  $\lambda\mu$-calculus manipulates output.  That is computing
  $t_1\,t_2$ can be viewed as filling the hole of the context
  $t_1\,\Box$ with the result of $t_2$ -- its value -- hence this
  value of $t_2$ is an input.  This seems dual to passing output
  values to output ports in the $\lambda\mu$-calculus.}
\end{quote}
This tells us that to switch from CBN to CBV we take the dual of the
$\mu$-abstraction suggesting that CBV is dual to CBN and vice versa.
This was the starting point of Curien and Herbelin's work.  They make
this relationship more precise by defining an extension of the
$\lambda\mu$-calculus with duals of $\lambda$-abstractions and
$\mu$-abstractions.  This requires the dual to implication.  Let's
define Curien and Herbelin's extension of the $\lambda\mu$-calculus
and then discuss how they used it to show that CBV is dual to CBN.
Curien and Herbelin called their extension of the $\lambda\mu$-calculus
the $\LBMMT$-calculus.  Despite the ugly name it is a beautiful type
theory.  Its syntax and reduction rules are in
Figure~\ref{fig:lbmmt_syntax} and its typing rules are in Figure~\ref{fig:lbmmt_typing}.
\begin{figure}
  \index{$\LBMMT$-Calculus}
  \begin{center}
    \begin{tabular}{lll}
      Syntax:
      \vspace{10px} \\
      \begin{math}
        \begin{array}{rll}
          [[T]],[[A]],[[B]],[[C]] & ::= & [[_|_]]\,|\,[[X]]\,|\,[[A -> B]]\,|\,[[A - B]]\\
          [[c]] & ::= &  [[< v | e >]]\\
          [[v]] & ::= &  [[x]]\,|\,[[\ x . v]]\,|\,[[\m a . c]]\,|\,[[e * v]]\\
          [[e]] & ::= &  [[a]]\,|\,[[\mt x . c]]\,|\,[[v * e]]\,|\,[[b \ . e]]\\
        \end{array}
      \end{math}\\
      & \\
      CBV reduction: \\
      \begin{mathpar}
        \LBMMTdruleRXXBeta{}   \and
        \LBMMTdruleRXXMu{}     \and
        \LBMMTdruleRXXMuT{}    \and
        \LBMMTdruleRXXCoBeta{} \and
        \LBMMTdruleEXXCtx{}      
      \end{mathpar}
    \end{tabular}
  \end{center}

  \caption{The Syntax and Reduction Rules for the $\LBMMT$-Calculus}
  \label{fig:lbmmt_syntax}
\end{figure}

\begin{figure}
  \begin{center}
    \begin{tabular}{lll}           
      \ \ \ \ \ \ \ Terms:\\      
        \begin{mathpar}
        \begin{array}{lll}
          \LBMMTdruleVar{} &
          \LBMMTdruleLam{} \\
          & \\
          \LBMMTdruleMu{}  &
          \LBMMTdruleCoCtx{}
        \end{array}
      \end{mathpar}
        \\
      &\\
      \ \ \ \ \ \ \ Contexts:\\
      \begin{mathpar}
            \LBMMTdruleCovar{} \and
            \LBMMTdruleComu{}  \and
            \LBMMTdruleCtx{}   \and
            \LBMMTdruleColam{}
        \end{mathpar}\\
        & \\
      \ \ \ \ \ \ \ Commands:\\
      \begin{mathpar}
        \LBMMTdruleCut{}
      \end{mathpar}\\     
    \end{tabular}
  \end{center}
  \caption{The Typing Rules for the $\LBMMT$-Calculus}
  \label{fig:lbmmt_typing}
\end{figure}

The new type $[[A - B]]$ is the dual to implication called
subtraction\index{Subtraction}.  It is logically equivalent to $[[A]] \land [[{-B}]]$
which is the dual to $[[{-A}]] \lor [[B]]$ which is logically
equivalent to $[[A -> B]]$.  The syntactic category $[[c]]$ are called
commands.  They have the form of $[[< v | e >]]$ where $[[v]]$ is a
computation and $[[e]]$ is its environment.  Commands essentially
encode an abstract stack machine directly in the type theory.  We can
think of $[[e]]$ as the stack of terms to which $[[v]]$ will be applied to. It
also turns out that logically commands denote cuts using the cut-rule
of the underlying sequent calculus.  Values defined by the syntactic
category $[[v]]$ come in three flavors: variables,
$\lambda$-abstractions, $\mu$-abstractions, and co-contexts denoted by
$[[e * v]]$.  These can be thought of as the computations to give to
the co-$\lambda$-abstraction and their output routed to an output port
bound by the co-$\lambda$-abstraction.  Finally, we have expressions
or co-terms which come in four flavors: co-variables (output ports),
$\tilde\mu$-abstractions, contexts, and co-$\lambda$-abstractions.
The $\tilde\mu$-abstraction is the encoding of the let-expression we
defined above.  We write $\textsf{let}\,x = \Box\,\textsf{in}\,[[v]]$
as $[[\mt x.<v|e>]]$ where $[[e]]$ is the evaluation context for
$[[v]]$.  Thus, the $\tilde\mu$-abstraction is the dual to the
$\mu$-abstraction.  Now contexts are commands.  These provide a way of feeding
input to programs.  Co-$\lambda$-abstractions denoted $[[b\.e]]$ are
the dual to $\lambda$-abstractions.  In stead of taking input
arguments they return outputs assigned to the output port bound by the
abstraction.  We can see that this is a rather large
reformulation/extension of the $\lambda\mu$-calculus.  Just to summarize:
Curien and Herbelin extended the $\lambda\mu$-calculus with all the
duals of the constructs of the $\lambda\mu$-calculus.

Now reduction amounts to cuts logically, and computationally as running
these abstract machine states we are building.  Programming and proving
amounts to the construction of these abstract machines.  Other then
this the reduction rules are straightforward. The typing algorithm
consists of three types of judgments:
\begin{center}
  \begin{math}
    \begin{array}{lll}
      \text{Commands:} & [[c : (G |- D)]]\\
      \text{Terms:} & [[G |- v : A | D]]\\
      \text{Contexts:} & [[G | e : A |- D]]\\
    \end{array}
  \end{math}
\end{center}
As we said early the command typing rule is cut while the judgment for
terms and contexts consist of the left rules and the right rules
respectively.  The bar $|$ separates input from output or left from
right.  Finally, using this type theory Curien and Herbelin define a
duality of the $\LBMMT$-calculus into itself.  Then using this duality
they show that starting with the CBN $\LBMMT$-calculus and taking the
dual one obtains the CBV $\LBMMT$-calculus.

%%% Local Variables: 
%%% mode: latex
%%% TeX-master: "paper"
%%% End:
