The first step in proving strong normalization of STLC using Tait's
method is to define the interpretation of types\index{Interpretation
  of Types}.  An interpretation of
a type $[[T]]$ is a set of closed terms closed under eliminations.  We
denote the set of strongly normalizing terms as $\mathsf{SN}$.
Defining the interpretation of types depends on an extension by Girard
which constrains the number of lemmas down to a minimal amount.  A
term is \emph{neutral}\index{Neutral Term} if it is of the form $[[t1 t2]]$ for some terms
$[[t1]]$ and $[[t2]]$.  Neutrality characterizes the terms which
cannot be easily seen to be normalizing.
\begin{definition}
  \label{def:stlc_interpretation_taits}
  The interpretation of types are defined as follows:
  \begin{center}
    \begin{math}
      \begin{array}{lll}
        \interp{[[X]]} & = & \{[[t]]\,|\, [[t]] \in \mathsf{SN} \}\\
        \interp{[[T1 -> T2]]} & = & \{ t\,|\,\forall t' \in \interp{[[T1]]}.
               [[t t']] \in \interp{[[T2]]} \}      
      \end{array}
    \end{math}
  \end{center}
\end{definition}
\noindent
The interpretation of types are known as reducibility sets.  We say a
term is reducible if it is a member of one of these sets.  Next we
have some constraints the previous definition must satisfy.  Girard
called these the \textbf{CR 1-4} properties.  Their proofs can be
found in \cite{Girard:1989}.
\begin{lemma}
  \label{lemma:stlc_sn_cr1}
  If $[[t]] \in \interp{[[T]]}$, then $[[t]] \in \mathsf{SN}$.
\end{lemma}

\begin{lemma}
  \label{lemma:stlc_sn_cr2}
  If $[[t]] \in \interp{[[T]]}$ and $[[t ~> t']]$ then $[[t']] \in \interp{[[T]]}$.
\end{lemma}

\begin{lemma}
  \label{lemma:stlc_sn_cr3}
  If $[[t]]$ is neutral, $[[t']] \in \interp{[[T]]}$ and $[[t ~> t']]$ then 
  $[[t]] \in \interp{[[T]]}$.
\end{lemma}

\begin{lemma}
  \label{lemma:stlc_sn_cr4}
  If $[[t]]$ is neutral and normal then $[[t]] \in \interp{[[T]]}$.
\end{lemma}

The proof that $\interp{[[T]]}$ defined in
Def.~\ref{def:stlc_interpretation_taits} satisfies these four
properties can be done by induction on the structure of $[[T]]$.
We need two additional lemmas to show that all typeable terms
of STLC are reducible.  
\begin{lemma}
  \label{lemma:sn_red_lam}
  If for all $[[t2]] \in \interp{[[T1]]}$ and $[[ [t2/x]t1]] \in \interp{[[T2]]}$ then
  $[[\x:T1.t1]] \in \interp{[[T1 -> T2]]}$.
\end{lemma}
\noindent
The proof is by case analysis on the possible reductions of
$[[(\x:T1.t1) t2]]$.  To prove that all terms are reducible we must
first define sets of well-formed substitutions.  We denote the empty
substitution as $\emptyset$.
\begin{definition}
  \label{def:stlc_well-formed_subs}
  %% Why not just define G |- \sigma instead of doing it two pieces?
  Well-formed substitutions are defined as follows:
  \begin{center}
    \begin{math}
      \begin{array}{lll}
        $$\mprset{flushleft}
        \inferrule* [right=] {
          \ 
        }{\vdash \emptyset}
        &
        $$\mprset{flushleft}
        \inferrule* [right=] {
          t \in \interp{[[T]]}
          \\
          \vdash \sigma
        }{\vdash \sigma \cup ([[x]],[[t]])}
      \end{array}
    \end{math}
  \end{center}
\end{definition}
We say a substitution is well-formed with respect to a context if the
substitution is well-formed, the domain of the substitution consists
of all the variables of the context, and the range of the substitution
consists of terms with the same type as the variable they are
replacing.  We denote this by $[[G]] \vdash \sigma$.  Thus, if $[[G]]
\vdash \sigma$ then the domain of $\sigma$ is the domain of $[[G]]$
and the range of $\sigma$ are reducible typeable terms with the same
type as the variable they are replacing.  We define the interpretation
of a context as $[[G]] = \{ \sigma\,|\,[[G]] \vdash \sigma\}$.  We now
have everything we need to show that all typeable terms are reducible,
hence, strongly normalizing.
\begin{thm}
  \label{thm:stlc_red}
  If $\sigma \in \interp{[[G]]}$ and $[[G |- t : T]]$ then $\sigma\,t \in \interp{[[T]]}$.
\end{thm}

\begin{corollary}
  \label{corollary:stlc_sn}
  If $[[. |- t : T]]$ then $[[t]] \in \mathsf{SN}$.
\end{corollary}
