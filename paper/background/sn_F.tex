Girard extended this method into a more powerful one called
reducibility candidates to be able to prove strong normalization for
system F\index{Girard/Reynolds System F}.  We first extend the definition of a neutral term to include
$[[ t [T] ]]$.  The definition of the interpretation of
types\index{Interpretation of Types} are
defined next.
\begin{definition}
  \label{def:stlc_interpretation_taits}
  The interpretation of types are defined as follows:
  \begin{center}
    \begin{math}
      \begin{array}{lll}
        \interp{[[X]]}_\rho & = & \rho([[X]])\\
        \interp{[[T1 -> T2]]}_\rho & = & \{ t\,|\,\forall t' \in \interp{[[T1]]}_\rho.
               [[t t']] \in \interp{[[T2]]}_\rho \}\\
        \interp{[[Forall X.T]]}_\rho & = & \{ t\,|\,\forall T'.
        t[T'] \in \interp{[[T]]}_{\rho\{X \mapsto \interp{[[T']]} \}} \}
      \end{array}
    \end{math}
  \end{center}
\end{definition}
The sets defined in the previous definition are called parameterized
reducibility sets\index{Reducibility Sets}.  Recall that system F has type variables so when we
interpret types we must interpret type variables.  The naive extension
would just replace type variables with types, but Girard quickly
realized this would not work, because the final case of the above
definition of the interpretation of types would then be
$\interp{[[Forall X.T]]} = \{ t\,|\,\forall T'.  t[T'] \in \interp{[[
  [T'/X]T]]} \} $ and we can no longer consider this a well-defined
definition, because it is not structurally recursive over the type.
So instead we replace type variables with reducibility candidates.  A
reducibility candidate is just a reducibility set which satisfies the
four \textbf{CR} properties above.  We denote the set of all
reducibility candidates as $\mathsf{Red}$.  Then in the definition of
$\interp{[[Forall X.T]]}_\rho$ we quantify over all reducibility sets.
This is where set comprehension is coming in; also note that this is a
second order quantification.  We need two forms of substitutions:
substitutions mapping term variables to terms and type variables to
types, but also substitutions mapping type variables to reducibility
candidates called reducibility candidate substitutions.  We denote the
former by $\sigma$ and the latter as $\rho$.  The following two
definitions derive when both of these types of substitutions are
well-formed.
\begin{definition}
  \label{def:red_subst}
  Well-formed substitutions:
  \begin{center}
    \begin{math}
      \begin{array}{lll}
        $$\mprset{flushleft}
        \inferrule* [right=] {
          \ 
        }{\vdash \emptyset}
        &
        $$\mprset{flushleft}
        \inferrule* [right=] {
          [[t]] \in \interp{[[T]]}_\sigma
          \\
          \vdash \sigma
        }{\vdash \sigma \cup \{([[x]],[[t]])\}}
        &
        $$\mprset{flushleft}
        \inferrule* [right=] {
          \vdash \sigma
        }{\vdash \sigma \cup \{([[X]],[[T]])\}}
      \end{array}
    \end{math}
  \end{center}
\end{definition}
\begin{definition}
  \label{def:red_subst}
  Well-formed reducibility candidate substitutions:
  \begin{center}
    \begin{math}
      \begin{array}{lll}
        $$\mprset{flushleft}
        \inferrule* [right=] {
          \ 
        }{\vdash \emptyset}
        &
        $$\mprset{flushleft}
        \inferrule* [right=] {
          \interp{\rho'\,[[T]]} \in \mathsf{Red}
          \\
          \vdash \rho
        }{\vdash \rho \cup \{([[X]],\interp{[[T]]}_{\rho'})\}}
      \end{array}
    \end{math}
  \end{center}
\end{definition}
\noindent 
The following lemmas depend on the following definition of well-formed substitutions with
respect to reducibility candidate substitutions.
\begin{definition}
  \label{def:sub_wrt_redsub}
  Well-formed substitution with respect to a well-formed reducibility candidate substitution:
  \begin{center}
    \begin{math}
      \begin{array}{lll}
        $$\mprset{flushleft}
        \inferrule* [right=] {
          \ 
        }{\emptyset \vdash \emptyset}
        &
        $$\mprset{flushleft}
        \inferrule* [right=] {
          \interp{\rho'\,[[T]]} \in \mathsf{Red}
          \\
          \rho \vdash \sigma
        }{\rho \cup \{([[X]],\interp{[[T]]}_{\rho'})\} \vdash \sigma \cup \{ ([[X]],[[T]])\}}
      \end{array}
    \end{math}
  \end{center}
\end{definition}
We are now set to prove some new lemmas.  The following proofs depend
on the lemmas we have proven above about STLC.  We do not repeat them
here.  We say a parameterized reducibility set $\interp{[[T]]}_\rho$
is a reducibility candidate\index{Reducibility Candidate} of type $\sigma\,[[T]]$ if and only if
$\interp{\sigma\,[[T]]} \in \mathsf{Red}$ and $\rho \vdash \sigma$.
First, we prove that parameterized reducibility sets are
members of $\mathsf{Red}$.

\begin{lemma}
  \label{lemma:sn_f_cand}
  $\interp{[[T]]}_\rho$ is a reducibility candidate of type $\sigma\,[[T]]$ 
  where $\rho \vdash \sigma$.
\end{lemma}
\noindent
Our second result shows that substitution can be pushed into the
parameter of the reducibility set.  Set comprehension is hiding in
the statement of the lemma.  In order to push the substitution down into
the parameter we must first know that $\interp{[[T]]}_\rho$ really is
a set.
\begin{lemma}
  \label{lemma:red_subst}
  If $\vdash \rho$ then $\interp{[[ [T/X]T']]}_\rho = \interp{T'}_{\rho\{ X \mapsto \interp{[[T]]}_\rho\}}$.
\end{lemma}
\noindent
These final two lemmas are straightforward.  They are similar to the
lemmas above for $\lambda$-abstraction and application.
\begin{lemma}
  \label{lemma:red_f_univ_abs}
  If for every type $[[T]]$ and reducibility candidate $R$, $[[ [T/X]t]] \in \interp{[[T']]}_{\rho\{X \mapsto R\}}$, then
  $[[\\ X.t]] \in \interp{[[Forall X.T']]}_\rho$.
\end{lemma}

\begin{lemma}
  \label{lemma:red_f_univ_app}
  If $[[t]] \in \interp{[[Forall X.T]]}_\rho$, then $[[t [T'] ]] \in \interp{[[ [T'/X]T]]}_\rho$ for every type $[[T']]$.
\end{lemma}
\noindent
Finally, we can prove type soundness and obtain strong normalization.
\begin{thm}
  \label{thm:soundness_f}
  If $[[G |- t : T]]$, $[[G]] \vdash \sigma$ and $[[G]] \vdash \rho$ then,
  $\sigma [[t]] \in \interp{[[T]]}_{\rho}$.
\end{thm}

\begin{corollary}
  \label{corollary:sn_f}
  If $[[. |- t : T]]$ then $[[t]] \in SN$.
\end{corollary}
\noindent
The proof of the corollary follows from Theorem~\ref{thm:soundness_f}
by using a certain $\rho$.  The $\rho$ one must use is the one were
every type variable is mapped to a subset of $\mathsf{SN}$.  This
extension is a giant leap forward and is the foundation of what we now
call logical relations.  The interpretation of types we defined above
are actually unary predicates defined by recursion on the type.  The
form we defined them in here are in logical relation form rather than
the syntax of Girard in \cite{Girard:1989}.
