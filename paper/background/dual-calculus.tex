Philip Wadler invented a type theory equivalent to Curien and
Herbelin's $\LBMMT$-calculus called the dual calculus
\cite{Wadler:2003}.  What we mean by equivalent here is that both
correspond to Gentzen's classical sequent calculus LK, but both type
theories are definitionally inequivalent.  The difference between the
two type theories is that the $\LBMMT$-calculus is defined with only
negation, implication, and subtraction.  Then using De Morgan's laws
we can define conjunction and disjunction.  However, the dual calculus
is defined with only negation, conjunction, and disjunction.  Then we
define implication, which implies we may define
$\lambda$-abstractions.  This is a truly remarkable feature of
classical logic.

The syntax of the dual calculus is defined in Fig.~\ref{fig:dc_syntax}.
\begin{figure}
  \begin{center}
    \begin{math}
      \begin{array}{rll}
  [[T]], [[A]], [[B]], [[C]] & ::= & [[X]]\,|\,[[A /\ B]]\,|\,[[A \/ B]]\,|\,[[- A]]\\
  [[t]], [[a]], [[b]], [[c]] & ::= & [[x]]\,|\,[[< a , b >]]\,|\,[[inl t]]\,|\,[[inr t]]\,|\,[[ [ k ] not]]\,|\,[[( s ) . o]]\\
  [[k]], [[l]]               & ::= & [[o]]\,|\,[[ [ k , l ] ]]\,|\,[[fst k]]\,|\,[[snd k]]\,|\,[[not [ t ] ]]\,|\,[[x . ( s )]]\\
  [[s]]                      & ::= & [[t * k]]
      \end{array}
    \end{math}
  \end{center}

  \caption{Syntax of the Dual Calculus}
  \label{fig:dc_syntax}
\end{figure}
It is similar to the $\LBMMT$-calculus, consisting of types,
terms, coterms (continuations), and statements.  As types we have
propositional variables, conjunction, disjunction, and negation.  Note
that negation must be a primitive in the dual calculus rather than
being defined.  Terms in the dual calculus are variables, the introduction
form for conjunction called pairs denoted $[[<a,b>]]$, the introduction
forms for disjunction denoted $[[inl t]]$ and $[[inr t]]$ which can
be read as inject left and inject right respectively.  The next term
is the introduction form of negation denoted $[[ [k] not]]$.  The final
term is a binder for coterms and is the computational correspondent to
the left-to-right rule.  It is denoted $[[(s).o]]$.  This can be thought
of as running the statement $[[s]]$ and then routing its output to the output
port $[[o]]$.  The continuations or coterms are the duals to terms and consist of covariables
denoted $[[o]]$, copairs denoted $[[ [k,l] ]]$, the duals of inject-left and
inject-right called first and second denoted $[[fst k]]$ and $[[snd k]]$ respectively.
The next coterm is the elimination form of negation denoted $[[ not[t] ]]$ which
can be thought of as the continuation which takes as input a term of a negative 
formula and routes its output to some output port.  Finally, the dual to binding
an output port is binding an input port.  This is denoted $[[x.(s)]]$.  Now statements
are the introduction of a cut and are denoted $[[t * k]]$.  Computationally, we
can think of this as a command which runs the term $[[t]]$ and routes its output
to the continuation $[[k]]$ which continues the computation.

The reduction rules for the dual calculus are in Fig.~\ref{fig:dc_red}
and the typing rules are in Fig.~\ref{fig:dc_typing}.  
\begin{figure}
  \begin{center}
    \begin{mathpar}
      \DCdruleEtaR{}  \and
      \DCdruleEtaL{}  \and
      \DCdruleBetaR{} \and 
      \DCdruleBetaL{} \and
      \DCdruleBetaNeg{} \and
      \DCdruleBetaCoProdOne{} \and
      \DCdruleBetaCoProdTwo{} \and
      \DCdruleBetaProdOne{}   \and
      \DCdruleBetaProdTwo{}      
    \end{mathpar}
  \end{center}
  \caption{Reduction Rules for the Dual Calculus}
  \label{fig:dc_red}
\end{figure}
The reduction rules correspond to cut-elimination and can be thought
of a simplification process on proofs.  Computationally they can be
thought of as running programs with their continuations.  We derive
three judgments from the typing rules for terms, coterms, and
statements.  They have the following forms:
\begin{center}
  \begin{math}
    \begin{array}{lll}
      \text{Terms}:      & [[G |- D <- t -> A]]\\
      \text{Coterms}:    & [[G |- D <- k <- A]]\\
      \text{Statements}: & [[G |- D <- s]]\\
    \end{array}
  \end{math}
\end{center}
The syntax of judgments are different from Wadler's original syntax.
Here we use the arrows to indicate data flow.  One meaning for the
judgment $[[G |- D <- t -> A]]$ is that when all the variables in
$[[G]]$ have an input in $[[t]]$ then computing $[[t]]$ either returns
a value of type $[[A]]$ or routes its output to a covariable in
$[[D]]$.  One meaning for the judgment $[[G |- D <- k <- A]]$ is when
the continuation gets input for all the variables in $[[G]]$ and gets
an input of $[[A]]$ it computes a value which is stored in an output
port in $[[D]]$.  Finally, the meaning of $[[G |- D <- s]]$ is that
after the $[[s]]$ is done computing it stores its output in an output
port in $[[D]]$.  Each judgment has a logical meaning.  The typing
rules for terms correspond to the right rules of LK, and the typing
rules correspond to the left rules of LK, while the judgment for
statements correspond to the cut rule of LK.
\begin{figure}
  \begin{center}
    \begin{tabular}{lll}
      Terms:\\
      \begin{mathpar}
        \DCdruletXXAxR{}     \and
        \DCdruletXXProd{}    \and
        \DCdruletXXCoProdl{} \and
        \DCdruletXXCoProdr{} \and   
        \DCdruletXXNegR{}    \and
        \DCdruletXXIR{}
      \end{mathpar}\\
      & \\
      Coterms:\\
      \begin{mathpar}
        \DCdrulectXXAxL{}     \and
        \DCdrulectXXCoProd{}  \and        
        \DCdrulectXXProdFst{} \and
        \DCdrulectXXProdSnd{} \and        
        \DCdrulectXXNegL{}    \and
        \DCdrulectXXIR{}
      \end{mathpar}\\
      & \\
      Statements:\\
      \begin{mathpar}
        \DCdrulestXXCut{}
      \end{mathpar}      
    \end{tabular}    
  \end{center}

  \caption{Typing Rules for the Dual Calculus}
  \label{fig:dc_typing}
\end{figure}

It has been said that Wadler invented the dual calculus when reading
Curien and Herbelin's paper and found the subtraction operator
confusing.  This was his reason for going with conjunction and
disjunction instead of implication.  He knew that conjunction and
disjunction are duals in a well-known way unlike implication and
subtraction.  Then using negation, conjunction, and disjunction he 
defined implication, $\lambda$-abstractions, and application.  Now
the definition of these differs depending on which reduction strategy
is used.
\begin{definition}
  \label{def:dc_cbv_implication}
  Under CBN Implication, $\lambda$-abstractions, and application are defined in the following way:
  \begin{center}
    \begin{math}
      \begin{array}{lll}
        [[A]] \to [[B]]     & := & [[(-A) \/ B]]\\
        \lambda [[x]].[[t]] & := & [[(inl ([x.((inr t) * o)] not) * o).o]]\\
        [[t]]\,[[k]]       & := &  [[ [not [t], k] ]]\\
      \end{array}
    \end{math}
  \end{center}

  \ \\
  Under CBV Implication, $\lambda$-abstractions, and application are defined in the following way:
  \begin{center}
    \begin{math}
      \begin{array}{lll}
        [[A]] \to [[B]]     & := & [[-(A /\ -B)]]\\
        \lambda [[x]].[[t]] & := & [[ [z.(z * fst (x.(z * snd(not [t]))))] not ]]\\
        [[t]]\,[[k]]       & := &  [[not [<t, [k]not>] ]]\\
      \end{array}
    \end{math}
  \end{center}
\end{definition}
Notice that the two ways of defining implication in the previous
definition are duals.  Wadler used the dual calculus to show that CBV
is dual to CBN in \cite{Wadler:2003} just like Curien and Herbelin did
in \cite{Curien:2000}.  However, in a follow up paper Wadler showed
that his duality of the dual calculus into itself is an involution
\cite{Wadler:2005}.  This was a step further than Selinger.  While
Curien and Herbelin's duality was an involution they did not prove it.
In his follow up paper Wadler also showed that the CBV
$\lambda\mu$-calculus is dual to the CBN $\lambda\mu$-calculus by
translating it into the dual calculus and taking the dual of the
translation, and then translating back to the $\lambda\mu$-calculus.

