Type safety is a property of a programming language which guarantees
that computation never gets stuck.  We can always either complete the
computation (hit a value) or continue computing (take another
step). Type safety is defined by the following theorem:
\begin{thm}
  \label{thm:type_safety}
  \begin{math}
    \cdot \vdash t : T \text{ and } t \redto^* t' \text{ then }
    \text{val}(t') \text{ or } \exists t''.t' \redto t''.
  \end{math}
\end{thm}
\noindent
Where $\text{val}(t)$ is a predicate on terms which is true iff $t$ is
a syntactic value.  Recall values are either a variable or a
$\lambda$-abstraction.  Usually, type safety is shown by proving type
preservation and progress theorems.  These theorems however are
corollaries of our type-safety theorem.  The reason it is usually done
this way is because it is thought to be easier than giving a direct
proof.  We will see here that it is actually pretty simple to give a
direct proof using the logical relations\footnote{This section is
  based off of a lecture given by Amal Ahmed at the 2011 Oregon
  Programming Language Summer School.}.  We do not show any proofs
here, but they can all be found in the extended version of
\cite{Ahmed:2006} which can be accessed through Ahmed's web
page\footnote{\url{http://www.ccs.neu.edu/home/amal/papers/lr-recquant-techrpt.pdf}}.

We begin with the proof of type safety of the call-by-value STLC, and
then extend this to include iso-recursive types.  To get the proof of
the type safety theorem to go through we need to first define the
logical relations.
\begin{definition}
  \label{def:logical_relations}
  We define logical relations for values and then we extend this definition to
  terms.\\
  \ \\
  Logical relations for values: \\
  \begin{math}
    \begin{array}{lllll}
      \mathcal{V}\interp{X} & = &  \{ v\,|\, \cdot \vdash v : X \} \\
      & \\
      \mathcal{V}\interp{T_1 \to T_2} & = & \{ \lambda x:T.t\ |\ 
      \cdot \vdash \lambda x:T_1 : T_1 \to T_2 \land
      \forall v.v \in \mathcal{V} \interp{T_1} \implies 
              [v/x]t \in \mathcal{E}\interp{T_2} \}\\
    \end{array}
  \end{math}
  \ \\
  \ \\
  Logical relations extended to terms:\\
  \begin{math}
    \begin{array}{lll}
      \mathcal{E}\interp{T} & = & \{ t\ |\ \cdot \vdash t:T \land 
      \exists v.t \redto^* v \land v \in \mathcal{V}\interp{T} \}
    \end{array}
  \end{math}
  \ \\
  \ \\
  Well-formed substitutions:\\
  \begin{math}
    \begin{array}{lllll}
        \mathcal{G}\interp{\Gamma,x:T} & = & 
        \{ \gamma[x \mapsto v]\ |\ \gamma \in \mathcal{G}\interp{\Gamma} \land 
        v \in \mathcal{V}\interp{T} \}
    \end{array}
  \end{math}
\end{definition}
%% FIX: Explain why the previous definition is well founded.
To express when a particular open term $t$ has meaning with respect to our chosen 
semantics we define a new judgment which has the form $\Gamma \models e : T$.
This judgment can be read as $t$ models type $T$ in context $\Gamma$.
\begin{center}
  \begin{math}
    \Gamma \models t : T =^{df} \forall \gamma \in \mathcal{G}\interp{\Gamma} \implies
    \gamma(t) \in \mathcal{E}\interp{T}.
  \end{math}
\end{center}
\noindent
We now turn to the fundamental property of logical relations.  We state this property as
follows:
\begin{lemma}
  \label{lemma:fundamental_property}
  If $\Gamma \vdash t : T$ then $\Gamma \models e : T$.
\end{lemma}
\begin{proof}
  By induction on the structure of the assumed typing derivation.
\end{proof}
\noindent
The fundamental property then allows us to prove our main theorem.  To make
expressing this result cleaner we define the following predicate:
\begin{center}
  \begin{math}
    \text{safe}(t) =^{def} \forall t'.t \redto^* t' \implies (\text{val}(t') \lor \exists t''.t' \redto t'').
  \end{math}
\end{center}

\begin{thm}
  \label{corollary:type_safety}
  If $\Gamma \vdash t : T$ then $\text{safe}(t)$.
\end{thm}
\begin{proof}
  By induction on the assumed typing derivation.
\end{proof}
To summarize we have shown how to prove type safety using logical
relations of CBV STLC.  Next we extend CBV STLC with iso-recursive
types.  To the types we add $\mu \alpha . T$ and type variables
$\alpha$.  Do not confuse this operator with that of the operator of
the $\lambda\mu$-calculus.  It is unfortunate, but this operator is used to
capture many different notions throughout the literature.  The terms
are extended to include $\text{fold}\ t$ and $\text{unfold}\ t$, and
values are extended to include $\text{fold}\ v$.  Finally, we add
$\text{fold}\ E$ and $\text{unfold}\ E$ to the syntax for evaluation
contexts.  To deal with free type variables we either can add them to
contexts $\Gamma$ or add a new context specifically for keeping track
of type variables.  We will do the latter and add the following to our
syntax:
\begin{center}
  \begin{math}
    \begin{array}{lllll}
      \Delta & := & \cdot & | & \Delta,\alpha
    \end{array}
  \end{math}
\end{center}
We need one additional rule to complete the operational semantics which is 
$\text{unfold}(\text{fold}\ v) \redto v$.  We complete the extension by adding
two new type checking rules.  They are defined as follows: 
\begin{center}
  \begin{math}
    \begin{array}{lll}
      $$\mprset{flushleft}
      \inferrule* [right=fold] {
        \Gamma,\Delta \vdash t : [\mu \alpha.T/\alpha]T
      }{\Gamma,\Delta \vdash \text{fold}\ t:\mu \alpha.T}
      &
      $$\mprset{flushleft}
      \inferrule* [right=unfold] {
        \Gamma,\Delta \vdash t : \mu \alpha.T
      }{\Gamma,\Delta \vdash \text{unfold}\ t:[\mu \alpha.T/\alpha]T}
    \end{array}
  \end{math}
\end{center}
Let's try and apply the same techniques we used in the previous section to prove
type safety of our extended language.

We first have to extend the definition of the logical relations to deal with recursive types.  
\begin{definition}
  \label{def:logical_relations}
  We define logical relations for values and then we extend this definition to
  expressions (terms t).\\
  \ \\
  Logical relations for values: \\
  \begin{math}
    \begin{array}{lllll}
      \mathcal{V}\interp{\alpha}_\rho & = & \rho(\alpha) \\
      & \\
      \mathcal{V}\interp{X}_\rho & = &  \{ v\,|\,\cdot \vdash v : X \} \\
      & \\
      \mathcal{V}\interp{T_1 \to T_2}_\rho & = & \{ \lambda x:T.t\ |\ 
      \cdot \vdash \lambda x:T_1 : T_1 \to T_2 \land
      \forall v.v \in \mathcal{V} \interp{T_1}_\rho \implies 
              [v/x]t \in \mathcal{E}\interp{T_2}_\rho \}\\
      & \\
      \mathcal{V}\interp{\mu \alpha.T}_\rho & = & 
      \{ \text{fold}\ v\ |\ \forall v.
        \text{unfold}\ (\text{fold}\ v) \in \mathcal{V}\interp{[\mu \alpha.T/\alpha]T}_\rho.\\
    \end{array}
  \end{math}
  \ \\
  \ \\
  Logical relations extended to expressions:\\
  \begin{math}
    \begin{array}{lll}
      \mathcal{E}\interp{T}_\rho & = & \{ t\ |\ \cdot \vdash t:T \land 
      \exists v.t \redto^* v \land v \in \mathcal{V}\interp{T}_\rho \}
    \end{array}
  \end{math}
  \ \\
  \ \\
  Well-formed substitutions:\\
  \begin{math}
    \begin{array}{lllll}
        \mathcal{G}\interp{\Gamma,x:T}_\rho & = & 
        \{ \gamma[x \mapsto v]\ |\ \gamma \in \mathcal{G}\interp{\Gamma}_\rho \land 
        v \in \mathcal{V}\interp{T}_\rho \}
    \end{array}
  \end{math}
\end{definition}

This definition is slightly different from the previous.  Since we
have type variables we need to use Girard's trick to handle
reducibility-candidates substitutions.  Then we added the case for
recursive types.  Here we took the usual idea of using the elimination
form for $\mu$-types.  Now is this definition well-founded?  Recall
that one of the main ideas pertaining to logical relations is that the
definitions are done by induction on the structure of the type.  Now
it is easy to see that the definition above is clearly well-founded in
all the previous cases, but it would seem not to be for the case of
the $\mu$-type.  The type $[\mu \alpha.T/\alpha]T$ increased in size
rather than decreasing.  So how can we fix this?  First we notice that
by the definition of our operational semantics
$\text{unfold}\ (\text{fold}\ v)) \redto v$, so we can replace
$\text{unfold}\ (\text{fold}\ v)$ with just $v$ in the definition.  So
that simplifies matters a bit, although this does not help us with
respect to well-foundedness.  One more attempt would be to take the
substitution and push it into $\rho$.  Let's see what happens when we
try this.  Take the following for our new definition of the logical
relation for $\mu$-types:
\begin{center}
  \begin{math}
    \mathcal{V}\interp{\mu \alpha.T}_\rho  = 
    \{ \text{fold}\ v\ |\ \forall v.
    v \in \mathcal{V}\interp{T}_{\rho[\alpha \mapsto \mathcal{V}\interp{\mu \alpha.T}_\rho]} \}.
  \end{math}
\end{center}
Now we can really see the problem.  This new definition is defined in
terms of itself!  This is a result of the recursive type being
recursive.  So how can we fix this?  To define a well-founded
definition for recursive types we need something a little more
powerful then just ordinary logical relations.  This is where step
indices come to the rescue.

We need to not only consider the structure of the type as the measure
of well-foundedness for our definition for recursive types, but also
the operational behavior defined by our operational semantics.  Let's
just dive right in and define a new definition of our logical
relations.  All of our logical relations are interpretations.

\begin{definition}
  \label{def:interpretations}
  We define an interpretation as
  \begin{math}
    \begin{array}{lll}
      \interpset & \in & \powerset{\nat \times Term}.\\
    \end{array}
  \end{math}
\end{definition}
\noindent
We say an interpretation is well-formed if its elements are all atoms
(members of the set Atom). An atom is a set of tuples of natural
numbers and closed terms. Additionally, we require an interpretation
to be an element of the set Type.  Atom and Type are defined by the
following definition.
\begin{definition}
  \label{def:wellformed_interpretations}
  \ \\
  \begin{math}
    \begin{array}{lllll}
      \text{Atom} & = & \{ (k,t) \,|\ k \in \nat \land t \in Closed^{Term} \}\\
      & \\
      \text{Atom}^{\text{value}} & = & \text{Atom restricted to values}\\
      & \\
      \text{Type} & = & \{ \mathcal{I} \subseteq Atom^{value}\ |\ 
      \forall (k,v) \in \mathcal{I} . \forall j \leq k.(j,v) \in \mathcal{I} \}
      & \\
    \end{array}
  \end{math}
\end{definition}
\noindent
One of the key concepts of step-indexed logical relations is the
notion of approximation.  Hence, we need to be able to take
approximations of interpretations.  This will be more clear below.
\begin{definition}
  \label{def:approximations_of_interpretations}
  The $n$-approximation function on interpretations is defined as follows:\\
  \begin{math}
    \begin{array}{lll}
      \napprox{\interpset}{n} & = & \{ (k,v) \in \interpset\ |\ k < n\}
    \end{array}
  \end{math}
\end{definition}
We are now in a position to start defining the interpretations of
types (logical relations).  
\begin{definition}
  \label{def:logical_relations}
  Logical relations for values: \\
  \begin{math}
    \begin{array}{lllll}
      \vinterp{\alpha}_\rho & = & \rho(\alpha) \\
      & \\
      \vinterp{X}_\rho & = &  \{ (k,v) \in Atom^{value}\ |\ \cdot \vdash v : X \} \\
      & \\
      \vinterp{T_1 \to T_2}_\rho & = & \{ (k, \lambda x:T.t) \in Atom^{value}\ |\ 
      \cdot \vdash \lambda x:T_1.t : T_1 \to T_2 \,\land \\
      & & \,\,\,\,\forall j \leq k.\forall v.(j, v) \in \vinterp{T_1}_\rho \implies 
              (j, [v/x]t) \in \tinterp{T_2}_{\rho} \}\\
      & \\
      \vbinterp{n}{\mu \alpha.T}_\rho & = & \{ (k,fold\ v) \in Atom^{value}\ |\ 
        k < n \land \forall j < k.(j,v) \in 
        \vinterp{T}_{\rho[\alpha \mapsto \vbinterp{k}{\mu \alpha.T}_\rho]}\}\\
      & \\
      \vinterp{\mu \alpha.T}_\rho & = & \bigcup_{n \geq 0} \vbinterp{n}{\mu \alpha.T}_\rho
    \end{array}
  \end{math}
\end{definition}
\noindent
The next definition extends the previous to terms.
\begin{definition}
  \label{def:lr_extended_to_terms}
    Logical relations extended to terms:\\
  \begin{math}
    \begin{array}{lll}
      \tinterp{T} & = & \{ (k,t) \in Atom\ |\ \cdot \vdash t:T \land 
      \forall j \leq k.\forall t'.t \redto^j t' \land \cdot \vdash t':T \land 
      (irred(t') \implies (j,t') \in \vinterp{T}_\rho) \}
    \end{array}
  \end{math}
\end{definition}
\noindent
We will need two types of substitutions one for term variables and one for type variables.
The following definitions tell us when they are well-formed.
\begin{definition}
  \label{def:well-formed_term_var_subs}
  Well-formed term-variable substitutions:\\
  \begin{math}
    \begin{array}{lllll}
      \ginterp{\cdot} & = & \{ (k,\emptyset) \}\\
      & \\
      \ginterp{\Gamma,x:T} & = & 
      \{ (k, \gamma[x \mapsto v])\ |\ k \in \nat \land (k, \gamma) \in \ginterp{\Gamma} \land 
      (k,v) \in \vinterp{T}_\emptyset \}
    \end{array}
  \end{math}
\end{definition}

\begin{definition}
  \label{def:well-formed_type_var_subs}
  Well-formed type-variable contexts:\\
  \begin{math}
    \begin{array}{lll}
      \dinterp{\cdot} & = & \{ \emptyset \}\\
      & \\
      \dinterp{\Delta,\alpha} & = & \{ \rho[\alpha \mapsto \interpset]\ |\ 
      \rho \in \dinterp{\Delta} \land \interpset \in Type \}\\
    \end{array}
  \end{math}
\end{definition}
\noindent
Finally, we define when term $t$ is in the interpretation of type $T$ as follows:
\begin{center}
  \begin{math}
    \Gamma \models t : T =^{def} \forall k \geq 0 . \forall \gamma.(k, \gamma) \in 
    \ginterp{\Gamma} \implies (k, \gamma(t)) \in \tinterp{T}_\emptyset.
  \end{math}
\end{center}

Let's take a step back and consider our new definition and use it to
define exactly what a we mean by step-indexed logical relations.
Instead of our logical relations being sets of closed terms they are
now tuples of natural numbers and closed terms.  This natural number
is called the step index.  This is the number of steps necessary for
the closed term to reach a value.  By steps we mean the number of rule
applications of our operational semantics.  
%% Explain this example better.
For example, $t\ t'
\redto^1 [t'/x]t =^1 t''$, where $t''$ is the actual result of the
substitution.  Thus, applications actually consumes two steps!

Now all of our definitions of the logical relations are well-defined
using an ordering consisting of only the type except for the
definition of the logical relation for $\mu$-types.  This is the case
as we saw earlier where we need the step index.  The main point of
this definition is that we take larger and larger approximations of
the runtime behavior of the elements of the $\mu$-type logical
relation.  So we define the logical relation for $\mu$-types in terms
of an auxiliary logical relation, where the number of steps the members
of the relation are allowed to take is bound by some natural number
$n$.  This corresponds to $\vbinterp{n}{\mu \alpha.T}_\rho$.  Then
we define the logical relation for $\mu$-types as the union of all the
approximations, i.e. $\vinterp{\mu \alpha.T}_\rho$.  

We can now conclude type safety for STLC with recursive types.  We
will need the following two lemmas in the proof of the fundamental
property of the logical relation.  We write $\Delta \vdash T$ to mean
that all the type variables in $\Delta$ are free in $T$.  The first
lemma is known as downward closure of the step-index logical relation.
The second is simple substitution commuting just as we saw for system
F above.
\begin{lemma}
  \label{lemma:1}
  If $\Delta \vdash T$, $\rho \in \dinterp{\Delta}$, $(k,v) \in \vinterp{T}_\rho$, and
  $j \leq k$ then $(j,v) \in \vinterp{T}_\rho$.
\end{lemma}

\begin{lemma}
  \label{lemma:2}
  If $\Delta, \alpha \vdash T$, $\rho \in \dinterp{\Delta}$ and 
  $\interpset = \napprox{\vinterp{\mu \alpha . T}_\rho}{k}$ then
  $\napprox{\vinterp{[\mu \alpha.T/\alpha]T}_\rho}{n} = 
   \napprox{\vinterp{T}_{\rho[\alpha \mapsto \interpset]}}{n}$.
\end{lemma}

\noindent
Finally, we conclude with the fundamental property of logical
relations.
\begin{thm}
  \label{thm:fundamental_property}
  If $\Gamma \vdash t : T$ then $\Gamma \models t:T$.
\end{thm}

From the fundamental property of the logical relations we can prove
type safety in a similar way as for standard STLC above.  Step-index
logical relations have been used with recursive types, but not for
general recursive programs and inconsistent type theories.  We said at
the beginning of this section that Casinghino does exactly this in
\cite{Casinghino:2012}, but their type theory does not contain
$\mathsf{Type:Type}$.
