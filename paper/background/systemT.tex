The two type theories we have considered above are not very
expressive.  In fact we cannot represent any decently complex
functions on the naturals within them.  This suggests it is quite
predictable that extensions of STLC would arise.  The first of these
is G\"odel's system T.  In this theory G\"odel extends STLC with
natural numbers and primitive recursion.  In \cite{Girard:1989} the
authors present system T with pairs and booleans, but we leave these
out here for clarity.  The big improvement is primitive recursion.
The syntax and reduction relation are defined in
Fig.~\ref{fig:T_syntax} and the type-checking relation is defined in
Fig.~\ref{fig:T_typing}.
\begin{figure}
  \begin{center}
    \begin{tabular}{lll}
      Syntax: & \\
      & 
      \begin{math}
        \begin{array}{lll}
          [[T]] & ::= & [[Nat]]\,|\,[[T -> T']]\\
          [[t]] & ::= & [[x]]\,|\,[[0]]\,|\,[[S]]\,|\,[[\x:T.t]]\,|\,[[t1 t2]]\,|\,[[R]]
        \end{array}
      \end{math}
      & \\
      Full $\beta$-reduction: & \\
      & 
      \begin{math}
        \begin{array}{lll}
          \TdruleRXXBeta{} & \TdruleRXXRecBase{}\\
          & \\
          \TdruleRXXRecStep{} & \TdruleRXXRecCongOne{}\\
          & \\
          \TdruleRXXRecCongTwo{} & \TdruleRXXRecCongThree{}\\
          & \\
          \TdruleRXXLam{} & \TdruleRXXAppOne{}\\
          &\\
          \TdruleRXXAppTwo{} & \TdruleRXXSucc{}
        \end{array}
      \end{math}
    \end{tabular}
  \end{center}

  \caption{Syntax and reduction rules for G\"odel's system T}
  \label{fig:T_syntax}
\end{figure}

\begin{figure}
  \begin{center}
    \begin{math}
      \begin{array}{ccc}
        \TdruleVar{} & \TdruleZero{}\\
        & \\
        \TdruleSucc{} & \TdruleRec{}\\ 
        & \\
        \TdruleLam{} & \TdruleApp{}
      \end{array}
    \end{math}
  \end{center}
  \caption{Type-checking algorithm for the G\"odel's system T}
  \label{fig:T_typing}
\end{figure}
We can easily see from the definition of the language that this is a
direct extension of STLC.  G\"odel extended the types STLC with a type
constant $[[Nat]]$ which is the type of natural numbers.  He then
extended the terms with a constant term $[[0]]$ denoting the natural
number zero, a term $[[S]]$ which is the successor function and
finally a recursor $[[R]]$ which corresponds to primitive recursion.  The
typing judgment is extended in the straightforward way.  We only explain the
typing rule for the recursor of system T.  Consider the rule:
\begin{center}
  \begin{math}
    \TdruleRec{}
  \end{math}
\end{center}
We can think of $[[R]]$ as a function which takes in a term of type
$[[T]]$, which will be the base case of the recursor, and then a term
of type $[[T -> (Nat -> T)]]$, which is the step case of the
recursion, and a second term of type $[[Nat]]$, which is the natural
number index of the recursion, i.e. with each recursive call this
number decreases.  Finally, when given these inputs $[[R]]$ will
compute a term by recursion of type $[[T]]$.  While the typing of
$[[R]]$ gives us a good picture of its operation the reduction rules
for $[[R]]$ give an even better one.  The rule
\begin{center}
  \begin{math}
    \TdruleRXXRecBase{}
  \end{math}
\end{center}
shows exactly that the first argument of $[[R]]$ is the base case.  Similarly,
the rule
\begin{center}
  \begin{math}
    \TdruleRXXRecStep{}
  \end{math}
\end{center}
shows how the step case is computed.  The type of $[[R]]$ tells us
that its second argument must be a function which takes in the
recursive call and the predecessor of the index of $[[R]]$.  These two
functions turn out to be all that is needed to compute all primitive
recursive functions \cite{Girard:1989}.

The authors of \cite{Girard:1989} consider system T to be a step
forward computationally, but a step backward logically.  We will see
in Sect.~\ref{sec:the_three_perspectives} how type theories can be
considered as logics, but for now it suffices to say that they claim
that system T has no such correspondence.  It turns out that system T
is expressive enough to define every primitive recursive function.  In
fact we can encode every ordinal from 0 to $\epsilon_0$ in system T.
This is quite an improvement from STLC.  We now pause to give a few
example terms corresponding to interesting functions and some example
computations.

\begin{example}
  \label{ex:T_functions}
  Some interesting functions in system T:
  \begin{center}
    \begin{math}
      \begin{array}{lll}
        \text{Addition:} & & [[add x y]] := 
             [[\x:Nat.(\y:Nat.(R x (\z:Nat.(\w:Nat.(S z))) y))]]\\
        \text{Multiplication:} & & 
             [[mult x y]] := [[\x:Nat.(\y:Nat.(R 0 (\z:Nat.(\w:Nat.(add x z))) y))]]\\ 
        \text{Exponentiation:} & & 
             [[exp x y]] := [[\x:Nat.(\y:Nat.(R (S 0) (\z:Nat.(\w:Nat.(exp x z))) y))]]\\ 
        \text{Predecessor:} & & 
             [[pred x]] := [[\x:Nat.(R 0 (\z:Nat.(\w:Nat.w)) x)]]\\ 
      \end{array}
    \end{math}
  \end{center}
\end{example}

\begin{example}
  \label{ex:T_reductions}
  We give an example reduction of addition.  We use numeral
  syntax for natural numbers, but this should be read as syntactic
  sugar.  That is, $1 \equiv [[S 0]]$, $2 \equiv [[S (S 0)]]$, etc.
  \begin{center}
    \begin{math}
      \begin{array}{lll}
        [[add 2 3]] & \redto^2 &  [[R 2 (\z:Nat.(\w:Nat.(S z))) 3]]\\
             & \redto   & [[((\z:Nat.(\w:Nat.(S z))) (R 2 (\z:Nat.(\w:Nat.(S z))) 2)) 3]]\\
             & \redto   & [[(\w:Nat.(S (R 2 (\z:Nat.(\w:Nat.(S z))) 2))) 3]]\\
             & \redto   & [[S (R 2 (\z:Nat.(\w:Nat.(S z))) 2)]]\\
             & \redto   & [[S (((\z:Nat.(\w:Nat.(S z))) (R 2 (\z:Nat.(\w:Nat.(S z))) 1)) 2)]]\\
             & \redto   & [[S ((\w:Nat.(S (R 2 (\z:Nat.(\w:Nat.(S z))) 1))) 2)]]\\
             & \redto   & [[S (S (R 2 (\z:Nat.(\w:Nat.(S z))) 1))]]\\
             & \redto   & [[S (S ((\z:Nat.(\w:Nat.(S z))) (R 2 (\z:Nat.(\w:Nat.(S z))) 0) 1))]]\\
             & \redto   & [[S (S ((\z:Nat.(\w:Nat.(S z))) 2 1))]]\\
             & \redto   & [[S (S ((\w:Nat.(S 2)) 1))]]\\
             & \redto   & [[S (S (S 2))]] \\
             & \equiv   & [[S (S (S (S (S 0))))]] \\
             & \equiv   & [[5]]
      \end{array}
    \end{math}
  \end{center}
\end{example}

Notice that the example reduction given in Ex.~\ref{ex:T_reductions}
is terminating.  A natural question one could ask is, are all
functions definable in system T terminating?  The answer is positive.
There is a detailed proof of termination of system T in
\cite{Girard:1989}.  The proof is similar to how we show strong
normalization for STLC in
Sect.~\ref{subsec:tait-griard_reduciblity}. Termination is in fact
guaranteed by the types of system T -- in fact it is guaranteed by the
types of all the type theories we have seen up till now.  Remember
types are used to enforce certain properties and termination is one of
the most popular properties types enforce.
%%% Local Variables: 
%%% mode: latex
%%% TeX-master: "paper"
%%% End:
