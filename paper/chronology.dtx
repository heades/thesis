% \iffalse meta-comment
%
% Copyright (C) 2010 by Levi Wiseman
%
% This work may be distributed and/or modified under the
% conditions of the LaTeX Project Public License, either version 1.3
% of this license or (at your option) any later version.
% The latest version of this license is in
%   http://www.latex-project.org/lppl.txt
% and version 1.3 or later is part of all distributions of LaTeX
% version 2005/12/01 or later.
%
% This work has the LPPL maintenance status `maintained'.
%
% The Current Maintainer of this work is Levi Wiseman.
%
% This work consists of the file chronology.dtx and the derived files
% chronology.ins, chronology.pdf, and chronology.sty.
%
% \fi
%
% \iffalse
%<*internal>
\iffalse
%</internal>
%<*readme>
A new timeline package. Allows labeling of events with per day granularity. Other features include relative positioning with unit specification, adjustable tick mark step size, and scaling to specified width.
%</readme>
%<*internal>
\fi
\def\nameofplainTeX{plain}
\ifx\fmtname\nameofplainTeX\else
  \expandafter\begingroup
\fi
%</internal>
%<*install>
\input docstrip.tex
\keepsilent
\askforoverwritefalse
\preamble

This is a generated file.

Copyright (C) 2010 by Levi Wiseman

This work may be distributed and/or modified under the
conditions of the LaTeX Project Public License, either version 1.3
of this license or (at your option) any later version.
The latest version of this license is in
  http://www.latex-project.org/lppl.txt
and version 1.3 or later is part of all distributions of LaTeX
version 2005/12/01 or later.

This work has the LPPL maintenance status `maintained'.

The Current Maintainer of this work is Levi Wiseman.

This work consists of the file chronology.dtx and the derived files
chronology.ins, chronology.pdf, and chronology.sty.

\endpreamble
\usedir{tex/latex/chronology}
\generate{\file{\jobname.sty}{\from{\jobname.dtx}{package}}}
%</install>
%<install>\endbatchfile
%<*internal>
\usedir{source/latex/chronology}
\generate{\file{\jobname.ins}{\from{\jobname.dtx}{install}}}
\nopreamble\nopostamble
\usedir{doc/latex/chronology}
\generate{\file{README.txt}{\from{\jobname.dtx}{readme}}}
\ifx\fmtname\nameofplainTeX
  \expandafter\endbatchfile
\else
  \expandafter\endgroup
\fi
%</internal>
%<*package>
\NeedsTeXFormat{LaTeX2e}
\ProvidesPackage{chronology}[2010/6/12 v1.0 Horizontal timeline]
\RequirePackage{calc}
\RequirePackage{tikz}
\RequirePackage{ifthen}
%</package>
%<*driver>
\documentclass{ltxdoc}
\usepackage{\jobname}
\usepackage{alltt}
\usepackage[numbered]{hypdoc}
\EnableCrossrefs
\CodelineIndex
\RecordChanges
\begin{document}
  \DocInput{\jobname.dtx}
\end{document}
%</driver>
% \fi
%
% \CheckSum{105}
%
% \CharacterTable
%  {Upper-case    \A\B\C\D\E\F\G\H\I\J\K\L\M\N\O\P\Q\R\S\T\U\V\W\X\Y\Z
%   Lower-case    \a\b\c\d\e\f\g\h\i\j\k\l\m\n\o\p\q\r\s\t\u\v\w\x\y\z
%   Digits        \0\1\2\3\4\5\6\7\8\9
%   Exclamation   \!     Double quote  \"     Hash (number) \#
%   Dollar        \$     Percent       \%     Ampersand     \&
%   Acute accent  \'     Left paren    \(     Right paren   \)
%   Asterisk      \*     Plus          \+     Comma         \,
%   Minus         \-     Point         \.     Solidus       \/
%   Colon         \:     Semicolon     \;     Less than     \<
%   Equals        \=     Greater than  \>     Question mark \?
%   Commercial at \@     Left bracket  \[     Backslash     \\
%   Right bracket \]     Circumflex    \^     Underscore    \_
%   Grave accent  \`     Left brace    \{     Vertical bar  \|
%   Right brace   \}     Tilde         \~}
%
% \changes{v1.0}{2010/6/12}{Initial version}
%
% \GetFileInfo{\jobname.sty}
%
% \DoNotIndex{\addtocounter,\begin,\draw,\else,\end,\fi,\foreach,\ifx,\newcommand,\newcounter,\newenvironment,\newlength,\newsavebox,\node,\pgfmathsetcounter,\pgfmathsetlength,\raisebox,\resizebox,\setcounter,\setlength,\thedeltayears,\thestep,\thestepstart,\thestepstop,\theyearstart,\theyearstop,\timelinebox,\timelinewidth,\unit,\usebox,\x,\xstart,\xstop}
%
% \title{The \textsf{\jobname} package\thanks{This document corresponds to \textsf{\jobname}~\fileversion, dated~\filedate.}}
% \author{Levi Wiseman \\ \texttt{levi.wiseman@gmail.com}}
%
% \maketitle
%
% \begin{abstract}
% A new timeline package. Allows labeling of events with per day granularity. Other features include relative positioning with unit specification, adjustable tick mark step size, and scaling to specified width.
% \end{abstract}
%
% \section{Introduction}
% Most timeline packages and solutions for {\LaTeX} are used to convey a lot of information and are therefore designed vertically. If you are just attempting to assign labels to dates, a more traditional timeline might be more appropriate. That's what {\jobname} is for.
%
% \section{Usage}
% \DescribeEnv{chronology}
% Declare a |chronology| as follows:
% \begin{alltt}\verb|\begin{chronology}|\oarg{step}\marg{startYear}\marg{endYear}\marg{unit}\marg{width}\\\meta{events}\\\verb|\end{chronology}|\end{alltt}
% Where \marg{width} is the final width of the timeline. The width can be specified as a command (e.g.\ \cmd{\textwidth}). \marg{unit} specifies the distance between minor ticks. \oarg{step} (unitless) specifies how many units between major ticks, where the first starts $\equiv_{\bmod {\langle step\rangle}}0$. The timeline runs from \marg{startYear} to \marg{endYear}.
% \DescribeMacro{\event}
% Label an \cmd{\event} as follows:
% \begin{alltt}\cmd{\event}\oarg{startDate}\marg{endDate}\marg{label}\end{alltt}
% An \cmd{\event} with the label \marg{label} is created. If \oarg{startDate} is specified the \cmd{\event} should span to \marg{endDate}, otherwise the \cmd{\event} specifies a specific date. All dates are unitless.
% \DescribeMacro{\decimaldate}
% Specify a \cmd{\decimaldate} as follows:
% \begin{alltt}\cmd{\decimaldate}\marg{day}\marg{month}\marg{year}\end{alltt}
%
% \StopEventually{\PrintChanges\PrintIndex}
%
% \section{Implementation}
% \iffalse
%<*package>
% \fi
%
% \begin{environment}{chronology}
% |chronology| creates the graphic first and then scales it to size. Using a large unit size results in a large prescaled graphic, and therefore finer postscaled details.
%    \begin{macrocode}
\newenvironment{chronology}[5][5]{%
  \newcounter{step}\newcounter{stepstart}\newcounter{stepstop}%
  \newcounter{yearstart}\newcounter{yearstop}\newcounter{deltayears}%
  \newlength{\xstart}\newlength{\xstop}%
  \newlength{\unit}\newlength{\timelinewidth}%
  \setcounter{step}{#1}%
  \setcounter{yearstart}{#2}\setcounter{yearstop}{#3}%
  \setcounter{deltayears}{\theyearstop-\theyearstart}%
  \setlength{\unit}{#4}%
  \setlength{\timelinewidth}{#5}%
  \pgfmathsetcounter{stepstart}%
    {\theyearstart+\thestep-mod(\theyearstart,\thestep)}%
  \pgfmathsetcounter{stepstop}{\theyearstop-mod(\theyearstop,\thestep)}%
  \addtocounter{step}{\thestepstart}%
  \newsavebox{\timelinebox}%
  \begin{lrbox}{\timelinebox}%
    \begin{tikzpicture}[baseline={(current bounding box.north)}]%
      \draw [|->] (0,0) -- (\thedeltayears*\unit+\unit, 0);%
      \foreach \x in {1,...,\thedeltayears}%
        \draw[xshift=\x*\unit] (0,-.1\unit) -- (0,.1\unit);%
      \addtocounter{deltayears}{1}%
      \foreach \x in {\thestepstart,\thestep,...,\thestepstop}{%
        \pgfmathsetlength\xstop{(\x-\theyearstart)*\unit}%
        \draw[xshift=\xstop] (0,-.3\unit) -- (0,.3\unit);%
        \node at (\xstop,0) [below=.2\unit] {\x};}}{%
    \end{tikzpicture}%
  \end{lrbox}%
  \raisebox{2ex}{\resizebox{\timelinewidth}{!}{\usebox{\timelinebox}}}}%
%    \end{macrocode}
% \end{environment}
%
% \begin{macro}{\event}
%    \begin{macrocode}
\newcommand{\event}[3][e]{%
  \pgfmathsetlength\xstop{(#2-\theyearstart)*\unit}%
  \ifx #1e%
    \draw[fill=black,draw=none,opacity=0.5]%
      (\xstop, 0) circle (.2\unit)%
      node[opacity=1,rotate=45,right=.5\unit] {#3};%
  \else%
    \pgfmathsetlength\xstart{(#1-\theyearstart)*\unit}%
    \draw[fill=black,draw=none,opacity=0.5,rounded corners=.2\unit]%
      (\xstart,-.2\unit) rectangle%
      node[opacity=1,rotate=45,right=.5\unit] {#3} (\xstop,.2\unit);%
  \fi}%
%    \end{macrocode}
% \end{macro}
%
% \begin{macro}{\decimaldate}
%    \begin{macrocode}
\newcommand{\decimaldate}[3]{(#1-1)/31/12+(#2-1)/12+#3}
%    \end{macrocode}
% \end{macro}
% \iffalse
%</package>
% \fi
%
% \Finale
\endinput
