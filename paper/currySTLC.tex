The Curry-style simply typed $\lambda$-calculus is exactly
Church-style simply type $\lambda$-calculus except there is no type
annotations on $\lambda$-abstractions.  That is we have $[[\x.t]]$
instead of $\lambda [[x]]:[[T]].[[t]]$ in the syntax for terms.  This
definition of STLC was an extension of Curry's work on combinator
logic. The syntax and reduction relation of this theory is defined in
Fig.~\ref{fig:cstlc_syntax} and the typing judgment is defined in
Fig.~\ref{fig:cstlc_typing}. We call the typing judgment defined
here an implicit typing paradigm.  The fact that it is implicit shows
up in the applicatin typing rule $\CSTLCdrulename{App}$:
\begin{center}
  \begin{math}
    \CSTLCdruleApp{}
  \end{math}
\end{center}
Recall that these rules are read bottom up.  Up until now we have
considered the typing judgment as simply a checking procedure with the
type as one of the inputs, but often this judgment is defined so that
the type is computed and becomes an output.  In theories like this the
above rule causes some trouble.  The type $[[T1]]$ is left implicit
that is by looking at only the conclusion of the rule one cannot tell
what the value of $[[T1]]$ must be.  This problem also exists for the
typing rule for $\lambda$-abstractions.  This is, however, not a
problem in Church style STLC because that type is annotated on
functions.  This suggest that for some Curry style type theories type
construction is undecidable.  Not all type theories have a Church
style and a Curry style formulations.  Therry Coquand's Calculus of
Constructions is an example of a type theory that is in the style of
Church, but it is also unclear how to define a Curry style version.
It is also unclear how to define a Church style version of the type
theory of intersection types \cite{Barendregt:1992}.

\begin{figure}
  \begin{center}
    \begin{tabular}{lll}
      Syntax: & \\
      & 
      \begin{math}
        \begin{array}{lll}
          [[T]] & ::= & [[X]]\,|\,[[T -> T']]\\
          [[t]] & ::= & [[x]]\,|\,[[\x.t]]\,|\,[[t1 t2]]
        \end{array}
      \end{math}
      & \\
      Full $\beta$-reduction: & \\
      & 
      \begin{math}
        \begin{array}{lll}
          \CSTLCdruleRXXBeta{} & \CSTLCdruleRXXLam{}\\
          & \\
          \CSTLCdruleRXXAppOne{} & \CSTLCdruleRXXAppTwo{}
        \end{array}
      \end{math}
    \end{tabular}
  \end{center}

  \caption{Syntax and reduction rules for the Curry-style simply-typed $\lambda$-calculus}
  \label{fig:cstlc_syntax}
\end{figure}

\begin{figure}
  \begin{center}
    \begin{math}
      \begin{array}{lll}
        \CSTLCdruleVar{} & \CSTLCdruleLam{} & \CSTLCdruleApp{}
      \end{array}
    \end{math}
  \end{center}
  \caption{Type-checking algoritym for the Curry-style simply typed $\lambda$-calculus}
  \label{fig:cstlc_typing}
\end{figure}

%%% Local Variables: 
%%% mode: latex
%%% TeX-master: "paper"
%%% End:
