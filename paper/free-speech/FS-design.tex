\input{free-speech-ott}

\renewcommand{\FSdrulename}[1]{\scriptsize \textsc{#1}}

\newcommand{\tvdash}[1]{\vdash^{#1}}
\newcommand{\arrowT}[5]{(#1 :^{#2} #3)^{#4} \to #5}
\newcommand{\rec}[3]{rec\ #1\ #2\ #3}
\newcommand{\recc}[3]{rec^{-}\ #1\ #3}

\begin{figure}
  \begin{center}
    \begin{tabular}{lll}
      Syntax:
      \vspace{10px} \\
      \begin{math}
        \arraycolsep=2pt\edef\arraystretch{1.2}
        \begin{array}{rllllllllllllllll}
          \text{(Classifiers)}  & [[th]] & ::= & [[L]]\,|\,[[C]]\\
          \text{(Stages)}       & [[ep]] & ::= & [[+]]\,|\,[[-]]\\
          \text{(Expressions)}  & [[e]]  & ::= & 
          [[x]]\,|\,[[Type]]\,|\,[[Nat]]\,|\,[[(x : th e1) ep -> e2]]\,|\,[[e1 = e2]]\,|\,
          [[S]]\,|\,[[Z]]\,|\,\\
          & & & [[\x.e]]\,|\,[[rec f x e]]\,|\,[[rec - f e]]\,|\,[[e1 e2]]\,|\,
                [[join]]\,|\,[[injdom]]\,|\,[[injran]]\,|\,\\
          & & & [[contra]]\,|\,[[abort]]\\
          \text{(Values)}       & [[v]] & ::= & 
          [[x]]\,|\,[[Type]]\,|\,[[Nat]]\,|\,[[(x : th v1) ep -> v2]]\,|\,
          [[e1 = e2]]\,|\,[[\x.e]]\,|\,\\
          & & & [[join]]\,|\,[[injdom]]\,|\,[[injran]]\,|\,[[rec f x v]]\,|\,[[rec - f v]]\\
          \text{(Evaluation Contexts)} & [[C]] & ::= & [[ [] ]]\,|\,[[( x : th C ) ep -> e2]]\,|\,[[( x : th e1 ) ep -> C]]\,|\,[[rec f x C]]\,|\,\\
          & & & [[rec - f C]]\,|\,[[e1 C]]\,|\,[[C v]]\\
          \text{(Typing Contexts)}     & [[G]] & ::= & [[.]]\,|\,[[x : th e]]\,|\,[[G1,G2]]\\        
        \end{array}
      \end{math}\\
      & \\
      CBV reduction:\\
      \small
      \begin{mathpar}
        \FSdruleCbvXXApp{}  \and
        \FSdruleCbvXXRec{}  \and
        \FSdruleRedXXCtxt{} \and
        \FSdruleRedXXAbort{} \and
        \FSdruleComputeJoin{}
      \end{mathpar}
    \end{tabular}
  \end{center}  
  \caption{Syntax and reduction rules for freedom of speech}
  \label{fig:FS-syn-red}
\end{figure}

\begin{figure}
  \begin{center}
    \begin{mathpar}
      \FSdruleKXXType{}        \and
      \FSdruleKXXNat{}         \and
      \FSdruleKXXPi{}          \and
      \FSdruleKXXEq{}          \and
      \FSdruleVar{}            \and
      \FSdruleLam{}            \and
      \FSdruleILam{}           \and
      \FSdruleAppPiTerm{}      \and
      \FSdruleAppAllTerm{}     \and
      \FSdrulejoin{}           \and
      \FSdruleConv{}           \and
      \FSdruleSucc{}           \and
      \FSdruleZero{}           \and
      \FSdruleAbort{}          \and
      \FSdruleContra{}         \and
      \FSdruleContraAbort{}    \and
      \FSdruleCoerce{}         \and
      \FSdruleRecNat{}         \and
      \FSdruleRecNatComp{}     \and
      \FSdruleRec{}
    \end{mathpar}
  \end{center}
  \caption{Typing rules for freedom of speech}
  \label{fig:FS-typing}
\end{figure}


To prevent equations between $\Pi$-types having different compiletime/runtime arguments or 
different consistency classifiers we add the following rules:

\begin{center}
  \begin{mathpar}
    \FSdruleContraPiTh{} \and
    \FSdruleContraPiEp{}
  \end{mathpar}
\end{center}

There are many types of problems that arise from the absence of $\FSdrulename{ContraPiTh}$
and $\FSdrulename{ContraPiEp}$.
For example, not having this resitrction would allow one to equate programmatic functions taking 
programmatic arguments to programmatic functions taking logical arguments, which would be the 
opposite of the freedom of speech property.  Even worse we could equate logical functions taking
programmatic arguments to logical functions taking logical arguments which breaks the freedom
of speech property.

Currently there exists a counter example to type preservation.  Let $\Gamma$ be 
$X:^L Type, Y:^L Type, x:^L X, $
$u:^L ((\arrowT{a}{\nat}{L}{+}{X}) = (\arrowT{a}{\nat}{L}{+}{Y}))$.  Then
if $\Gamma \tvdash{L} \lambda a.x:\arrowT{a}{\nat}{L}{+}{X}$ and $\Gamma \tvdash{L} Zero:Y$
then $\Gamma \tvdash{L} (\lambda a.x)\ Zero:X$.  By applying $\FSdrulename{Conv}$ and using
the assumption $u$, $\Gamma \tvdash{L} (\lambda a.x)\ Zero:Y$, but 
$(\lambda a.x)\ Zero \redto x$ and $\Gamma \tvdash{L} x:X$.

An easy solution to this problem is to add injectivity axioms for $\Pi$-types.  We define the 
injectivity axioms as follows:
\begin{center}  
  \begin{mathpar}
    \FSdruleInjDom{} \and
    \FSdruleInjRan{}
  \end{mathpar}
\end{center}

There are two rules one for the domain types and a second for the range types.  The second rule 
may seem a bit strange especially the second premise, because we are substiuting a value $v$ for
two free variables of possibily different types.  This is however okay, because from the premises
of $\FSdrulename{InjRan}$ and $\FSdrulename{InjDom}$ and $\FSdrulename{Conv}$ it is easy to
conclude that $\Gamma \tvdash{\theta} v:e'_1$.  By $\FSdrulename{InjDom}$, 
$\Gamma \tvdash{L} injdom:e_1 = e'_1$ and clearly $\Gamma \tvdash{\theta} e_1$ is equivalent to 
$\Gamma \tvdash{\theta} [e_1/x]x$.  Thus by $\FSdrulename{Conv}$,
$\Gamma \tvdash{\theta} [e'_1/x]x$, which is equivalent to
$\Gamma \tvdash{\theta} v:e'_1$.  So $\FSdrulename{InjRan}$ does seem to be sound.
