In the previous section we introduced classical type theories and
defined the $\lambda\mu$-calculus.  We saw that it was a sequent style
logic.  In this section we define the natural deduction equivalent of
$\lambda\mu$-calculus called the $\lambda\Delta$-calculus.  After we
define the type theory we give a brief explanation of its equivalence
to the $\lambda\mu$-calculus, but we do not prove its equivalence.
The $\lambda\Delta$-calculus was defined by Jakob Rehof and Morten
S\o{}rensen in Rehof's thesis \cite{Rehof:1994}.  Their work on the
$\lambda\Delta$-calculus was done independently of the
$\lambda\mu$-calculus and they were not aware of their equivalence
until Parigot pointed it out.  To our knowledge no actual proof was
ever published, but the proof is rather straightforward.  The
$\lambda\Delta$-calculus is an extension of STLC with the LDN in the
form of a control operator called $\Delta$.  Unlike the other type
theories we have seen we are going to first present the language and
then the typing rules.  Lastly, we will define the reduction rules.
It is our belief that the reduction rules may be more clear after the
reader sees the typing rules.

The language is defined in Fig.~\ref{fig:lamd_syntax} and the typing rules in
Fig.~\ref{fig:lamd_typing}.  
\begin{figure}
  \begin{center}
    \begin{tabular}{lll}
      Syntax: & \\
      & 
      \begin{math}
        \begin{array}{lll}
          [[T]],[[A]],[[B]],[[C]] & ::= & [[X]]\,|\,[[_|_]]\,|\,[[A -> B]]\\
          [[t]] & ::= & [[x]]\,|\,[[\x:T.t]]\,|\,[[\d x:T.t]]\,|\,[[t1 t2]]\\
        \end{array}
      \end{math}      
    \end{tabular}
  \end{center}

  \caption{Syntax and reduction rules for the $\lambda\Delta$-calculus}
  \label{fig:lamd_syntax}
\end{figure}
\begin{figure}
  \begin{center}
    \begin{math}
      \begin{array}{lll}
        \LamddruleVar{} & \LamddruleLam{}\\
          & \\
          \LamddruleDelta{} & \LamddruleApp{}\\
      \end{array}
    \end{math}
  \end{center}
  \caption{Type-checking algorithm for the $\lambda\Delta$-calculus}
  \label{fig:lamd_typing}
\end{figure}
We give the formulation a la Church, but the formulation a la Curry
does exist.  We can see that the syntax really is just the extension
of STLC with the $[[\d x:T.t]]$ control operator.  This operator is
the elimination form for absurdity $[[_|_]]$.  We can see this
connection by looking at its typing rule $\Lamddrulename{Delta}$.
Here we assume $[[{-A}]]$ and show $[[_|_]]$, and obtain
$[[A]]$.  We can use this rule to prove LDN:
\begin{definition}
  \label{def:lamd_ldn}
  The proof of $[[{-{-A}} -> A]]$ in the $\lambda\Delta$-calculus:
  \begin{center}
    \begin{math}
      $$\mprset{flushleft}
      \inferrule* [right=Lam] {
        $$\mprset{flushleft}
        \inferrule* [right=Delta] {
          $$\mprset{flushleft}
          \inferrule* [right=App] {
            $$\mprset{flushleft}
            \inferrule* [right=Var] {
              \ 
            }{[[x : {-{-A}}, y : {-A} |- x : {-{-A}}]]}
            \\
            $$\mprset{flushleft}
            \inferrule* [right=Var] {
              \ 
            }{[[x : {-{-A}}, y : {-A} |- y : {-A}]]}
          }{[[x : {-{-A}}, y : {-A} |- x y : _|_]]}
        }{[[x : {-{-A}} |- \d y : {-A}.(x y) : A]]}
      }{[[. |- \x:{-{-A}}.(\d y:{-A}.(x y)) : {-{-A}} -> A]]}
    \end{math}
  \end{center}
\end{definition}

The $\lambda\Delta$-calculus is equivalent to the $\lambda\mu$-calculus.  The
following definition gives an embedding from the $\lambda\mu$-calculus to
the $\lambda\Delta$-calculus.
\begin{definition}
  \label{def:lamd_embed_lamu}
  The following embeds the Church style formulation of the $\lambda\mu$-calculus into
  the $\lambda\Delta$-calculus:
  
  \ \\
  Context:
  \begin{center}
    \begin{math}
      \begin{array}{lll}
        |[[G]],[[A]]^[[x]]| & := & |[[G]]|,[[x]]:[[A]]\\
        &\\
        |\Delta,[[A]]^\alpha| & := & |\Delta|,[[x]]:[[{-A}]], \text{where } [[x]] \text{ is fresh in } |\Delta|\\
        &\\
        |[[A]],\Delta| & := & |\Delta|\\
      \end{array}
    \end{math}
  \end{center}

  \ \\
  Terms and Statements:
  \begin{center}
    \begin{math}
      \begin{array}{lll}
        |[[x]]|  & := & [[x]]\\
        &\\
        |\alpha| & := & [[y]] \text{, for some fresh variable } [[y]]\\
        &\\
        |\lambda [[x]]:[[A]].[[t]]| & := & \lambda [[x]]:[[A]].|[[t]]|\\
        &\\
        |[[t1 t2]]|  & := & |[[t1]]|\,|[[t2]]|\\
        &\\
        |\mu \alpha:[[A]].s| & := & \Delta [[x]]:[[{-A}]].|[[s]]| \\
        &\\
        |[\alpha][[t]]| & := & [[z]]\,|[[t]]|
          \text{, where } [[z]] \text{ is fresh in } [[t]]
      \end{array}
    \end{math}
  \end{center}
\end{definition}
Using the previous definition we can now prove that if a term is typeable in the $\lambda\mu$-calculus
then we can construct a corresponding term of the same type in the $\lambda\Delta$-calculus. 
\begin{lemma}
  \label{lemma:lamu_is_lamd}
  \begin{itemize}
  \item[i.] If $[[t]]:\Gamma \vdash [[A]],\Delta$ then $[[G]],|\Delta| \vdash |[[t]]|:[[A]]$.
  \item[ii.] If $[[t]]:\Gamma \vdash .,\Delta$ then $[[G]],|\Delta| \vdash |[[t]]|:[[_|_]]$.
  \end{itemize}
\end{lemma}
\noindent
The previous lemma establishes that the $\lambda\Delta$-calculus is at
least as expressive -- in terms of typeability -- than the
$\lambda\mu$-calculus.  It so happens that we can prove that the
$\lambda\mu$-calculus is at least as strong as the
$\lambda\Delta$-calculus which implies that both type theories are
equivalent with respect to typeability.  We assume without loss of
generality that the typing context $[[G]]$ is sorted so that all
negative types come after all positive types.  A negative type is of
the form $[[{-A}]]$.  
\begin{definition}
  \label{def:embed_lamd_lamu}
  The following embeds the Church style formulation of the $\lambda\Delta$-calculus into the
  $\lambda\mu$-calculus:
  
  \ \\
  Contexts:\\  
  If $[[G]] \equiv [[x1]] : [[A1]], \ldots, [[xi]] : [[Ai]], [[y1]] : [[{-B1}]], \ldots, [[yj]] : [[{-Bj}]]$, then
  \begin{center}
    \begin{math}
      \begin{array}{lll}
        |[[x1]] : [[A1]], \ldots, [[xi]] : [[Ai]]| & := & [[A1]]^{[[x1]]}, \ldots, [[Ai]]^{[[xi]]} \equiv [[G]]_\mu\\
        &\\
        |[[y1]] : [[{-B1}]], \ldots, [[yj]] : [[{-Bj}]]| & := & [[B1]]^{\alpha_1},\ldots,[[Bj]]^{\alpha_j} \equiv \Delta\\
      \end{array}
    \end{math}
  \end{center}
  
  \ \\
  Terms:
  \begin{center}
    \begin{math}
      \begin{array}{lll}
        |[[x]]|  & := & [[x]]\\
        &\\
        |\lambda [[x]]:[[A]].[[t]]| & := & \lambda [[x]]:[[A]].|[[t]]|\\
        &\\
        |[[t1 t2]]|  & := & |[[t1]]|\,|[[t2]]|\\
        &\\
        |\Delta x:[[{-A}]].[[t]]| & := & \mu \alpha.[ \beta ]|[[t]]|
          \text{, where } \alpha \not \equiv \beta
      \end{array}
    \end{math}
  \end{center}
\end{definition}
\noindent
Similar to the previous embedding we can now prove that all inhabited
types of the $\lambda\Delta$-calculus are inhabited in the
$\lambda\mu$-calculus.
\begin{lemma}
  \label{lemma:lamd_is_lamu}
  If $[[G |- t : A]]$ then $|[[t]]| : [[G]]_\mu \vdash A,\Delta$.
\end{lemma}
\noindent 
Both of the above lemmas can be proven by induction on the form of the
assumed typing derivations.  This equivalence extends to the reduction
rules as well, but the reduction rules are not step by step
equivalent, but rather terms of the $\lambda\Delta$-calculus will have
to do more reduction than the corresponding terms of the
$\lambda\mu$-calculus.  We do not show this here.
%%% Local Variables: 
%%% mode: latex
%%% TeX-master: "paper"
%%% End:
