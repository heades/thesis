Russell called impredicativity vicious circularity and found it
appalling.  He actually took steps to remove it from his type theories
all together.  To remove impredicativity -- that is enforce
predicativity -- from his type theories he added a second level of
types which where used to organize the types of his theory.  This
organization made it impossible to instantiate a type with itself.
Predicative systems are less expressive than impredicative systems
\cite{Leivant:1991}.  This means that there are functions definable
in an impredicative theory which are not definable in its predicative
version.  In \cite{Leivant:1991,Danner:1999a} Daniel Leivant and
Norman Danner define and analyze a predicative version of
Reynolds-Girard's system F called Stratified System F (SSF).  They show
that SSF is substantially weaker than system F.  In fact we will
discuss the fact that SSF can be proven terminating by a much simpler
proof technique then system F suggesting that it is indeed weaker in
Sec.~\ref{subsec:hereditary_substitution}.  The syntax and reduction
rules for SSF are defined in Fig.~\ref{fig:SSF_syntax}, kinding rules
in Fig.~\ref{fig:SSF_kinding}, and typing rules in
Fig.~\ref{fig:SSF_typing}.
\begin{figure}
  \begin{center}
    \begin{tabular}{lll}
      Syntax: & \\
      & 
      \begin{math}
        \begin{array}{lll}
          [[K]] & ::= & 1\,|\,2\,|\,\ldots\\
          [[T]] & ::= & [[X]]\,|\,[[T -> T']]\,|\,[[Forall X:K.T]]\\
          [[t]] & ::= & [[x]]\,|\,[[\x:T.t]]\,|\,[[\\X:K.t]]\,|\,[[t1 t2]]\,|\,[[t [T] ]]
        \end{array}
      \end{math}
      & \\
      Full $\beta$-reduction: & \\
      & 
      \begin{math}
        \begin{array}{lll}
          \SSFdruleRXXBeta{} & \SSFdruleRXXTypeRed{}\\
          & \\
          \SSFdruleRXXLam{} & \SSFdruleRXXTypeAbs{}\\
          & \\
          \SSFdruleRXXAppOne{} & \SSFdruleRXXAppTwo{}\\
          & \\
          \SSFdruleRXXTypeApp{}
        \end{array}
      \end{math}
    \end{tabular}
  \end{center}

  \caption{Syntax and reduction rules for SSF}
  \label{fig:SSF_syntax}
\end{figure}
\begin{figure}
  \begin{center}
    \begin{math}
      \begin{array}{ccc}
        \SSFdruleKXXVar{} & \SSFdruleKXXArrow{}\\
        & \\
        \SSFdruleKXXForall{} 
      \end{array}
    \end{math}
  \end{center}
  \caption{Kind-checking rules for the SSF}
  \label{fig:SSF_kinding}
\end{figure}
\begin{figure}
  \begin{center}
    \begin{math}
      \begin{array}{ccc}
        \SSFdruleVar{} & \SSFdruleLam{}\\
        & \\
        \SSFdruleApp{} & \SSFdruleTypeAbs{}\\
        & \\
        \SSFdruleTypeApp{}
      \end{array}
    \end{math}
  \end{center}
  \caption{Type-checking rules for the SSF}
  \label{fig:SSF_typing}
\end{figure}
The objective of SSF is to enforce the property of predicativity on
the types of system F.  To accomplish this Leivant took the same path
as Russell in that he added a second layer of typing to system F. This
second layer is known as the kind level.  Kinds are the types of
types.  The kinds of SSF are the elements of the syntactic category
$[[K]]$ in the syntax for SSF.  These are simply all the natural
numbers.  We call these type levels.  To stratify the types of system
F we use kinding rules to organize the types into levels making sure
that polymorphic types reside in a higher level than the types allowed
to instantiate these polymorphic types.  The kinding rules are pretty
straightforward the one of interest is
\begin{center}
  \begin{math}
    \SSFdruleKXXForall{}.
  \end{math}
\end{center}
This is the rule which enforces predicativity. It does this by making
sure the level of $[[Forall X:K.T]]$ is at a larger level than
$[[X]]$.  This works, because all the types we instantiate this type
with must have the same level as $[[X]]$. We can easily see that
$[[K]] < max([[K]]+1,[[K']])$ for all $[[K]]$ and $[[K']]$.  Hence,
resulting in the enforcement of our desired property.

A understandable question one could ask at this point is, are
predicative theories enough? Unfortunately there is no correct answer
at this time.  This is a debatable question.  Some believe
predicative systems are enough and that impredicative systems are to
paradoxical \cite{Feferman:2005}.  In fact Hermann Weyl proposed a
predicativist foundation of mathematics.  In his book \cite{Weyl:1918}
he developed a predicative analysis using stratification to enforce
predicativity.  He goes on to show that a substantial amount of
mathematics can be done predictively.

We believe that impredicativity is not something that should be
abolished, but embraced.  It gives theories more expressive power in
an elegant way. This power comes at a cost that reasoning about
impredicative theories is more complex then predicative theories, but
this we think is to be expected.  However, we do believe that
impredicativity needs to be better understood.  At least in a
computational light.  

\begin{openproblem}
  It seems as if there are degrees of impredicativity.  For example,
  system F has a weaker form of impredicativity, because no paradoxes
  exist in system F, this follows from the fact that we know it is
  consistent, but there are other impredicative systems which do
  contain paradoxes.  In fact we will see such a system in
  Sect.~\ref{sec:dependent_type_theory}.  How many degrees of
  impredicativity are there and how much is to much?
\end{openproblem}

\noindent
We do not know of any research investigating the previous open problem,
but it seems to us that it is an important question.
%%% Local Variables: 
%%% mode: latex
%%% TeX-master: "paper"
%%% End:
