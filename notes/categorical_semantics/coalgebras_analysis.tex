It is widely known that the natural numbers and induction come hand in hand.  That is induction is used to both define and prove
predicates on the natural numbers.  That is elementary arithmetic. So we have the following picture:
\begin{center}
  \begin{math}
    $$\mprset{flushleft}
    \inferrule* [right=] {
      \text{Induction}
    }{\text{arithmetic (Natural Numbers)}}
  \end{math}
\end{center}
We then dare pose the question, how do we fill in the hole in the following picture?
\begin{center}
  \begin{math}
    $$\mprset{flushleft}
    \inferrule* [right=] {
      \text{Coinduction}
    }{\text{?}}
  \end{math}
\end{center}
Well D. Pavlovi\'{c}, V. Pratt, and M. Escardo have come up with an answer for us.  They answer with the following:
\begin{center}
  \begin{math}
    $$\mprset{flushleft}
    \inferrule* [right=] {
      \text{Coinduction}
    }{\text{analysis (real numbers)}}
  \end{math}
\end{center}
In this section we will slowly uncover how this works by closely examining their work.  One thing to notice regarding the 
relationship between the above pictures is that induction and coinduction are duels in a very precise way.  We pose yet another
question, does this imply that arithmetic and analysis are duels?  Are natural numbers and real numbers duals?  These are unknown
questions as of right now.

\subsection{Real Numbers}
\label{subsec:real_numbers}
\input{real_nums}
% subsection real_numbers (end)