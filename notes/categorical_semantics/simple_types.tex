
\section{A Metaframework}
\label{sec:a_metaframework}
We will use Martin-L\"of's Type Theory for our metaframework throughout this chapter.  
This will allow use to give precise types to all of the structures in our object languages.
In fact all of the type theories discussed in this chapter can be rigorously defined within
this framework where binding can be encoded using either de Bruijn indecies or using the
locally nameless representation \footnote{We prefer the latter.} \cite{??}.
% section a_metaframework (end)


\section{The Theory of Constants}
\label{sec:the_theory_of_constants}
We begin our journey into the world of categorical semantics of type theories by first 
showing how to interpret a simple algebraic theory consisting of a countably infinite set
of variables, a finite set of constant types, a finite set of $[[i]]$-ary function symbols,
a typing judgment, and a definitional equality judgment.  This theory is called the
theory of constants. We first give a formal definition of this theory in the presentation we
will adopt for the reminder of this document.  The syntax for the theory of constants is 
defined in Figure~\ref{fig:syntax_theory_constants}.  

The free variables of the theory can be defined at the metalevel as de
Bruijn indices, but we will use mathematical notation to simplify the
presentation.  They have type $[[Term]]$ and are classified by
constant types of type $[[Type]]$.  The $[[i]]$-ary function symbols
have type $[[Term i => Term]]$.  The constant types of the theory of constants
have meta-type $[[Type]]$.  Judgments are metastatements describing
what type a term can be assigned.  All of the judgements we will
define can be defined at the metalevel as an inductive datatype where
each rule of the judgment is defined as a constructor. If we call the
typing judgment $\text{has}\_\text{type}$ at the type level then its
type is $[[ (G:[Term x Type]) => (t:Term) => (U:Type) => Type]]$.  We
denote this judgment by $[[G |- t : U]]$. The type of the definitional
equality judgment is similar.  We define the type assignment judgment
in Figure~\ref{fig:typing_theory_constants} and the definitional
equality judgment in Figure~\ref{fig:eq_theory_constants}.

\begin{figure}
  \begin{center}
    \begin{math}
      \begin{array}{rlllll}
        (Types)    & [[T]] & ::= & [[S]]\,|\,[[U]]\\
        (Terms)    & [[t]] & ::= & [[x]]\,|\,[[c]]\,|\,[[f x1 ... xi]]\\
        (Contexts) & [[G]] & ::= & [[x:T]]\,|\,[[G1,G2]]
      \end{array}
    \end{math}
  \end{center}
  \caption{Syntax of the theory of constants}
  \label{fig:syntax_theory_constants}
\end{figure}

\begin{figure}
  \begin{center}
    \begin{math}
      \begin{array}{cc}
        \STdruleVar{} & \STdruleFun{}\\
        & \\
        \STdruleConst{}
      \end{array}
    \end{math}
  \end{center}
  \caption{Type assignment for the theory of constants}
  \label{fig:typing_theory_constants}
\end{figure}

\begin{figure}
  \begin{center}
    \begin{math}
      \begin{array}{cc}
        \STdruleRefl{} & \STdruleSym{}\\
        & \\
        \STdruleTrans{} & \STdruleSubst{}\\
      \end{array}
    \end{math}
  \end{center}
  \caption{Definitional equality for the theory of constants}
  \label{fig:eq_theory_constants}
\end{figure}

% section the_theory_of_constants (end)
