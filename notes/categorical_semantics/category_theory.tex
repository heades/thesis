\section{Natural Transformations}
\label{sec:natural_transformations}

\begin{lemma}[Composing Natural Transformations]
  \label{lemma:composing_natural_transformations}
  Suppose $\phi : F_1 \to F_2$ and $\psi : F_2 \to F_3$ are natural transformations between
  the functors $F_1$, $F_2$, and $F_3$.  Then $\psi \circ \phi : F_1 \to F_3$ is a natural
  transformation.
\end{lemma}
\begin{proof}
  Suppose $\phi : F_1 \to F_2$ and $\psi : F_2 \to F_3$ are natural transformations between
  the functors $F_1$, $F_2$, and $F_3$.  Then we know the following by the definition of a 
  natural transformation:
  \begin{center}
    \begin{math}
      \begin{array}{lll}
        \begin{diagram}
          [[a]] \\
          \dTo{f} \\
          [[b]]
        \end{diagram}
        & \,\,\,\, &
        \begin{diagram}
          F_1([[a]])   & \rTo{\phi_{[[a]]}} & F_2([[a]])\\
          \dTo{F_1(f)} &                   & \dTo_{F_2(f)}\\
          F_1([[b]])   & \rTo_{\phi_{[[b]]}} & F_2([[b]])\\
        \end{diagram}
      \end{array}
    \end{math}
  \end{center}
  and
  \begin{center}
    \begin{math}
      \begin{array}{lll}
        \begin{diagram}
          [[a']] \\
          \dTo{g} \\
          [[b']]
        \end{diagram}
        & \,\,\,\, &
        \begin{diagram}
          F_2([[a']])   & \rTo{\psi_{[[a']]}} & F_3([[a']])\\
          \dTo{F_2(g)} &                   & \dTo_{F_3(g)}\\
          F_2([[b']])   & \rTo_{\psi_{[[b']]}} & F_3([[b']])\\
        \end{diagram}
      \end{array}
    \end{math}
  \end{center}
  where all of the above diagrams commute.  Now it suffices to show that they following commutes:
  \begin{center}
    \begin{math}
      \begin{array}{lll}
        \begin{diagram}
          [[c]] \\
          \dTo{h} \\
          [[d]]
        \end{diagram}
        & \,\,\,\, &
        \begin{diagram}
          F_1([[c]])   & \rTo{\psi_{[[c]]} \circ \phi_{[[c]]}} & F_3([[c]])\\
          \dTo{F_1(h)} &                   & \dTo_{F_3(h)}\\
          F_1([[d]])   & \rTo_{\psi_{[[d]]} \circ \phi_{[[d]]}} & F_3([[d]])\\
        \end{diagram}
      \end{array}
    \end{math}
  \end{center}
  The fact that the previous diagrams commute follows from the following equational reasoning:
  \begin{center}
    \begin{math}
      \begin{array}{lll}
        (\psi_{[[d]]} \circ \phi_{[[d]]}) \circ F_1(h) & = & \psi_{[[d]]} \circ (\phi_{[[d]]} \circ F_1(h))\\
                                                     & = & \psi_{[[d]]} \circ (F_2(h) \circ \phi_{[[C]]})\\
                                                     & = & (\psi_{[[d]]} \circ F_2(h)) \circ \phi_{[[C]]}\\
                                                     & = & (F_3(h) \circ \psi_{[[C]]}) \circ \phi_{[[C]]}\\
                                                     & = & F_3(h) \circ (\psi_{[[C]]} \circ \phi_{[[C]]})\\
      \end{array}
    \end{math}
  \end{center}
  Therefore, the composition of two natural transformations is also a natural transformation.
\end{proof}
% section natural_transformations (end)

\section{Monoidal Categories}
\label{sec:monoidal_categories}

In this section we develope the notions of monoidal categories and symmetric monoidal categories.  
These types of categories are used extensively in the study of linear logic and linear type theories.
We first define each of these types of categories and then prove the well-known coherence property. 
The following defines monoidal categories.
\begin{definition}
  \label{def:monoidal_category}
  A \textbf{monoidal category} is a tuple $([[C]], \otimes, T, \alpha_{A,B,C}, \lambda_{A}, \rho_{A})$,
  where $[[C]]$ is a category, $\otimes$ is called the multiplication, $T \in [[C]]_0$ is the unit, $\alpha_{A,B,C}$ is
  a natural isomorphism called the associator, $\lambda_{A}$ is a natural isomorphism called the left unitor, 
  and $\rho_{A}$ is a natural isomorphism called the right unitor. The components of the monoidal category have the following types:
  \begin{center}
    \begin{tabular}{cc}
    \begin{diagram}
    [[C]] \times [[C]] & \rTo{\otimes} & [[C]] \\
  \end{diagram}
    & \ \ \ \ \ \ 
  \begin{diagram}
    (A \otimes B) \otimes C & \rTo{\alpha_{A,B,C}} & A \otimes (B \otimes C)\\
  \end{diagram}\\
  & \\
  \begin{diagram}
    T \otimes A & \rTo{\lambda_{A}} & A\\
  \end{diagram}
    & \ \ \ \ \ \ 
  \begin{diagram}
    A \otimes T & \rTo{\rho_{A}} & A\\
  \end{diagram}
  \end{tabular}
  \end{center}
  Monoidal categories have the following coherence conditions:
  \begin{diagram}
                          &               & (A \otimes B) \otimes (C \otimes D) & \\
           & \ruTo{\alpha_{A \otimes B,C,D}} &                                     & \rdTo{\alpha_{A,B,C \otimes D}} & \\
    ((A \otimes B) \otimes C) \otimes D & &                                     & & A \otimes (B \otimes (C \otimes D))\\
          \dTo{\alpha_{A,B,C} \otimes id_D} &  &                                  & & \uTo_{id_A \otimes \alpha_{B,C,D}}\\
    (A \otimes (B \otimes C)) \otimes D & & \rTo{\alpha_{A,B \otimes C, D}}         & & A \otimes ((B \otimes C) \otimes D)\\
  \end{diagram}
  and 
  \begin{diagram}
    (A \otimes T) \otimes B & & \rTo{\alpha_{A,T,B}}  & & A \otimes (T \otimes B)\\
    & \rdTo_{\rho_A \otimes id_B} &                & \ldTo_{id_A \otimes \rho_B} & \\
    & & A \otimes B & & \\
  \end{diagram}
  The first coherence condition is known as the pentagon identity and the second the triangle
  identity. We also need the following equality to hold:
  \begin{center}
    \begin{math}
      \lambda_T = \rho_T : T \otimes T \to T
    \end{math}
  \end{center}
\end{definition}
Now a symmetric monoidal category is one in which the monoidal multiplcation is symmetric. 
\begin{definition}
  \label{def:SMC}
  A \textbf{symmetric monoidal category (SMC)} is a monoidal category with an additional 
  natural isomorphism:
  \begin{diagram}
    A \otimes B & \rTo{\beta_{A,B}} & B \otimes A\\
  \end{diagram}
  Satisfying the following coherence conditions:
  \begin{diagram}
    (A \otimes B) \otimes C & \rTo{\alpha_{A,B,C}} & A \otimes (B \otimes C) & \rTo{\beta_{A,B \otimes C}} & (B \otimes C) \otimes A\\
    \dTo{\beta_{A,B} \otimes id_C}            &                     &                         &    & \dTo_{\alpha_{B,C,A}}\\
   (B \otimes A) \otimes C & \rTo{\alpha_{B,A,C}} & B \otimes (A \otimes C) & \rTo{id_B \otimes \beta_{A,C}}  & B \otimes (C \otimes A)\\
  \end{diagram}
and
\begin{center}
  \begin{math}
    \beta_{B,A} \circ \beta_{A,B} = id_{A \otimes B}
  \end{math}
\end{center}
\end{definition}

\subsection{Free Symmetric Monoidal Categories}
\label{subsec:free_symmetric_monoidal_categories}
First, we define free monoidal categories in the style of Mac Lane in \cite{MacLane:1998}. 
The language of \textbf{binary words} is defined by the following grammar:
\begin{center}
  \begin{math}
    \begin{array}{llllllll}
      \text{(binary words)} & [[w]], [[v]] & ::= & [[em]]\,|\,[[-]]\,|\,[[w [] v]]\\
    \end{array}
  \end{math}
\end{center}
The length of a binary word is defined by recursion in the following way:
\begin{center}
  \begin{math}
    \begin{array}{lll}
      length([[em]])  & = & 0\\
      length([[-]]) & = & 1\\
      length([[w [] v]]) & = & length([[w]]) + length([[v]])\\
    \end{array}
  \end{math}
\end{center}
An example of a binary word of length $4$ is $[[(- [] -) [] (- [] (- [] em))]]$.  

The cateogry $W$ of binary words is defined by taking for any two binary words $[[v]]$ and $[[w]]$
such that $length([[v]]) = length([[w]])$ exactly one arrow $[[v]] \to [[w]]$.  Certainly, this definition includes
identity arrows $id_w : [[v]] \to [[v]]$.  Composition also clearly exists, because having two arrows
$f : [[w]] \to [[w']]$ and $g : [[w']] \to [[v]]$ means that the length of the words $[[w]]$,$[[w']]$, and
$[[v]]$ all coincide, thus, the morphism $g \circ f : [[w]] \to [[v]]$ must exist.  Identities are certainly unique
by construction.  Thus, identitiy axioms hold.  The only thing left to verify is associativity.   There is exactly
one arrow between any two objects, thus, it must be the case that $(g \circ f) \circ j = g \circ (f \circ j)$.  

It is easy to see that the previous construction makes $W$ a preorder.  Furthermore, notice that all arrows are invertible. We know
for any two objects $[[w]]$ and $[[v]]$ there is an arrow $f : [[w]] \to [[v]]$, but we also know there is an
arrow $f^{-1} : [[v]] \to [[w]]$ by construction.  In addition we know that there is exactly one arrow between
any two objects, thus, their composition must be the identitiy.  Keeping this in mind it is easy to see that $W$
is also a monoidal category with the multiplcation being $[[w]],[[v]] \mapsto [[w [] v]]$, $[[em]]$ being the 
unit, and the isomorphisms $\lambda,\rho$, and $\alpha$ defined to be their respective arrows in $W$. The
coherence conditions for monoidal categories are guaranteed to hold by the fact that every diagram in $W$ is
commutative by construction.  Therefore, we have arrived at the following result.

\begin{theorem}[The Free Monoidal Category]
  \label{thm:the_free_monoidal_category}
  For any monoidal category $[[B]]$ and any object $[[b]] \in [[B]]_0$, there is a unique functor
  $W \to [[B]]$ with $- \mapsto [[b]]$.
\end{theorem}
\begin{proof}
  We can think of the objects of $W$ has objects of $[[B]]$ with ``holes''.  The holes are indicated
  by the $[[-]]$ word.  So to construct an object of $[[B]]$ from an object $W$ we have to ``fill
  in these holes''.  For some $[[b]] \in [[B]]_0$ we write the desired functor as $[[w]] \mapsto [[w b]]$,
  denoted $([[w]])_{[[b]]}$.  We write $[[w b]]$ to suggest that we are filling the holes in $[[w]]$ with the object $[[b]]$. Now 
  on objects of $W$ we define the desired functor by:
  \begin{center}
    \begin{math}
      \begin{array}{ccccc}
        ([[em]])_{[[b]]} = T & ([[-]])_{[[b]]} = b & ([[w [] v]])_{[[b]]} = ([[w]])_{[[b]]}\,\Box\,([[v]])_{[[b]]}
      \end{array}
    \end{math}
  \end{center}
  We now prove that this functor uniquely determines every object $(w)_[[b]] \in [[B]]$.  Suppose
  $([[w]])_{[[b]]} = ([[v]])_{[[b]]}$ for some words $[[w]]$ and $[[v]]$, and we must show that $[[w]] = [[v]]$.  
  Clearly, $[[w]]$ and $[[v]]$ must have the same length $n$.  So we proceed by induction on $n$.  Suppose
  $n = 0$, then $([[w]])_{[[b]]} = ([[v]])_{[[b]]} = ([[em]])_{[[b]]} = T$, thus, $[[w]] = [[v]] = [[em]]$. 
  Now suppose $n = 1$, this case is similiar to the previous. Finally, suppose $n = m > 1$, then 
  $[[w]] = [[w1 [] w2]]$ and $[[v]] = [[v1 [] v2]]$.  Then 
  \begin{center}
    \begin{math}
      \begin{array}{lll}
        ([[w1 [] w2]])_{[[b]]} & = & ([[v1 [] v2]])_{[[b]]}\\ 
                              & = & ([[w1]])_{[[b]]}\,\Box\,([[w2]])_{[[b]]}\\
                              & = & ([[v1]])_{[[b]]}\,\Box\,([[v2]])_{[[b]]}.
      \end{array}
    \end{math}
  \end{center}
  Hence it must be the case that $([[w1]])_{[[b]]} = ([[v1]])_{[[b]]}$ and $([[w2]])_{[[b]]} = ([[v2]])_{[[b]]}$. 
  By the induction hypthesis $[[w1]] = [[v1]]$ and $[[w2]] = [[v2]]$.  Therefore, $[[w1 [] w2]] = [[v1 [] v2]]$.

  Next we prove the functor $(\,)_{[[b]]}$ is surjective. 
  Suppose $[[b]],[[c]] \in [[B]]_0$. We proceed by induction on the length $[[c]]$. 
  If $[[c]] = T$, then $([[em]])_{[[b]]} = T = [[c]]$.  Now suppose
  $[[c]] = [[b]]$, then $([[-]])_{[[c]]} = [[c]] = [[b]]$. Lastly, suppose $[[c]] = [[c1]]\,\Box\,[[c2]]$, we
  know by induction there exsits two binary words $[[w1]]$ and $[[w2]]$ such that $([[w1]])_{[[b]]} = [[c1]]$
  and $([[w2]])_{[[b]]} = [[c2]]$.  Therefore, $([[w1 [] w2]])_{[[b]]} = [[c1]]\,\Box\,[[c2]]$, and we have
  shown that we have a bijection from $W$ to $[[B]]$.  
\end{proof}

% subsection free_symmetric_monoidal_categories (end)

% section monoidal_categories (end)

\section{Monads and Comonads}
\label{sec:monads_and_comonads}

In the sequel $[[T]]^n$, for $n \in \mathbb{N}$, is defined to be
the $n$-fold composition of $[[T]]$, e.g. $[[T]]^2 = [[T]] \circ [[T]]$.

\begin{definition}
  \label{def:monad}
  A \textbf{monad} of a category $[[C]]$ consists of an endofunctor
  $[[T]] : [[C]] \to [[C]]$ and two natural transformations
  $\eta : id_{[[C]]} \to [[T]]$ and $\mu : [[T]]^2 \to [[T]]$
  satisfying the following commutative diagrams:
  \begin{center}
    \begin{tabular}{lll}
      \begin{diagram}
        [[T]]^3    & \rTo{\mu_{[[T]]}} & [[T]]^2\\
        \dTo{T\mu} &                  & \dTo{\mu}\\
        [[T]]^2    & \rTo{\mu}        & [[T]]
      \end{diagram}
      &
      \begin{diagram}
        [[T]]       & \rTo{\eta_{T}} & [[T]]^2\\
        \dTo{T\eta} & \rdTo{id}      & \dTo_{\mu}\\
        [[T]]^2     & \rTo{\mu}      & [[T]]
      \end{diagram}
    \end{tabular}
  \end{center}
\end{definition}

\newcommand{\opeta}[0]{\tilde\eta}
\newcommand{\opmu}[0]{\tilde\mu}

\begin{definition}
  \label{def:comonad}
  A \textbf{comonad} of a category $[[C]]$ consists of an endofunctor
  $[[T]] : [[C]] \to [[C]]$ and two natural transformations
  $\opeta : [[T]] \to id_{[[C]]}$ and $\opmu : [[T]] \to [[T]]^2$
  satisfying the following commutative diagrams:
  \begin{center}
    \begin{tabular}{lll}
      \begin{diagram}
        [[T]]      & \rTo{\opmu}        & [[T]]^2\\
        \dTo{\mu}  &                    & \dTo{T\mu}\\
        [[T]]^2    & \rTo{\opmu_{[[T]]}} & [[T]]^3\\
      \end{diagram}
      &
      \begin{diagram}
        [[T]]   & \rTo{\opmu}      & [[T]]^2\\
        \dTo{\opmu} & \rdTo{id}      & \dTo_{T\opeta}\\
        [[T]]^2 & \rTo{\opeta_{T}} & [[T]]\\
      \end{diagram}
    \end{tabular}
  \end{center}
\end{definition}
% section monads_and_comonads (end)

\section{Initial Algebras}
\label{sec:initial_algebras}

\begin{definition}[Algebra]
  \label{def:algebra}
  An \textbf{algebra} or \textbf{$[[F]]$-algebra} is a pair $([[F]], [[sm]]_{[[a]]})$ 
  with respect to some category $[[C]]$ consisting of an object $[[a]]$ called the 
  carrier object of the algebra, an endofunctor $[[F]] : [[C]] \to [[C]]$,
  and a morphism $[[sm]]_{[[a]]} : [[F]]([[a]]) \to [[a]]$ called the structure of the
  algebra.
\end{definition}

\begin{definition}[Homomorphism]
  \label{def:homo_algebras}
  A \textbf{homomorphism} between two algebras $[[sm1]]:[[F]]([[a]]) \to [[a]]$ and $[[sm2]]:[[F]]([[b]]) \to [[b]]$ is 
  a morphism $[[fm]] : [[a]] \to [[b]]$ such that the following diagram commutes:
  \begin{diagram}
    [[a]]          & & \rTo{[[fm]]}          &  &  [[b]]\\
    \uTo{[[sm1]]}  & &                       &  & \uTo_{[[sm2]]}\\
    [[F]]([[a]])   & &  \rTo_{[[F]]([[fm]])} &  & [[F]]([[b]])       
  \end{diagram}
  The above diagram expresses that $[[fm]]$ commutes with the operations of the two algebras.
\end{definition}

\begin{definition}[Initial Algebras]
  \label{def:initial_algebras}
  A \textbf{initial algebra} $[[sm1]] : [[F]]([[a]]) \to [[a]]$ is one such that for any other algebra 
  $[[sm2]] : [[F]]([[b]]) \to [[b]]$ there is a unique homomorphism between them.
\end{definition}

Lets consider a simple example.  Suppose $[[C]]$ is some category with an object $[[Nat]]$, and morphisms
$[[zm]] : 1 \to [[Nat]]$ and $[[sm]] : [[Nat]] \to [[Nat]]$.  Then it is possible to construct any natural number
$[[i]]$ using just these two morphisms.  That is $[[i]]$ is defined by $[[sm]]^{[[i]]}([[zm]])$, hence $[[zm]]$ is 
the natural number zero and $[[sm]]$ is the successor operator.  Now coparing these two operations up into a single
morphism results in $[ [[zm]],[[sm]] ] : 1 + [[Nat]] \to [[Nat]]$.  Defining the functor $[[F]]([[x]]) := 1 + [[x]]$ 
we then obtain $[ [[zm]],[[sm]] ] : [[F]]([[Nat]]) \to [[Nat]]$.  Clearly, this is an $[[F]]$-algebra.  Next we show
that the algebra $[ [[zm]],[[sm]] ] : [[F]]([[Nat]]) \to [[Nat]]$ is initial.

\begin{lemma}[Initiality of the Natural Number Algebra]
  \label{lemma:initiality_nat_alg}
  Let $[[F]]([[x]]) := 1 + [[x]]$. Then algebra $[ [[zm]], [[sm]] ] : [[F]]([[Nat]]) \to [[Nat]]$ is initial.
\end{lemma}
\begin{proof}
  Suppose $[[F]]([[x]]) := 1 + [[x]]$ and $[ [[zm']],[[sm']] ] : [[F]]([[a]]) \to [[a]]$ is an arbitrary algebra. 
  It suffices to show that there exists a unique morphism $[[fm]] : [[Nat]] \to [[a]]$ such that the following diagram commutes:
  \begin{diagram}
    [[Nat]]          & & \rTo{[[fm]]}          &  &  [[a]]\\
    \uTo{[ [[zm]], [[sm]] ]}  & &                       &  & \uTo_{[ [[zm']],[[sm']] ]}\\
    [[F]]([[Nat]])   & &  \rTo_{[[F]]([[fm]])} &  & [[F]]([[a]])       
  \end{diagram}
  \noindent
  This diagram is equivalent to the following:
  \begin{diagram}
    [[Nat]]          & & \rTo{[[fm]]}          &  &  [[a]]\\
    \uTo{[ [[zm]], [[sm]] ]}  & &                       &  & \uTo_{[ [[zm']],[[sm']] ]}\\
    1 + [[Nat]]   & &  \rTo_{\mathit{id} + [[fm]]} &  & 1 + [[a]]       
  \end{diagram}
  We know $[[F]]([[fm]]) := \mathit{id} + [[fm]]$, because the only morphisms from $1$ to $1$ are the identity morphisms and they
  are unique. 
  
  \noindent
  Take $[[fm]](n) := [[sm']]^{(n)} ([[zm']](*))$.  Where $(n)$ is a inductively defined by the following equations:
  \begin{center}
    \begin{math}
      \begin{array}{lll}
        ([[zm]](*)) & := & 0\\
        ([[sm]](a)) & := & 1+(a).\\
      \end{array}
    \end{math}
  \end{center}
  
  \noindent
  At this point we can see that:
  \begin{center}
    \begin{tabular}{clc}
      \begin{math}
        \begin{array}{lll}
          ([[fm]] \circ [[zm]])(*) & = & [[fm]]([[sm]](*))\\
                                   & = & [[sm']]^{(z(*))}([[zm']](*))\\
                                   & = & [[sm']]^{0}([[zm']](*))\\
                                   & = & [[zm']](*)\\
                                   & = & ([[zm']] \circ \mathit{id})(*)\\
        \end{array}
      \end{math}
      & \text{ and } &
      \begin{math}
        \begin{array}{lll}
          ([[fm]] \circ [[sm]])(n) & = & [[fm]]([[sm]](n))\\
                                   & = & [[sm']]^{1+(n)}([[zm']](*))\\
                                   & = & [[sm']]([[sm']]^{(n)}([[zm']](*)))\\
                                   & = & [[sm']]([[fm]](n))\\
                                   & = & ([[sm']] \circ [[fm]])(n).\\
        \end{array}
      \end{math}
    \end{tabular}
  \end{center}
  Thus, 
  \begin{center}
    \begin{math}
      \begin{array}{lll}
        [[fm]] \circ [ [[zm]],[[sm]] ]  & = & [ [[zm']], [[sm']] ] \circ (\mathit{id} + [[fm]]).\\
      \end{array}
    \end{math}
  \end{center}
  The previous results shows that the above diagram commutes. However, we still must show that $[[fm]]$ is unique in order for it to be
  a homomorphism of algebras.  Suppose $[[gm]] : [[Nat]] \to [[A]]$ such that 
  $[[gm]] \circ [ [[zm]],[[sm]] ]  = [ [[zm']], [[sm']] ] \circ (\mathit{id} + [[gm]])$.  Then it must be the case that 
  $[[gm]](z(*)) = [[zm']](*)$ and $[[gm]]([[sm]](a)) = \mathit{id} + [[gm]]$, which can be easily shown to be equivalent
  to $[[fm]]$ by structural induction on the input to $[[fm]]$ and $[[gm]]$.  Therefore, $[[fm]]$ is a homomorphism
  of algebras, and the algebra $[ [[zm]], [[sm]] ] : [[F]]([[Nat]]) \to [[Nat]]$ for functor $[[F]]([[x]]) := 1 + [[x]]$ is
  initial.
\end{proof}

\begin{lemma}[Uniqueness of Initial Algebras]
  \label{lemma:uniqueness_of_initial_algebras}
  If $[[sm]] : [[F]]([[a]]) \to [[a]]$ is an initial algebra of the
  functor $[[F]]$, then $[[sm]]$ is unique up to isomorphism.
\end{lemma}
\begin{proof}
  Suppose $[[sm]] : [[F]]([[a]]) \to [[a]]$ is an initial algebra for the
  functor $[[F]]$.  It suffices to show that the mediating homomorphism between
  $[[sm]]$ and any other initial algebra of the functor $[[F]]$ is an isomorphism.
  Suppose $[[tm]] : [[F]]([[b]]) \to [[b]]$ is another initial algebra of the functor
  $[[F]]$. By initiality of these two algebras we know there exists a unique algebra homomorphism
  $[[fm]] : [[a]] \to [[b]]$ between $[[sm]]$ and $[[tm]]$.
  Similarly we know there exists a unique algebra homomorphism $[[fm']] : [[b]] \to [[a]]$ between
  $[[sm]]$ and $[[tm]]$.  Hence we have the following diagrams:
  \begin{diagram}
    [[a]]        & & \rTo{[[fm]]!}         && [[b]]            && \rTo{[[fm']]!}        & & [[a]]\\
    \uTo{[[sm]]} & &                       && \uTo{[[tm]]}     &&                       & & \uTo{[[sm]]}\\
    [[F]]([[a]]) & & \rTo_{[[F]]([[fm]])}  && [[F]]([[b]])     && \rTo_{[[F]]([[fm']])} & & [[F]]([[a]])
  \end{diagram}
  and
  \begin{diagram}
    [[b]]        & & \rTo{[[fm']]!}         && [[a]]            && \rTo{[[fm]]!}        & & [[b]]\\
    \uTo{[[tm]]} & &                        && \uTo{[[sm]]}     &&                       & & \uTo{[[tm]]}\\
    [[F]]([[b]]) & & \rTo_{[[F]]([[fm']])}  && [[F]]([[a]])     && \rTo_{[[F]]([[fm]])} & & [[F]]([[b]])
  \end{diagram}
  Now we know by initiality that $[[fm]]$ and $[[fm']]$ are unique and hence their compositions
  $[[fm]] \circ [[fm']] : [[a]] \to [[a]]$ and $[[fm']] \circ [[fm]] : [[b]] \to [[b]]$ are also
  unique.  Thus, it must be the case that $[[fm]] \circ [[fm']] = [[id]]_{[[a]]}$ and $[[fm']] \circ [[fm]] = [[id]]_{[[b]]}$.
  So $[[fm]]$ and $[[fm']]$ are mutual inverses.  Therefore, $[[fm]]$ is an isomorphism.
\end{proof}

\begin{lemma}[Initial Algebras are Isomorphisms]
  \label{lemma:initial_algebras_are_isomorphisms}
  If $[[sm]] : [[F]]([[a]]) \to [[a]]$ is an initial algebra for some functor $[[F]]$, then
  it has an inverse $[[sm]]^{-1} : [[a]] \to [[F]]([[a]])$.
\end{lemma}
\begin{proof}
  Suppose $[[sm]] : [[F]]([[a]]) \to [[a]]$ is an initial algebra for the functor $[[F]]$.
  We must show there exists an inverse $[[sm]]^{-1} : [[a]] \to [[F]]([[a]])$ of $[[sm]]$.
  This can be shown by cleverly finding another algebra such that $[[sm]]^{-1}$ is 
  the algebra homomorphism between $[[sm]]$ and this other algebra.  To do this we must
  fill in the holes of the following diagram:
  \begin{diagram}
    [[a]]         & \rTo{[[sm']]} & [[F]]([[a]])\\
    \uTo{[[sm]]}  &         & \uTo_{?}\\
    [[F]]([[a]])  & \rTo_{[[F]](?)} & [[F]](?)
  \end{diagram}
  So our goal is to find an algebra $[[tm]] : [[F]](?) \to [[F]]([[a]])$ such that there is an
  algebra homomorphism $[[sm']] : [[a]] \to [[F]]([[a]])$, and the above diagram commutes.  Choose
  $[[F]]([[sm]]) : [[F]]([[F]]([[a]])) \to [[F]]([[a]])$.  Hence, we obtain the following diagram:
  \begin{diagram}
    [[a]]         & \rTo{[[sm']]}         & [[F]]([[a]])\\
    \uTo{[[sm]]}  &                       & \uTo_{[[F]]([[sm]])}\\
    [[F]]([[a]])  & \rTo_{[[F]]([[sm']])} & [[F]]([[F]]([[a]]))
  \end{diagram}
  Now we know $[[sm]] : [[F]]([[a]]) \to [[a]]$ is an initial algebra, and that 
  $[[F]]([[sm]]) : [[F]]([[F]]([[a]])) \to [[F]]([[a]])$ is clearly an algebra.   
  Thus, by initiality (Definition~\ref{def:initial_algebras}) we know that $[[sm']] : [[a]] \to [[F]]([[a]])$ must
  exist and is an algebra homomorphism.  Therefore, it is unique by Definition~\ref{def:homo_algebras}.
  Hence, $[[sm]] \circ [[sm']] : [[F]]([[a]]) \to [[F]]([[a]])$ is unique.  Furthermore,
  $[[sm']] \circ [[sm]] = [[id]]_{[[F]]([[a]])}$ and $[[sm]] \circ [[sm']] = [[id]]_{[[a]]}$.  
  Therefore, we obtain the following:
  \begin{center}
    \begin{math}
      \begin{array}{lll}
        [[sm']] \circ [[sm]] & = & [[id]]_{[[F]]([[a]])}\\
                             & = & [[F]]([[id]]_{[[a]]})\\
                             & = & [[F]]([[sm]] \circ [[sm']])\\
                             & = & [[F]]([[sm]]) \circ [[F]]([[sm']]).
      \end{array}
    \end{math}
  \end{center}
  Therefore, the inverse of $[[sm]]$ is $[[sm']]$. 
\end{proof}

Since our main example has been the natural number initial algebra $[
[[zm]],[[sm]] ] : 1+[[Nat]] \to [[Nat]]$ we will continue developing
this example by showing how to use initiality to define functions on
the natural numbers by induction.  Consider the doubling function,
$[[double]]$, which takes a natural number as input, and if it is zero 
returns zero, otherwise, it returns the number summed with itself.  We can define
this function by induction using initiality as follows.

\begin{diagram}
    [[Nat]]                  & \rTo{[[double]]}   & [[Nat]]\\
    \uTo{[ [[zm]], [[sm]] ]} &                 & \uTo_{[ [[zm]], [[sm]] \circ [[sm]] ]}\\
    1+[[Nat]]                & \rTo_{[[id]]+[[double]]} & 1+[[Nat]]
  \end{diagram}
Initiality tells us that $[[double]] : [[Nat]] \to [[Nat]]$ is unique and exists.  Furthermore it tells us the following:
\begin{center}
  \begin{math}
    \begin{array}{rll}
      [[double]] \circ [[zm]] & = & [[zm]] \circ [[id]] = [[zm]]\\
      [[double]] \circ [[sm]] & = & [[sm]] \circ [[sm]] \circ [[double]]\\
    \end{array}
  \end{math}
\end{center}
This is equivalent to the following:
\begin{center}
  \begin{math}
    \begin{array}{rll}
      [[double]]([[zm]](*)) &  = & [[zm]](*)\\
      [[double]]([[sm]](n)) & = &  [[sm]]([[sm]]([[double]](n)))\\
    \end{array}
  \end{math}
\end{center}
Therefore inductively defined functions are no more than algebra homomorphisms between initial algebras!  

\begin{diagram}
    [[Nat]]                  & \rTo{[[plus]]}   & [[Nat]]^{[[Nat]]}\\
    \uTo{[ [[zm]], [[sm]] ]} &                  & \uTo_{[ [[\xm.xm]], [[\fm.h(\xm.h(sm(fm(xm))))]] ]}\\
    1+[[Nat]]                & \rTo_{[[plus]]}    & 1+[[Nat]]^{[[Nat]]}
  \end{diagram}
Initiality tells us that $[[plus]] : [[Nat]] \to [[Nat]]^{[[Nat]]}$ is unique and exists.  Furthermore it tells us the following:
\begin{center}
  \begin{math}
    \begin{array}{rllll}
      [[plus]] \circ [[zm]] & = & [[\fm.h(\xm.xm)]] \circ [[plus]]\\
      [[plus]] \circ [[sm]] & = & [[\fm.h(\xm.h(sm(fm(xm))))]] \circ [[plus]]\\
    \end{array}
  \end{math}
\end{center}
This is equivalent to the following:
\begin{center}
  \begin{math}
    \begin{array}{rll}
      [[plus]]([[zm]](*)) & = & ([[\fm.h(\xm.xm)]])([[plus]]([[zm]])) = [[\xm.xm]]\\
      [[plus]]([[sm]](n)) & = & ([[\fm.h(\xm.h(sm(fm(xm))))]])([[plus]]([[sm]](n))) = \lambda [[xm]].[[sm]]([[plus]]([[sm]](n))([[xm]]))\\
    \end{array}
  \end{math}
\end{center}
% section initial_algebras (end)

\section{Final Coalgebras}
\label{sec:final_coalgebras}
Lets consider the infinite stream of ones. 
\begin{center}
  \begin{math}
    [[1 :: 1 :: 1 :: 1]] \cdots
  \end{math}
\end{center}
Now what can be said about such an object?  We definitely can make observations about it.
That is, we can observe that, say, the first element is a $1$.  In addition to that we can 
observe that the second element is also a $1$.  Furthermore, we can make the observation 
that what follows the second element is still an infinite streams of ones.
We can define this stream using a few morphisms.  That is $[[head]] : [[A]] \to \mathbb{N}$ and
$[[next]] : [[A]] \to [[A]]$ are the morphisms, where $[[A]]$ is some fixed set.
Now notice that for any $a \in [[A]]$ we have $[[head]](a) = 1$, and 
$[[next]](a) = a' \in A$.  Now if we compose these two we obtain 
$[[head]]([[next]](a)) = [[head]](a') = 1$.  Thus, using these morphisms we can completely
define our stream above by taking for each position $n \in \mathbb{N}$ in the stream 
$[[head]]([[next]]^{n}(a))$ to be the value at that position for any $a \in [[A]]$.  These
two morphisms $[[head]]$ and $[[next]]$ are the observations we can make about the element 
$a$. It just happens, in this simple example, that the observations are the same for all 
elements of $[[A]]$.

At this point we can take the product of the two morphisms $[[head]]$ and $[[next]]$ to
obtain the morphism $\langle [[head]],[[next]] \rangle : [[A]] \to \mathbb{N} \times [[A]]$.
If we define the functor $[[F]]([[x]]) := \mathbb{N} \times [[x]]$ then we can redefine the
product as $\langle [[head]],[[next]] \rangle : [[A]] \to [[F]]([[A]])$.  Furthermore,
if we make $\mathbb{N}$ arbitrary and call it say $[[B]]$ then we obtain 
the functor $[[F]]([[x]]) := [[B]] \times [[x]]$.  Using this new definition of the functor
$[[F]]$ and $\langle [[head]],[[next]] \rangle : [[A]] \to [[F]]([[A]])$ we can define any
infinite stream of elements of $[[B]]$.  Taking the functor $[[F]]$ and the pair
$([[A]], \langle [[head]], [[next]] \rangle)$ we obtain what is called a coalgebra.

\begin{definition}[Coalgebra]
  \label{def:coalgebra}
  A \textbf{coalgebra} or \textbf{$[[F]]$-coalgebra} is a pair $([[F]], [[sm]]_{[[A]]})$ 
  with respect to some category $[[C]]$ consisting of an object $[[A]]$ called the 
  carrier object of the coalgebra, an endofunctor $[[F]] : [[C]] \to [[C]]$,
  and a morphism $[[sm]]_{[[A]]} : [[A]] \to [[F]]([[A]])$ called the structure of the
  coalgebra.
\end{definition}
It is often the case that coalgebras are denoted simply by specifying the type of the structure of
the coalgebra.  That is $[[sm]]_{[[A]]} : [[A]] \to [[F]]([[A]])$.  This is precise, because it gives all
of the elements of a coalgebra, but the category, but this is usually determinable from the context.  So, in our
example above we can see that $\langle [[head]],[[next]] \rangle : [[A]] \to [[F]]([[A]])$ is indeed a coalgebra with
respect to the category $[[Set]]$.

Now consider the infinite stream of twos.
\begin{center}
  \begin{math}
    [[2 :: 2 :: 2 :: 2]] \cdots
  \end{math}
\end{center}
This is definable by a coalgebra $\langle \hat{[[head]]},\hat{[[next]]} \rangle : [[B]] \to [[F]]([[B]])$ with respect to the 
category $[[Set]]$.  Suppose further that there is a function $[[fm]] : [[A]] \to [[B]]$ such that 
$\hat{[[head]]} \circ [[fm]] = [[head]]$ and $\hat{[[next]]} \circ [[fm]] = [[fm]] \circ [[next]]$.  In this case $[[fm]]$ is called
a homomorphism between coalgebras.  
\begin{definition}[Homomorphism]
  \label{def:homo_coalgebras}
  A \textbf{homomorphism} between two coalgebras $[[sm1]]:[[A]] \to [[F]]([[A]])$ and $[[sm2]]:[[B]] \to [[F]]([[B]])$ is 
  a morphism $[[fm]] : [[A]] \to [[B]]$ such that the following diagram commutes:
  \begin{diagram}
    [[A]]          & & \rTo{[[fm]]}          &  &  [[B]]\\
    \dTo{[[sm1]]}  & &                       &  & \dTo_{[[sm2]]}\\
    [[F]]([[A]])   & &  \rTo_{[[F]]([[fm]])} &  & [[F]]([[B]])       
  \end{diagram}
  The above diagram expresses that $[[fm]]$ commutes with the operations of the two coalgebras.
\end{definition}

\begin{definition}[Final Coalgebras]
  \label{def:final_coalgebras}
  A \textbf{final coalgebra} $[[sm1]] : A \to [[F]]([[A]])$ is one such that for any other coalgebra 
  $[[sm2]] : [[B]] \to [[F]]([[B]])$ there is a unique homomorphism between them.
\end{definition}

% section final_coalgebras (end)
